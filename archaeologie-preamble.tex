\usepackage{libertine}
\renewcommand*\ttdefault{lmvtt}

\usepackage[					% use  for bibliography
	backend=biber,
	style=archaeologie,
	lstpublishers,
	lstlocations,
	lstabbrv,
]{biblatex}
\renewcommand\bibfont{\normalfont
\footnotesize}
\usepackage{metalogo}
\usepackage{hologo}
\usepackage{babel}
\usepackage{coolthms}
\usepackage[					% advanced quotes
%	strict=true,					% 	- warning are errors now
autostyle=true,%
]{csquotes}
\usepackage{multicol}
\setlength{\columnsep}{1.5cm}
\setlength{\columnseprule}{0.2pt}

\usepackage[ 
%	headsepline, 
%	footsepline,
%	plainfootsepline, 
%markcase=upper, 
automark, 
draft=false,
]{scrlayer-scrpage} 
\pagestyle{scrheadings}
\clearscrheadfoot
	\ihead{\normalfont\footnotesize \texttt{bib}\LaTeX-style \texttt{archaeologie 2.0} (c) by Lukas C. Bossert | Johannes Friedl}%
\rofoot{\large\footnotesize  \textbf{\sffamily \thepage}}

\usepackage[%
			%flushmargin, %
%			marginal,
			ragged,%
			hang, %
			bottom%
			]{footmisc} %Fussnoten


\usepackage{framed}
\usepackage{enumitem}
\setlength{\parindent}{0pt}
\setlength{\parskip}{6pt plus 2pt minus 2pt}
\setenumerate[1]{label=(\alph*),leftmargin=*,nolistsep,parsep=\parskip}
\usepackage{changepage}
\makeindex

\defbibheading{empty}{}


\listfiles
%\EnableCrossrefs
%\CodelineIndex
%\RecordChanges
\addbibresource{archaeologie.bib}
%\addbibresource{archaeologie-ancient.bib}
%\addbibresource{antike-corpora.bib} %Version 2.0
\usepackage{caption}

\usepackage{listings}
\usepackage{chngcntr}
\AtBeginDocument{\counterwithin{lstlisting}{section}}

\usepackage{xcolor}
\definecolor{codeblue}{RGB}{0,65,137}
\definecolor{codegreen}{RGB}{147,193,26}
\definecolor{codegray}{rgb}{0.5,0.5,0.5}
\definecolor{codepurple}{rgb}{0.58,0,0.82}
\definecolor{backcolour}{rgb}{0.95,0.95,0.92}

\newenvironment{bsp}{\begin{framed}\begin{footnotesize}
\begin{adjustwidth}{.3cm}{.3cm}}{\end{adjustwidth}
\end{footnotesize}\end{framed}}
%------
\usepackage{environ}
\usepackage[tikz]{bclogo}
\usepackage{tikz}
\usetikzlibrary{calc}
%\NewEnviron{bibbsp}[1]
%  {\par\medskip\noindent\footnotesize
%  \begin{tikzpicture}
%    \node[inner sep=0pt] (box) {\parbox[t]{\textwidth}{%
%      \begin{minipage}{.15\textwidth}
%      \hspace{.5em}\scriptsize \textbf{\cref{#1}}\hfill
%      \end{minipage}%
%      \begin{minipage}{.83\textwidth}
%      \BODY
%      \end{minipage}\hfill}%
%    };
%    \draw[codeblue,line width=2pt] 
%      ( $ (box.north east) + (-5pt,3pt) $ ) -- ( $ (box.north east) + (0,3pt) $ ) -- ( $ (box.south east) + (0,-3pt) $ ) -- + (-5pt,0);
%    \draw[codeblue,line width=2pt] 
%      ( $ (box.north west) + (5pt,3pt) $ ) -- ( $ (box.north west) + (0,3pt) $ ) -- ( $ (box.south west) + (0,-3pt) $ ) -- + (5pt,0);
%  \end{tikzpicture}\par\medskip%
%}
%-----

\newcommand{\printbib}[2][4em]{%
\begingroup
\begin{bibbsp}{#2}
\footnotesize
\begin{refsection}
\setlength{\labwidthsameline}{#1} 
\nocite{#2}
\printbibliography[heading=none]
\end{refsection}
\end{bibbsp}
\endgroup
}

\newcommand{\printbiball}[2][4em]{%
\begingroup
\setlength{\labwidthsameline}{#1} 
\begin{bibbsp}{#2}
\footnotesize%
\begin{itemize}%[\bfseries]
\begin{refsection}%
\nocite{#2}%
\item[English:]{\printbibliography[heading=none]}
\item[German:]\foreignlanguage{ngerman}{\printbibliography[heading=none]}
\item[Italian:]\foreignlanguage{italian}{\printbibliography[heading=none]}
\item[French:]\foreignlanguage{french}{\printbibliography[heading=none]}
\item[Spanish:]\foreignlanguage{spanish}{\printbibliography[heading=none]}
\end{refsection}
\end{itemize}%
\end{bibbsp}
\endgroup
}

 \DeclareCaptionFormat{listing}{#1#2#3}
 \captionsetup[lstlisting]{format=listing,
 singlelinecheck=false, margin=0pt, font={sf},size=footnotesize}
\lstdefinestyle{bibentry}{%
	language=[LaTeX]TeX,
    backgroundcolor=\color{backcolour},   
    commentstyle=\color{codegreen},
    keywordstyle=\color{codeblue},
    numberstyle=\tiny\color{codegray},
    stringstyle=\color{codepurple},
    escapeinside={*@}{@*},          % if you want to add LaTeX within your code
    texcsstyle=*\color{codeblue},
    morekeywords={cites, parencites, parencite, textcite, textcites, citeauthor, citetitle,@String,
    						@Article, @Book,@Collection,@Proceedings,@Reference,@Thesis,
    						@Inproceedings,@Talk,@Review,@Inreference,@Incollection,
    						},
    basicstyle=\ttfamily\footnotesize,
    breakatwhitespace=false,         
    breaklines=true,   
    numberbychapter=false,   
    captionpos=b,                    
    keepspaces=true,               
	%framexleftmargin=5mm, 
	frame=shadowbox,
	rulesepcolor=\color{codeblue},
    numbers=left,                    
    numbersep=5pt,            
    showspaces=false,                
    showstringspaces=false,
    showtabs=false,                  
    tabsize=2,
    literate=
            *{\{}{{{\color{codegreen}{\{}}}}{1}
            {\}}{{{\color{codegreen}{\}}}}}{1}
            {[}{{{\color{codegreen}{[}}}}{1}
            {]}{{{\color{codegreen}{]}}}}{1},
}
\renewcommand{\lstlistingname}{Example}%rename caption
\renewcommand{\lstlistlistingname}{List of examples}%rename caption

 \lstdefinestyle{code}{%
	language=[LaTeX]TeX,
    backgroundcolor=\color{white},   
    commentstyle=\color{codegreen},
    keywordstyle=\color{codeblue},
    numberstyle=\small\color{codegray},
    stringstyle=\color{codepurple},
	escapeinside={*@}{@*},
    texcsstyle=*\color{codeblue},
    morekeywords={},
    basicstyle=\ttfamily\footnotesize,
    breakatwhitespace=false,         
    breaklines=true,                 
    captionpos=b,                    
    keepspaces=true,                 
    %numbers=left,                    
    stepnumber=1,
    numbersep=5pt,            
    showspaces=false,                
    showstringspaces=false,
    showtabs=false,                  
    tabsize=2,
      morekeywords={cites, parencites, parencite, textcite, textcites, citeauthor, citetitle,
    						@String,
    						@Article, @Book,@Collection,@Proceedings,@Reference,@Thesis,
    						@Inproceedings,@Talk,@Review,@Inreference,@Incollection,
    						},
    literate=
            *{\{}{{{\color{codeblue}{\{}}}}{1}
            {\}}{{{\color{codeblue}{\}}}}}{1}
            {[}{{{\color{codeblue}{[}}}}{1}
            {]}{{{\color{codeblue}{]}}}}{1},
}
%%% Always I forget this so I created some aliases
\def\ContinueLineNumber{\lstset{firstnumber=last}}
\def\StartLineAt#1{\lstset{firstnumber=#1}}
\let\numberLineAt\StartLineAt

\lstset{style=code}
\lstMakeShortInline[style=code]{|}


\usepackage{hyperxmp}
%\usepackage{hyperref}
\hypersetup{					% setup the hyperref-package options
	pdftitle={bib\LaTeX-archaeologie},	% 	- title (PDF meta)
	pdfsubject={This citation-style covers the citation and bibliography rules of the German Archaeological Institute. 
Various options are available to change and adjust the outcome according to one's own preferences. 
The style is compatible with the English, German, Italian, Spanish and French languages, since all bibstrings used are defined in each language.},% 	- subject (PDF meta)
	pdfauthor={Lukas C. Bossert, Johannes Friedl},	% 	- author (PDF meta)
	pdfauthortitle={},
	pdfcopyright={This work may be distributed and/or modified under the
conditions of the LaTeX Project Public License, either version 1.3
of this license or (at your option) any later version.},
	pdfhighlight=/N,
	pdfdisplaydoctitle=true,
	pdfdate={\the\year-\the\month-\the\day}
	pdflang={de},
	pdfcaptionwriter={Lukas C. Bossert},
	pdfkeywords={biblatex, archaeology, humanities},
%	pdfproducer={XeLaTeX},
	pdflicenseurl={http://www.latex-project.org/lppl.txt},
	plainpages=false,			% 	- 
  colorlinks   = true, %Colours links instead of ugly boxes
  urlcolor     =  codeblue, %Colour for external hyperlinks
  linkcolor    = codeblue, %Colour of internal links
  citecolor   = black, %Colour of citations
  	linktoc=page,
  	pdfborder={0 0 0},			% 	-
	breaklinks=true,			% 	- allow line break inside links
	bookmarksnumbered=true,		%
	bookmarksopenlevel=4,
	bookmarksopen=true,		%
	final=true	% = true, nur bei web-Dokument!! (wichtig!!)
}
 \crefformat{lstlisting}{#2example\ #1#3}
 \Crefformat{lstlisting}{#2 #1#3}
\crefmultiformat{lstlisting}{#2examples #1#3}{; #2#1#3}{; #2#1#3}{; #2#1#3}
\Crefmultiformat{lstlisting}{#2 #1#3}{; #2#1#3}{; #2#1#3}{; #2#1#3}
\Crefrangeformat{lstlisting}{#3examples #1#4--#5examples #2#6}
\crefrangeformat{lstlisting}{#3examples #1#4--#5examples #2#6}
