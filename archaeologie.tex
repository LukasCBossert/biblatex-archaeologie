% archaeologie --%
%               biblatex fuer Archaeologen, Historiker und Philologen
% Copyright (c) 2016 Lukas C. Bossert | Johannes Friedl
%  
% This work may be distributed and/or modified under the
% conditions of the LaTeX Project Public License, either version 1.3
% of this license or (at your option) any later version.
% The latest version of this license is in
%   http://www.latex-project.org/lppl.txt
% and version 1.3 or later is part of all distributions of LaTeX
% version 2005/12/01 or later.

\documentclass[a4paper,
10pt,
greek,
french,
italian, %ngerman,
english,
ngerman
]{ltxdoc}
\CodelineNumbered
\AtBeginDocument{\RecordChanges}
\AtEndDocument{\PrintChanges}
  \usepackage[T1,LGR]{fontenc}		% font types and character verification
\usepackage{ebgaramond}
	\usepackage{metalogo}
	\usepackage{hologo}
 \usepackage{babel}
\usepackage{coolthms}
\usepackage[					% advanced quotes
	strict=true,					% 	- warning are errors now
	style=ngerman,					% 	- german quotes
]{csquotes}
\usepackage{multicol}
\setlength{\columnsep}{1.5cm}
\setlength{\columnseprule}{0.2pt}
\usepackage{framed}
\usepackage{enumitem}
\setlength{\parindent}{0pt}
\setlength{\parskip}{6pt plus 2pt minus 2pt}
\setenumerate[1]{label=(\alph*),leftmargin=*,nolistsep,parsep=\parskip}
\usepackage{changepage}

\makeindex
   
\newenvironment{bsp}{\begin{framed}\begin{small}
\begin{adjustwidth}{1cm}{1cm}}{\end{adjustwidth}
\end{small}\end{framed}}


\listfiles
\EnableCrossrefs
\CodelineIndex
\RecordChanges


 \usepackage[disable]{todonotes} % notes not showed
 
 

\usepackage[					% use  for bibliography
	backend=biber,
	style=archaeologie,
%	eprint=false,
%	doi=false,
%	url=false,
%inreferences,
	%translation,
	%bibfullname,
	%noabbrevs,
	%publisher,
	%edby,
	%yearseries,
%yearinparens,
	%lastnames,
	%fullnames,
%	scshape,
%	width=6em,
%counter,
strings,
	%%%%%%%%%%%%%%%%%%%%%%%
]{biblatex}


\addbibresource{archaeologie.bib}
%\addbibresource{antike-corpora.bib} %Version 2.0
\renewcommand\bibfont{\normalfont\small}



\usepackage{listings}
\lstMakeShortInline[basicstyle=\small\ttfamily]{|}

\hypersetup{  colorlinks   = true, %Colours links instead of ugly boxes
  urlcolor     =  blue, %Colour for external hyperlinks
  linkcolor    = black, %Colour of internal links
  citecolor   = blue, %Colour of citations
  }	
\begin{document}
%  \DocInput{archaeologie.dtx}

%\MakeShortVerb{\|}
 \newenvironment{syntax}{\begin{small}\medskip\hspace*{2em}}{\par\medskip\end{small}}
 %\def\verbatimchar{;}



 \title{\texttt{archaeologie} -- \\\texttt{bib\LaTeX} for archaeologists\footnote{Also very handy as well for (ancient) History or Classics.
The development of the code is done at \url{https://github.com/LukasCBossert/biblatex-archaeologie}: 
Comments and critics are welcome. 
We thank  ›moewew‹ und Herbert Voß for their big help on the code.
} }
 \author{Lukas C. Bossert\thanks{\url{lukas@digitales-altertum.de}} \and Johannes Friedl}
 \date{Version: 1.42 (2016-03-17)}
 
 
 \maketitle
 \begin{abstract}
 \begin{multicols}{2}
\noindent \foreignlanguage{english}{This citation-style covers the citation and bibliography rules of the German Archaeological Institute. Various options are available to change and adjust the outcome according to one's own preferences. 
The style is compatible with the English, Italian and French languages, since all |bibstrings| used are defined in each language.}\columnbreak\\
\noindent Der Stil setzt die Zitations- und Bibliographievorgaben  des Deutschen Archäologischen Instituts (DAI) (Stand 2014) um. Verschiedene zusätzliche Optionen erlauben das Erscheinungsbild jedoch auch zu verändern, um eigenen Vorlieben anzupassen.
Der Stil ist nicht nur für Dokumente in deutscher Sprache geeignet, sondern wurde ebenso für die Sprachen Englisch, Italienisch und Französisch angepasst.

 \end{multicols}
 \end{abstract}
\section{Letzte Änderungen}
Änderungen seit Version 1.4
\begin{itemize}
\item Verbesserung der |\textcite|-Befehle
\item Vereinheitlichung des Abstands zwischen Label und Bibliographieeintrag (s. \cref{width}).
\item Optionales An- oder Abschalten der extra Bibliographie |archaeologie-abbrv.bib| (s. \cref{strings}).
\item Anpassung an |biblatex 3.3|, was dazu führte, dass alle gängigen |\cite|-Befehle verwendet werden können.
\item Umstellung der  Optionsnamen auf englische Bezeichnungen (s. \cref{preamble_options}).
\item Hinzufügen der Paketoption |counter| (s. \cref{counter}).
\end{itemize}

\changes{v2.0}{2015/11/10}{Ergänzung: Feld |number| bei |@inreference| wird jetzt ausgelesen.}

\newpage
\begin{multicols}{2}
{\parskip=0mm \tableofcontents}
\end{multicols}

\section{Verwendung}
 \DescribeMacro{archaeologie}  Der Name des Stils ist |archaeologie| und wird an entsprechender Stelle geladen.

\begin{syntax} |\usepackage[style=archaeologie,%|\\
 |            |\meta{weitere Optionen}|]{biblatex}|\\
 |\bibliography|\marg{|bib|-Datei}\end{syntax}
 
Dabei kann man weitere der \enquote{konventionellen} |biblatex|-Optionen oder der -- weiter unten beschriebenen -- von |archaeologie| zur Verfügung gestellten Optionen laden.

 |archaeologie| lädt standardmäßig den DAI-Stil im Autor-Jahr-System. 
 Um schnell und einfach im DAI-Stil zu zitieren, benötigt es keine weiteren Einstellungen und Optionen.

Am Ende des Dokuments oder an gewünschter Stelle muss der  |\printbibliography|-Befehl stehen, sodass  eine Bibliographie angelegt und ausgegeben wird. 
 Da |archaeologie| unterschiedliche Zitierweisen von Textsorten wie antiker Primärliteratur oder wissenschaftlicher Sekundärliteratur unterstützt, empfiehlt es sich, die Bibliographieaufteilung dementsprechend anzupassen. Verschiedene Möglichkeiten zur Gestaltung der Bibliographie  (\cref{bibliographie}).

\section{Übersicht}
Im Folgenden sind   kurz die möglichen Optionen, mit denen der Stil |archaeologie| aufgerufen werden kann, aufgeführt. 
 Dazu kann man -- quasi auf eigene Gefahr -- noch die konventionellen |biblatex|-Optionen (insbesondere zur Formatierung der Abstände etc. des Literaturverzeichnisses) verwenden. Näheres zu diesen findet man in der Dokumentation von |biblatex|.

 \subsection{Paketoptionen (Präambel)}\label{preamble_options}
 \begin{description}
 \item[strings] Aktiviert die extra Bibliographiedatei |archaeologie-abbrv.bib|, mit der man Zeitschrifen oder Reihen mit entsprechenden Abkürzungen zitieren kann. s. \cref{strings}. 
 \item[edby] (zuvor |hrsgv|) Bei Sammelbänden steht anstatt \enquote{Hrsg.} nun \enquote{hrsg. v.}. Siehe \cref{edby}.


\item[yearseries] (zuvor |jahrreihe|) Die Reihe wird erst nach der Jahreszahl ausgegeben.  \cref{yearseries}.
 
 \item[yearinparens] (zuvor |jahrinklammer|)  Die Jahreszahl wird in Klammern gesetzt.  \cref{yearinparens}.

\item[translation] (zuvor |uebersetzung|)  Anzeige der Angaben des Originaltitels, Übersetzers und der Sprache, aus der übersetzt wurde. Bei Einträgen, die |options=antik| haben, ist dies Standard.  \cref{translation}. 

\item[noabbrevs] (zuvor |keineabkuerzung|)  Die Abkürzungen von Zeitschriften und Serien (|shortjournal|\linebreak |shortseries|) werden ausgeschrieben, wofür die Felder |journaltitle| und |series| ausgelesen werden.  \cref{noabbrevs}.

\item[publisher] (zuvor |verlag|)  Angabe aller Verlagsorte und Verlag selbst. Ändert die Formatierung der Edition und Erstausgabe.  \cref{publisher}.

\item[bibfullname] (zuvor |bibvollername|)  Schreibt die Autoren/-Herausgeber mit vollem Namen in der Bibliographie.  \cref{bibfullname}.

\item[inreferences] (zuvor |lexika|)  Bibliographieeinträge mit |@inreference| werden bei aktivierter Funktion vollreferenziert geschrieben. Siehe \cref{inreferences}.

\item[lastnames] (zuvor |nurnachname|)  Schreibt die Autoren/-Herausgeber, die über  |citeauthor|\-\marg{key}  im Fließtext aufgerufen werden, nur mit Nachnamen (= Erscheinen in Fußnoten).  \cref{lastnames}.

\item[fullnames] (zuvor |vollername|)  Schreibt die Autoren/-Herausgeber, die über |citeauthor|\-\marg{key} im Fließtext aufgerufen werden,  mit vollem Vor- und Nachnamen -- sofern diese im Bibliographieeintrag vorhanden sind.  \cref{fullnames}.


\item[scshape] (zuvor |kapitaelchen|)  Die Namen in den Fußnoten werden in Kapitälchen gesetzt \cref{scshape}, ausgenommen sind Werke unbekannter Herkunft (\cref{unbekannt}) und Werke antiker Autoren (\cref{antik,frgantik}).


\item[width] (zuvor |abstand|)  Mit dieser Option wird in der Bibliographie der Abstand zwischen dem Label und dem Vollzitat nach Belieben gesteuert.
\cref{width}.

\item[counter] Angabe über die Anzahl an Zitationen des Werks im Text. \cref{counter}.
\end{description}


 \subsection{Literaturoptionen (Bibliographieeintrag)}
 Zusätzlich kann ein einzelner Eintrag durch folgende Werte in seinem |options|-Feld manipuliert werden. Siehe dazu auch \cref{optionen-literatur} und \cref{beispiele}. 

 \begin{description}
 \item[antik] Zeichnet den Eintrag als antike Quelle aus. Siehe \cref{antik}.
% \item[frg] Zeichnet den Eintrag als Fragment aus. Siehe \cref{frg}.
 \item[frgantik] Zeichnet den Eintrag als antikes Fragment aus. Siehe \cref{frgantik}.
 \item[corpus] Nur das |shorthand|-Feld wird beim Folgezitat ausgegeben. Wichtig für beispielsweise Inschriften- oder Münzcorpora (CIL, AE, RIC, etc.). Siehe \cref{corpus}.
% \item[lexikon] Zeichnet den Eintrag als ein zitierfähiges Lexikon aus, das über den abgekürzten Haupttitel zitiert wird (RE, DNP, LTUR, LIMC, etc.). Siehe \cref{lexikon}.
 %\item[unbekannt] Zeichnet den Eintrag als anonymes Werk aus, sodass nach dem Feld |shorthand| zitiert wird. Siehe \cref{unbekannt}.
 \end{description}

 \changes{v1.1}{2015/06/04}{Neue Optionen in Zusammenfassung ergänzt.}


 


 \subsection{cite-Befehle}\label{cite-befehle}
 \DescribeMacro{\cite}
Die einfachste Weise zum Zitieren wird mit |\cite| bewerkstelligt:\par
 \begin{syntax}|\cite|\oarg{prenote}\oarg{postnote}\marg{Schlüssel}\end{syntax}
 wobei \meta{prenote} eine einleitende Bemerkung (z.B. \enquote{Vgl.}) ist und \meta{postnote} für gewöhnlich die Seitenzahl. Wenn nur ein optionales Argument gegeben wird, so ist das die Seitenzahl:\par
 \begin{syntax}|\cite|\oarg{postnote}\marg{Schlüssel}\end{syntax}
 \meta{Schlüssel} ist dabei in jedem Fall der Schlüssel des Eintrags aus der |bib|-Datei.

 \DescribeMacro{\cites}
 Möchte man mehrere Autoren/Werke zugleich zitieren, eignet sich am besten der |\cites|-Befehl:\par
 \begin{syntax}\noindent|\cites|(pre-prenote)(post-postnote)\oarg{prenote}\oarg{postnote}\marg{Schlüssel}\%\\
 				\indent\oarg{prenote}\oarg{postnote}\marg{Schlüssel}\%\\%
 				\indent\oarg{prenote}\oarg{postnote}\marg{Schlüssel}\ldots
 				\end{syntax}
\begin{center} * * * \end{center}
 
 \DescribeMacro{\parencite}
 Möchte man Literaturangaben (bspw. in den Fußnoten) in Klammern setzen, dann empfiehlt sich dies mittels\par
   \begin{syntax}|\parencite|\oarg{prenote}\oarg{postnote}\marg{Schlüssel}\end{syntax} zu tun.
 Dieser Befehl berücksichtigt die Klammerregelung, die besagt, dass runde Klammern innerhalb einer Klammerumgebung als eckigen Klammern geschrieben werden müssen.
 Dies ist vor allem bei Lexikaeinträgen der Fall, wie das Beispiel unter \cref{inreference} zeigt.

 \DescribeMacro{\parencites}
 Möchte man mehrere Literaturangaben (bspw. in den Fußnoten) in Klammern setzen, dann empfiehlt sich dies mittels\par
 \begin{syntax}\noindent|\parencites|(pre-prenote)(post-postnote)\oarg{prenote}\oarg{postnote}\marg{Schlüssel}\%\\
 				\indent\oarg{prenote}\oarg{postnote}\marg{Schlüssel}\%\\%
 				\indent\oarg{prenote}\oarg{postnote}\marg{Schlüssel}\ldots
 				\end{syntax} zu tun.
 Dieser Befehl berücksichtigt die Klammerregelung, die besagt, dass runde Klammern innerhalb einer Klammerumgebung als eckigen Klammern geschrieben werden müssen.
 Dies ist vor allem bei Lexikaeinträgen der Fall, wie das Beispiel unter \cref{inreference} zeigt.
 \begin{center} * * * \end{center}
 
 \DescribeMacro{\textcite}
Zu den bisher aufgeführten |\cite|-Befehlen gibt es zusätzlich die Möglichkeit einen Eintrag bspw. im Fließtext mit |\textcite|  zu zitieren: \par
 \begin{syntax}\noindent|\textcite|\oarg{prenote}\oarg{postnote}\marg{Schlüssel}
 				\end{syntax} 

\DescribeMacro{\textcites}
Wiederum gibt es die Möglichkeit mehrere Werke mit |\textcites| anzugeben: \par
 \begin{syntax}\noindent|\textcites|(pre-prenote)(post-postnote)\oarg{prenote}\oarg{postnote}\marg{Schlüssel}\\%
 \indent\oarg{prenote}\oarg{postnote}\marg{Schlüssel}\%\\%
 				\indent\oarg{prenote}\oarg{postnote}\marg{Schlüssel}\ldots
 				\end{syntax} 




\begin{center} * * * \end{center}
 \DescribeMacro{\citeauthor}   \DescribeMacro{\citetitle}
Zum ›normalen‹ |\cite|-Befehl kann man im Fließtext und in den Anmerkungen auf den Autor/Herausgeber und das Werk verweisen.
Dies wird über den Befehl\par
  \begin{syntax}|\citeauthor|\oarg{prenote}\oarg{postnote}\marg{Schlüssel}\end{syntax} \par 
  und \par 
   \begin{syntax}|\citetitle|\oarg{prenote}\oarg{postnote}\marg{Schlüssel}\end{syntax}ausgeführt. Weitere Informationen unter \cref{fullnames}.
   

   

\subsection{Einträge mit @string}
Die Zitationsrichtlinie des DAI sieht vor, dass vor allem Zeitschriften und Reihen nach einer vorgegebenen Liste abgekürzt werden.\footnote{\url{www.dainst.org/documents/10180/70593/02_Liste-Abkürzungen_quer.pdf}  (\today)}

 \DescribeMacro{@string}Der Stil |archaeologie| geht auf diese Vorgabe ein und kürzt Zeitschriften/Reihen ab.
Um Fehler zu vermeiden und unnötiges Tippen der teilweise langen Zeitschriftennamen, bzw. Reihen zu ersparen, arbeitet |archaeologie| mit |@string|-Angaben.
Dies hat den Vorteil, dass  mittels |@string| mehreren Bibliographieeinträgen einen zentral definierten Wert zugewiesen wird.
Der |@string| wird in der |bib|-Datei am Anfang geladen, daher muss er vor alle anderen Bibliographieeinträge stehen.

Um dieses Angebot an ›Vereinfachung‹ zu nutzen,  müssen Zeitschriftennamen (|journaltitle|) und ihre Kurzformen (|shortjournal|), sowie Reihen (|series|) und deren Kurzformen (|shortseries|) mittels |@string| geschrieben werden.
In \cref{listen} sind die Abkürzungen nach DAI-Vorgabe aufgeführt, sodass der |@string| (mit den Ergänzungen |-kurz|, bzw. |-lang|) (linke Spalte) abgelesen werden kann.
Dieser |@string|
% (bspw. |AyasofyaMuezYil|; siehe \ref{AyasofyaMuezYil-kurz}, bzw. \ref{AyasofyaMuezYil-lang}) 
wird dann OHNE Klammersetzung in die Felder (bspw. |journaltitle|) geschrieben.\footnote{Verwendet man bspw. das Programm JabRef in der Fensteransicht, dann muss \#|AyasofyaMuezYil|\# geschrieben werden. JabRef konvertiert dies intern zu einem |@string|.}

Ein Beispiel zeigt den Umgang:

|@article{Koyunlu_1990,|\\
|author = {Koyunlu, A.},|\\
|title = {Die Bodenbelage und der Errichtungsort der Hagia Sophia},|\\
|pages = {147--156},|\\
|volume = {11},|\\
|year = {1990},|\\
|journaltitle = AyasofyaMuezYil-lang,|\\
|shortjournal = AyasofyaMuezYil-kurz|\\
|}|\\

Dieser Artikel erschien in einer nicht geläufigen Zeitschrift, die zudem nach DAI-Vorgabe mit ›AyasofyaMüzYıl‹ abgekürzt werden soll.
Um sich bei diesem Beispiel die Arbeit zu sparen den Buchstaben ›ı‹ manuell einzufügen, wird er im |@string| mit ›i‹ geschrieben (weitere Besonderheiten siehe \cref{listen}) und erscheint dann nach dem Kompilieren in der korrekten Schreibweise:


\begin{bsp}
\begin{refsection}
\nocite{Koyunlu_1990}
\printbibliography[title={Forschungsliteratur}]
\end{refsection}
\end{bsp}


Die standardmäßige Abkürzung kann optional ausgeschaltet werden.
Verwendet man die Option |noabbrevs| (siehe \cref{noabbrevs}), dann ändert sich die Ausgabe entsprechend:
\begin{bsp}\todo[noline]{manuelles Beispiel}
A. Koyunlu, Die Bodenbelage und der Errichtungsort der Hagia Sophia, {\color{red}Ayasofia Müzesi yıllığı. Annual of Ayasofya Museum} 11, 1990, 147–156
%\fullcite{Koyunlu_1990}
\end{bsp}

Ein Vorteil ist, dass man eine eigene Bibliographie mit den Abkürzungen der Zeitschriften und Reihen machen kann. Einzelheiten dazu siehe \cref{bibliographie}.

Findet sich die Zeitschrift oder Reihe {\color{red}nicht} in der unter \cref{listen} aufgeführten Liste, dann wird diese  {\color{red}nicht} abgekürzt und ganz normal in |{}| bei |journaltitle=|\marg{Zeitschriftentitel} gesetzt.
In den folgenden Beispielen wird, wo möglich, mit |@string| gearbeitet.

\section{Optionale Steuerung}
Nachfolgend werden die Optionen des Stils |archaeologie| immer auch konkret an Beispielen erläutert.
Änderungen, die durch die Optionen erfolgen, sind {\color{red}rot} markiert.
In {\color{blue}blau} sind bibliographische Einträge, die mit dem nächsten Literaturverzeichnis verknüpft sind.

 \subsection{Optionen in der Präambel}\label{optionen-preamble}
 
 In der Präambel kann man, wie folgt, den Stil |archaeologie| über das Paket |biblatex| laden:
 
| \usepackage[%|\\						
|		| |backend=biber,			% aktiviert biber|\\
|		| |style=archaeologie, 	% lädt den Stil 'archaeologie'|\\
|		| |inreferences=true, 				%true =  bspw. LTUR 2 (1994) 123 s. v.|\\
|		| |]{biblatex}|
 
In diesem Beispiel wurde der Stil |archaeologie| mit der Option |inreferences| geladen (|=true|). 
Die Ausgabe in Fußnote und in Bibliographie wird jedoch maßgeblich durch weitere verschiedene Optionen manipuliert, die entsprechend ebenfalls in der Präambel geladen werden können.

\subsubsection{strings}\label{strings}
 \DescribeMacro{strings}
 Möchte man die oben erwähnte Methode mit |@string| verwenden, muss in der Präambel die Option |strings| aktiviert sein.
 Damit wird die Bibliographie |archaeologie-abbrv.bib| geladen, in der die unter \cref{listen} aufgeführten Abkürzungen gespeichert sind.
Verwendet man keine  |@string|-Einträge, muss nichts weiter gemacht werden.


\subsubsection{dai-footnote}\label{dai-footnote}
 \DescribeMacro{dai-footnote}
 Diese Option ermöglicht in den Fußnoten die Angabe eines Rückverweises zur Erstnennung des zitierten Bibliographieeintrags.\\
{\color{red} |#### Work in Progress #### Lauffähig in Version 2.0|}
 

\subsubsection{translation}\label{translation}
\DescribeMacro{translation}
Wenn diese Option aktiviert wird, erfolgt die Angabe  eines Originaltitels, die Sprache, aus welcher übersetzt wurde und des Übersetzers des Werkes.
Werke, bei denen |options={antik}| oder |options={frgantik}| (also antike Autoren und Fragmente) gesetzt wurde, werden stehts mit Originaltitel, Sprache und Übersetzer angegeben.

Die Option |translation|  ist standardmäßig auf |=false| gesetzt. 
Ein Beispiel verschafft Klarheit. Folgender Bibliographieeintrag \\
 
  |@Book{Lefebvre_2011,|\\
|  Title 			= {The Production of Space},|\\
|  Author		= {Henri Lefebvre},|\\
|  Publisher 	= {Blackwell Publishing Ltd},|\\
|  Year			= {2011},|\\
|  Edition		= {30},|\\
|  Location	= {Maien, MA and Oxford and Victoria},|\\
|  Origlocation	= {Oxford},|\\
|  Origyear	= {1991},|\\
|  Origlanguage	= {french},|\\
|  Origtitle	= {La production de l’espace},|\\
|  Translator	= {Donald Nicholson-Smith}|\\
| }|

wird in der Bibliographie zunächst auf diese Weise umgesetzt:
%\begin{bsp}
%%Henri Lefebvre, The Production of Space \textsuperscript{30}(Maien, MA 1991; Nachdr. Maien, MA 2011)
%%\todo[noline]{dynamisches Beispiel}\fullcite{Lefebvre_2011}
%\end{bsp}
 \begin{bsp}
\begin{refsection}
\nocite{Lefebvre_2011}
\printbibliography[title={Forschungsliteratur}]
\end{refsection}
\end{bsp}


 mit der aktivierten Option |translation| wird daraus: 
\begin{bsp}%
%\fullcite{Lefebvre_2011}
H. Lefebvre, The Production of Space, {\color{red} \emph{La production de l’espace}, aus dem Französischen übers. von D. Nicholson-Smith} \textsuperscript{30}(Oxford 1991; Nachdr. Maien, MA 2011) 
\end{bsp}
 
Dies funktioniert nicht nur bei den Einträgen |@book|, sondern auch bei |@article|:

| @Article{Lefebvre_1977,| \\
|   Title                    = {Die Produktion des städtischen Raums},| \\
|   Author                   = {Lefebvre, Henri},| \\
|   Journaltitle             = {ARCH+},| \\
|   Number                   = {9},| \\
|   Pages                    = {52--57},| \\
|   Volume                   = {34},| \\
|   Year                     = {1977},| \\
|   Origlanguage             = {french},| \\
|   Origtitle                = {Introduction à l'espace urbain},| \\
|   Translator               = {Franz Hiss and Hans-Ulrich Wegener}| \\
| }| \\

Daraus wird dann:

\begin{bsp}%
%\fullcite{Lefebvre_1977}
 H. Lefebvre, Die Produktion des städtischen Raums, \emph{Introduction à l’espace urbain}, aus dem Französischen übers. von F. Hiss – H.-U. Wegener, ARCH+ 34/9, 1977, 52–57
\end{bsp}




\subsubsection{inreferences}\label{inreferences}
\DescribeMacro{inreferences}  Lexikoneinträge können in den Fußnoten in verschiedenen Zitationsformen dargestellt werden.
 Voraussetzung ist, dass  es sich um ein |@inreference| handelt (siehe \cref{inreference}).
 Zudem   bietet sich optional an ebenso  |keywords = {lexikon}| zu setzen, um diese Einträge dann in der Bibliographie auszuschließen (über |notkeyword=lexikon|, bzw. |keyword=lexikon|).
 
 Am folgenden Beispielsdatensatz wird die Option verdeutlicht:\\
 
%  \begin{bsp}
|	@Inreference{Nieddu_1995,|\\
|  Title			= {Dei Consentes},|\\
|  Author		= {Nieddu, Giuseppe},|\\
|  Year			= {1995},|\\
|  Booktitle	= {LTUR},|\\
|  Pages 		= {9--10},|\\
|  Volume		= {2},|\\
|  Bookpagination	= {column},|\\
|  Keywords	= {lexikon},|\\
|  }|\\
%  \end{bsp}
  \begin{refsection}
  Die Ausgabe von |\cite{Nieddu_1995}| ist nun auf zwei Arten möglich:
   \begin{bsp}

 \begin{enumerate}
 \item standardmäßig wird daraus:  %Nieddu 1995
 \cite{Nieddu_1995}
 \item mit der Option |inreferences| wird dies zu:
 LTUR 2 (1995) 9-10 s. v. Dei Consentes (G. Nieddu)
 % \cite{Nieddu_1995}
  \end{enumerate}
   \end{bsp}
 Trägt man eine konkrete Spalten/Seitenzahl ein (|\cite[9]{Nieddu_1995}|), dann wird diese Angabe in der Fußnote  angegeben:
 \begin{bsp}
\begin{enumerate} 
 \item standardmäßig wird daraus:  %Nieddu 1995
 \cite[9]{Nieddu_1995}
 \item mit der Option |inreferences| wird dies zu:
  LTUR 2 (1995) 9 s. v. Dei Consentes (G. Nieddu)
  \end{enumerate}
\end{bsp}

Verwendet man diese Option (|inreferences|), dann werden die als |@inreference| gekennzeichneten Einträge automatisch von der (End-)Bibliographie ausgeschlossen,  da sie bereits in den Fußnoten vollständig bibliographiert sind.
Verwendet man die Option nicht, dann sieht  der Eintrag wie folgt aus:
  \begin{bsp}
\nocite{Nieddu_1995}
\printbibliography[title={Forschungsliteratur}]
\end{bsp}
 \end{refsection}
 
 
\subsubsection{yearseries}\label{yearseries}
 \DescribeMacro{yearseries}
 Mit der Option |yearseries| kann man bewirken, dass die Reihe (Felder |series| und |number|) erst \emph{nach} dem Jahr ausgegeben werden, was von der Richtlinie des DAI abweicht. 
 Bei Sammelbänden kann diese Option zum Tragen kommen:
 
| @Incollection{Mundt_2015,|\\
|  Author  = {Mundt, Felix},|\\
|  Title   = {Der Mensch, das Licht und die Stadt},|\\
|  Subtitle  = {Rhetorische Theorie und Praxis antiker|\\
|								|| und humanistischer Städtebeschreibung},|\\
|  Pages   = {179--206},|\\
|  Editor  = {Therese Fuhrer and Felix Mundt and Jan Stenger},|\\
| Booktitle = {Cityscaping},|\\
| Booksubtitle             = {Constructing and Modelling Images of the City},|\\
|  Publisher = {de Gruyter},|\\
|  Year    = {2015},|\\
|  Number  = {3},|\\
|  Series  = Philologus-lang # |''| Supplement|'' |,|\footnote{Hier zeigt sich eine Besonderheit der |@string|: Zuerst wird die Langform der Serie mittels |Philologus-lang| aufgerufen, dann mit einem |\#| erweitert und mit ''| Supplement|'' ergänzt. In der Ausgabe wird dann die Ergänzung zum |@string| hinzugezogen. }\\
|  Shortseries          		= Philologus-kurz # |''| Suppl.|'' |,|\\
|  Location  = {Berlin and Boston},|\\
|}|\\
 
 Ohne eine Option wird in der Bibliographie daraus:
% \begin{bsp}\todo[noline]{dynamisches Beispiel}\fullcite{Mundt_2015}\end{bsp}
  \begin{bsp}
\begin{refsection}
\nocite{Mundt_2015}
\printbibliography[title={Forschungsliteratur}]
\end{refsection}
\end{bsp}
 
 mit der aktivierten Option |yearseries| verändert sich die Reihenfolge:
 \begin{bsp}
F. Mundt, Der Mensch, das Licht und die Stadt. Rhetorische Theorie und Praxis antiker und humanistischer Städtebeschreibung, in: T. Fuhrer – F. Mundt – J. Stenger (Hrsg.), Cityscaping. Constructing and Modelling Images of the City (Berlin 2015) {\color{red}Philologus Suppl. 3,} 179–206
\end{bsp}

\subsubsection{fullnames / lastnames}\label{fullnames}\label{lastnames}
\DescribeMacro{fullnames}
\DescribeMacro{lastnames}
Im Fließtext kann direkt auf Autoren (und bei fehlender Autorenangabe wird der oder die Herausgeber genannt) der Forschungsliteratur verwiesen werden. 
Autoren/Herausgeber werden über |\citeauthor|\marg{Schlüssel} aufgerufen.
Ebenso ist auch das Auslesen der Werktitel über |\citetitle|\marg{Schlüssel} möglich, wobei der Titel in eine |emph{}|-Umgebung gesetzt wird und das Erscheinungsjahr in Klammern dahinter.

Zunächst erfolgt die Ausgabe der Autoren, bzw. Herausgeber mit den Initialen des Vornamens und mit dem Nachnamen. 
Die Darstellung der Namen kann jedoch noch auf zwei andere  Arten geschehen und sind stets mit ihrem Bibliographieeintrag zum Bibliographie-\marg{Schlüssel} via |hyperref| verlinkt. Die zwei Arten sind: mit dem vollen Vor- und Zunamen und nur mit dem Nachnamen. In einer Fußnote jedoch wird stets nur der/die Nachnamen gesetzt.

Ein Beispiel macht dies klarer.
Der Bibliographieeintrag lautet:

|@Article{Boehmer_1985,|\\
|  Title   = {Astragalspiele in und um Warka},|\\
|  Author  = {Boehmer, Rainer Michael and Wrede, Nadja},|\\
|  Journaltitle = BaM-lang,|\\
|  Shortjournal  = BaM-kurz,|\\
|  Pages   = {399--404},|\\
|  Volume  = {16},|\\
|  Year    = {1985},|\\
|  }|\par

Im Fließtext schreibt man:
\begin{refsection}
\begin{bsp}
\ldots , dies behaupten ebenso  |\citeauthor{Boehmer_1985}| in ihrem jüngsten Werk |\citetitle{Boehmer_1985}|.
\end{bsp}

Und nach dem Kompilieren wird dann in der Standardeinstellung (ohne weitere Optionen):\footnote{Wird der Befehel |citeauthor| in einer Fußnote verwendet, dann werden nur die Nachnamen ausgegeben, unabhängig der Optionen |fullnames| (\cref{name:a}) und |lastnames| (\cref{name:b}): \begin{bsp}
\ldots , dies behaupten ebenso  |\citeauthor{Boehmer_1985}| in ihrem jüngsten Werk |\citetitle{Boehmer_1985}|.
\end{bsp} }

\begin{bsp} \todo[noline]{dynamisches Beispiel}\ldots , dies behaupten ebenso  \citeauthor{Boehmer_1985} in ihrem jüngsten Werk \citetitle{Boehmer_1985}.\end{bsp}

Oder mit den Optionen |fullnames| (\cref{name:a}) und |lastnames| (\cref{name:b}):
\begin{bsp}
 \begin{enumerate}
\item\label{name:a} 
 \ldots , dies behaupten ebenso {\color{red}Rainer Michael Boehmer  und Nadja Wrede} in ihrem jüngsten Werk \emph{Astragalspiele in und um Warka} (1985).
\item\label{name:b}  
\ldots , dies behaupten ebenso  {\color{red}Boehmer und  Wrede} in ihrem jüngsten Werk \emph{Astragalspiele in und um Warka} (1985).

 \end{enumerate}
\end{bsp}

\begin{bsp}
\nocite{Boehmer_1985}
\printbibliography[title={Forschungsliteratur}]
\end{bsp}
\end{refsection}


Werden allerdings mit |\citeauthor|, bzw. |\citetitle|  antike Autoren und ihre Werktitel aufgerufen (dafür muss im Bibliographieeintrag |options=antik| geschrieben werden), dann wird für den Autorenname das Feld |shortauthor| ausgelesen, in dem der deutsche Rufnamen des Autors steht. 

Um Fehler zu vermeiden ist es absolut wichtig, dass dieses Feld wie die anderen Namensfelder behandelt wird. Möchte man »Plinius, d. Ä.« als deutschen Rufnamen ausgeben, dann muss man schreiben: |shortauthor = {Plinius, {d.\,Ä.}}|.

Bei antiken Werktitel wird keine Jahreszahl dazugeschrieben.
Somit wird aus dem Bibliographieeintrag

|@Book{Quint_inst,|\\
|  Title   = {Ausbildung des Redners},|\\
|  Author  = {Fabius Quintilianus, Marcus},|\\
|  Translator  = {Rahn, Helmut},|\\
|  Year    = {2015},|\\
|  Edition = {6},|\\
|  Keywords  = {Quelle},|\\
|  Location  = {Darmstadt},|\\
|  Options = {antik},|\\
|  Origlanguage  = {latin},|\\
|  Shorthand = {Quint. inst.},|\\
|  Subtitle  = {Institutio oratoria},|\\
|  Shortauthor   = {Quintilian}|\\
|}|

wie folgt  Autor und  Buchttitel ausgelesen:
\begin{refsection}

 \ldots\ Auch |\citeauthor{Quint_inst}| nennt in |\citetitle{Quint_inst}| die notwendigen physischen Qualitäten eines Redners.\par
 
 \begin{bsp}
 \ldots\ \todo[noline]{dynamisches Beispiel}Auch \citeauthor{Quint_inst} nennt in \citetitle{Quint_inst} die notwendigen physischen Qualitäten eines Redners.\par
\end{bsp}


\begin{bsp}
\nocite{Quint_inst}
\printbibliography[title={Antike Quellen}]
\end{bsp}
\end{refsection}



\subsubsection{yearinparens}\label{yearinparens}
\DescribeMacro{yearinparens}%
Setzt die Jahreszahlen aller Einträge in der Fußnote und in der Bibliographie in runde Klammern.
Anstatt wie standardmäßig aus  |\cite[475]{Ball_2013}|
\begin{bsp} Ball – Dobbins 2013, 475 \end{bsp}
 wird, wird mit der Option: 
\begin{bsp}
Ball – Dobbins {\color{red}(}2013{\color{red})}, 475
\end{bsp}

%und in der Bibliographie:
%
%\begin{bsp}
%\begin{refsection}
%\nocite{Ball_2013}
%\printbibliography[title={Forschungsliteratur}]
%\end{refsection}
%\end{bsp}




\subsubsection{scshape}\label{scshape}
\DescribeMacro{scshape}
Die Option ändert die Formatierung der Zitation, sodass die Namen in den Anmerkungen/Fußnoten und in der Bibliographie in Kapitälchen gesetzt werden. Ausgenommen sind hierbei die Einträge, die über ein |label| (\cref{unbekannt}) ausgegeben werden, da es sich dabei nicht um einen Nachnamen, sondern eine selbst gewählte Bezeichnung handelt. 
Ebenso sind antike Autoren (|options=antik|, bzw. |options={frgantik}|) von dieser Option ausgenommen.
Anstatt wie standardmäßig aus  |\cite[475]{Ball_2013}|
\begin{bsp} \cite[475]{Ball_2013} \end{bsp}
 wird, wird mit der Option: 
\begin{bsp}
{\scshape {\color{red}Ball – Dobbins}} 2013, 475
\end{bsp}

Entsprechend in der Bibliographie:
\begin{bsp}
{\scshape {\color{red}Ball – Dobbins}} 2013\\
\hspace*{3em}L. F. Ball – J. J. Dobbins, Pompeii Forum Project. Current Thinking on the Pompeii Forum, AJA 117/3, 2013, 461–492

\end{bsp}



\subsubsection{bibfullname}\label{bibfullname}
\DescribeMacro{bibfullname}
Ausgabe des vollen Namens (soweit Vor- und Zuname ausgeschrieben wurden) in der Bibliographie.

\begin{bsp}
\begin{refsection}
\nocite{Ball_2013}
\printbibliography[title={Forschungsliteratur}]
\end{refsection}
\end{bsp}

Mit aktivierter Option wird dies zu:

\begin{bsp}
{\color{red}Larry F. Ball – John J. Dobbins}, Pompeii Forum Project. Current Thinking on the Pompeii Forum, AJA 117/3, 2013, 461–492
%\fullcite{Ball_2013}

\end{bsp}


\subsubsection{noabbrevs}\label{noabbrevs}
\DescribeMacro{noabbrevs}
Die DAI-Vorgabe sieht vor, Zeitschriften nur abgekürzt wiederzugeben, dafür wird das Feld |shortjournal| vom Bibliographieeintrag ausgelesen. Gibt es keine Abkürzung, also wurde das Feld |shortjournal| leer gelesen, wird automatisch das Feld |journal| ausgelesen.
Möchte man hingegen den vollen Zeitschriftennamen in der Bibliographie haben, dann muss man die Option |noabbrevs| aktivieren.

|@Article{Ball_2013,|\\
|  Title   = {Pompeii Forum Project},|\\
|  Author  = {Larry F. Ball and John J. Dobbins},|\\
|  Journaltitle = AJA-lang,|\\
|  Pages   = {461--492},|\\
|  Volume  = {117},|\\
|  Year    = {2013},|\\
|  Number  = {3},|\\
|  Shortjournal   = AJA-kurz,|\\
|  Subtitle  = {Current Thinking on the Pompeii Forum}|\\
|  }|\\

Ohne eine Zusätzliche Option wird der Eintrag in der Bibliographie wie folgt umgesetzt:

%\begin{bsp}
%\todo[noline]{dynamisches Beispiel}\fullcite{Ball_2013}
%\end{bsp}
\begin{bsp}
\begin{refsection}
\nocite{Ball_2013}
\printbibliography[title={Forschungsliteratur}]
\end{refsection}
\end{bsp}


Mit der aktivierten angesprochenen Funktion |noabbrevs| wird daraus:

\begin{bsp}\todo[noline]{manuelles Beispiel}L. F. Ball – J. J. Dobbins, Pompeii Forum Project. Current Thinking on the Pompeii Forum,  {\color{red}American Journal of Archaeology} 117/3, 2013, 461–492\end{bsp}

\subsubsection{publisher}\label{publisher}
\DescribeMacro{publisher} 
Angabe aller Erscheinungsorte und Verlagsort. Damit geht auch eine Änderung der Auflagezahl einher, die dann direkt vor das Erscheinungsjahr gesetzt wird.
Erstauflage wird in eckiger Klammer nach dem Erscheinungsjahr gesetzt.

|@Book{Emme_2013,|\\
|  Author  = {Burkhard Emme},|\\
|  Title   = {Peristyl und Polis},|\\
|  Subtitle  = {Entwicklung und Funktionen %|\\
|								||öffentlicher griechischer Hofanlagen},|\\
|  Number  = {1},|\\
|  Publisher = {Walter de Gruyter},|\\
|  Series  = {Urban Spaces},|\\
|  Year    = {2013},|\\
|  Location  = {Berlin and New York},|\\
|}|\\

 Ohne eine Option wird in der Bibliographie daraus:
% \begin{bsp}\todo[noline]{dynamisches Beispiel}\fullcite{Emme_2013}\end{bsp}
 \begin{bsp}
\begin{refsection}
\nocite{Emme_2013}
\printbibliography[title={Forschungsliteratur}]
\end{refsection}
\end{bsp}
 
 mit der aktivierten Option |publisher| verändert sich die Reihenfolge:
 \begin{bsp}
 \todo[noline]{manuelles Beispiel}
 B. Emme, Peristyl und Polis. Entwicklung und Funktionen öffentlicher griechischer Hofanlagen, Urban Spaces 1 (Berlin {\color{red} – New York: Walter de Gruyter} 2013)\end{bsp}
 
 
 
 
\subsubsection{edby}\label{edby}
\DescribeMacro{edby}
Herausgeber werden nicht mehr zu Beginn des Sammelbandes aufgelistet und mit einem |(Hrsg.)| gekennzeichnet, sondern nach dem Titel des Sammelbandes mit dem Zusatz |hrsg. v.|


|@Inproceedings{Wulf-Rheidt_2013,|\\
|  Author  = {Wulf-Rheidt, Ulrike},|\\
|  Title   = {Der Palast auf dem Palatin -- Zentrum im Zentrum},|\\
|  Subtitle  = {Geplanter Herrschersitz oder|\\
|									 Produkt eines langen Entwicklungsprozesses?},|\\
| Pages   = {277--289},|\\
|  Editor  = {Dally, Ortwin and Fless, Friederike and Haensch,|\\
|									Rudolf and Pirson, Felix and Sievers, Susanne},|\\
|  Booktitle = {Politische Räume in vormodernen Gesellschaften},|\\
|  Booksubtitle             = {Gestaltung – Wahrnehmung – Funktion},|\\
|  Year    = {2013},|\\
|  Eventdate = {2009-11-18/2009-11-22},|\\
|  Eventtitle               = {Internationale Tagung des DAI|\\
|							 und des DFG-Exzellenzclusters TOPOI},|\\
| Venue   = {Berlin},|\\
|  Publisher = {Verlag Marie Leidorf},|\\
|  Location  = {Rahden/Westf.},|\\
|  Series  = MKT-lang,|\\
| Shortseries 		= MKT-kurz,|\\
|  Number  = {6},|\\
|}|\\


 Ohne eine Option wird in der Bibliographie daraus:
 %\begin{bsp}\todo[noline]{dynamisches Beispiel}\fullcite{Wulf-Rheidt_2013}\end{bsp}
  \begin{bsp}
\begin{refsection}
\nocite{Wulf-Rheidt_2013}
\printbibliography[title={Forschungsliteratur}]
\end{refsection}
\end{bsp}
 
 mit der aktivierten Option |edby| verändert sich die Reihenfolge:
 \begin{bsp} \todo[noline]{manuelles Beispiel}
U. Wulf-Rheidt, Der Palast auf dem Palatin – Zentrum im Zentrum. Geplanter Herrschersitz oder Produkt eines langen Entwicklungsprozesses?, in:  {\color{red}Politische Räume in vormodernen Gesellschaften. Gestaltung – Wahrnehmung – Funktion, hrsg. v. O. Dally – F. Fless – R. Haensch – F. Pirson – S. Sievers}. Internationale Tagung des DAI und des DFG-Exzellenzclusters TOPOI Berlin 18.–22. November 2009, MKT 6 (Rahden/Westf. 2013) 277–289
 \end{bsp}
 
 
\subsubsection{width}\label{width}
\DescribeMacro{width}
In der Bibliographie ist der Abstand zwischen dem Label (Autor und Jahr)  und der vollständigen bibliographischen Angabe mit dem Wert |4em| angegeben.
Wünscht man einen anderen Abstand, dann kann dieser selbst gewählt werden:

|width = XY|

Dafür kann XY für jede Länge stehen (bspw. |3em|, |7pt| oder |4cm|).

\subsubsection{counter}\label{counter}
 \DescribeMacro{counter} 
 Möchte man die Anzahl an Zitationen eines Werks im Text erfahren, 
 dann reicht es wenn man in der Präambel die Option |counter| aktiviert (|counter=true|).\footnote{Idee basiert auf \href{http://tex.stackexchange.com/a/14159/98739}{http://tex.stackexchange.com/a/14159/98739} und wurde entsprechend angepasst.}
 Dadurch wird am Ende eines jeden Literatureintrages die Anzahl der zitierten Einträge geschrieben. 
 Die Ausgabe erfolgt bei eingestellter Sprache |ngerman|(im Paket |babel| oder in der |documentclass|)  entsprechend auf Deutsch,
 ansonsten auf Englisch.
\begin{bsp}
Böhm – Eickstedt 2001\\
\indent S. Böhm – K.-V. v. Eickstedt (Hrsg.), Ithake. Festschrift Jörg Schäfer (Würzburg 2001)  $\vert$  {\scshape  wurde 1-mal zitiert.}
\end{bsp} 

Ist ein Eintrag ohne Zitation in die Bibliographie geraten, wird dies entsprechend angezeigt:
\begin{bsp}
Böhm – Eickstedt 2001\\
\indent S. Böhm – K.-V. v. Eickstedt (Hrsg.), Ithake. Festschrift Jörg Schäfer (Würzburg 2001)  $\vert$  {\scshape  wurde {\color{red}{keinmal}} zitiert.}
\end{bsp} 



Für alle Sprachen außer Deutsch:
 \begin{bsp}
Böhm – Eickstedt 2001\\
 \indent S. Böhm – K.-V. v. Eickstedt (ed.), Ithake. Festschrift Jörg Schäfer (Würzburg 2001) $\vert$  {\scshape cited {{\color{red}{not once}}}.}
 \end{bsp}
  
 \begin{bsp}
Böhm – Eickstedt 2001\\
 \indent S. Böhm – K.-V. v. Eickstedt (ed.), Ithake. Festschrift Jörg Schäfer (Würzburg 2001) $\vert$  {\scshape cited 1 time.}
 \end{bsp}

Bei mehrmaligeb Zitationen ändert sich die Ausgabe entsprechend:
  \begin{bsp}
 Böhm – Eickstedt 2001\\
 \indent S. Böhm – K.-V. v. Eickstedt (ed.), Ithake. Festschrift Jörg Schäfer (Würzburg 2001) $\vert$  {\scshape cited 3 times.}
\end{bsp}
 
 

%%\subsubsection{miturl}\label{miturl}
%%\DescribeMacro{miturl}
%%Für Bibliographieeinträge die online sind (|@Online|) werden  die URLs/DOI immer mit angegeben.
%%
%%| @Online{Selle_2008,|\\
%%| Title   = {Raum und Verhalten},|\\
%%| Author  = {Selle, Klaus},|\\
%%| Url     = {http://www.pt.rwth-aachen.de/%|\\
%%|dokumente/lehre_materialien/c8_raum_verhalten.pdf},|\\
%%| Urldate 				= {2015-06-26},|\\
%%| Year    = {2008},|\\
%%| Subtitle  = {Grundlagen und erste Folgerungen für das Wohnen in der Stadt},|\\
%%| }|\\
%%
%%%\begin{bsp}\todo[noline]{dynamisches Beispiel}\fullcite{Selle_2008}\end{bsp}
%% \begin{bsp}
%%\begin{refsection}
%%\nocite{Selle_2008}
%%\printbibliography[title={Forschungsliteratur}]
%%\end{refsection}
%%\end{bsp}
%%
%%Handelt es sich jedoch um einen Artikel o. ä., dann kann man die Angabe der URL/DOI/eprint optional steuern:
%%
%%| @Article{Mastrocinque_2012,|\\
%%| Title   = {Magnetometry at Grumentum in Ancient Lucania},|\\
%%| Author           = {Mastrocinque, Attilio and Saggioro, Fabio},|\\
%%| Journaltitle           = {The Journal of Fasti Online},|\\
%%| Pages   = {1--5},|\\
%%| Year    = {2012},|\\
%%| Url     = {www.fastionline.org/docs/FOLDER-it-2012-245.pdf},|\\
%%| urldate				 = {2015-07-21},|\\
%%| }|\\
%%
%%
%%\begin{bsp}\todo[noline]{manuelles Beispiel}
%%A. Mastrocinque – F. Saggioro, Magnetometry at Grumentum in Ancient Lucania, The Journal of Fasti Online, 2012, 1–5,\\
%% {\color{red}<|www.fastionline.org/docs/FOLDER-it-2012-245.pdf|> (21. 07. 2015)}
%%\end{bsp}

\subsection{Optionen der Literatureinträge}\label{optionen-literatur}
\subsubsection{antik}\label{antik}
 \DescribeMacro{antik} \emph{Die Formatierungen von |antik| und |frgantik| sind vom Bibliographiestil |geschichtsfrkl| v.1.1 (Jonathan Zachhuber) inspiriert und modifiziert worden.}
 
Bei dem Zitieren antiker Autoren empfiehlt es sich diese Werke mit der Option |antik| zu versehen. Wir betrachten wieder ein Beispiel:\\[3pt]
|@Book{Cic_Att,|\\
|  Title = {Atticus-Briefe},|\\
|  Author = {Tullius Cicero, Marcus},|\\
|  Editor  = {Kasten, Helmut},   % wird bei @book nicht ausgelesen|\\
|  Publisher = {Artemis {\&} Winkler},|\\
|  Series  = {Tusculum Bücherei},|\\
|  Year = {1980},|\\
|  Edition = {3},|\\
|  Keywords = {Quelle},|\\
|  Location = {Düsseldorf and Zürich},|\\
|  Options = {antik},|\\
|  Origlanguage = {latin},|\\
|  Origtitle = {epistulae ad Atticum	},|\\
|  Origyear = {1959},|\\
|  Shorthand  = {Cic. Att.},|\\
|  Translator = {Kasten, Helmut},|\\
|  Shortauthor = {Cicero}     % relevant für \citeauthor|\\
|	}|\\[3pt]


 Beim Zitieren wird  nur das Feld |shorthand| berücksichtigt: 
 |\cite[1, 3,3]{Cic_Att}| liefert
 \begin{refsection}
\begin{bsp}
\cite[1, 3,3]{Cic_Att}
%Cic. Att. 1, 3,3
\end{bsp}
Dieses Feld |shorthand| wird für das Literaturverzeichnis verwendet, wo es als ›Schlüssel‹ auftaucht.: 
\begin{bsp}
\printbibliography[title={Antike Quellen}]
\end{bsp}
\end{refsection}



Es gibt auch antike Texte, die in einem Sammelband (|@incollection|) herausgegeben sind. 
Dieser Fall stellt jedoch kein Problem dar und wird analog zu |@book| geplottet.
Ein Beispiel verschafft Klarheit:

|@Incollection{Cic_Sest,|\\
|  Title = {Rede für P.\ Sestius},|\\
|  Author = {Tullius Cicero, Marcus},|\\
|  Editor  = {Fuhrmann, Manfred},|\\
|  Pages = {110--185},|\\
|  Publisher = {Artemis \& Winkler},|\\
|  Year = {1993},|\\
|  Series = {Sammlung Tusculum},|\\
|  Volume = {II},|\\
|  Keywords = {Quelle},|\\
|  Location = {München},|\\
|  Booktitle= {Die politischen Reden},|\\
|  Options = {antik},|\\
|  Origlanguage = {latin},|\\
|  Origtitle = {pro P.\ Sestio},|\\
|  Shorthand = {Cic. Sest.},|\\
|  Translator = {Fuhrmann, Manfred},|\\
|  Shortauthor = {Cicero}|\\
|}|\\


\begin{bsp}
\begin{refsection}
\nocite{Cic_Sest}
\printbibliography[title={Antike Quellen}]
\end{refsection}
\end{bsp}


%\subsubsection{frg}\label{frg}
% \DescribeMacro{frg}
%%% Wenn man Fragmente zitiert kann man dazu die Option |frg| bzw. |frgantik| wählen. Das wirkt sich dann leicht unterschiedlich auf die Zitatsform aus. Wir betrachten das am besten wieder an Beispielen: Der |bib|-Eintrag\\[3pt]
%|@book{alkaios,|\\
%|	author={{Alkaios}},|\\
%|	shorthand={Alk.},|\\
%|	sortname={Edgar Lobel and Denys Page},|\\
%|	maintitle={Poetarum Lesbiorum Fragmenta},|\\
%|	editor={Edgar Lobel and Denys Page},|\\
%|	shorteditor={LP},|\\
%|	address={Oxford},|\\
%|	year={1955},|\\
%|	keywords={quelle},|\\
%|	options={frg}|\\
%|}|\\[3pt]
%
%\begin{refsection} 
% \begin{bsp}
% \cite[2]{alkaios}
% \end{bsp}
% 
% In der Bibliographie  unterscheidet sich der Eintrag gegenüber |antik|  geringfügig:
%\begin{bsp}
%\printbibliography[title={Antike Quellen}]
%\end{bsp}
%\end{refsection}

%% Das |sorttitle|-Feld sorgt hier dafür, dass der Eintrag nicht unter \enquote{Alkaios} sondern den Namen der Herausgeber sortiert wird. Im Literaturverzeichnis sieht das dann folgendermaßen aus:
%%\begin{bsp} {\sc Lobel}, Edgar / {\sc Page}, Denys (Hrsg.): Poetarum Lesbiorum Fragmenta, Oxford 1955.\end{bsp}
%% Wenn man ihn zitiert, erscheint bei |\cite[2]{alkaios}|
%% \begin{bsp} Alk. frg. 2 LP.\end{bsp}
%% Bemerkenswert ist insbesondere die Verwendung des |shorteditor|-Feldes. Ansonsten wird nur der Nachname des Herausgebers angegeben. Details findet man in \cref{driver}.
%%

\subsubsection{frgantik}\label{frgantik}
 \DescribeMacro{frgantik}
 Mit der Option |frgantik| werden die Bibliographieeinträge versehen, die antike Fragmente beinhalten, da die Herausgeber dieser Fragmente relevant sind. Dies wird in der Zitierweise berücksichtigt. 
 \\[3pt]
|@Book{Fest,|\\
|  Title = {De verborum significatu quae supersunt cum Pauli epitome},|\\
|  Author = {Pompeius Festus, {\relax Sex}tus},|\\
|  Editor = {Lindsay, Wallace Martin},|\\
|  ShortEditor = {L},|\\
|  Publisher = {Teubner},|\\
|  Series = {Bibliotheca scriptorum et Graecorum et Romanorum Teubneriana},|\\
|  Year = {1965},|\\
|  Keywords = {Quelle},|\\
|  Location = {Leipzig},|\\
|  Options = {frgantik},|\\
|  Origyear = {1913},|\\
|  Shorthand = {Fest.},|\\
|  Shortauthor = {Festus}|\\
|}|\\[3pt]
 Zitiert man diesen Eintrag durch |\cite[3]{Fest}|, so wird der oder die Herausgeber genannt; ist das Feld |Shorteditor| ausgefüllt, dann wird dieses angegeben, ansonsten die Nachnamen aus |Editor|.
\begin{refsection} 
 \begin{bsp}\todo[noline]{dynamisches Beispiel}
 \cite[3]{Fest}
 \end{bsp}
 
 In der Bibliographie  unterscheidet sich der Eintrag gegenüber |antik|  geringfügig:
\begin{bsp}
\printbibliography[title={Antike Quellen}]
\end{bsp}
\end{refsection}
 


\subsubsection{corpus}\label{corpus}
\DescribeMacro{corpus}
Für bestimmte Corpora (Inschriften, Münzen, etc.) wird für gewöhnlich mit einer gängigen Abkürzung zitiert.
Diese Abkürzung des Corpus wird im Bibliographieeintrag unter |shorthand| eingetragen.
Nun kann man sehr einfach das gewünschte Corpus in der Fußnote zitieren, mit  |prenote| und |postnote|-Feldern. Bei anderen Autoren-Einträge, die mittels |shorthand| zitiert werden, wird ein Komma zwischen Nachname und |postnote| gesetzt. Dank |options=corpus| fällt dieses Komma weg.

Das Beispiel zeigt die Option für die lateinischen Inschriften:

|@Book{CIL,|\\
| Title   = CIL-lang,|\\
|  Keywords  = {Sigel},|\\
|  Options = {corpus},|\\
|  Shorthand = CIL-kurz,|\\
|}|\\

Zitiert wird, wie gewöhnlich, mit |\cite[06, 01234]{CIL}|. Daraus wird:

 \begin{refsection}
\begin{bsp}
\cite[06, 01234]{CIL}
%Cic. Att. 1, 3,3
\end{bsp}
Dieses Feld |shorthand| wird für das Literaturverzeichnis verwendet, wo es als ›Schlüssel‹ auftaucht.: 
\begin{bsp}
\printbibliography[keyword=Sigel,%
title={Abkürzungen und Sigel}]\end{bsp}
\end{refsection}



Aufgrund der Setzung von |keywords=Sigel| können diese Art von Corpora in einer separaten Bibliographie aufgeführt werden. Siehe dazu \cref{bibliographie}

\changes{v1.1}{2015/06/15}{Modifikation der Option |corpus|.}


 \section{Beschreibung der Eintragtypen (Beispiele)} \label{beispiele}


 Der |archaeologie|-Zitierstil definiert unterschiedliche bibliography driver, die es erlauben verschiedene Arten Werke zu zitieren. 
 Diese werden im Folgenden zusammen mit den für sie relevanten Optionen beschrieben.



 \subsection{Typ \texttt{@book}}\label{book}
 \DescribeMacro{@book}
 \DescribeMacro{@collection}\footnote{Der Typ |@collection| entspricht hier dem Typ |@book|.}
 Fangen wir ganz einfach an: Zu einem einfachen Buch sieht der Eintrag in der |bib|-Datei ungefähr folgendermaßen aus:\\[3pt]


|@Book{Mann_2011,|\\
|  Author  = {Mann, {\relax Chr}istian},|\\
|  Title   = {\enquote{Um keinen Kranz, um das Leben kämpfen wir!}},|\\
|  Subtitle  = {Gladiatoren im Osten des Römischen Reiches| \\| und die Frage der Romanisierung},|\\
| Series  = {Studien zur Alten Geschichte},|\\
|  Number  = {14},|\\
|  Location  = {Berlin},|\\
|  Publisher = {Verlag Antike},|\\
|  Year    = {2011},|\par
|}|\\
\begin{refsection}
Zitiert man daraus mit |\footnote{\cite[Vgl.][142--144]{Mann_2011}.}|, dann erscheint in der Fußnote\footnote{\cite[Vgl.][142--144]{Mann_2011}.}.
In der Bibliographie wird der Eintrag wiedergegeben mit:
%\begin{bsp}\fullcite{Mann_2011}\end{bsp}
\begin{bsp}
\printbibliography[title={Forschungsliteratur}]
\end{bsp}
 \end{refsection}

\subsubsection{Festschriften u.ä.}
Um Festschriften/Gedenkschriften/Ausstellungskataloge/Auktionskataloge entsprechend zu zitieren, gehört der Zusatz ins Feld |titleaddon|, bzw. wenn es sich um ein |@inbook| oder |@inproceedings| handelt, entsprechend ins Feld |Booktitleaddon| (\cref{inbook}).


|@Book{Boehm_2001,|\\
 | Title   = {Ithake},|\\
 | Editor  = {Böhm, Stephanie and Eickstedt, Klaus-Valtin von},|\\
 | Titleaddon = {Festschrift Jörg Schäfer},|\\
 | Publisher = {Ergon-Verlag},|\\
  |Year    = {2001},|\\
 | Location = {Würzburg},|\\
|}|\par

%\begin{bsp}\fullcite{Boehm_2001}\end{bsp}
\begin{refsection}
\begin{bsp}
\nocite{Boehm_2001}
\printbibliography[title={Forschungsliteratur}]
\end{bsp}
 \end{refsection}
 
 \subsubsection{Übersetzungen u.ä.}
Liegt ein Werk in Übersetzung vor, gibt es die Möglichkeit den Übersetzer/in, Ausgangssprache und ursprünglichen Titel automatisch anzeigen zu lassen.
Dies funktioniert mit den Literatureinträgen |related| und |relatedtype|  (\cref{review}).

Im Beispiel wird die Verwendung klar:
Das Ausgangswerk ist |Zanker_2014|.

| @Book{Zanker_2014,|\\
|   Title                    = {Die römische Stadt},|\\
|   Author                   = {Zanker, Paul},|\\
|   Location                 = {München},|\\
|   Publisher                = {CHB},|\\
|   Subtitle                 = {Eine kurze Geschichte},|\\
|   Year                     = {2014},|\\
|   Language                 = {german},|\\
| }|\\

Die Übersetzung davon ist |Zanker_2013|:

| @Book{Zanker_2013,|\\
|   Title                    = {La città romana},|\\
|   Author                   = {Zanker, Paul},|\\
|   Location                 = {Roma-Bari},|\\
|   Publisher                = {GLF},|\\
|   Series                   = {Storia della città},|\\
|   Year                     = {2013},|\\
|   Language                  = {italian},|\\
|   Relatedtype              = {translationof},|\\
|   Related                  = {Zanker_2014},|\\
|   Translator               = {Senatore, Anna Maria}|\\
| }|\\



 \begin{refsection}
\begin{bsp}
\nocite{Zanker_2013,Zanker_2014}
\printbibliography[title={Forschungsliteratur}]
\end{bsp}
 \end{refsection}

 \subsubsection{Mehrbändige Monographien mit Untertitel}
Es kommt vor, dass man einen Band einer mehrbändigen Monographie zitiert, der einen eigenen Untertitel hat.
Damit die jeweilige Zahl des Bandes an korrekter Stelle ausgegen wird, 
muss ein Bibliographieeintrag wie folgt aussehen:

|@Book{MacDonald_1986,|\\
|  Title                    = {An urban Appraisal},|\\
| Author                   = {MacDonald, William L.},|\\
| Location                 = {New Haven and London},|\\
| Number                   = {35},|\\
| Publisher                = {YUP},|\\
| Series                   = {Yale Publictions in the History of Art},|\\
| Year                     = {1986},|\\
| Volume                   = {II},|\\
| Maintitle                = {The Architecture of the Roman Empire}|\\
|}|\\

Damit werden Haupttitel der Monographie (|Maintitle|) und Titel des Bandes (|Titel|) getrennt voneinander eingegeben, sodass die Bandzahl (|volume|) vor den Titel gesetzt werden kann.

\begin{refsection}
\begin{bsp}
\nocite{MacDonald_1986}
\printbibliography[title={Forschungsliteratur}]
\end{bsp}
 \end{refsection}



 \subsection{Typ \texttt{@inbook / @incollection}}\label{inbook}

 \DescribeMacro{@incollection}
 Kapitel aus Sammelbändern macht man am Besten mit dem Typ |@incollection|. 
 Am besten sieht man das wieder an Hand eines Beispiels:\\[3pt]
 
|@Incollection{Carter_2014,|\\
|  Title   = {Spectacle in Rome, Italy, and the Provinces},|\\
|  Author  = {Carter, Michael J. and Edmondson, Jonathan},|\\
|  Editor  = {Bruun, Christer and Edmondson, Jonathan},|\\
|  Pages   = {537--558},|\\
|  Publisher = {Oxford University Press},|\\
|  Year    = {2014},|\\
|  Location  = {Oxford},|\\
|  Booktitle = {The Oxford Handbook of Roman Epigraphy},|\\
|}|\\

%\begin{bsp}\todo[noline]{dynamisches Beispiel}\fullcite{Carter_2014}\end{bsp}
\begin{refsection}
\begin{bsp}
\nocite{Carter_2014}
\printbibliography[title={Forschungsliteratur}]
\end{bsp}
 \end{refsection}
 
Zitiert man einen Beitrag aus einer Festschrift, o.ä., dann kann das folgendermaßen aussehen:

|@Incollection{Hoelscher_2001,|\\
 | Author  = {Hölscher, Tonio},|\\
 | Title   = {Schatzhäuser -- Banketthäuser?},|\\
 | Pages   = {143--152},|\\
 | Editor  = {Böhm, Stephanie and Eickstedt, Klaus-Valtin von},|\\
 | Booktitle = {Ithake},|\\
 | Booktitleaddon = {Festschrift Jörg Schäfer},|\\
 | Publisher = {Ergon-Verlag},|\\
  |Year    = {2001},|\\
 | Location = {Würzburg},|\\
|}|\par

In der Bibliographie ist die Darstellung:

%\begin{bsp}\fullcite{Hoelscher_2001}\end{bsp}
\begin{refsection}
\begin{bsp}
\nocite{Hoelscher_2001}
\printbibliography[title={Forschungsliteratur}]
\end{bsp}
 \end{refsection}
 \subsubsection{Kleinere Kurzreihen}
Es kommt auch vor, dass man einen Band einer kleinen Reihe (nicht Serie) zitieren muss.
Zum Beispiel dieses Buch:

|@Incollection{Fentress_2003,|\\
|  Title                    = {Cosa in the Republic and Early Empire},|\\
|  Author                   = {Fentress, Elizabeth and John Bodel and Adam Rabinowitz and Rabun Taylor},|\\
|  Booksubtitle             = {Excavations 1991--1997},|\\
|  Booktitle                = {An Intermittent Town},|\\
|  Editor                   = {Fentress, Elizabeth},|\\
|  Location                 = {Ann Arbor, Mich.},|\\
|  Number                   = {2},|\\
|  Pages                    = {13--62},|\\
|  Publisher                = UMP,|\\
|  Series                   = MemAmAc-lang,|\\
|  Shortseries              = MemAmAc-kurz,|\\
|  Year                     = {2003},|\\
|  Volume                   = {V},|\\
|  Maintitle                = {Cosa},|\\
|}|\par

Es handelt sich also um den fünften Band (Volume) mit dem Titel (Title) ›Cosa in the Republic and Early Empire‹ der Reihe ›Cosa‹ (Maintitle) darstellt, was wiederum den zweiten Band (Number) der Serie ›MemAmAc‹ ist.

\begin{refsection}
\begin{bsp}
\nocite{Fentress_2003}
\printbibliography[title={Forschungsliteratur}]
\end{bsp}
 \end{refsection}
 
 
 \subsubsection{Bestandskatalog}
 Die Angabe von Bestandskatalogen weicht in einer Kleinigkeit von Sammelbänden o.ä. ab:
 Es wird kein Titel genannt. Damit fällt auch das Komma nach dem Autor weg.
Zwei Beispiele verdeutlichen dies.

|@inbook{Kohlmeyer_1983,|\\
|author = {K. Kohlmeyer},|\\
|editor = {U. Gehrig},|\\
|booktitle = {Tierbilder aus vier Jahrtausenden},|\\
|booksubtitle = {Antiken der Sammlung Mildenberg},|\\
|location = {Mainz},|\\
|pages = {20 Nr. 9},|\\
|year = {1983}|\\
|}|\\

und


|@inbook{Parlasca_1969,|\\
|author = {K. Parlasca},|\\
|booktitle = {Helbig},|\\
|volume = {III},|\\
|edition = {4},|\\
|location = {Tübingen},|\\
|pages = {98 f. Nr. 2176},|\\
|year ={1969},|\\
|}|\\

pages = {98 f. Nr. 2176},

 \begin{refsection}
\begin{bsp}
\nocite{Kohlmeyer_1983,Parlasca_1969}
\printbibliography[title={Forschungsliteratur}]
\end{bsp}
 \end{refsection}
 
 Der Typ |@inbook| entspricht hier dem Typ |@incollection|.


 \subsection{Typ \texttt{@article}}\label{article}
%%
\DescribeMacro{@article}

|@Article{Evangelidis_2014,|\\
|  Title   = {Agoras {and} Fora},|\\
|  Author  = {Evangelidis, Vasilis},|\\
|  Journaltitle         = BSA-lang,|\\
|  Pages   = {335--356},|\\
|  Volume  = {109},|\\
|  Year    = {2014},|\\
|  Doi     = {10.1017/s006824541400015x},|\\
|  Shortjournal             = BSA-kurz,|\\
|  Subtitle  = {Developments in the Central Public Space of|\\
|										 the Cities of Greece during the {Roman} Period},|\\
|}|\\


%\begin{bsp}\todo[noline]{dynamisches Beispiel}\fullcite{Evangelidis_2014}\end{bsp}
\begin{refsection}
\begin{bsp}
\nocite{Evangelidis_2014}
\printbibliography[title={Forschungsliteratur}]
\end{bsp}
 \end{refsection}

Es gibt die Möglichkeit bei Zeitschriftenartikel den Publikationszustand anzugeben.
Die beliebige Eingabe erfolgt im Feld |pubstate| und wird ans Ende des bibliographischen Eintrages gesetzt.

Es empfiehlt sich allerdings sich auf ein paar Standarts zu beschränken, da diese sprachabhängig ausgegeben werden:
\begin{description}
\item[inpreparation] Typoskript wird für die Publikation vorbereitet. \\(|pubstate = {inpreparation}|)
\item[submitted] Typoskript wurde bei der Zeitschrift eingereicht.\\(|pubstate = {submitted}|)
\item[forthcoming] Typoskript wurde von der Zeitschrift akzeptiert.\\(|pubstate = {forthcoming}|)
\item[inpress] Typoskript liegt bereits als fertig gesetzter Artikel vor.\\(|pubstate = {inpress}|)
\item[prepublished] Artikel liegt in einer (online) Vorabversion vor. \\(|pubstate = {prepublished}|)
\end{description}

Ein Beispiel zeigt die Anwendung für |pubstate = {forthcoming}|

|@Article{Bossert_2016,|\\
|  Title                    = {\ldots\ \textsc{in formam anitqvam restitvto}?}, |\\
| Subtitle = {Überlegungen zur Inschrift der ›Porticus Deorum Consentium‹ (CIL\,VI 102)|\\
|					 und ihren Ergänzungen im 19.{\,}Jahrhundert},|\\
 | Author                   = {Lukas C. Bossert},|\\
 | Volume                   = {2},|\\
 | Year                     = {2016},|\\
 | Journaltitle                  = {BeStAR. Berliner Studien zum Antiken Rom},|\\
 | shortjournal = {BeStAR},|\\
| pubstate = {forthcoming},|\\
|}|\\

\begin{refsection}
\begin{bsp}
\nocite{Bossert_2016}
\printbibliography[title={Forschungsliteratur}]
\end{bsp}
 \end{refsection}



\subsection{Typ \texttt{@proceedings}}\label{proceedings}

 Für Beiträge innerhalb eines Konferenzbandes müssen die Felder |venue|, |eventdate| und |eventtitle| ausgefüllt werden. Ansonsten alle anderen Felder entsprechend wie bei |@book|:
 
| @Proceedings{Kurapkat_2014,|\\
|Title   = {Die Architektur des Weges},|\\
|Editor  = {Kurapkat, Dietmar and Schneider, Peter I.|\\
|									and Wulf-Rheidt, Ulrike},|\\
|Year    = {2014},|\\
|Eventdate = {2012-02-08/2012-02-11},|\\
|Eventtitle = {Kolloquium Architekturreferat des DAI},|\\
|Location  = {Regensburg},|\\
|Number  = {11},|\\
|Publisher = {Schnell + Steiner},|\\
|Series  = DiskAB-lang,|\\
|Shortseries  = DiskAB-kurz,|\\
|Subtitle  = {Gestaltete Bewegung im gebauten Raum},|\\
|Venue   = {Berlin},|\\
| }|\\

So wird daraus: 
 
%\begin{bsp}\todo[noline]{dynamisches Beispiel}
%\fullcite{Kurapkat_2014}
%\end{bsp}

\begin{refsection}
\begin{bsp}
\nocite{Kurapkat_2014}
\printbibliography[title={Forschungsliteratur}]
\end{bsp}
 \end{refsection}
 
\subsection{Typ \texttt{@inproceedings}}\label{inproceedings}

Wie bei |@proceedings| so auch hier:


| @Inproceedings{Torelli_1991,|\\
|  Author			= {Torelli, Mario},|\\
|  Title				= {Il \enquote{diribitorium} di Alba Fucens e il \enquote{campus} eroico di Herdonia},|\\
|  Pages			= {39--63},|\\
|  Editor			= {Mertens, Josef},|\\
|  Booktitle		= {Comunitá indigene e problemi della romanizzazione|\\
|							 nell’Italia centro-meridionale (IV--III sec. a.C.)},|\\
|  Year				= {1991},|\\
|  Eventtitle		= {Actes du Colloque International Organisé|\\
|							à l'Occasion du 50. Anniversaire de l'Academia Belgica et|\\
|							du 40. Anniversaire des Fouilles Belges en Italie},|\\
|  Venue			= {Roma},|\\
|  Eventdate	= {1990-02-01/1990-02-03},|\\
|  Publisher		= {Institut Historique Belge de Rome},|\\
|  Location		= {Bruxelles},|\\
|  Series			= {Études de philologie, d'archéologie et d'histoire anciennes},|\\
|  Number		= {29},|\\
|  Hyphenate 	= {italian},|\\
|  Shorttitle		= {Il \enquote{diribitorium}}|\\
|}| \\

Und daraus wird:

\begin{refsection}
\begin{bsp}
\nocite{Torelli_1991}
\printbibliography[title={Forschungsliteratur}]
\end{bsp}
 \end{refsection}

 \subsection{Typ \texttt{@inreference}}\label{inreference}

 \DescribeMacro{@inreference}
 Mit dem Typ |@inreference| können  Lexikonartikel zitiert werden.
 Dabei gibt es generall zwei Möglichkeiten wie der Eintrag dargestellt wird:
 Entweder gleich wie alle anderen Einträge mit ›Autor Jahr‹-Angabe, oder speziell nach den Vorgaben des Deutschen Archäologischen Instituts, sodass der Eintrag in der Fußnote vollzitiert und -referenziert ist:
›Lexikon Band (Jahr) Seitenzahlen s.\,v. Titel des Artikels (Autor)‹.
Für die erste Variante bedarf es keine Änderung, präferiert man die zweite Variante, so muss in der Präambel die Option |inreferences=true| gesetzt werden. \DescribeMacro{inreferences=true}

An einem Beispiel wird die Anwendung deutlich.
 
 |@Inreference{Neils_1994,|\\
 |  Title   = {Theseus},|\\
 |  Author  = {Neils, Jenifer},|\\
 |  Year    = {1994},|\\
 |  Booktitle = LIMC-kurz,        |\emph{Angabe des @string}\\
 | Options	= {lexikon},|\\
 |  Pages   = {922--951},|\\
 |  Related   = {LIMC},|\\
 |  Volume  = {7.1},|\\
 |} |

Zusätzlich lohnt es sich den Eintrag über |related| mit dem Hauptwerk (in diesem Falle ›LIMC‹) zu verküpfen:

 |@Reference{LIMC, |\\
 | Title                    = LIMC-lang,|\\
 |  Keywords                 = {Sigel},|\\
 |  Options                  = {corpus},|\\
 |  Shorthand                = LIMC-kurz |\\
 |}|\\
 
 Dies hat den Vorteil, dass man die Abkürzung ›LIMC‹ in der Teilbibliographie ›Lexika und Corpora‹ (\cref{bib:sigel}) danke des |Keywords = {Sigel}| automatisiert auflösen kann. Dafür muss der Bibliographieeintrag |@reference{LIMC}| nicht manuell zitiert werden, da dies über den Eintrag |related| bei |@inreference{Neils_1994}| dynamischt erfolgt.
 
 

 Verwendet man den normalen |\cite|\oarg{prenote}\oarg{postnote}\marg{Schlüssel}-Befehl, dann wird bei eingetragener Option (|lexikon|) die \oarg{postnote} nicht nachgestellt, sondern nach der Jahresangabe des Lexikons gesetzt.
 Ist die \oarg{postnote} leer, dann werden die Seiten aus dem Bibliographieeintrag ausgelesen und an diese Stelle gesetzt:
  
 \begin{bsp}
	\ldots\ (|\cite[vgl.][930 Nr. 283]{Neils_1994}|).
	
	...\ (vgl. LIMC 7.1 (1994) 930 Nr. 283 s. v. Theseus (J. Neils)).
	% \ldots (\fullcite[vgl.][930 Nr. 283]{Neils_1994}).
 \end{bsp}
 
\DescribeMacro{\parencite} 
Dieses Beispiel zeigt, dass keine Klammerregelung angewendet wurde, was nicht gewünscht ist. 
Es ist zu beachten, dass bei gewählter Option  |inreferences=true| (in der Präambel) und bei einer gewünschten  Zitation in Klammern (), der Befehl |\parencite|\marg{Schlüssel} verwendet wird (\ref{cite-befehle}), damit die Klammerregelung automatisch angewendet wird.
Dies funktioniert so: 
 
 \begin{bsp}
	\ldots\ |\parencite[vgl.][930 Nr. 283]{Neils_1994}.|
 
 	...\ (vgl. LIMC 7.1 [1994] 930 Nr. 283 s. v. Theseus [J. Neils]).

	%\ldots \parencite{\fullcite[vgl.][930 Nr. 283]{Neils_1994}}.
 \end{bsp}

\DescribeMacro{\parencites}
Gleiches gilt auch für das Zitieren von mehreren Einträgen in Klammern: Dafür wird analog der Befehl |\parencites|\marg{Schlüssel} verwendet:
 \begin{bsp}
	\ldots\ |\parencites[vgl.][930 Nr. 283]{Neils_1994}[9]{Nieddu_1995}.|
 
 	...\ (vgl. Neils 1994, 930 Nr. 283; Nieddu 1995, 9).
 	 \end{bsp}

 	bzw. mit der Präambel-Option |inreferences=true|:
 	 \begin{bsp}

 	\ldots (vgl. LIMC 7.1 [1994] 930 Nr. 283 s. v. Theseus [J. Neils]; LTUR 2 [1995] 9 s. v. Dei Consentes [G. Nieddu])

	%\ldots\ \parencites[vgl.][930 Nr. 283]{Neils_1994}[9]{Nieddu_1995}
	%\ldots \parencite{\fullcite[vgl.][930 Nr. 283]{Neils_1994}}.
 \end{bsp}


Da der Bibliographieeintrag ein Lexikonartikel ist (|@inreference|), werden bei gewählter Präambel-Option |inreferences=true| alle Lexikoneinträge von der Bibliographie ausgeschlossen, da sie ja in der Fußnote vollzitiert werden.

Ansonsten  (Präambel-Option |inreferences=false| = default)  sieht der Eintrag in der Bibliographie so aus:

\begin{refsection}
\begin{bsp}
\nocite{Neils_1994,Nieddu_1995}
\printbibliography[title={Forschungsliteratur}]
\end{bsp}
 \end{refsection}
 



 Diese Handhabe gilt nicht nur für ›kanonische Lexika‹ der Altertumswissenschaften (RE, LIMC, DNP, LTUR, LÄ, etc.) sondern kann auf alle Lexika angewendet werden.
 
 Da es nicht für alle Lexika eine Abkürzungskonvention  des  Titels gibt, können und sollten diese selbst abkürzt werden.\footnote{Dies ist sehr emphehlenswert, da in der Bibliographie der Abstand zwischen der Autor-Jahr-Abkürzung und dem Vollzitat sich nach der längsten ›Shorthand‹ richtet.}
 
 Ein weiteres Beispiel:
| @Inreference{Weinbrenner_1914,|\\
|  Title                    = {Rennbahn},|\\
| Author                   = {Weinbrenner},|\\
|  Number                   = {2},|\\
|  Pages                    = {636--637},|\\
|  Volume                   = {9},|\\
|  Year                     = {1914},|\\
|  related = {Lexikon-der-Technik},|\\
|  Booktitle                = {Lexikon d. T.},|\\
|}|\\

Der Lexikoneintrag ist über |related| verknüpft mit:

|@reference{Lexikon-der-Technik,|\\
| editor = {Otto Lueger},|\\
| location = {Stuttgart},|\\
|year = {1904--1920},|\\
|edition = {2},|\\
|shorthand = {Lexikon d. T.},|\\
| Keywords                 = {Sigel},|\\
| title = {Lexikon der gesamten Technik und ihrer Hilfswissenschaften},|\\
|}|\\

Wichtig ist, dass |booktitle| des |@inreference| und |shorthand| des |@reference| gleich sind, damit der Titel des Lexikons entsprechend aufgelöst werden kann.


\begin{refsection}
\begin{bsp}
\nocite{Lexikon-der-Technik,Weinbrenner_1914}
 \printbibliography[keyword=Sigel,title={Abkürzungen und Sigel}]
\printbibliography[title={Forschungsliteratur},notkeyword=Sigel]
\end{bsp}
 \end{refsection}


 
 \subsection{Typ \texttt{@review}}\label{review}
\DescribeMacro{@review}
Rezensionen, die in Zeitschriften erschienen sind, werden als |@review| verarbeitet.
Oft zitiert man die Rezension und das rezensierte Werk in seiner Arbeit.
Das folgende Beispiel geht auf diesen Fall ein und erläutert die relevanten Punkte und Besonderheiten.

Dafür sind zwei Bibliographieeinträge notwendig.
Zuerst das rezensierte Werk:

|@Book{Welch_2007,|\\
|  Title                    = {The {Roman} Amphitheatre},|\\
|  Author                   = {Welch, Katherine E.},|\\
|  Location                 = {Cambridge and New York},|\\
|  Publisher                = CUP,|\\
|  Subtitle                 = {From its Origins to the Colosseum},|\\
|  Year                     = {2007}|\\
|}|\\

und schließlich die dazugehörige Rezension:

|@Review{Bell_2011,|\\
|  Author                   = {Bell, Sinclair},|\\
|  Related = {Welch_2007},%| \emph{Besonderheit!!}\\
|  Relatedtype = {reviewof},%| \emph{Besonderheit!!}\\
|  Number                   = {1},|\\
|  Volume                   = {115},|\\
|  Pages                    = {1--4},|\\
|  Year                     = {2011},|\\
|  Journaltitle             = AJA-lang,|\\
|  Publisher                = {Archaeological Institute of America},|\\
|  Shortjournal             = AJA-kurz,|\\
|}|\\

Im Bibliographieeintrag der Rezension (|Bell_2011|) wird im Feld |Related| mittels Eintrag |{Welch_2007}| Bezug auf das rezensierte Werk (|Welch_2007|) genommen.
Das Feld |Relatedtype| enthält die Information in welcher Beziehung das Werk (|Bell_2011|) zu |Welch_2007| steht. Die Angabe |reviewof| ist der |bibstring| für Rezensionen und enthält sprachabhängig (Deutsch und Englisch) die Bezeichnung |Rez. zu| bzw. |Review of|.
Dies ist für die Bibliographie relevant: Es ist nun nicht mehr notwendig das rezensierte Werk mit allen dazugehörigen Angaben einzutippen, da  dies automatisch über die |Related|-Funktion erfolgt.

\begin{refsection}
\begin{bsp}
\nocite{Bell_2011}
\printbibliography[title={Forschungsliteratur}]
\end{bsp}
 \end{refsection}
 
Der Vorteil mit |Related| und |Relatedtype| zu arbeiten liegt in der Dynamik: Ändert sich eine Angabe im rezensierten Werk, wird dies bei der Rezension automatisch geändert.
Zudem wird das rezensierte Werk bei der Erwähung in der Rezension nicht automatisch in die Bibliographie als eigener Eintrag aufgenommen, sondern erst dann, wenn es selbst explizit zitiert wird.

\subsubsection{Rezensionen mit eigenem Titel}
Manche Rezensionen sind ausführlicher und haben daher einen eigenen Titel, der angegeben werden soll.
In diesem Fall wird das Feld |Title| ausgefüllt, ansonsten bleibt alles gleich: 

|@Review{Hufschmid_2010,|\\
|  Title                    = {Von Caesars \emph{theatron kynegetikon}|\\
| zum \emph{amphitheatrum novum} Vespasians},|\\
|  Related = {Welch_2007},%| \emph{Besonderheit!!}\\
|  Relatedtype = {reviewof},%| \emph{Besonderheit!!}\\
|  Author                   = {Hufschmid, {\relax Th}omas},|\footnote{Mit |relax| wird geregelt, dass |Th.| als Initiale ausgegeben wird.}\\
|  Pages                    = {487--504},|\\
|  Volume                   = {23},|\\
|  Year                     = {2010},|\\
|  Journaltitle             = JRA-lang,|\\
|  Shortjournal             = JRA-kurz,|\\
|}|\\

In der Bibliographie wird dann zuerst der eigene Titel der Rezension ausgegeben, dann die Angaben zum rezensierten Werk.

\begin{refsection}
\begin{bsp}
\nocite{Hufschmid_2010}
\printbibliography[title={Forschungsliteratur}]
\end{bsp}
 \end{refsection}


\subsubsection{Sammelrezensionen}
Es kann vorkommen, dass Rezensionen nicht nur ein Werk unter die Lupe nehmen, sondern zwei oder noch mehr.
Dies ist jedoch kein Problem, da das Vorgehen fast analog zu den oben genannten Beispielen ist.
\citeauthor{Taylor_2008} hat nicht nur das bereits erwähnte Buch \citetitle{Welch_2007} von \citeauthor{Welch_2007} rezensiert, sondern es zugleich mit \citeauthor{Sear_2006}s \citetitle{Sear_2006} verglichen.


|@Review{Taylor_2008,|\\
|  Author                   = {Taylor, Rabun},|\\
| Number                   = {3},|\\
| Pages                    = {443--445},|\\
|  Related                  = {Sear_2006,Welch_2007},|\\
| Relatedtype              = {reviewof},|\\
|  Journaltitle             = {Journal of the Society of Architectural Historians},|\\
|  Volume                   = {67},|\\
|  Year                     = {2008},|\\
|}|\\


|@Book{Sear_2006,|\\
|  Title                    = {Roman Theatres},|\\
|  Author                   = {Sear, Frank},|\\
|  Location                 = {Oxford},|\\
|  Publisher                = OUP,|\\
|  Series                   = {Oxford Monographs on Classical Archeology},|\\
|  Subtitle                 = {An Architectural Study},|\\
|  Year                     = {2006},|\\
|}|\\


\begin{refsection}
\begin{bsp}
\nocite{
%Sear_2006,
Taylor_2008,
%Welch_2007
}
\printbibliography[title={Forschungsliteratur}]
\end{bsp}
 \end{refsection}


 \subsection{Typ \texttt{@thesis}}\label{thesis}
Master- und (unpublizierte) Doktorarbeiten sind als |@thesis| aufzunehmen.
Wichtige Felder sind |type=|\marg{|phdthesis|} bzw. \marg{|mathesis|}  und |institution=|\marg{Universität}.

Beispiel:

| 	@Thesis{Arnolds_2005,|\\
|   Title      	= {Funktionen republikanischer und frühkaiserzeitlicher|\\ 
|					Forumsbasiliken in Italien},|\\
|  	Author	= {Markus Arnolds},|\\
|  	Date			= {2005-05-31},|\\
|  	Institution	= {Ruprecht-Karls-Universität zu Heidelberg},|\\
|  	Type		= {phdthesis},|\\
|  	Year			= {2005},|\\
|  eprint       = {urn:nbn:de:bsz:16-heidok-74406}, |\\
|  eprinttype = {urn},|\\
|	}|\

In der Bibiographie wird das zu:

%\begin{bsp}\todo[noline]{dynamisches Beispiel}
%\fullcite{Arnolds_2005}
%\end{bsp}

\begin{refsection}
\begin{bsp}
\nocite{Arnolds_2005}
\printbibliography[title={Forschungsliteratur}]
\end{bsp}
 \end{refsection}
 
 
 \changes{v1.1}{2015/06/04}{Umsetzung von |@thesis| im Stil.}


\section{Verschiedenes}
\subsection{Anonymes Werk}\label{unbekannt}

 Für manche Artikel oder Bücher lässt sich kein Autor oder Herausgeber ermitteln. 
 Diese Werke werden dann als anonym gekennzeichnet und nicht nach dem (anonymen) Autor/Herausgeber zitiert, sondern nach einem gewählten |label|. 

 
| @Article{Cosa_1949,|\\
|  Title   = {Cosa},|\\
|  Journaltitle  = ClJ-lang,|\\
|  Shortjournal  = ClJ-kurz,|\\
|  Pages   = {141--149},|\\
|  Volume  = {45},|\\
|  Year    = {1949},|\\
|  Number  = {1},|\\
|  Label = {Cosa},|\\
|  Subtitle  = {Republican Colony in Etruria},|\\
|}|\\

 
 Das |Label| wurde in diesem Fall analog zum Titel gewählt (Cosa).
 Zitiert man dieses Werk in einer Fußnote, dann wird

 |\cite[vgl.][145--146]{Cosa_1949}| 

 zu
 \begin{refsection}
 \begin{bsp} 
 \cite[vgl.][145--146]{Cosa_1949}
% vgl.  [Cosa 1949], 145--146
 \end{bsp}
 
Und für die Bibliographie:
\begin{bsp}
%\nocite{Cosa_1949}
\printbibliography[title={Forschungsliteratur}]
\end{bsp}
 \end{refsection}

%%
%%



\newpage
 \section{Bibliographie}\label{bibliographie}
 \DescribeMacro{\printbibliography}
Wie bei jedem Dokument mit im Text zitierten Werken bedarf es einer Stelle, an der diese auch aufgeschlüsselt werden: die Bibliographie. 
Für Altertumswissenschaftler (und auch andere) ist es manchmal hilfreich verschiedene Bibliographien im Dokument zu haben, die unterschiedliche Arten von Werke beinhalten, bspw. ein Quellenverzeichnis, Abkürzungen und Forschungsliteratur. 
Nachfolgend wird gezeigt, wie dies berwerkstelligt werden kann. 
Zunächst sollten alle Quellen in der |bib|-Datei mit dem Feld
 |keyword={Quelle},|
 versehen werden. 
Es bietet sich  an, mit (nummerierten) Unterbibliographien zu arbeiten, die über die Option  |heading=bibnumbered|, bzw. |heading=subbibnumbered| geladen werden.

|\printbibheading[%|\\
|            ||heading=bibnumbered,|\\
|            ||title={Bibliographie}] %| \emph{Überschrift für Bibliographieumgebung}\\
|\printbibliography[%|\\
|            ||keyword=Quelle,|\\
|            ||heading=subbibnumbered,%|\\
|            ||title={Antike Quellen}]|\\
|\printbibliography[%|\\
|            ||notkeyword=Quelle,|\\
|            ||notkeyword=Sigel,%|\\
|            ||heading=subbibnumbered,%|\\
|            ||title={Forschungsliteratur}]| 

\begin{refsection}
\nocite{*}

Damit wird zuerst die Quellen und danach das \enquote{gewöhnliche} Literaturverzeichnis getrennt voneinander ausgegeben. 
\setcounter{section}{0}
\begin{bsp}
\renewcommand\bibfont{\normalfont\footnotesize}
\printbibheading[heading=bibnumbered,
            title={Bibliographie}]
\printbibliography[keyword=Quelle,
            heading=subbibnumbered,
            title={Antike Quellen}]
\printbibliography[%
		notkeyword=Quelle,%
		notkeyword=Sigel,%
           	heading=subbibnumbered,
            title={Forschungsliteratur}]
\end{bsp}


Es können mehrere Bibliographien über |\printbibliography| erstellt werden, die jeweils unterschiedliche Einträge haben können.
Beispielsweise kann man eine Unterbibliographie erstellen, in der nur die Sigeln (Lexika, Handbücher, Inschriftencorpora, etc) aufgeführt werden, sodass diese dann aus der |Forschungsliteratur| herausfliegen (dort  |notkeyword=Sigel| ergänzen). Dafür wird das Feld |keyword| auf den Inhalt |Sigel| ausgelesen:

|\printbibliography[keyword=Sigel,%|\\
|            ||heading=subbibnumbered,%|\\
|            ||title={Abkürzungen und Sigel}]|\\

Die Teilbibliographie umfasst dann nur Einträge, die unter |keywords = {Sigel}| stehen haben:
\begin{bsp}
\printbibliography[keyword={Sigel},
           heading=subbibnumbered,
            title={Abkürzungen und Sigel}]\label{bib:sigel}
\end{bsp}

Ebenso hilfreich kann es sein, dass die verwendeten Abkürzungen der Zeitschriften und Reihen aufgelöst werden.
Dies geschieht  für die Zeitschriften mit:

|\printbiblist[%|\\
|            ||heading=subbibnumbered,%|\\
|            ||title={Zeitschriftenabkürzungenl}]{shortjournal}|\\
\begin{bsp}
\printbiblist[heading=subbibnumbered,
title={Zeitschriftenabkürzungen}]{shortjournal}
\end{bsp}

bzw. für die Reihen:

|\printbiblist[%|\\
|            ||heading=subbibnumbered,%|\\
|            ||title={Reihenabkürzungen}]{shortseries}|\\

\begin{bsp}
\printbiblist[heading=subbibnumbered,title={Reihenabkürzungen}]{shortseries}\end{bsp}

\end{refsection}
\setcounter{section}{5}
\section{Abkürzungen nach DAI-Richtlinie}\label{listen}

Da die Schreibweise von manchen Kürzeln der Zeitschriften oder Reihen nicht als |@string| übernommen werden konnten, musste die Liste an die \TeX-verarbeitende Lesart angepasst werden.
Generell gilt, dass Akzente weggelassen wurden, Umlaute ausgeschrieben (ä--> ae), ähnliche Buchstaben mit Äquivalenten ersetzt wurden (ı --> i).\footnote{Für problemloses Kompilieren und Verarbeiten der nichtlateinischen Buchstaben wird \hologo{XeLaTeX} oder  \hologo{LuaTeX} empfohlen.}
Nachfolgend werden zwei Listen angeführt, die die Auflistung nach @string (links) und Ausgabe (rechts) zeigen.
Die erste Liste enthält die Abkürzungen (\cref{liste-kurz}), die zweite Liste die ausgeschriebenen Namen (\cref{liste-lang}).
Es empfiehlt sich, in diesen Listen die |@string|-Angaben nachzuschauen und dann für |journaltitle| und |shortjournal|, bzw. |series| und |shortseries| einzutragen.

Die Unterscheidung, ob in der Bibliographie die Abkürzung (Standard) oder der voller Titel genannt werden soll, erfolgt automatisch über die Option |noabbrevs| (Siehe \cref{noabbrevs}). Ist diese Option nicht aktiviert, dann werden standardmäßig die Abkürzungen angegeben.

%%Beim Kompilieren wird eine Bibliographiedatei (|dai-abkuerzungen.bib|) erstellt, in der die Angaben der |@strings| gespeichert sind.
%%Diese Bibliographiedatei wird zusätzlich zur eigenen Bibliographie geladen.
%%Ist |dai-abkuerzungen.bib| bereits vorhanden, dann wird sie nicht neu generiert.


\subsection{Kurzformen\label{liste-kurz}}
\begin{multicols}{2}
\begin{footnotesize}
\begin{description}[%
			%	style=multiline,
				style=nextline,
				leftmargin=3cm,
				font=\normalfont]
%\begin{description}
 \item[AA-kurz] AA 
 \item[AAA-kurz] AAA 
 \item[AAcque-kurz] AAcque 
 \item[AAdv-kurz] AAdv 
 \item[AAJ-kurz] AAJ 
 \item[AAlpi-kurz] AAlpi 
 \item[AarbKob-kurz] AarbKøb %*Abweichung!
 \item[AArchit-kurz] AArchit 
 \item[AAS-kurz] AAS 
 \item[AASOR-kurz] AASOR 
 \item[AAusgrBadWuert-kurz] AAusgrBadWürt %*Abweichung!
 \item[AAustr-kurz] AAustr 
 \item[ABADY-kurz] ABADY 
 \item[AbhBerlin-kurz] AbhBerlin 
 \item[AbhDuesseldorf-kurz] AbhDüsseldorf %*Abweichung!
 \item[AbhGoettingen-kurz] AbhGöttingen %*Abweichung!
 \item[AbhLeipzig-kurz] AbhLeipzig 
 \item[AbhMainz-kurz] AbhMainz 
 \item[AbhMuenchen-kurz] AbhMünchen %*Abweichung!
 \item[ABret-kurz] ABret 
 \item[Abr-Nahrain-kurz] Abr-Nahrain 
 \item[ABulg-kurz] ABulg 
 \item[ABV-kurz] ABV 
 \item[ACalc-kurz] ACalc 
 \item[ACamp-kurz] ACamp 
 \item[ACant-kurz] ACant 
 \item[AcBibl-kurz] AcBibl 
 \item[Achse-kurz] Achse 
 \item[Acme-kurz] Acme 
 \item[Acontia-kurz] Acontia 
 \item[ACors-kurz] ACors 
 \item[ActaAArtHist-kurz] ActaAArtHist 
 \item[ActaAArtHist-sa-kurz] ActaAArtHist s.a. %*Abweichung!
 \item[ActaAcAbo-kurz] ActaAcAbo 
 \item[ActaACarp-kurz] ActaACarp 
 \item[ActaALov-kurz] ActaALov 
 \item[ActaALovMono-kurz] ActaALovMono 
 \item[ActaAntHung-kurz] ActaAntHung 
 \item[ActaArch-kurz] ActaArch 
 \item[ActaArchHung-kurz] ActaArchHung 
 \item[ActaAth-kurz] ActaAth 
 \item[ActaCl-kurz] ActaCl 
 \item[ActaClDebrec-kurz] ActaClDebrec 
 \item[ActaHistDac-kurz] ActaHistDac 
 \item[ActaHyp-kurz] ActaHyp 
 \item[ActaInstRomFin-kurz] ActaInstRomFin 
 \item[ActaMusNapoca-kurz] ActaMusNapoca 
 \item[ActaMusPorol-kurz] ActaMusPorol 
 \item[ActaNum-kurz] ActaNum 
 \item[ActaOr-kurz] ActaOr 
 \item[ActaOrHung-kurz] ActaOrHung 
 \item[ActaPhilSocDac-kurz] ActaPhilSocDac 
 \item[ActaPraehistA-kurz] ActaPraehistA 
 \item[ActaTorunA-kurz] ActaTorunA 
 \item[ActaTorunHist-kurz] ActaTorunHist 
 \item[AD-kurz] AD 
 \item[ADAIK-kurz] ADAIK 
 \item[Adalya-kurz] Adalya 
 \item[ADelt-A-kurz] ADelt A %*Abweichung!
 \item[ADelt-B-kurz] ADelt B %*Abweichung!
 \item[ADerg-kurz] ADerg 
 \item[ADFU-kurz] ADFU 
 \item[AdI-kurz] AdI 
 \item[ADOG-kurz] ADOG 
 \item[Adumatu-kurz] Adumatu 
 \item[AE-kurz] AE 
 \item[AeA-kurz] AeA 
 \item[Aegaeum-kurz] Aegaeum 
 \item[AegLev-kurz] ÄgLev %*Abweichung!
 \item[AEmil-kurz] AEmil 
 \item[AEphem-kurz] AEphem 
 \item[AeR-kurz] AeR 
 \item[AErgoMak-kurz] AErgoMak 
 \item[AErt-kurz] AErt 
 \item[AEspA-kurz] AEspA 
 \item[Aevum-kurz] Aevum 
 \item[AevumAnt-kurz] AevumAnt 
 \item[AF-kurz] AF 
 \item[AfO-kurz] AfO 
 \item[Africa-kurz] Africa 
 \item[AGD-kurz] AGD 
 \item[AGeo-kurz] AGeo 
 \item[Agora-kurz] Agora 
 \item[AgoraPB-kurz] AgoraPB 
 \item[AHist-kurz] AHist 
 \item[AHistStAlex-kurz] AHistStAlex 
 \item[AHw-kurz] AHw 
 \item[AiD-kurz] AiD 
 \item[AInf-kurz] AInf 
 \item[AIONArch-kurz] AIONArch 
 \item[AIONFil-kurz] AIONFil 
 \item[AIONLing-kurz] AIONLing 
 \item[AIPhOr-kurz] AIPhOr 
 \item[Aitna-kurz] Aitna 
 \item[AJA-kurz] AJA 
 \item[AJahrBay-kurz] AJahrBay 
 \item[AJPh-kurz] AJPh 
 \item[AJug-kurz] AJug 
 \item[AKorrBl-kurz] AKorrBl 
 \item[AlbaRegia-kurz] AlbaRegia 
 \item[AlmaMaterSt-kurz] AlmaMaterSt 
 \item[AlmanachWien-kurz] AlmanachWien 
 \item[AlonJisrael-kurz] AlonJisrael 
 \item[Al-Qannis-kurz] Al-Qanniš
 \item[Altamura-kurz] Altamura 
 \item[AltoMed-kurz] AltoMed 
 \item[Alt-Paphos-kurz] Alt-Paphos 
 \item[AltThuer-kurz] AltThür %*Abweichung!
 \item[AM-kurz] AM 
 \item[AMediev-kurz] AMediev 
 \item[AMethTh-kurz] AMethTh 
 \item[AMI-kurz] AMI 
 \item[AMIT-kurz] AMIT 
 \item[AmJAncHist-kurz] AmJAncHist 
 \item[AmJNum-kurz] AmJNum 
 \item[AMold-kurz] AMold 
 \item[AMosel-kurz] AMosel 
 \item[AMS-kurz] AMS 
 \item[AmStP-kurz] AmStP 
 \item[AMuGS-kurz] AMuGS 
 \item[ANachr-kurz] ANachr 
 \item[ANachrBad-kurz] ANachrBad 
 \item[AnadoluAra-kurz] AnadoluAra 
 \item[AnadoluKonf-kurz] AnadoluKonf 
 \item[AnadoluYil-kurz] AnadoluYıl 
 \item[AnAe-kurz] AnAe 
 \item[Anagennesis-kurz] Anagennesis 
 \item[AnalBolland-kurz] AnalBolland 
 \item[AnalP-kurz] AnalP 
 \item[AnalRom-kurz] AnalRom 
 \item[AnArqAnd-kurz] AnArqAnd 
 \item[Anas-kurz] Anas 
 \item[AnatA-kurz] AnatA 
 \item[Anatolia-kurz] Anatolia 
 \item[ANaturwiss-kurz] ANaturwiss 
 \item[AncCivScytSib-kurz] AncCivScytSib 
 \item[AncHistB-kurz] AncHistB 
 \item[AncInd-kurz] AncInd 
 \item[AncNearEastSt-kurz] AncNearEastSt 
 \item[AnCord-kurz] AnCord 
 \item[AncSoc-kurz] AncSoc 
 \item[AncW-kurz] AncW 
 \item[AncWestEast-kurz] AncWestEast 
 \item[AnDubr-kurz] AnDubr 
 \item[ANews-kurz] ANews 
 \item[ANilMoy-kurz] ANilMoy 
 \item[ANL-kurz] ANL 
 \item[AnMunFaro-kurz] AnMunFaro 
 \item[AnMurcia-kurz] AnMurcia 
 \item[AnnAcEtr-kurz] AnnAcEtr 
 \item[AnnAcTorino-kurz] AnnAcTorino 
 \item[AnnAStorAnt-kurz] AnnAStorAnt 
 \item[AnnBari-kurz] AnnBari 
 \item[AnnBenac-kurz] AnnBenac 
 \item[AnnBiblAModena-kurz] AnnBiblAModena 
 \item[AnnBiblARom-kurz] AnnBiblARom 
 \item[AnnByzConf-kurz] AnnByzConf 
 \item[AnnCagl-kurz] AnnCagl 
 \item[AnnCaglMag-kurz] AnnCaglMag 
 \item[AnnEconSocCiv-kurz] AnnEconSocCiv 
 \item[AnnEgBibl-kurz] AnnEgBibl 
 \item[AnnEth-kurz] AnnEth 
 \item[AnnFaina-kurz] AnnFaina 
 \item[AnnHistA-kurz] AnnHistA 
 \item[AnnHistScSoc-kurz] AnnHistScSoc 
 \item[AnnIstGiapp-kurz] AnnIstGiapp 
 \item[AnnIstItNum-kurz] AnnIstItNum 
 \item[AnnLecce-kurz] AnnLecce 
 \item[AnnLeedsUnOrSoc-kurz] AnnLeedsUnOrSoc 
 \item[AnnMacerata-kurz] AnnMacerata 
 \item[AnnMessMag-kurz] AnnMessMag 
 \item[AnnMusRov-kurz] AnnMusRov 
 \item[AnnNap-kurz] AnnNap 
 \item[AnnNivern-kurz] AnnNivern 
 \item[AnnNoment-kurz] AnnNoment 
 \item[AnnOrNap-kurz] AnnOrNap 
 \item[AnnotNum-kurz] AnnotNum 
 \item[AnnPerugia-kurz] AnnPerugia 
 \item[AnnPisa-kurz] AnnPisa 
 \item[AnnPontAcRom-kurz] AnnPontAcRom 
 \item[AnnRepBSA-kurz] AnnRepBSA 
 \item[AnnRepCypr-kurz] AnnRepCypr 
 \item[AnnRepFoggArtMus-kurz] TAnnRepFoggArtMus 
 \item[AnnSiena-kurz] AnnSiena 
 \item[AnnuarioAcLinc-kurz] AnnuarioAcLinc 
 \item[AnnuarioLecce-kurz] AnnuarioLecce 
 \item[AnnUnBud-kurz] AnnUnBud 
 \item[AnnWorcArtMus-kurz] AnnWorcArtMus 
 \item[Anodos-kurz] Anodos 
 \item[AnOr-kurz] AnOr 
 \item[ANRW-kurz] ANRW 
 \item[Anschnitt-kurz] Anschnitt 
 \item[ANSMusNotes-kurz] ANSMusNotes 
 \item[AnSt-kurz] AnSt 
 \item[Antaeus-kurz] Antaeus 
 \item[AntAfr-kurz] AntAfr 
 \item[AntChr-kurz] AntChr 
 \item[AntCl-kurz] AntCl 
 \item[AnthrAChron-kurz] AnthrAChron 
 \item[Anthropos-kurz] Anthropos 
 \item[Antichthon-kurz] Antichthon 
 \item[AntigCr-kurz] AntigCr 
 \item[Antipolis-kurz] Antipolis 
 \item[Antiqua-kurz] Antiqua 
 \item[Antiquity-kurz] Antiquity 
 \item[AntJ-kurz] AntJ 
 \item[AntK-kurz] AntK 
 \item[AntNat-kurz] AntNat 
 \item[AntPisa-kurz] AntPisa 
 \item[AntPl-kurz] AntPl 
 \item[AntSurv-kurz] AntSurv 
 \item[AntTard-kurz] AntTard 
 \item[AnzAW-kurz] AnzAW 
 \item[AnzWien-kurz] AnzWien 
 \item[AOAT-kurz] AOAT 
 \item[AoF-kurz] AoF 
 \item[AOtkryt-kurz] AOtkryt 
 \item[APamKiiv-kurz] APamKiiv 
 \item[APh-kurz] APh 
 \item[APol-kurz] APol 
 \item[Apollo-kurz] Apollo 
 \item[ApolloLond-kurz] ApolloLond 
 \item[APort-kurz] APort 
 \item[AppRomFil-kurz] AppRomFil 
 \item[APregl-kurz] APregl 
 \item[Apulum-kurz] Apulum 
 \item[AquiLeg-kurz] AquiLeg 
 \item[AquilNost-kurz] AquilNost 
 \item[Aquitania-kurz] Aquitania 
 \item[ArabAEpigr-kurz] ArabAEpigr 
 \item[ARadRaspr-kurz] ARadRaspr 
 \item[ArbFBerSaechs-kurz] ArbFBerSaechs 
 \item[Archaeographie-kurz] Archäographie %*Abweichung!
 \item[Archaeologia-kurz] Archäographie 
 \item[Archaeology-kurz] Archaeology 
 \item[Archaeometry-kurz] Archaeometry 
 \item[Archaiognosia-kurz] Archaiognosia 
 \item[ArchBegriffsGesch-kurz] ArchBegriffsGesch 
 \item[ArchByzMnem-kurz] ArchByzMnem 
 \item[ArchCl-kurz] ArchCl 
 \item[Archeo-kurz] Archeo 
 \item[ArcheogrTriest-kurz] ArcheogrTriest 
 \item[ArcheologiaParis-kurz] ArcheologiaParis 
 \item[ArcheologiaRoma-kurz] ArcheologiaRoma 
 \item[ArcheologiaWarsz-kurz] ArcheologiaWarsz 
 \item[ArcheologijaKiiv-kurz] ArcheologijaKiiv 
 \item[ArcheologijaSof-kurz] ArcheologijaSof 
 \item[ArchEubMel-kurz] ArchEubMel 
 \item[ArchHom-kurz] ArchHom 
 \item[Architectura-kurz] Architectura 
 \item[Archivi-kurz] Archivi 
 \item[ArchPF-kurz] ArchPF 
 \item[ArchPrehistLev-kurz] ArchPrehistLev 
 \item[ArchRel-kurz] ArchRel 
 \item[ArchStorCal-kurz] ArchStorCal 
 \item[ArchStorPugl-kurz] ArchStorPugl 
 \item[ArchStorRom-kurz] ArchStorRom 
 \item[ArchStorSicOr-kurz] ArchStorSicOr 
 \item[ArchStorSir-kurz] ArchStorSir 
 \item[Arctos-kurz] Arctos 
 \item[ARepLond-kurz] ARepLond 
 \item[Argo-kurz] Argo 
 \item[ArOr-kurz] ArOr 
 \item[ArOrMono-kurz] ArOrMono 
 \item[ArOrSuppl-kurz] ArOrSuppl 
 \item[ARozhl-kurz] ARozhl 
 \item[ArqBeja-kurz] ArqBeja 
 \item[Arse-kurz] Arse 
 \item[ArsGeorg-kurz] ArsGeorg 
 \item[ArtAntMod-kurz] ArtAntMod 
 \item[ArtARhone-kurz] ArtARhône %*Abweichung!
 \item[ArtAs-kurz] ArtAs 
 \item[ArtB-kurz] ArtB 
 \item[ArtJ-kurz] ArtJ 
 \item[ArtLomb-kurz] ArtLomb 
 \item[ArtMediev-kurz] ArtMediev 
 \item[ArtVirg-kurz] ArtVirg 
 \item[ARV2-kurz] ARV2 
 \item[ASachs-kurz] ASachs 
 \item[ASAE-kurz] ASAE 
 \item[ASammlUnZuerch-kurz] ASammlUnZürch %*Abweichung!
 \item[ASAtene-kurz] ASAtene 
 \item[ASbor-kurz] ASbor 
 \item[ASchw-kurz] ASchw 
 \item[ASoc-kurz] ASoc 
 \item[ASoloth-kurz] ASoloth 
 \item[ASR-kurz] ASR 
 \item[Assaph-kurz] Assaph 
 \item[AssyrMisc-kurz] AssyrMisc 
 \item[AST-kurz] AST 
 \item[ASub-kurz] ASub 
 \item[ASubacq-kurz] ASubacq 
 \item[Athenaeum-kurz] Athenaeum 
 \item[Atiqot-kurz] Atiqot 
 \item[AtiqotHeb-kurz] AtiqotHeb 
 \item[Atlal-kurz] Atlal 
 \item[AttiAcPontan-kurz] AttiAcPontan 
 \item[AttiAcRov-kurz] AttiAcRov 
 \item[AttiAcTorino-kurz] AttiAcTorino 
 \item[AttiCAntCl-kurz] AttiCAntCl 
 \item[AttiCItRom-kurz] AttiCItRom 
 \item[AttiMemBologna-kurz] AttiMemBologna 
 \item[AttiMemDal-kurz] AttiMemDal 
 \item[AttiMemFirenze-kurz] AttiMemFirenze 
 \item[AttiMemIstria-kurz] AttiMemIstria 
 \item[AttiMemMagnaGr-kurz] AttiMemMagnaGr 
 \item[AttiMemModena-kurz] AttiMemModena 
 \item[AttiMemTivoli-kurz] AttiMemTivoli 
 \item[AttiMusTrieste-kurz] AttiMusTrieste 
 \item[AttiPalermo-kurz] AttiPalermo 
 \item[AttiRovigno-kurz] AttiRovigno 
 \item[AttiSocFriuli-kurz] AttiSocFriuli 
 \item[AttiVenezia-kurz] AttiVenezia 
 \item[AuA-kurz] AuA 
 \item[AulaOr-kurz] AulaOr 
 \item[AusgrFu-kurz] AusgrFu 
 \item[AusgrFuWestf-kurz] AusgrFuWestf 
 \item[AustrRom-kurz] AustrRom 
 \item[AUTerr-kurz] AUTerr 
 \item[AUWE-kurz] AUWE 
 \item[AV-kurz] AV 
 \item[AVen-kurz] AVen 
 \item[AVes-kurz] AVes 
 \item[AViva-kurz] AViva 
 \item[AvP-kurz] AvP 
 \item[AW-kurz] AW 
 \item[AyasofyaMuezYil-kurz] AyasofyaMüzYıl \label{AyasofyaMuezYil-kurz} %*Abweichung!
 \item[AZ-kurz] AZ 
 \item[Azotea-kurz] Azotea 
 \item[BA-kurz] BA 
 \item[Baalbek-kurz] Baalbek 
 \item[BAAlger-kurz] BAAlger 
 \item[BABarcel-kurz] BABarcel 
 \item[BABesch-kurz] BABesch 
 \item[BAcRHist-kurz] BAcRHist 
 \item[BACopt-kurz] BACopt 
 \item[BadFuBer-kurz] BadFuBer 
 \item[Baetica-kurz] Baetica 
 \item[BaF-kurz] BaF 
 \item[BalacaiKoez-kurz] BalacaiKöz %*Abweichung!
 \item[BalkSt-kurz] BalkSt 
 \item[BALond-kurz] BALond 
 \item[BALux-kurz] BALux 
 \item[BaM-kurz] BaM 
 \item[BAMaroc-kurz] BAMaroc 
 \item[BAmSocP-kurz] BAmSocP 
 \item[BAncOrMus-kurz] BAncOrMus 
 \item[BAngers-kurz] BAngers 
 \item[BAngloIsrASoc-kurz] BAngloIsrASoc 
 \item[BAnnMusFerr-kurz] BAnnMusFerr 
 \item[BAntFr-kurz] BAntFr 
 \item[BAntLux-kurz] BAntLux 
 \item[BAParis-kurz] BAParis 
 \item[BAProv-kurz] BAProv 
 \item[BAR-kurz] BAR 
 \item[BArchAlex-kurz] BArchAlex 
 \item[BArchit-kurz] BArchit 
 \item[BARIntSer-kurz] BARIntSer 
 \item[BASard-kurz] BASard 
 \item[BAsEspA-kurz] BAsEspA 
 \item[BAsInst-kurz] BAsInst 
 \item[BASOR-kurz] BASOR 
 \item[BAssBude-kurz] BAssBudé %*Abweichung!
 \item[BAssMosAnt-kurz] BAssMosAnt 
 \item[BASub-kurz] BASub 
 \item[BASudEstEur-kurz] BASudEstEur 
 \item[BATarr-kurz] BATarr 
 \item[BAur-kurz] BAur 
 \item[BAVA-kurz] BAVA 
 \item[BayVgBl-kurz] BayVgBl 
 \item[BBasil-kurz] BBasil 
 \item[BBelgRom-kurz] BBelgRom 
 \item[BBolsena-kurz] BBolsena 
 \item[BBrByzSt-kurz] BBrByzSt 
 \item[BCamuno-kurz] BCamuno 
 \item[BCASic-kurz] BCASic 
 \item[BCercleNum-kurz] BCercleNum 
 \item[BCH-kurz] BCH 
 \item[BCircNumNap-kurz] BCircNumNap 
 \item[BCl-kurz] BCl 
 \item[BClevMus-kurz] BClevMus 
 \item[BCom-kurz] BCom 
 \item[BCord-kurz] BCord 
 \item[BdA-kurz] BdA 
 \item[BdE-kurz] BdE 
 \item[BdEC-kurz] BdEC 
 \item[BdI-kurz] BdI 
 \item[BDirRom-kurz] BDirRom 
 \item[BeazleyAddenda2-kurz] Beazley Addenda\textsuperscript{2}%*Abweichung!
 \item[BeazleyPara-kurz] Beazley, Para. %*Abweichung!
 \item[BEcAntNimes-kurz] BEcAntNîmes %*Abweichung!
 \item[BediKart-kurz] BediKart 
 \item[BEFAR-kurz] BEFAR 
 \item[BeitrESkAr-kurz] BeitrESkAr 
 \item[BeitrNamF-kurz] BeitrNamF 
 \item[BeitrSudanF-kurz] BeitrSudanF 
 \item[BelArt-kurz] BelArt 
 \item[Belleten-kurz] Belleten 
 \item[Benacus-kurz] Benacus 
 \item[BerBayDenkmPfl-kurz] BerBayDenkmPfl 
 \item[BerDFG-kurz] BerDFG 
 \item[BerlBeitrArchaeom-kurz] BerlBeitrArchäom %*Abweichung!
 \item[BerlBlVFruehGesch-kurz] BerlBlVFrühGesch %*Abweichung!
 \item[BerlJbVFruehGesch-kurz] BerlJbVFrühGesch %*Abweichung!
 \item[BerlMus-kurz] BerlMus 
 \item[BerlNumZ-kurz] BerlNumZ 
 \item[BerOudhBod-kurz] BerOudhBod 
 \item[BerRGK-kurz] BerRGK 
 \item[BerVerhLeipz-kurz] BerVerhLeipz 
 \item[Berytus-kurz] Berytus 
 \item[BEspA-kurz] BEspA 
 \item[BEspOr-kurz] BEspOr 
 \item[BEtOr-kurz] BEtOr 
 \item[BFilGrPadova-kurz] BFilGrPadova 
 \item[BFilLingSic-kurz] BFilLingSic 
 \item[BFlegr-kurz] BFlegr 
 \item[BFoligno-kurz] BFoligno 
 \item[BHarvMus-kurz] BHarvMus 
 \item[BIasos-kurz] BIasos 
 \item[BibAr-kurz] BibAr 
 \item[BiblClOr-kurz] BiblClOr 
 \item[BiblSymb-kurz] BiblSymb 
 \item[BIBulg-kurz] BIBulg 
 \item[BICS-kurz] BICS 
 \item[BIFAO-kurz] BIFAO 
 \item[BInfCentumcellae-kurz] BInfCentumcellae 
 \item[BInfCESDAE-kurz] BInfCESDAE 
 \item[BiogrZbor-kurz] BiogrZbor 
 \item[BiOr-kurz] BiOr 
 \item[BIstOrvieto-kurz] BIstOrvieto 
 \item[BJaen-kurz] BJaen %*Abweichung!
 \item[BJb-kurz] BJb 
 \item[BJerus-kurz] BJerus 
 \item[BLaborMusLouvre-kurz] BLaborMusLouvre 
 \item[BLazioMerid-kurz] BLazioMerid 
 \item[BLikUm-kurz] BLikUm 
 \item[BlMueFreundeF-kurz] BlMüFreundeF %*Abweichung!
 \item[BLugo-kurz] BLugo 
 \item[BMCGreekCoins-kurz] BMC Greek Coins %*Abweichung!
 \item[BMCOR-kurz] BMCOR 
 \item[BMCRE-kurz] BMCRE 
 \item[BMCRRI-III-kurz] BMCRR I--III %*Abweichung!
 \item[BMetrMus-kurz] BMetrMus 
 \item[BMon-kurz] BMon 
 \item[BMonMusPont-kurz] BMonMusPont 
 \item[BMQ-kurz] BMQ 
 \item[BMQNSuppl-kurz] BMQNSuppl 
 \item[BMusBeyrouth-kurz] BMusBeyrouth 
 \item[BMusBrux-kurz] BMusBruxs 
 \item[BMusCadiz-kurz] BMusCadiz 
 \item[BMusCivRom-kurz] BMusCivRom 
 \item[BMusFA-kurz] BMusFA 
 \item[BMusHongr-kurz] BMusHongr 
 \item[BMusMadr-kurz] BMusMadr 
 \item[BMusMich-kurz] BMusMich 
 \item[BMusMonaco-kurz] BMusMonaco 
 \item[BMusPadova-kurz] BMusPadova 
 \item[BMusPBelArt-kurz] BMusPBelArt 
 \item[BMusRom-kurz] BMusRom 
 \item[BMusVars-kurz] BMusVars 
 \item[BMusZaragoza-kurz] BMusZaragoza 
 \item[BNumParis-kurz] BNumParis 
 \item[BNumRoma-kurz] BNumRoma 
 \item[Bogazkoey-Hattusa-kurz] Boğazköy-Hattuša %*Abweichung!
 \item[Bolskan-kurz] Bolskan 
 \item[BonnHVg-kurz] BonnHVg 
 \item[BOntMus-kurz] BOntMus 
 \item[Boreas-kurz] Boreas 
 \item[BoreasUpps-kurz] BoreasUpps 
 \item[BPeintRom-kurz] BPeintRom 
 \item[BPI-kurz] BPI 
 \item[BPrehistAlp-kurz] BPI %*Abweichung!
 \item[BProAvent-kurz] BProAvent 
 \item[BProvidence-kurz] BProvidence 
 \item[BracAug-kurz] BracAug 
 \item[BracaraAugusta-kurz] BracaraAugusta 
 \item[BremABl-kurz] BremABl 
 \item[BRest-kurz] BRest 
 \item[Brigantium-kurz] Brigantium 
 \item[BrMusYearbook-kurz] BrMusYearbook 
 \item[BSA-kurz] BSA 
 \item[BSAA-kurz] BSAA 
 \item[BSFE-kurz] BSFE 
 \item[BSiena-kurz] BSiena 
 \item[BSOAS-kurz] BSOAS 
 \item[BSocAChamp-kurz] BSocAChamp 
 \item[BSocBiblReinach-kurz] BSocBiblReinach 
 \item[BSocNumRom-kurz] BSocNumRom 
 \item[BSR-kurz] BSR 
 \item[BStLat-kurz] BStLat 
 \item[BStorArt-kurz] BStorArt 
 \item[BTextilAnc-kurz] BStorArt 
 \item[BTorino-kurz] BTorino 
 \item[BTravTun-kurz] BTravTun 
 \item[BudReg-kurz] BudReg 
 \item[BulletinGetty-kurz] BulletinGetty 
 \item[BulletinNorthampton-kurz] BulletinNorthampton 
 \item[BVallad-kurz] BVallad 
 \item[BVitoria-kurz] BVitoria 
 \item[BWaltersArtGal-kurz] BWaltersArtGal 
 \item[BWPr-kurz] BWPr 
 \item[Byzantina-kurz] Byzantina
 \item[ByzF-kurz] ByzF 
 \item[ByzJb-kurz] ByzJb 
 \item[ByzZ-kurz] ByzZ 
 \item[BZ-kurz] BZ 
 \item[CAA-kurz] CAA 
 \item[CAD-kurz] CAD 
 \item[CadA-kurz] CadA 
 \item[Caesaraugusta-kurz] Caesaraugusta 
 \item[Caesarodunum-kurz] Caesarodunum 
 \item[CAH-kurz] CAH 
 \item[CahArmeeRom-kurz] CahArmeeRom 
 \item[CahASubaqu-kurz] CahASubaqu 
 \item[CahByrsa-kurz] CahByrsa 
 \item[CahCEC-kurz] CahCEC 
 \item[CahCerEg-kurz] CahCerEg 
 \item[CahDelFrIran-kurz] CahDelFrIran 
 \item[CahGlotz-kurz] CahGlotz 
 \item[CahKarnak-kurz] CahKarnak 
 \item[CahLig-kurz] CahLig 
 \item[CahMariemont-kurz] CahMariemont 
 \item[CahMusChampollion-kurz] CahMusChampollion 
 \item[CahPEg-kurz] CahPEg 
 \item[CahRhod-kurz] CahRhod 
 \item[CahTun-kurz] CahTun 
 \item[CalifStClAnt-kurz] CalifStClAnt 
 \item[CambrAJ-kurz] CambrAJ 
 \item[CArch-kurz] CArch 
 \item[CarinthiaI-kurz] Carinthia I %*Abweichung!
 \item[CarnuntumJb-kurz] CarnuntumJb 
 \item[Carpica-kurz] Carpica 
 \item[Carrobbio-kurz] Carrobbio 
 \item[CAT-kurz] CAT 
 \item[CE-kurz] CE 
 \item[CEDAC-kurz] CEDAC 
 \item[CEFR-kurz] CEFR 
 \item[Celticum-kurz] Celticum 
 \item[CercA-kurz] CercA 
 \item[CercNum-kurz] CercNum 
 \item[Chiron-kurz] Chiron 
 \item[ChronEg-kurz] ChronEg 
 \item[CIA-kurz] CIA 
 \item[CIE-kurz] CIE 
 \item[CIG-kurz] CIG 
 \item[CIH-kurz] CIH 
 \item[CIL-kurz] CIL 
 \item[CincArtB-kurz] CincArtB 
 \item[CIS-kurz] CIS 
 \item[CIstAMilano-kurz] CIstAMilano 
 \item[CivClCr-kurz] CivClCr 
 \item[CivPad-kurz] CivPad 
 \item[ClAnt-kurz] ClAnt 
 \item[ClevStHistArt-kurz] ClevStHistArt 
 \item[Clio-kurz] Clio 
 \item[ClIre-kurz] ClIre 
 \item[ClJ-kurz] ClJ 
 \item[ClMediaev-kurz] ClMediaev 
 \item[ClPhil-kurz] ClPhil 
 \item[ClQ-kurz] ClQ 
 \item[ClR-kurz] ClR 
 \item[ClRh-kurz] ClRh 
 \item[CMatAOr-kurz] CMatAOr 
 \item[CMGr-kurz] CMGr 
 \item[CMS-kurz] CMS 
 \item[CoinHoards-kurz] Coin Hoards %*Abweichung!
 \item[ColloquiSod-kurz] ColloquiSod 
 \item[CommunicAHung-kurz] CommunicAHung 
 \item[Complutum-kurz] Complutum 
 \item[Conoscenze-kurz] Conoscenze 
 \item[Corduba-kurz] Corduba 
 \item[Corinth-kurz] Corinth 
 \item[CRAI-kurz] CRAI 
 \item[CretAnt-kurz] CretAnt 
 \item[CretSt-kurz] CretSt 
 \item[CronA-kurz] CronA 
 \item[CronErcol-kurz] CronErcol 
 \item[CronPomp-kurz] CronPomp 
 \item[CRPetersbourg-kurz] CRPetersbourg %*Abweichung!
 \item[CSE-kurz] CSE 
 \item[CSIR-kurz] CSIR 
 \item[CSSpecPisa-kurz] CSSpecPisa 
 \item[CuadAMed-kurz] CuadAMed 
 \item[CuadArquitRom-kurz] CuadArquitRom 
 \item[CuadCastellon-kurz] CuadCastellon 
 \item[CuadCat-kurz] CuadCat 
 \item[CuadFilCl-kurz] CuadFilCl 
 \item[CuadGallegos-kurz] CuadGallegos 
 \item[CuadGranada-kurz] CuadGranada 
 \item[CuadNavarra-kurz] CuadNavarra 
 \item[CuadPrehistA-kurz] CuadPrehistA 
 \item[CuadRom-kurz] CuadRom 
 \item[CuPaUAM-kurz] CuPaUAM 
 \item[CVA-kurz] CVA 
 \item[CZero-kurz] CZero 
 \item[DAA-kurz] DAA 
 \item[Dacia-kurz] Dacia 
 \item[DACL-kurz] DACL 
 \item[Dacoromania-kurz] Dacoromania 
 \item[DaF-kurz] DaF 
 \item[Daidalos-kurz] Daidalos 
 \item[DAIGeschDok-kurz] DAIGeschDok 
 \item[DaM-kurz] DaM 
 \item[Daremberg-Saglio-kurz] Daremberg -- Saglio %*Abweichung!
 \item[DebrecMuzEvk-kurz] DebrecMuzEvk 
 \item[Dedalo-kurz] Dédalo %*Abweichung!
 \item[Delos-kurz] Delos %*Abweichung!
 \item[DeltChrA-kurz] DeltChrA 
 \item[Demircihueyuek-kurz] Demircihüyük %*Abweichung!
 \item[DeMuseus-kurz] DeMuseus 
 \item[DenkmPflBadWuert-kurz] DenkmPflBadWürt %*Abweichung!
 \item[DenkschrWien-kurz] DenkschrWien 
 \item[Diadora-kurz] Diadora 
 \item[DialA-kurz] DialA 
 \item[DialHistAnc-kurz] DialHistAnc 
 \item[Dike-kurz] Dike 
 \item[Dioniso-kurz] Dioniso 
 \item[DiskAB-kurz] DiskAB 
 \item[DKuDenkmPfl-kurz] DKuDenkmPfl 
 \item[DLZ-kurz] DLZ 
 \item[DNP-kurz] DNP 
 \item[DocAlb-kurz] DocAlb 
 \item[DocALouv-kurz] DocALouv 
 \item[DocAMerid-kurz] DocAMerid 
 \item[DocEmRom-kurz] DocEmRom 
 \item[Dodone-kurz] Dodone 
 \item[DOP-kurz] DOP 
 \item[DossAlet-kurz] DossAlet 
 \item[DossAParis-kurz] DossAParis 
 \item[Dura-Europos-kurz] Dura-Europos 
 \item[EAA-kurz] EAA 
 \item[EAE-kurz] EAE 
 \item[EastWest-kurz] EastWest 
 \item[EcAntNimes-kurz] EcAntNimes 
 \item[EchosCl-kurz] EchosCl 
 \item[eDAI-F-kurz] eDAI-F 
 \item[eDAI-J-kurz] eDAI-J 
 \item[EgA-kurz] EgA 
 \item[Egnatia-kurz] Egnatia 
 \item[EgVicOr-kurz] EgVicOr 
 \item[Eikasmos-kurz] Eikasmos 
 \item[Eirene-kurz] Eirene 
 \item[Elenchos-kurz] Elenchos 
 \item[Ellenika-kurz] Ellenika 
 \item[Emerita-kurz] Emerita 
 \item[EmPrerom-kurz] EmPrerom 
 \item[Empuries-kurz] Empúries %*Abweichung!
 \item[Enalia-kurz] Enalia 
 \item[EnaliaAnn-kurz] EnaliaAnn 
 \item[Enchoria-kurz] Enchoria 
 \item[Eos-kurz] Eos 
 \item[EpetBoiotMel-kurz] EpetBoiotMel 
 \item[EpetByzSpud-kurz] EpetByzSpud 
 \item[EpetKyklMel-kurz] EpetKyklMel 
 \item[EphemDac-kurz] EphemDac 
 \item[EphemNapoc-kurz] EphemNapoc 
 \item[EpigrAnat-kurz] EpigrAnat 
 \item[EpistEpetAth-kurz] EpistEpetAth 
 \item[EpistEpetPolytThess-kurz] EpistEpetPolytThess 
 \item[EpistEpetThess-kurz] EpistEpetThess 
 \item[EPRO-kurz] EPRO 
 \item[Eranos-kurz] Eranos 
 \item[EranosJb-kurz] EranosJb 
 \item[Eretria-kurz] Eretria 
 \item[Eretz-Israel-kurz] Eretz-Israel 
 \item[Ergon-kurz] Ergon 
 \item[ESA-kurz] ESA 
 \item[EspacioHist-kurz] EspacioHist 
 \item[EstMadr-kurz] EstMadr 
 \item[EstZaragoza-kurz] EstZaragoza 
 \item[EtACl-kurz] EtACl 
 \item[EtCl-kurz] EtCl 
 \item[EtClAix-kurz] EtClAix 
 \item[EtCret-kurz] EtCret 
 \item[Ethnos-kurz] Ethnos 
 \item[EtP-kurz] EtP 
 \item[EtPezenas-kurz] EtPézenas %*Abweichung!
 \item[EtrSt-kurz] EtrSt 
 \item[Etruscans-kurz] Etruscans 
 \item[EtTrav-kurz] EtTrav 
 \item[Eulimene-kurz] Eulimene 
 \item[Eunomia-kurz] Eunomia 
 \item[Euphrosyne-kurz] Euphrosyne 
 \item[EurAnt-kurz] EurAnt 
 \item[EurRHist-kurz] EurRHist 
 \item[Eutopia-kurz] Eutopia 
 \item[EVP-kurz] EVP 
 \item[ExcIsr-kurz] ExcIsr 
 \item[Expedition-kurz] Expedition 
 \item[ExtremA-kurz] ExtremA 
 \item[FA-kurz] FA 
 \item[FAAK-kurz] FAAK 
 \item[Faenza-kurz] Faenza 
 \item[FAVA-kurz] FAVA 
 \item[Faventia-kurz] Faventia 
 \item[FBerBadWuert-kurz] FBerBadWürt %*Abweichung!
 \item[FdC-kurz] FdC 
 \item[FdD-kurz] FdD 
 \item[FdX-kurz] FdX 
 \item[FeddersenWierde-kurz] Feddersen Wierde %*Abweichung!
 \item[FelRav-kurz] FelRav 
 \item[FGrHist-kurz] FGrHistr 
 \item[FHG-kurz] FHG 
 \item[FiA-kurz] FiA 
 \item[FichEpigr-kurz] FichEpigr 
 \item[FiE-kurz] FiE 
 \item[FIFAO-kurz] FIFAO 
 \item[Figlina-kurz] Figlina 
 \item[Florentia-kurz] Florentia 
 \item[FlorIl-kurz] FlorIl 
 \item[FMRD-kurz] FMRD 
 \item[FMROe-kurz] FMRÖ %*Abweichung!
 \item[FoggArtMusAcqu-kurz] FoggArtMusAcqu 
 \item[FolA-kurz] FolA 
 \item[FolOr-kurz] FolOr 
 \item[Fonaments-kurz] Fonaments 
 \item[Fondamenti-kurz] Fondamenti 
 \item[FontAPos-kurz] FontAPos 
 \item[Fontes-kurz] Fontes 
 \item[Forlimpopoli-kurz] Forlimpopoli 
 \item[Fornvaennen-kurz] Fornvännen %*Abweichung!
 \item[Forum-kurz] Forum 
 \item[FR-kurz] FR 
 \item[FruehMitAltSt-kurz] FrühMitAltSt %*Abweichung!
 \item[FuAusgrTrier-kurz] FuAusgrTrier 
 \item[FuB-kurz] FuB 
 \item[FuBerBadWuert-kurz] FuBerBadWürt %*Abweichung!
 \item[FuBerHessen-kurz] FuBerHessen 
 \item[FuBerOe-kurz] FuBerÖ %*Abweichung!
 \item[FuBerSchwab-kurz] FuBerSchwab 
 \item[FuF-kurz] FuF 
 \item[FuWien-kurz] FuWien 
 \item[GacNum-kurz] GacNum 
 \item[Gades-kurz] Gades 
 \item[Gallaecia-kurz] Gallaecia 
 \item[Gallia-kurz] Gallia 
 \item[GalliaInf-kurz] GalliaInf 
 \item[GalliaInfAReg-kurz] GalliaInfAReg 
 \item[GalliaPrehist-kurz] GalliaPrehist 
 \item[GaR-kurz] GaR 
 \item[GazBA-kurz] GazBA 
 \item[Genava-kurz] Genava 
 \item[GeoAnt-kurz] GeoAnt 
 \item[Germania-kurz] Germania 
 \item[Gesta-kurz] Gesta 
 \item[GettyMusJ-kurz] GettyMusJ 
 \item[GFA-kurz] GFA 
 \item[GGA-kurz] GGA 
 \item[GiornFilFerr-kurz] GiornFilFerr 
 \item[GiornItFil-kurz] GiornItFil 
 \item[GiornStorLun-kurz] GiornStorLun 
 \item[GiRoccPalermo-kurz] GiRoccPalermo 
 \item[Gladius-kurz] Gladius 
 \item[GlasAJ-kurz] GlasAJ 
 \item[GlasBeograd-kurz] GlasBeograd 
 \item[GlasSarajevo-kurz] GlasSarajevo 
 \item[Glotta-kurz] Glotta 
 \item[Gnomon-kurz] Gnomon 
 \item[GodDepA-kurz] GodDepA 
 \item[GodMuzPlov-kurz] GodMuzPlov 
 \item[GodMuzSof-kurz] GodMuzSof 
 \item[GodZborSkopje-kurz] GodZborSkopje 
 \item[GorLet-kurz] GorLet 
 \item[GoettMisz-kurz] GöttMisz %*Abweichung!
 \item[GraRaspr-kurz] GraRaspr 
 \item[GrazBeitr-kurz] GrazBeitr 
 \item[GrLatOr-kurz] GrLatOr 
 \item[GrLatPrag-kurz] GrLatPrag 
 \item[GrRomByzSt-kurz] GrRomByzSt 
 \item[Gymnasium-kurz] Gymnasium 
 \item[Habis-kurz] Habis 
 \item[HallWPr-kurz] HallWPr 
 \item[Hama-kurz] Hama 
 \item[HambBeitrA-kurz] HambBeitrA 
 \item[HambBeitrNum-kurz] HambBeitrNum 
 \item[Handlingar-kurz] Handlingar 
 \item[HarvStClPhil-kurz] HarvStClPhil 
 \item[HarvTheolR-kurz] HarvTheolR 
 \item[HASB-kurz] HASB 
 \item[HAW-kurz] HAW 
 \item[HdArch-kurz] HdArch 
 \item[Head-kurz] Head 
 \item[Helbig-kurz] Helbig 
 \item[Helike-kurz] Helike 
 \item[Helikon-kurz] Helikon 
 \item[Helinium-kurz] Helinium 
 \item[Helios-kurz] Helios 
 \item[HellenikaJb-kurz] HellenikaJb 
 \item[HelvA-kurz] HelvA 
 \item[Hephaistos-kurz] Hephaistos 
 \item[Hermes-kurz] Hermes 
 \item[Herrscherbild-kurz] Herrscherbild 
 \item[Hesperia-kurz] Hesperia 
 \item[Hispania-kurz] Hispania 
 \item[HispAnt-kurz] HispAnt 
 \item[HispAntEpigr-kurz] HispAntEpigra 
 \item[HispEpigr-kurz] HispEpigr 
 \item[HistAnthr-kurz] HistAnthr 
 \item[HistArt-kurz] HistArt 
 \item[Historia-kurz] Historia 
 \item[Historica-kurz] Historica 
 \item[Histria-kurz] Histria 
 \item[HistriaA-kurz] HistriaA 
 \item[HistriaAnt-kurz] HistriaAnt 
 \item[HistSprF-kurz] HistSprF 
 \item[HKL-kurz] HKL 
 \item[Horos-kurz] Horos 
 \item[HSS-kurz] HSS 
 \item[HuelvaA-kurz] HuelvaA 
 \item[HumBild-kurz] HumBild 
 \item[Hyp-kurz] Hyp 
 \item[HZ-kurz] HZ 
 \item[IA-kurz] IA 
 \item[Iberia-kurz] Iberia 
 \item[IEJ-kurz] IEJ 
 \item[IG-kurz] IG 
 \item[IGCH-kurz] IGCH 
 \item[IGR-kurz] IGR 
 \item[IK-kurz] IK 
 \item[Ilerda-kurz] Ilerda 
 \item[Iliria-kurz] Iliria 
 \item[IllinClSt-kurz] IllinClSt 
 \item[ILN-kurz] ILN 
 \item[ILS-kurz] ILS 
 \item[IndexQuad-kurz] IndexQuad 
 \item[IndogermF-kurz] IndogermF 
 \item[IndUnArtB-kurz] IndUnArtB 
 \item[InsFulc-kurz] InsFulc 
 \item[InstNautAQ-kurz] InstNautAQ 
 \item[IntJClTrad-kurz] IntJClTrad 
 \item[IntJNautA-kurz] IntJNautA 
 \item[IntZSchauBibelWiss-kurz] IntZSchauBibelWiss 
 \item[InvLuc-kurz] InvLuc 
 \item[Ipek-kurz] Ipek 
 \item[Iran-kurz] Iran 
 \item[IrAnt-kurz] IrAnt 
 \item[IsrMusJ-kurz] IsrMusJ 
 \item[IsrMusN-kurz] IsrMusN 
 \item[IsrMusStA-kurz] IsrMusStA 
 \item[IsrNumJ-kurz] IsrNumJ 
 \item[IstanbAMuezYil-kurz] IstanbAMüzYıl %*Abweichung!
 \item[IstForsch-kurz] IstForsch 
 \item[Isthmia-kurz] Isthmia 
 \item[IstMitt-kurz] IstMitt 
 \item[Italica-kurz] Italica 
 \item[ItNostr-kurz] ItNostr 
 \item[IzvBurgas-kurz] IzvBurgas 
 \item[IzvMuzJuzBalg-kurz] IzvMuzJužBalg %*Abweichung!
 \item[IzvVarna-kurz] IzvVarna 
 \item[Jabega-kurz] Jábega %*Abweichung!
 \item[JadrZbor-kurz] JadrZbor 
 \item[JAOS-kurz] JAOS 
 \item[JARCE-kurz] JARCE 
 \item[JASc-kurz] JASc 
 \item[JbAC-kurz] JbAC 
 \item[JbAkMainz-kurz] JbAkMainz 
 \item[JbBadWuert-kurz] JbBadWürt %*Abweichung!
 \item[JbBerlMus-kurz] JbBerlMus 
 \item[JbBernHistMus-kurz] JbBernHistMus 
 \item[JberAugst-kurz] JberAugst 
 \item[JberBasel-kurz] JberBasel 
 \item[JberBayDenkmPfl-kurz] JberBayDenkmPfl 
 \item[JberProVindon-kurz] JberProVindon 
 \item[JberVgFrankf-kurz] JberVgFrankf 
 \item[JberZuerich-kurz] JberZürich %*Abweichung!
 \item[JbGoett-kurz] JbGött %*Abweichung!
 \item[JbHambKuSamml-kurz] JbHambKuSamml 
 \item[JbKHMWien-kurz] JbKHMWien 
 \item[JbKHSWien-kurz] JbKHSWien 
 \item[JbKleinasF-kurz] JbKleinasF 
 \item[JbMuench-kurz] JbMünch %*Abweichung!
 \item[JbMusKGHamb-kurz] JbMusKGHamb 
 \item[JbMusLinz-kurz] JbMusLinz 
 \item[JbOeByz-kurz] JbÖByz %*Abweichung!
 \item[JbPreussKul-kurz] JbPreussKul 
 \item[JbRGZM-kurz] JbRGZM 
 \item[JbSchwUrgesch-kurz] JbSchwUrgesch 
 \item[JCS-kurz] JCS 
 \item[JdI-kurz] JdI 
 \item[JEA-kurz] JEA 
 \item[JEChrSt-kurz] JEChrSt 
 \item[JEOL-kurz] JEOL 
 \item[JewelSt-kurz] JewelSt 
 \item[JFieldA-kurz] JFieldA 
 \item[JGS-kurz] JGS 
 \item[JHS-kurz] JHS 
 \item[JIbA-kurz] JIbA 
 \item[JJurP-kurz] JJurP 
 \item[JKuGesch-kurz] JKuGesch 
 \item[JMedA-kurz] JMedA 
 \item[JMedAnthrA-kurz] JMedAnthrA 
 \item[JMithrSt-kurz] JMithrSt 
 \item[JNES-kurz] JNES 
 \item[JNG-kurz] JNG 
 \item[JPrehistRel-kurz] JPrehistRel 
 \item[JRA-kurz] JRA 
 \item[JRomMilSt-kurz] JRomMilSt 
 \item[JRomPotSt-kurz] JRomPotSt 
 \item[JRS-kurz] JRS 
 \item[JSav-kurz] JSav 
 \item[JSchrVgHalle-kurz] JSchrVgHalle 
 \item[JSS-kurz] JSS 
 \item[JTheorA-kurz] JTheorA 
 \item[Jura-kurz] Jura 
 \item[JWaltersArtGal-kurz] JWaltersArtGal 
 \item[JWCI-kurz] JWCI 
 \item[Kadmos-kurz] Kadmos 
 \item[Kairos-kurz] Kairos 
 \item[Kalapodi-kurz] Kalapodi 
 \item[Kalathos-kurz] Kalathos 
 \item[Kalos-kurz] Kalós %*Abweichung!
 \item[Karthago-kurz] Karthago 
 \item[Kemi-kurz] Kêmi %*Abweichung!
 \item[Kenchreai-kurz] Kenchreai 
 \item[KentAR-kurz] KentAR 
 \item[Keos-kurz] Keos 
 \item[Kerameikos-kurz] Kerameikos 
 \item[Kernos-kurz] Kernos 
 \item[Klearchos-kurz] Klearchos 
 \item[Kleos-kurz] Kleos 
 \item[Klio-kurz] Klio 
 \item[Kodai-kurz] Kodai 
 \item[KoelnJb-kurz] KölnJb %*Abweichung!
 \item[KoelnMusB-kurz] KölnMusB %*Abweichung!
 \item[Kokalos-kurz] Kokalos 
 \item[KollAVA-kurz] KollAVA 
 \item[Kratylos-kurz] Kratylos 
 \item[KretChron-kurz] KretChron 
 \item[KSIA-kurz] KSIA 
 \item[KSIAKiev-kurz] KSIAKiev 
 \item[KST-kurz] KST 
 \item[Ktema-kurz] Ktema 
 \item[KuGeschAnz-kurz] KuGeschAnz 
 \item[Kuml-kurz] Kuml 
 \item[Kunstchronik-kurz] Kunstchronik 
 \item[KuOr-kurz] KuOr 
 \item[Kush-kurz] Kush 
 \item[KuWeltBerlMus-kurz] KuWeltBerlMus 
 \item[KypA-kurz] KypA 
 \item[KypSpud-kurz] KypSpud 
 \item[Labeo-kurz] Labeo 
 \item[LAe-kurz] LÄ %*Abweichung!
 \item[Laietania-kurz] Laietania 
 \item[Lampas-kurz] Lampas 
 \item[Lancia-kurz] Lancia 
 \item[LandKunVierJBl-kurz] LandKunVierJBl 
 \item[LangOrAnc-kurz] LangOrAnc 
 \item[Latinitas-kurz] Latinitas 
 \item[Latomus-kurz] Latomus 
 \item[Laverna-kurz] Laverna 
 \item[LCS-kurz] LCS 
 \item[Levant-kurz] Levant 
 \item[Lexis-kurz] Lexis 
 \item[LF-kurz] LF 
 \item[LibSt-kurz] LibSt 
 \item[LibyaAnt-kurz] LibyaAnt 
 \item[LibycaBServAnt-kurz] LibycaBServAnt 
 \item[LibycaTrav-kurz] LibycaTrav 
 \item[Liddell-­Scott-Jones-kurz] Liddell -- ­Scott -- Jones %*Abweichung!
 \item[LIMC-kurz] LIMC 
 \item[Limesforschungen-kurz] Limesforschungen 
 \item[Lindos-kurz] Lindos 
 \item[LingIt-kurz] LingIt 
 \item[LTUR-kurz] LTUR 
 \item[Lucentum-kurz] Lucentum 
 \item[LundAR-kurz] LundAR 
 \item[Lustrum-kurz] Lustrum 
 \item[Lykia-kurz] Lykia 
 \item[MacActaA-kurz] MacActaA 
 \item[Maecenas-kurz] Maecenas 
 \item[MAGesGraz-kurz] MAGesGraz 
 \item[MAGesStei-kurz] MAGesStei 
 \item[Maia-kurz] Maia 
 \item[Mainake-kurz] Mainake 
 \item[MAInstUngAk-kurz] MAInstUngAk 
 \item[MainzZ-kurz] MainzZ 
 \item[MakedNasl-kurz] MakedNasl 
 \item[Makedonika-kurz] Makedonika 
 \item[MAMA-kurz] MAMA 
 \item[MAnthrWien-kurz] MAnthrWien 
 \item[MAR-kurz] MAR 
 \item[MarbWPr-kurz] MarbWPr 
 \item[Marche-kurz] Marche 
 \item[Mari-kurz] Mari 
 \item[Marisia-kurz] Marisia 
 \item[MarNero-kurz] MarNero 
 \item[Marsyas-kurz] Marsyas 
 \item[MascaJ-kurz] MascaJ 
 \item[MascaP-kurz] MascaP 
 \item[Mastia-kurz] Mastia 
 \item[MatABSSR-kurz] MatABSSR 
 \item[MatASevPri-kurz] MatASevPri 
 \item[MatCercA-kurz] MatCercA 
 \item[MatIsslA-kurz] MatIsslA 
 \item[MatStar-kurz] MatStar 
 \item[MatStarWczes-kurz] MatStarWczes 
 \item[MatTestiCl-kurz] MatTestiCl 
 \item[MatWczes-kurz] MatWczes 
 \item[MAVA-kurz] MAVA 
 \item[MB-kurz] MB 
 \item[MBAH-kurz] MBAH 
 \item[MBlVFruehGesch-kurz] MBlVFrühGesch %*Abweichung!
 \item[MDAIK-kurz] MDAIK 
 \item[MDAVerb-kurz] MDAVerb 
 \item[MdI-kurz] MdI 
 \item[MDOG-kurz] MDOG 
 \item[Meander-kurz] Meander 
 \item[MedA-kurz] MedA 
 \item[MedAnt-kurz] MedAnt 
 \item[MeddelGlypt-kurz] MeddelGlypt 
 \item[MeddelLund-kurz] MeddelLund 
 \item[MeddelThor-kurz] MeddelThor 
 \item[MededRom-kurz] MededRom 
 \item[MedelhavsMusB-kurz] MedelhavsMusB 
 \item[MedHistR-kurz] MedHistR 
 \item[MedievA-kurz] MedievA 
 \item[MediSec-kurz] MediSec 
 \item[MEFRA-kurz] MEFRA 
 \item[MelBeyrouth-kurz] MelBeyrouth 
 \item[MelCasaVelazquez-kurz] MelCasaVelazquez 
 \item[MemAcInscr-kurz] MemAcInscr 
 \item[MemAmAc-kurz] MemAmAc 
 \item[MemAnt-kurz] MemAnt 
 \item[MemAntFr-kurz] MemAntFr 
 \item[MemBarcelA-kurz] MemBarcelA 
 \item[MemBologna-kurz] MemBologna 
 \item[MemHistAnt-kurz] MemHistAnt 
 \item[MemInstNatFr-kurz] MemInstNatFr 
 \item[MemLinc-kurz] MemLinc 
 \item[MemNap-kurz] MemNap 
 \item[MemPontAc-kurz] MemPontAc 
 \item[MemStor-kurz] MemStor 
 \item[MemStorFriuli-kurz] MemStorFriuli 
 \item[Merida-kurz] Mérida %*Abweichung!
 \item[MeridaMem-kurz] MéridaMem %*Abweichung!
 \item[Meroitica-kurz] Meroitica 
 \item[Mesopotamia-kurz] Mesopotamia 
 \item[Messana-kurz] Messana 
 \item[Metis-kurz] Métis %*Abweichung!
 \item[MetrMusJ-kurz] MetrMusJ 
 \item[MetrMusSt-kurz] MetrMusSt 
 \item[MF-kurz] MF 
 \item[MFruehChrOe-kurz] MFrühChrÖ %*Abweichung!
 \item[Milet-kurz] Milet 
 \item[MilForsch-kurz] MilForsch 
 \item[MinEpigrP-kurz] MinEpigrP 
 \item[Minerva-kurz] Minerva 
 \item[Minos-kurz] Minos 
 \item[MInstWasser-kurz] MInstWasser 
 \item[MIO-kurz] MIO 
 \item[MiscCrAnt-kurz] MiscCrAnt 
 \item[MiscStStor-kurz] MiscStStor 
 \item[MitChrA-kurz] MitChrA 
 \item[MKT-kurz] MKT 
 \item[MKuHistFlorenz-kurz] MKuHistFlorenz 
 \item[MKul-kurz] MKul 
 \item[MM-kurz] MM 
 \item[Mnemosyne-kurz] Mnemosyne 
 \item[MOeNumGes-kurz] MÖNumGes %*Abweichung!
 \item[MonAnt-kurz] MonAnt 
 \item[MonInst-kurz] MonInst 
 \item[MonPiot-kurz] MonPiot 
 \item[MonPitt-kurz] MonPitt 
 \item[Mozia-kurz] Mozia 
 \item[MPraehistKomWien-kurz] MPraehistKomWien %*Abweichung!
 \item[MSAtene-kurz] MSAtene 
 \item[MSchliemann-kurz] MSchliemann 
 \item[MSchwUrFruehGesch-kurz] MSchwUrFrühGesch %*Abweichung!
 \item[MSpaetAByz-kurz] MSpätAByz %*Abweichung!
 \item[MueJb-kurz] MüJb %*Abweichung!
 \item[MuenchBeitrVFG-kurz] MünchBeitrVFG %*Abweichung!
 \item[MuenchStSprWiss-kurz] MünchStSprWiss %*Abweichung!
 \item[MuM-kurz] MuM 
 \item[MusAfr-kurz] MusAfr 
 \item[MusBenaki-kurz] MusBenaki 
 \item[MusCrit-kurz] MusCrit 
 \item[Muse-kurz] Muse 
 \item[Museon-kurz] Muséon %*Abweichung!
 \item[MuseumUnesco-kurz] MuseumUnesco 
 \item[MusFerr-kurz] MusFerr 
 \item[MusGalIt-kurz] MusGalIt 
 \item[MusHaaretz-kurz] MusHaaretz 
 \item[MusHelv-kurz] MusHelv 
 \item[MusKoeln-kurz] MusKöln %*Abweichung!
 \item[MusNotAmNumSoc-kurz] MusNotAmNumSoc 
 \item[MusPontevedra-kurz] MusPontevedra 
 \item[MusRiv-kurz] MusRiv 
 \item[MusTusc-kurz] MusTusc 
 \item[MuzEvkSzeged-kurz] MuzEvkSzeged 
 \item[MuzNa-kurz] MuzNa 
 \item[MuzPamKul-kurz] MuzPamKul 
 \item[NachrArbUWA-kurz] NachrArbUWA 
 \item[NapNobil-kurz] NapNobil 
 \item[NassAnn-kurz] NassAnn 
 \item[NAWG-kurz] NAWG 
 \item[NBWorcArtMus-kurz] NBWorcArtMus 
 \item[NEphemSemEpigr-kurz] NEphemSemEpigr 
 \item[NewsletterAthen-kurz] NewsletterAthen 
 \item[NewsletterPotTech-kurz] NewsletterPotTech 
 \item[NGWG-kurz] NGWG 
 \item[NigCl-kurz] NigCl 
 \item[Nikephoros-kurz] Nikephoros 
 \item[Nin-kurz] Nin 
 \item[NNM-kurz] NNM 
 \item[NomChron-kurz] NomChron 
 \item[Norba-kurz] Norba 
 \item[NordNumArs-kurz] NordNumArs 
 \item[NotABerg-kurz] NotABerg 
 \item[NotAHisp-kurz] NotAHisp 
 \item[NotAHispPrehistoria-kurz] NotAHispPrehistoria 
 \item[NotAllumiere-kurz] NotAllumiere 
 \item[NotALomb-kurz] NotALomb 
 \item[NotMilano-kurz] NotMilano 
 \item[NouvClio-kurz] NouvClio 
 \item[Novaensia-kurz] Novaensia 
 \item[NSc-kurz] NSc 
 \item[NStFan-kurz] NStFan 
 \item[NubChr-kurz] NubChr 
 \item[NubLet-kurz] NubLet 
 \item[NueBlA-kurz] NüBlA %*Abweichung!
 \item[NumAntCl-kurz] NumAntCl 
 \item[Numantia-kurz] Numantia 
 \item[NumChron-kurz] NumChron 
 \item[Numen-kurz] Numen 
 \item[NumEpigr-kurz] NumEpigr 
 \item[Numisma-kurz] Numisma 
 \item[NumismaticaRom-kurz] NumismaticaRom 
 \item[Numizmaticar-kurz] Numizmatičar %*Abweichung!
 \item[Nummus-kurz] Nummus 
 \item[NumZ-kurz] NumZ 
 \item[NuovDidask-kurz] NuovDidask 
 \item[OccasPublClSt-kurz] OccasPublClSt 
 \item[OccOr-kurz] OccOr 
 \item[OGIS-kurz] OGIS 
 \item[OeJh-kurz] ÖJh %*Abweichung!
 \item[OF-kurz] OF 
 \item[Offa-kurz] Offa 
 \item[Ogam-kurz] Ogam 
 \item[Oikumene-kurz] Oikumene 
 \item[OIP-kurz] OIP 
 \item[Olba-kurz] Olba 
 \item[OlBer-kurz] OlBer 
 \item[Olympia-kurz] Olympia 
 \item[Olynthus-kurz] Olynthus 
 \item[OLZ-kurz] OLZ 
 \item[OpArch-kurz] OpArch 
 \item[OpAth-kurz] OpAth 
 \item[OpFin-kurz] OpFin 
 \item[OpPomp-kurz] OpPomp 
 \item[OpRom-kurz] OpRom 
 \item[Opus-kurz] Opus 
 \item[Or-kurz] Or 
 \item[OrA-kurz] OrA 
 \item[OrAnt-kurz] OrAnt 
 \item[OrbTerr-kurz] OrbTerr 
 \item[OrChr-kurz] OrChr 
 \item[OrChrPer-kurz] OrChrPer 
 \item[Ordona-kurz] Ordona 
 \item[Orient-kurz] Orient 
 \item[Origini-kurz] Origini 
 \item[Orizzonti-kurz] Orizzonti 
 \item[Orpheus-kurz] Orpheus 
 \item[OrpheusThracSt-kurz] OrpheusThracSt 
 \item[OrSu-kurz] OrSu 
 \item[OsjZbor-kurz] OsjZbor 
 \item[OstbGrenzm-kurz] OstbGrenzm 
 \item[Ostraka-kurz] Ostraka 
 \item[OudhMeded-kurz] OudhMeded 
 \item[OxfJA-kurz] OxfJA 
 \item[OxfStPhilos-kurz] OxfStPhilos 
 \item[Pact-kurz] Pact 
 \item[Padusa-kurz] Padusa 
 \item[PagA-kurz] PagA 
 \item[PAI-kurz] PAI 
 \item[Paideuma-kurz] Paideuma 
 \item[Palaeohistoria-kurz] Palaeohistoria 
 \item[Paleohispanica-kurz] Paleohispánica %*Abweichung!
 \item[Palladio-kurz] Palladio 
 \item[Pallas-kurz] Pallas 
 \item[Palmet-kurz] Palmet 
 \item[PamA-kurz] PamA 
 \item[Pan-kurz] Pan 
 \item[PapBilb-kurz] PapBilb 
 \item[Papyri-kurz] Papyri 
 \item[Parthica-kurz] Parthica 
 \item[Partenope-kurz] Partenope 
 \item[PAS-kurz] PAS 
 \item[PBF-kurz] PBF 
 \item[Pegasus-kurz] Pegasus 
 \item[PEQ-kurz] PEQ 
 \item[Peristil-kurz] Peristil 
 \item[Persica-kurz] Persica 
 \item[Peuce-kurz] Peuce 
 \item[PF-kurz] PF 
 \item[Pharos-kurz] Pharos 
 \item[Philologus-kurz] Philologus 
 \item[Phoenix-kurz] Phoenix 
 \item[PhoenixExOrLux-kurz] PhoenixExOrLux 
 \item[Phoibos-kurz] Phoibos 
 \item[Phronesis-kurz] Phronesis 
 \item[Picus-kurz] Picus 
 \item[PIR-kurz] PIR 
 \item[PKG-kurz] PKG 
 \item[Platon-kurz] Platon 
 \item[PLup-kurz] PLup 
 \item[PolAMed-kurz] PolAMed 
 \item[Polemon-kurz] Polemon 
 \item[Polis-kurz] Polis 
 \item[PompHercStab-kurz] PompHercStab 
 \item[Pontica-kurz] Pontica 
 \item[Portugalia-kurz] Portugalia 
 \item[PP-kurz] PP 
 \item[PPM-kurz] PPM 
 \item[PraceA-kurz] PraceA 
 \item[PraceMatLodz-kurz] PraceMatŁodz %*Abweichung!
 \item[Prakt-kurz] Prakt 
 \item[PraktAkAth-kurz] PraktAkAth 
 \item[PreistAlp-kurz] PreistAlp 
 \item[PriloziZagreb-kurz] PriloziZagreb 
 \item[PrincViana-kurz] PrincViana 
 \item[ProblIsk-kurz] ProblIsk 
 \item[ProcAfrClAss-kurz] ProcAfrClAss 
 \item[ProcCambrPhilSoc-kurz] ProcCambrPhilSoc 
 \item[ProcDanInstAth-kurz] ProcDanInstAth 
 \item[ProcPrehistSoc-kurz] ProcPrehistSoc 
 \item[Prometheus-kurz] Prometheus 
 \item[ProspAQuad-kurz] ProspAQuad 
 \item[Prospettiva-kurz] Prospettiva 
 \item[Prospezioni-kurz] Prospezioni 
 \item[ProvHist-kurz] ProvHist 
 \item[ProvLucca-kurz] ProvLucca 
 \item[PublInstTTMeneses-kurz] PublInstTTMeneses 
 \item[Pulpudeva-kurz] Pulpudeva 
 \item[Puteoli-kurz] Puteoli 
 \item[Pyrenae-kurz] Pyrenae 
 \item[PZ-kurz] PZ 
 \item[QDAP-kurz] QDAP 
 \item[QuadABarcel-kurz] QuadABarcel 
 \item[QuadACagl-kurz] QuadACagl 
 \item[QuadACal-kurz] QuadACal 
 \item[QuadALibya-kurz] QuadALibya 
 \item[QuadAMant-kurz] QuadAMant 
 \item[QuadAMess-kurz] QuadAMess 
 \item[QuadAOst-kurz] QuadAOst 
 \item[QuadAPiem-kurz] QuadAPiem 
 \item[QuadAquil-kurz] QuadAquil 
 \item[QuadAReggio-kurz] QuadAReggio 
 \item[QuadAVen-kurz] QuadAVen 
 \item[QuadCast-kurz] QuadCast 
 \item[QuadCat-kurz] QuadCat 
 \item[QuadChieti-kurz] QuadChieti 
 \item[QuadErb-kurz] QuadErb 
 \item[QuadFriulA-kurz] QuadFriulA 
 \item[QuadGerico-kurz] QuadGerico 
 \item[QuadIstFilGr-kurz] QuadIstFilGr 
 \item[QuadIstLat-kurz] QuadIstLat 
 \item[QuadLecce-kurz] QuadLecce 
 \item[QuadMusPontecorvo-kurz] QuadMusPontecorvo 
 \item[QuadMusSalinas-kurz] QuadMusSalinas 
 \item[QuadProtost-kurz] QuadProtost 
 \item[QuadStLun-kurz] QuadStLun 
 \item[QuadStor-kurz] QuadStor 
 \item[QuadStorici-kurz] QuadStorici 
 \item[QuadUrbin-kurz] QuadUrbin 
 \item[Quaternaria-kurz] Quaternaria 
 \item[RA-kurz] RA 
 \item[RAArtLouv-kurz] RAArtLouv 
 \item[RAC-kurz] RAC 
 \item[RACFr-kurz] RACFr 
 \item[RAComo-kurz] RAComo 
 \item[RACr-kurz] RACr 
 \item[RadAkZadar-kurz] RadAkZadar 
 \item[Radiocarbon-kurz] Radiocarbon 
 \item[RadSplit-kurz] RadSplit 
 \item[RAE-kurz] RAE 
 \item[Raggi-kurz] Raggi 
 \item[RAMadrid-kurz] RAMadrid 
 \item[RANarb-kurz] RANarb 
 \item[RAPon-kurz] RAPon 
 \item[RapWyk-kurz] RapWyk 
 \item[RArchBiblMus-kurz] RArchBiblMus 
 \item[RArcheom-kurz] RArcheom 
 \item[RArtMus-kurz] RArtMus 
 \item[RassAPiomb-kurz] RassAPiomb 
 \item[RassLazio-kurz] RassLazio 
 \item[RassStorSalern-kurz] RassStorSalern 
 \item[RassVolt-kurz] RassVolt 
 \item[RAssyr-kurz] RAssyr 
 \item[Ratiariensia-kurz] Ratiariensia 
 \item[RAtlMed-kurz] RAtlMed 
 \item[RavStRic-kurz] RavStRic 
 \item[Raydan-kurz] Raydan 
 \item[RB-kurz] RB 
 \item[RBelgNum-kurz] RBelgNum 
 \item[RBelgPhilHist-kurz] RBelgPhilHist 
 \item[RBK-kurz] RBK 
 \item[RCulClMedioev-kurz] RCulClMedioev 
 \item[RdA-kurz] RdA 
 \item[RDAC-kurz] RDAC 
 \item[RdE-kurz] RdE 
 \item[RDroitsAnt-kurz] RDroitsAnt 
 \item[RE-kurz] RE 
 \item[REA-kurz] REA 
 \item[REByz-kurz] REByz 
 \item[RecConstantine-kurz] RecConstantine 
 \item[RechACrac-kurz] RechACrac 
 \item[RechAlb-kurz] RechAlb 
 \item[RecMusAlcoi-kurz] RecMusAlcoi 
 \item[RecTrav-kurz] RecTrav 
 \item[REG-kurz] REG 
 \item[ReiCretActa-kurz] ReiCretActa 
 \item[ReiCretCommunic-kurz] ReiCretCommunic 
 \item[REL-kurz] REL 
 \item[Rema-kurz] Rema 
 \item[RendBologna-kurz] 	RendBologna 
 \item[RendIstLomb-kurz] RendIstLomb 
 \item[RendLinc-kurz] RendLinc 
 \item[RendNap-kurz] RendNap 
 \item[RendPontAc-kurz] RendPontAc 
 \item[RepMalta-kurz] RepMalta 
 \item[Reppal-kurz] Reppal 
 \item[RepSocLibSt-kurz] RepSocLibSt 
 \item[RES-kurz] RES 
 \item[REstIber-kurz] REstIber 
 \item[REtArm-kurz] REtArm 
 \item[RevEg-kurz] RevEg 
 \item[RFil-kurz] RFil 
 \item[RGeorgCauc-kurz] RGeorgCauc 
 \item[RGF-kurz] RGF 
 \item[RGuimar-kurz] RGuimar 
 \item[RHA-kurz] RHA 
 \item[RheinMusBonn-kurz] RheinMusBonn 
 \item[RHistArmees-kurz] RHistArmees %*Abweichung!
 \item[RHistRel-kurz] RHistRel 
 \item[RhM-kurz] RhM 
 \item[RIA-kurz] RIA 
 \item[RIC-kurz] RIC 
 \item[RicEgAntCopt-kurz] RicEgAntCopt 
 \item[RicognA-kurz] RicognA 
 \item[RicStBrindisi-kurz] RicStBrindisi 
 \item[RIngIntem-kurz] RIngIntem 
 \item[RItNum-kurz] RItNum 
 \item[RlA-kurz] RlA 
 \item[RLouvre-kurz] RLouvre 
 \item[RM-kurz] RM 
 \item[RNum-kurz] RNum 
 \item[RoczMuzWarsz-kurz] RoczMuzWarsz 
 \item[RoemOe-kurz] RömÖ %*Abweichung!
 \item[RoemQSchr-kurz] RömQSchr %*Abweichung!
 \item[Romanobarbarica-kurz] Romanobarbarica 
 \item[RomGens-kurz] RomGens 
 \item[RoscherML-kurz] Roscher, ML %*Abweichung!
 \item[RossA-kurz] RossA 
 \item[RPC-kurz] RPC 
 \item[RPhil-kurz] RPhil 
 \item[RPortA-kurz] RPortA 
 \item[RPorto-kurz] RPorto 
 \item[RRC-kurz] RRC 
 \item[RSaintonge-kurz] RSaintonge 
 \item[RScPreist-kurz] RScPreist 
 \item[RSO-kurz] RSO 
 \item[RStBiz-kurz] RStBiz 
 \item[RStCl-kurz] RStCl 
 \item[RStFen-kurz] RStFen 
 \item[RStLig-kurz] RStLig 
 \item[RStMarch-kurz] RStMarch 
 \item[RStorAnt-kurz] RStorAnt 
 \item[RStorCal-kurz] RStorCal 
 \item[RStPomp-kurz] RStPomp 
 \item[RStPun-kurz] RStPun 
 \item[RTopAnt-kurz] RTopAnt 
 \item[Rudiae-kurz] Rudiae 
 \item[SaalbJb-kurz] SaalbJb 
 \item[SaarBeitr-kurz] SaarBeitr 
 \item[SaarStMat-kurz] SaarStMat 
 \item[Sacer-kurz] Sacer 
 \item[Saeculum-kurz] Saeculum 
 \item[SAGA-kurz] SAGA 
 \item[SaggiFen-kurz] SaggiFen 
 \item[Saguntum-kurz] Saguntum 
 \item[Saitabi-kurz] Saitabi 
 \item[SAK-kurz] SAK 
 \item[Salduie-kurz] Salduie 
 \item[Samothrace-kurz] Samothrace 
 \item[Sandalion-kurz] Sandalion 
 \item[Sardis-kurz] Sardis 
 \item[Sargetia-kurz] Sargetia 
 \item[SarkSt-kurz] SarkSt 
 \item[Sautuola-kurz] Sautuola 
 \item[Savaria-kurz] Savaria 
 \item[SBBerlin-kurz] SBBerlin 
 \item[SBLeipzig-kurz] SBLeipzig 
 \item[SBMuenchen-kurz] SBMünchen %*Abweichung!
 \item[SborBrno-kurz] SborBrno 
 \item[SBWien-kurz] SBWien 
 \item[ScAnt-kurz] ScAnt 
 \item[SCE-kurz] SCE 
 \item[SchildStei-kurz] SchildStei 
 \item[Scholia-kurz] Scholia 
 \item[SchwMueBl-kurz] SchwMüBl %*Abweichung!
 \item[SchwNumRu-kurz] SchwNumRu 
 \item[ScrCiv-kurz] ScrCiv 
 \item[ScrClIsr-kurz] ScrClIsr 
 \item[ScrHieros-kurz] ScrHieros 
 \item[ScrMed-kurz] ScrMed 
 \item[SDAIK-kurz] SDAIK 
 \item[SEG-kurz] SEG 
 \item[SeminRom-kurz] SeminRom 
 \item[Semitica-kurz] Semitica 
 \item[SetubalA-kurz] SetubalA 
 \item[Sibrium-kurz] Sibrium 
 \item[SicA-kurz] SicA 
 \item[SicGymn-kurz] SicGymn 
 \item[SIG-kurz] SIG 
 \item[Sileno-kurz] Sileno 
 \item[SilkRoadArtA-kurz] SilkRoadArtA 
 \item[SIMA-kurz] SIMA 
 \item[Simblos-kurz] Simblos 
 \item[Skyllis-kurz] Skyllis 
 \item[SlovA-kurz] SlovA 
 \item[SlovNum-kurz] SlovNum 
 \item[SMEA-kurz] SMEA 
 \item[SNG-kurz] SNG 
 \item[SocGeoAOran-kurz] SocGeoAOran 
 \item[SoobErmit-kurz] SoobErmit 
 \item[SoobMuzMoskva-kurz] SoobMuzMoskva 
 \item[SovA-kurz] SovA 
 \item[Spal-kurz] Spal 
 \item[SpNov-kurz] SpNov 
 \item[Spoletium-kurz] Spoletium 
 \item[StA-kurz] StA 
 \item[Stadion-kurz] Stadion 
 \item[StaedelJb-kurz] StädelJb %*Abweichung!
 \item[StAeg-kurz] StAeg 
 \item[StAlb-kurz] StAlb 
 \item[StAnt-kurz] StAnt 
 \item[Starinar-kurz] Starinar 
 \item[StAWarsz-kurz] StAWarsz 
 \item[StBiFranc-kurz] StBiFranc 
 \item[StBitont-kurz] StBitont 
 \item[StBoT-kurz] StBoT 
 \item[StCercIstorV-kurz] StCercIstorV 
 \item[StCercNum-kurz] StCercNum 
 \item[StCl-kurz] StCl 
 \item[StClOr-kurz] StClOr 
 \item[StDocA-kurz] StDocA 
 \item[StDocHistIur-kurz] StDocHistIur 
 \item[StEbla-kurz] StEbla 
 \item[StEgAntPun-kurz] StEgAntPun 
 \item[SteMat-kurz] SteMat 
 \item[StEpigrLing-kurz] StEpigrLing 
 \item[StEtr-kurz] StEtr 
 \item[StGenu-kurz] StGenu 
 \item[StHist-kurz] StHist 
 \item[StiftHambKuSamml-kurz] StiftHambKuSamml 
 \item[StItFilCl-kurz] StItFilCl 
 \item[StLatIt-kurz] StLatIt 
 \item[StMagreb-kurz] StMagreb 
 \item[StMatStorRel-kurz] StMatStorRel 
 \item[StOliv-kurz] StOliv 
 \item[StOr-kurz] StOr 
 \item[StOrCr-kurz] StOrCr 
 \item[StP-kurz] StP 
 \item[StrennaRom-kurz] StrennaRom 
 \item[StRom-kurz] StRom 
 \item[StRomagn-kurz] StRomagn 
 \item[StSalent-kurz] StSalent 
 \item[StSard-kurz] StSard 
 \item[StStorRel-kurz] StStorRel 
 \item[StTardoant-kurz] StTardoant 
 \item[StTrentStor-kurz] StTrentStor 
 \item[StTroica-kurz] StTroica 
 \item[StUrbin-kurz] StUrbin 
 \item[Sumer-kurz] Sumer 
 \item[SylvaMala-kurz] Sylva Mala %*Abweichung!
 \item[SymbOslo-kurz] SymbOslo 
 \item[Syria-kurz] Syria 
 \item[SyrMesopSt-kurz] SyrMesopSt 
 \item[TAD-kurz] TAD 
 \item[Talanta-kurz] Talanta 
 \item[TAM-kurz] TAM 
 \item[Taras-kurz] Taras 
 \item[Tarsus-kurz] ETarsus 
 \item[TAVO-kurz] TAVO 
 \item[TeherF-kurz] TeherF 
 \item[Teiresias-kurz] Teiresias 
 \item[TelAvivJA-kurz] TelAvivJA 
 \item[TerraAntBalc-kurz] TerraAntBalc 
 \item[TerraVolsci-kurz] TerraVolsci 
 \item[Teruel-kurz] Teruel 
 \item[TextilAnc-kurz] TextilAnc 
 \item[TheolRu-kurz] TheolRu 
 \item[ThesCRA-kurz] ThesCRA 
 \item[Thessalika-kurz] Thessalika 
 \item[Thessalonike-kurz] Thessalonike 
 \item[Thieme-Becker-kurz] Thieme -- Becker %*Abweichung!
 \item[ThrakChron-kurz] ThrakChron 
 \item[ThrakEp-kurz] ThrakEp 
 \item[TIB-kurz] TIB 
 \item[Tibiscus-kurz] Tibiscus 
 \item[TiLeon-kurz] TiLeon 
 \item[Tiryns-kurz] Tiryns 
 \item[TMA-kurz] TMA 
 \item[Topoi-kurz] Topoi 
 \item[Torretta-kurz] Torretta 
 \item[TourOrleOr-kurz] TourOrleOr 
 \item[TrabAntrEtn-kurz] TrabAntrEtn 
 \item[TrabArq-kurz] TrabArq 
 \item[TrabAssArqPort-kurz] TrabAssArqPort 
 \item[TrabNavarra-kurz] TrabNavarra 
 \item[TrabPrehist-kurz] TrabPrehist 
 \item[Traditio-kurz] Traditio 
 \item[TransactAmPhilAss-kurz] TransactAmPhilAss 
 \item[TransactAmPhilosSoc-kurz] TransactAmPhilosSoc 
 \item[TransactLond-kurz] TransactLond 
 \item[TravMem-kurz] TravMem 
 \item[TravToulouse-kurz] TravToulouse 
 \item[TreMonet-kurz] TreMonet 
 \item[TribArq-kurz] TribArq 
 \item[TrudyErmit-kurz] Trudy 
 \item[TrWPr-kurz] TrWPr 
 \item[TrZ-kurz] TrZ 
 \item[TTKY-kurz] TTKY 
 \item[TueBA-Ar-kurz] TüBA-Ar %*Abweichung!
 \item[Tyche-kurz] Tyche 
 \item[UF-kurz] UF 
 \item[UPA-kurz] UPA 
 \item[LUrbe-kurz] L’Urbe %*Abweichung!
 \item[UrSchw-kurz] UrSchw 
 \item[UVB-kurz] UVB 
 \item[VarSpom-kurz] VarSpom 
 \item[VDI-kurz] VDI 
 \item[Vekove-kurz] Vekove 
 \item[Veleia-kurz] Veleia 
 \item[VenArt-kurz] VenArt 
 \item[VerAmstMeded-kurz] VerAmstMeded 
 \item[Verbanus-kurz] Verbanus 
 \item[VeteraChr-kurz] VeteraChr 
 \item[VGesVind-kurz] VGesVind 
 \item[Vichiana-kurz] Vichiana 
 \item[VicOr-kurz] VicOr 
 \item[VigChr-kurz] VigChr 
 \item[Viminacium-kurz] Viminacium 
 \item[VisRel-kurz] VisRel 
 \item[Vitudurum-kurz] Vitudurum 
 \item[VivScyl-kurz] VivScyl 
 \item[VizVrem-kurz] VizVrem 
 \item[VjesAMuzZagreb-kurz] VjesAMuzZagreb 
 \item[VjesDal-kurz] VjesDal 
 \item[Wad-al-Hayara-kurz] Wad-al-Hayara 
 \item[WeltGesch-kurz] WeltGesch 
 \item[WiadA-kurz] WiadA 
 \item[WissMBosn-kurz] WissMBosn 
 \item[WissZBerl-kurz] WissZBerl 
 \item[WissZHalle-kurz] WissZHalle 
 \item[WissZJena-kurz] WissZJena 
 \item[WissZRostock-kurz] WissZRostock 
 \item[WO-kurz] WO 
 \item[WorldA-kurz] WorldA 
 \item[WSt-kurz] WSt 
 \item[WuerzbJb-kurz] WürzbJb %*Abweichung!
 \item[WVDOG-kurz] WVDOG 
 \item[WZKM-kurz] WZKM 
 \item[Xenia-kurz] Xenia 
 \item[XeniaAnt-kurz] XeniaAnt 
 \item[XeniaKonst-kurz] XeniaKonst 
 \item[YaleClSt-kurz] YaleClSt 
 \item[YaleUnivB-kurz] YaleUnivB 
 \item[ZA-kurz] ZA 
 \item[ZAAK-kurz] ZAAK 
 \item[ZAeS-kurz] ZÄS %*Abweichung!
 \item[ZAKSSchriften-kurz] ZAKSSchriften 
 \item[ZAntChr-kurz] ZAntChr 
 \item[ZAW-kurz] ZAW 
 \item[ZborMuzBeograd-kurz] ZborMuzBeograd 
 \item[ZborRadBeograd-kurz] ZborRadBeograd 
 \item[ZborZadar-kurz] ZborZadar 
 \item[ZDMG-kurz] ZDMG 
 \item[ZDPV-kurz] ZDPV 
 \item[Zephyrus-kurz] Zephyrus 
 \item[ZEthn-kurz] ZEthn 
 \item[ZfA-kurz] ZfA 
 \item[ZfNum-kurz] ZfNum 
 \item[ZivaAnt-kurz] ZivaAnt 
 \item[ZKuGesch-kurz] ZKuGesch 
 \item[ZNW-kurz] ZNW 
 \item[ZPE-kurz] ZPE 
 \item[ZSav-kurz] ZSav 
 \item[ZSchwA-kurz] ZSchwA 
 \item[ZVerglSprF-kurz] ZVerglSprF
\end{description}
\end{footnotesize}
\end{multicols}


\subsection{Langformen\label{liste-lang}}
Die Ausgabe erfolgt über die Option |noabbrevs|.
%\begin{multicols}{1}
\begin{footnotesize}
\begin{description}[%
			%	style=multiline,
				style=nextline,
				leftmargin=3cm,
				font=\normalfont]
\item[AA-long] Archäologischer Anzeiger 
\item[AAA-long] Αρχαιολογικά Ανάλεκτα εξ Αθηνών 
\item[AAcque-long] Archeologia delle acque. Rivista semestrale di antropologia, archeologia, etnografia, storia dell'acqua 
\item[AAdv-long] The Archaeological Advertiser 
\item[AAJ-long] Annual of the Department of Antiquities of Jordan 
\item[AAlpi-long] Archeologia delle Alpi 
\item[AarbKob-long] Aarbøger for nordisk oldkyndighed og historie %*Abweichung!
\item[AArchit-long] Archeologia dell'architettura 
\item[AAS-long] Les annales archéologiques arabes syriennes 
\item[AASOR-long] The Annual of the American Schools of Oriental Research 
\item[AAusgrBadWuert-long] Archäologische Ausgrabungen in Baden-Württemberg %*Abweichung!
\item[AAustr-long] Archaeologia Austriaca 
\item[ABADY-long] Archäologische Berichte aus dem Yemen 
\item[AbhBerlin-long] Abhandlungen der Deutschen Akademie der Wissenschaften zu Berlin 
\item[AbhDuesseldorf-long] Abhandlungen der Rheinisch-Westfälischen Akademie der Wissenschaften %*Abweichung!
\item[AbhGoettingen-long] Abhandlungen der Akademie der Wissenschaften zu Göttingen. Philologisch-Historische Klasse %*Abweichung!
\item[AbhLeipzig-long] Abhandlungen der Sächsischen Akademie der Wissenschaften zu Leipzig. Philologisch-Historische Klasse 
\item[AbhMainz-long] Akademie der Wissenschaften und der Literatur in Mainz. Abhandlungen der Geistes- und Sozialwissenschaftlichen Klasse 
\item[AbhMuenchen-long] Bayerische Akademie der Wissenschaften. Philosophisch-Historische Klasse. Abhandlungen %*Abweichung!
\item[ABret-long] Archéologie en Bretagne. Bulletin d'information 
\item[Abr-Nahrain-long] Abr-Nahrain. An Annual Published by the Department of Middle Eastern Studies, University of Melbourne 
\item[ABulg-long] Archaeologia Bulgarica 
\item[ABV-long] J. D. Beazley, Attic Black-figure Vase-painters (Oxford 1956) 
\item[ACalc-long] Archeologia e calcolatori 
\item[ACamp-long] Archeologia in Campania. Bolletino di informazioni a cura della Soprintendenza archeologica delle province di Napoli e Caserta 
\item[ACant-long] Archaeologia Cantiana 
\item[AcBibl-long] Accademie e biblioteche d'Italia 
\item[Achse-long] Achse, Rad und Wagen. Beiträge zur Geschichte der Landfahrzeuge 
\item[Acme-long] Acme. Annali della Facoltà di lettere e filosofia dell'Università degli studi di Milano 
\item[Acontia-long] Acontia. Revista de arqueología 
\item[ACors-long] Archeologia corsa 
\item[ActaAArtHist-long] Acta ad archaeologiam et artium historiam pertinentia 
\item[ActaAArtHist-sa-long] Acta ad archaeologiam et artium historiam pertinentia. Series altera in 8 %*Abweichung!
\item[ActaAcAbo-long] Acta Academiae Aboensis 
\item[ActaACarp-long] Acta archaeologica Carpathica 
\item[ActaALov-long] Acta archaeologica Lovaniensia 
\item[ActaALovMono-long] Acta archaeologica Lovaniensia. Monographiae 
\item[ActaAntHung-long] Acta antiqua Academiae scientiarum Hungaricae 
\item[ActaArch-long] Acta archaeologica. København 
\item[ActaArchHung-long] Acta archaeologica Academiae scientiarum Hungaricae 
\item[ActaAth-long] Acta Instituti Atheniensis regni Sueciae 
\item[ActaCl-long] Acta classica. Proceedings of the Classical Association of South Africa 
\item[ActaClDebrec-long] Acta classica Universitatis scientiarum Debreceniensis 
\item[ActaHistDac-long] Acta historica. Societas academica Dacoromana 
\item[ActaHyp-long] Acta hyperborea. Danish Studies in Classical Archaeology 
\item[ActaInstRomFin-long] Acta Instituti Romani Finlandiae 
\item[ActaMusNapoca-long] Acta Musei Napocensis 
\item[ActaMusPorol-long] Acta Musei Porolissensis 
\item[ActaNum-long] Acta numismática (Barcelona) 
\item[ActaOr-long] Acta orientalia (Kopenhagen) 
\item[ActaOrHung-long] Acta orientalia Academiae scientiarum Hungaricae 
\item[ActaPhilSocDac-long] Acta philologica. Societas academica Dacoromana 
\item[ActaPraehistA-long] Acta praehistorica et archaeologica 
\item[ActaTorunA-long] Acta Universitatis Nicolai Copernici. Archaeologia 
\item[ActaTorunHist-long] Acta Universitatis Nicolai Copernici. Historia 
\item[AD-long] Antike Denkmäler 
\item[ADAIK-long] Abhandlungen des Deutschen Archäologischen Instituts, Abteilung Kairo 
\item[Adalya-long] Adalya. Annual of the Suna \& Inan Kiraç-Research Institute on Mediterranean Civilizations 
\item[ADelt-A-long] Αρχαιολογικόν Δελτίον (Μελέτες) %*Abweichung!
\item[ADelt-B-long] Αρχαιολογικόν Δελτίον (Χρονικά) %*Abweichung!
\item[ADerg-long] Arkeoloji dergisi. Ege Üniversitesi Edebiyat Fakültesi 
\item[ADFU-long] Ausgrabungen der Deutschen Forschungsgemeinschaft in Uruk-Warka 
\item[AdI-long] Annali dell'Instituto di corrispondenza archeologica 
\item[ADOG-long] Abhandlungen der Deutschen Orient-Gesellschaft 
\item[Adumatu-long] Adumatu. A Semi-Annual Archeological Refereed Journal on the Arab World 
\item[AE-long] L'année épigraphique 
\item[AeA-long] Aegean Archaeology 
\item[Aegaeum-long] Aegaeum. Annales d'archéologie égéenne de l'Université de Liège 
\item[AegLev-long] Ägypten und Levante. Egypt and the Levant. Internationale Zeitschrift für ägyptische Archäologie und deren Nachbargebiete %*Abweichung!
\item[AEmil-long] Archeologia dell'Emilia-Romagna 
\item[AEphem-long] Αρχαιολογική Eφημερίς 
\item[AeR-long] Atene e Roma 
\item[AErgoMak-long] Το Αρχαιολογικό Έργο στη Μακεδονία και Θράκη 
\item[AErt-long] Archaeologiai értesitő
\item[AEspA-long] Archivo español de arqueología 
\item[Aevum-long] Aevum. Rassegna di scienze storiche linguistiche e filologiche 
\item[AevumAnt-long] Aevum antiquum 
\item[AF-long] Archäologische Forschungen 
\item[AfO-long] Archiv für Orientforschung 
\item[Africa-long] Africa. Institut national d'archéologie et d'art, Tunis 
\item[AGD-long] Antike Gemmen in deutschen Sammlungen 
\item[AGeo-long] Archaeologia geographica 
\item[Agora-long] The Athenian Agora 
\item[AgoraPB-long] Excavations of the Athenian Agora. Picture Book 
\item[AHist-long] Arqueologia e história 
\item[AHistStAlex-long] Archaeological and Historical Studies. The Archaeological Society of Alexandria 
\item[AHw-long] W. von Soden, Akkadisches Handwörterbuch (Wiesbaden 1965--1981) 
\item[AiD-long] Archäologie in Deutschland 
\item[AInf-long] Archäologische Informationen. Mitteilungen zur Ur- und Frühgeschichte 
\item[AIONArch-long] Annali dell'Istituto universitario orientale di Napoli. Dipartimento di studi del mondo classico e del Mediterraneo antico. Sezione di archeologia e storia antica 
\item[AIONFil-long] Annali dell'Istituto universitario orientale di Napoli. Dipartimento di studi del mondo classico e del Mediterraneo antico. Sezione filologicoletteraria 
\item[AIONLing-long] Annali dell'Istituto universitario orientale di Napoli. Dipartimento di studi del mondo classico e del Mediterraneo antico. Sezione linguistica 
\item[AIPhOr-long] Annuaire de l'Institut de philologie et d'histoire orientales et slaves (Université Libre de Bruxelles) 
\item[Aitna-long] Aitna. Quaderni di topografia antica 
\item[AJA-long] American Journal of Archaeology 
\item[AJahrBay-long] Das archäologische Jahr in Bayern 
\item[AJPh-long] American Journal of Philology 
\item[AJug-long] Archaeologia Jugoslavica 
\item[AKorrBl-long] Archäologisches Korrespondenzblatt 
\item[AlbaRegia-long] Alba Regia. Annales Musei Stephani regis 
\item[AlmaMaterSt-long] Alma mater studiorum Almanacco calabrese Almoraima 
\item[AlmanachWien-long] Österreichische Akademie der Wissenschaften. Almanach 
\item[AlonJisrael-long] Alon mahleqat ha-'atiqot šel medinat Jisra'el 
\item[Al-Qannis-long] Al-Qanniš. Boletín del Taller de arqueología de Alcañiz
\item[Altamura-long] Altamura. Bollettino dell'Archivio-biblioteca-museo civico 
\item[AltoMed-long] Alto medioevo 
\item[Alt-Paphos-long] Ausgrabungen in Alt-Paphos auf Cypern 
\item[AltThuer-long] Alt-Thüringen %*Abweichung!
\item[AM-long] Mitteilungen des Deutschen Archäologischen Instituts, Athenische Abteilung 
\item[AMediev-long] Archaeologia medievale. Cultura materiale, insediamenti, territorio 
\item[AMethTh-long] Advances in Archaeological Method and Theory 
\item[AMI-long] Archäologische Mitteilungen aus Iran 
\item[AMIT-long] Archäologische Mitteilungen aus Iran und Turan 
\item[AmJAncHist-long] American Journal of Ancient History 
\item[AmJNum-long] American Journal of Numismatics 
\item[AMold-long] Arheologia Moldovei 
\item[AMosel-long] Archaeologia Mosellana 
\item[AMS-long] Asia Minor Studien 
\item[AmStP-long] American Studies in Papyrology 
\item[AMuGS-long] Antike Münzen und geschnittene Steine 
\item[ANachr-long] Archäologisches Nachrichtenblatt 
\item[ANachrBad-long] Archäologische Nachrichten aus Baden 
\item[AnadoluAra-long] Anadolu araştırmaları. Jahrbuch für kleinasiatische Forschung 
\item[AnadoluKonf-long] Anadolu Medeniyetleri Müzesi konferansları 
\item[AnadoluYil-long] Anadolu Medeniyetleri Müzesi yıllığı 
\item[AnAe-long] Analecta Aegyptiaca 
\item[Anagennesis-long] Anagennesis. A Papyrological Journal 
\item[AnalBolland-long] Analecta Bollandiana 
\item[AnalP-long] Analecta papyrologica 
\item[AnalRom-long] Analecta Romana Instituti Danici 
\item[AnArqAnd-long] Anuario arqueológico de Andalucía 
\item[Anas-long] Anas. Museo nacional de arte romano de Mérida 
\item[AnatA-long] Anatolian Archaeology. Reports on Research Conducted in Turkey 
\item[Anatolia-long] Anatolia. Revue annuelle de l’Institut d’archéologie de l’Université d’Ankara 
\item[ANaturwiss-long] Archäologie und Naturwissenschaften 
\item[AncCivScytSib-long] Ancient Civilizations from Scythia to Siberia. An International Journal of Comparative Studies in History and Archaeology 
\item[AncHistB-long] The Ancient History Bulletin 
\item[AncInd-long] The Ancient India 
\item[AncNearEastSt-long] Ancient Near Eastern Studies. An Annual 
\item[AnCord-long] Anales de arqueología cordobesa 
\item[AncSoc-long] Ancient Society 
\item[AncW-long] The Ancient World 
\item[AncWestEast-long] Ancient West and East 
\item[AnDubr-long] Annali Zavoda za povijesne znanosti Istraivakog centra Jugoslavenske akademije znanosti i umjetnosti u Dubrovniku 
\item[ANews-long] Archaeological News 
\item[ANilMoy-long] Archéologie du Nil moyen 
\item[ANL-long] The Archaeological News Letter 
\item[AnMunFaro-long] Anais do municípo de Faro 
\item[AnMurcia-long] Anales de prehistoria y arqueología. Universidad de Murcia 
\item[AnnAcEtr-long] Annuario. Accademia etrusca di Cortona 
\item[AnnAcTorino-long] Annuario della Accademia delle scienze di Torino 
\item[AnnAStorAnt-long] Annali. Sezione di archeologia e storia antica. Istituto universitario orientale di Napoli. Dipartimento di studi del mondo classico e del Mediterraneo antico 
\item[AnnBari-long] Annali della Facoltà di lettere e filosofia, Università degli Studi, Bari 
\item[AnnBenac-long] Annali Benacensi 
\item[AnnBiblAModena-long] Annuario bibliografico di archeologia. Modena 
\item[AnnBiblARom-long] Annuario bibliografico di archeologia. Nuove accessioni del \ldots. Biblioteca dell’Istituto nazionale di archeologia e storia dell’arte, Roma 
\item[AnnByzConf-long] Annual Byzantine Studies Conference. Abstracts of Papers 
\item[AnnCagl-long] Annnali della Facoltà di lettere e filosofia dell'Università di Cagliari 
\item[AnnCaglMag-long] Annali della Facoltà di magistero dell'Università di Cagliari 
\item[AnnEconSocCiv-long] Annales. Economies, sociétés, civilisations 
\item[AnnEgBibl-long] Annual Egyptological Bibliography 
\item[AnnEth-long] Annales d'Éthiopie 
\item[AnnFaina-long] Annali della Fondazione per il Museo Claudio Faina 
\item[AnnHistA-long] Annales d'histoire et d'archéologie 
\item[AnnHistScSoc-long] Annales. Histoire, sciences sociales 
\item[AnnIstGiapp-long] Annuario. Istituto giapponese di cultura in Roma 
\item[AnnIstItNum-long] Annali. Istituto italiano di numismatica 
\item[AnnLecce-long] Annali dell'Università di Lecce. Facoltà di lettere e filosofia e di magistero 
\item[AnnLeedsUnOrSoc-long] The Annual of Leeds University Oriental Society 
\item[AnnMacerata-long] Annali della Facoltà di lettere e filosofia, Università di Macerata 
\item[AnnMessMag-long] Nuovi annali della Facoltà di magistero dell'Università di Messina 
\item[AnnMusRov-long] Annali del Museo civico di Rovereto. Sezione archeologia, storia, scienze naturali 
\item[AnnNap-long] Annali della Facoltà di lettere e filosofia, Università di Napoli 
\item[AnnNivern-long] Les annales des pays Nivernais 
\item[AnnNoment-long] Annali. Associazione nomentana di storia e archeologia 
\item[AnnOrNap-long] Annali. Rivista del Dipartimento di studi asiatici e del Dipartimento di studi e ricerche su Africa e paesi arabi, Istituto universitario orientale di Napoli 
\item[AnnotNum-long] Annotazioni numismatiche 
\item[AnnPerugia-long] Annali della Facoltà di lettere e filosofia, Università degli studi di Perugia, 1. Studi classici 
\item[AnnPisa-long] Annali della Scuola normale superiore di Pisa 
\item[AnnPontAcRom-long] Annuario della Pontificia accademia romana di archeologia 
\item[AnnRepBSA-long] Annual Report of Council. British School of Archaeology at Athens 
\item[AnnRepCypr-long] Annual Report of the Department of Antiquities, Republic of Cyprus 
\item[AnnRepFoggArtMus-long] The Annual Report of the Fogg Art Museum 
\item[AnnSiena-long] Annali della Facoltà di lettere e filosofia, Università di Siena 
\item[AnnuarioAcLinc-long] Annuario della Accademia nazionale dei Lincei 
\item[AnnuarioLecce-long] Annuario. Liceo-ginnasio statale G. Palmieri, Lecce 
\item[AnnUnBud-long] Annales Universitatis scientiarum Budapestinensis de Rolando Eötvös nominatae 
\item[AnnWorcArtMus-long] Annual (Report). Worcester Art Museum 
\item[Anodos-long] Anodos. Studies of Ancient World 
\item[AnOr-long] Analecta orientalia. Commentationes scientificae de rebus orientis antiqui 
\item[ANRW-long] Aufstieg und Niedergang der römischen Welt 
\item[Anschnitt-long] Der Anschnitt. Mitteilungsblatt der Vereinigung der Freunde von Kunst und Kultur im Bergbau 
\item[ANSMusNotes-long] Museum Notes. American Numismatic Society 
\item[AnSt-long] Anatolian Studies 
\item[Antaeus-long] Antaeus. Communicationes ex Instituto archaeologico Academiae scientiarum hungaricae 
\item[AntAfr-long] Antiquités africaines 
\item[AntChr-long] Antike und Christentum 
\item[AntCl-long] L'antiquité classique 
\item[AnthrAChron-long] Ανθρωπολογικά και Αρχαιολογικά Χρονικά 
\item[Anthropos-long] Άνθρωπος. Όργανο της Ανθρωπολογικής Εταιρείας Ελλάδος 
\item[Antichthon-long] Antichthon. Journal of the Australian Society for Classical Studies 
\item[AntigCr-long] Antigüedad y cristianismo. Monografías históricas sobre la antigüedad tardía 
\item[Antipolis-long] Antipolis. A Journal of Mediterranean Archaeology 
\item[Antiqua-long] Antiqua. Rivista dell'Archeoclub d'Italia 
\item[Antiquity-long] Antiquity. A Quarterly Review of Archaeology 
\item[AntJ-long] The Antiquaries Journal 
\item[AntK-long] Antike Kunst 
\item[AntNat-long] Antiquités nationales. Saint-Germain-en-Laye 
\item[AntPisa-long] Antichità pisane 
\item[AntPl-long] Antike Plastik 
\item[AntSurv-long] Antiquity and Survival 
\item[AntTard-long] Antiquité tardive. Revue internationale d'histoire et d'archéologie 
\item[AnzAW-long] Anzeiger für die Altertumswissenschaft 
\item[AnzWien-long] Anzeiger. Österreichische Akademie der Wissenschaften, Philosophisch-Historische Klasse 
\item[AOAT-long] Alter Orient und Altes Testament. Veröffentlichungen zur Kultur und Geschichte des Alten Orients und des Alten Testaments 
\item[AoF-long] Altorientalische Forschungen 
\item[AOtkryt-long] Archeologičeskie otkrytija 
\item[APamKiiv-long] Archeologični pamjatki URSR 
\item[APh-long] L'année philologique 
\item[APol-long] Archaeologia Polona 
\item[Apollo-long] Apollo. Bolletino dei musei provinciali del Salernitano 
\item[ApolloLond-long] Apollo. The International Magazine of the Arts 
\item[APort-long] O arqueólogo português 
\item[AppRomFil-long] Appunti romani di filologia. Studi e comunicazioni di filologia, linguistica e letteratura greca e latina 
\item[APregl-long] Arheološki pregled. Arheološko društvo Jugoslavije 
\item[Apulum-long] Apulum. Acta Musei Apulensis 
\item[AquiLeg-long] Aquila legionis. Cuadernos des estudios sobre el ejército romano 
\item[AquilNost-long] Aquileia nostra. Bollettino dell'Associazione nazionale per Aquileia 
\item[Aquitania-long] Aquitania. Une revue inter-régionale d'archéologie 
\item[ArabAEpigr-long] Arabian Archaeology and Epigraphy 
\item[ARadRaspr-long] Arheolokiradovi i rasprave 
\item[ArbFBerSaechs-long] Arbeits- und Forschungsberichte zur sächsischen Bodendenkmalpflege 
\item[Archaeographie-long] Archäographie. Archäologie und elektronische Datenverarbeitung %*Abweichung!
\item[Archaeologia-long] Archaeologia or Miscellaneous Tracts Relating to Antiquity Published by the Society of Antiquaries of London 
\item[Archaeology-long] Archaeology. A Magazine Dealing with the Antiquity of the World 
\item[Archaeometry-long] Archaeometry. Bulletin of the Research Laboratory for Archaeology and History of Art, Oxford University 
\item[Archaiognosia-long] Αρχαιογνωσία 
\item[ArchBegriffsGesch-long] Archiv für Begriffsgeschichte 
\item[ArchByzMnem-long] Αρχείον των Βυζαντινών Μνημείων της Ελλάδος  
\item[ArchCl-long] Archeologia classica 
\item[Archeo-long] Archeo. Attualità del passato 
\item[ArcheogrTriest-long] Archeografo triestino 
\item[ArcheologiaParis-long] Archeologia, Paris. L'archéologie dans le monde et tout ce qui concerne les recherches historiques, artistiques et scientifiques sur terre et dans les mers 
\item[ArcheologiaRoma-long] Archeologia. Rivista bimestrale. Roma 
\item[ArcheologiaWarsz-long] Archeologia. Rocznik Instytutu archeologii i etnologii, Polskiej akademii nauk 
\item[ArcheologijaKiiv-long] Archeologija. Nacional'na akademija nauk Ukraini. Institut archeologii 
\item[ArcheologijaSof-long] Archeologija. Organ na Archeologičeskija institut i muzej (pri Bălgarskata akademii nauk) 
\item[ArchEubMel-long] Αρχείον Ευβοϊκών Μελετών 
\item[ArchHom-long] Archaeologia Homerica 
\item[Architectura-long] Architectura. Zeitschrift für Geschichte der Baukunst 
\item[Archivi-long] Archivi. Archivi d'Italia e rassegna internazionale degli archivi 
\item[ArchPF-long] Archiv für Papyrusforschung und verwandte Gebiete 
\item[ArchPrehistLev-long] Archivo de prehistoria levantina 
\item[ArchRel-long] Archiv für Religionsgeschichte 
\item[ArchStorCal-long] Archivio storico per la Calabria e la Lucania 
\item[ArchStorPugl-long] Archivio storico pugliese 
\item[ArchStorRom-long] Archivio della Società romana di storia patria 
\item[ArchStorSicOr-long] Archivio storico per la Sicilia orientale 
\item[ArchStorSir-long] Archivio storico siracusano 
\item[Arctos-long] Arctos. Acta philologica Fennica 
\item[ARepLond-long] Archaeological Reports 
\item[Argo-long] Argo. časopis slovenskih muzejev, Narodni Muzej Slovenije 
\item[ArOr-long] Archív orientální. Quarterly Journal of African and Asian Studies 
\item[ArOrMono-long] Archív orientální. Quarterly Journal of African, Asian and Latin American Studies. Monografie Archívu orientálního 
\item[ArOrSuppl-long] Archív orientální. Quarterly Journal of African, Asian and Latin American Studies. Supplementa 
\item[ARozhl-long] Archeologické rozhledy 
\item[ArqBeja-long] Arquivo de Beja. Boletim, estudos, arquivo 
\item[Arse-long] Arse. Boletín del Centro arqueológico saguntino 
\item[ArsGeorg-long] Ars Georgica 
\item[ArtAntMod-long] Arte antica e moderna 
\item[ArtARhone-long] Art et archéologie en Rhône-Alpes %*Abweichung!
\item[ArtAs-long] Artibus Asiae 
\item[ArtB-long] The Art Bulletin 
\item[ArtJ-long] Art Journal 
\item[ArtLomb-long] Arte lombarda 
\item[ArtMediev-long] Arte medievale 
\item[ArtVirg-long] Arts in Virginia 
\item[ARV2-long] J. D. Beazley, Attic Red-figure Vase-painters \textsuperscript{2}(Oxford 1963) 
\item[ASachs-long] Archäologie in Sachsen-Anhalt 
\item[ASAE-long] Annales du Service des antiquités de l’Égypte 
\item[ASammlUnZuerch-long] Archäologische Sammlung der Universität Zürich %*Abweichung!
\item[ASAtene-long] Annuario della Scuola archeologica di Atene e delle missioni italiane in Oriente 
\item[ASbor-long] Archeologičeskij sbornik. Gosudarstvennyj ordena Lenina Ermitaž 
\item[ASchw-long] Archäologie der Schweiz. Mitteilungsblatt der Schweizerischen Gesellschaft für Ur- und Frühgeschichte 
\item[ASoc-long] Archeologia e società 
\item[ASoloth-long] Archäologie des Kantons Solothurn 
\item[ASR-long] Die antiken Sarkophagreliefs 
\item[Assaph-long] Assaph. Studies in Art History 
\item[AssyrMisc-long] Assyriological Miscellanies 
\item[AST-long] AratirmaSonuçlariToplantisi 
\item[ASub-long] L'archeologo subacqueo. Quadrimestrale di archeologia subacquea e navale 
\item[ASubacq-long] Archeologia subacquea. Studi, ricerche e documenti 
\item[Athenaeum-long] Athenaeum. Studi di letteratura e storia dell'antichità 
\item[Atiqot-long] `Atiqot. Journal of the Israel Department of Antiquities 
\item[AtiqotHeb-long] `Atiqot. Journal of the Israel Department of Antiquities. Hebrew Series 
\item[Atlal-long] Atlal. The Journal of Saudi Arabian Archaeology 
\item[AttiAcPontan-long] Atti della Accademia pontaniana 
\item[AttiAcRov-long] Atti della Accademia Roveretana degli Agiati. Contributi della classe di scienze umane, di lettere ed arti 
\item[AttiAcTorino-long] Atti della Accademia delle scienze di Torino, 2. Classe di scienze morali, storiche e filologiche 
\item[AttiCAntCl-long] Atti. Centro ricerche e documentazione sull'antichità classica 
\item[AttiCItRom-long] Atti. Centro studi e documentazione sull'Italia romana 
\item[AttiMemBologna-long] Atti e memorie. Deputazione di storia patria per le province di Romagna 
\item[AttiMemDal-long] Atti e memorie della Società dalmata di storia patria 
\item[AttiMemFirenze-long] Atti e memorie dell'Academia toscana di scienze e lettere »La Columbaria« 
\item[AttiMemIstria-long] Atti e memorie della Società istriana di archeologia e storia patria 
\item[AttiMemMagnaGr-long] Atti e memorie della Società Magna Grecia 
\item[AttiMemModena-long] Atti e memorie. Deputazione di storia patria per le antiche provincie modenesi 
\item[AttiMemTivoli-long] Atti e memorie della Società tiburtina di storia e d'arte 
\item[AttiMusTrieste-long] Atti dei Civici musei di storia ed arte di Trieste 
\item[AttiPalermo-long] Atti della Accademia di scienze, lettere e arti di Palermo 
\item[AttiRovigno-long] Atti. Centro di ricerche storiche, Rovigno 
\item[AttiSocFriuli-long] Atti della Società per la preistoria e protostoria della regione Friuli -- Venezia Giulia 
\item[AttiVenezia-long] Atti. Istituto veneto di scienze, lettere ed arti 
\item[AuA-long] Antike und Abendland 
\item[AulaOr-long] Aula orientalis. Revista de estudios del Próximo Oriente antiguo 
\item[AusgrFu-long] Ausgrabungen und Funde. Nachrichtenblatt der Landesarchäologie 
\item[AusgrFuWestf-long] Ausgrabungen und Funde in Westfalen-Lippe 
\item[AustrRom-long] Pro Austria Romana 
\item[AUTerr-long] Archeologia, uomo, territorio. Rivista dei Gruppi archeologici Nord Italia 
\item[AUWE-long] Ausgrabungen in Uruk-Warka. Endberichte 
\item[AV-long] Archäologische Veröffentlichungen. Deutsches Archäologisches Institut, Abteilung Kairo 
\item[AVen-long] Archeologia veneta 
\item[AVes-long] Arheološki vestnik (Ljubljana) 
\item[AViva-long] Archeologia viva 
\item[AvP-long] Altertümer von Pergamon 
\item[AW-long] Antike Welt. Zeitschrift für Archäologie und Kulturgeschichte 
\item[AyasofyaMuezYil-long] Ayasofia Müzesi yıllığı. Annual of Ayasofya Museum \label{AyasofyaMuezYil-lang} %*Abweichung!
\item[AZ-long] Archäologische Zeitung 
\item[Azotea-long] Azotea. Revista de cultura del Ayuntamiento de Coria del Río 
\item[BA-long] Bollettino di archeologia 
\item[Baalbek-long] Baalbek. Ergebnisse der Ausgrabungen und Untersuchungen in den Jahren 1898 bis 1905 
\item[BAAlger-long] Bulletin d'archéologie algérienne 
\item[BABarcel-long] Butlletí informatiu de l'Institut de prehistòria i arqueologia de la Diputació provincial de Barcelona 
\item[BABesch-long] Bulletin antieke beschaving. Annual Papers on Classical Archaeology 
\item[BAcRHist-long] Boletín de la Real academia de la historia 
\item[BACopt-long] Bulletin de la Société d'archéologie copte 
\item[BadFuBer-long] Badische Fundberichte 
\item[Baetica-long] Baetica. Estudios de arte, geografía e historia 
\item[BaF-long] Baghdader Forschungen 
\item[BalacaiKoez-long] Balácai közlemények %*Abweichung!
\item[BalkSt-long] Balkan Studies 
\item[BALond-long] Bulletin of the Institute of Archaeology, University of London 
\item[BALux-long] Bulletin d'archéologie luxembourgeoise 
\item[BaM-long] Baghdader Mitteilungen 
\item[BAMaroc-long] Bulletin d'archéologie marocaine 
\item[BAmSocP-long] The Bulletin of the American Society of Papyrologists 
\item[BAncOrMus-long] Bulletin of the Ancient Orient Museum (Tokyo) 
\item[BAngers-long] Bulletin du Centre de recherches et d'enseignement de l'antiquité, Angers 
\item[BAngloIsrASoc-long] Bulletin of the Anglo-Israel Archaeological Society 
\item[BAnnMusFerr-long] Bollettino annuale. Musei ferraresi 
\item[BAntFr-long] Bulletin de la Société nationale des antiquaires de France 
\item[BAntLux-long] Bulletin des antiquités luxembourgeoises 
\item[BAParis-long] Bulletin archéologique du Comité des travaux historiques et scientifiques. Antiquités nationales 
\item[BAProv-long] Bulletin archéologique de Provence 
\item[BAR-long] British Archaeological Reports. British Series 
\item[BArchAlex-long] Bulletin. Société archéologique d'Alexandrie 
\item[BArchit-long] Bollettino del Centro di studi per la storia dell'architettura 
\item[BARIntSer-long] British Archaeological Reports. International Series 
\item[BASard-long] Nuovo bullettino archeologico sardo 
\item[BAsEspA-long] Boletín. Asociación española de amigos de la arqueología 
\item[BAsInst-long] Bulletin of the Asia Institute 
\item[BASOR-long] Bulletin of the American Schools of Oriental Research 
\item[BAssBude-long] Bulletin de l'Association Guillaume Budé %*Abweichung!
\item[BAssMosAnt-long] Bulletin d'information de l'Association internationale pour l'étude de la mosaïque antique 
\item[BASub-long] Bollettino di archeologia subacquea 
\item[BASudEstEur-long] Bulletin d'archéologie sud-est européenne 
\item[BATarr-long] Bulletí arqueològic. Reial societat arqueològica tarraconense. Boletín arqueológico. Real sociedad arqueológica tarragonese 
\item[BAur-long] Boletín auriense 
\item[BAVA-long] Beiträge zur Allgemeinen und Vergleichenden Archäologie 
\item[BayVgBl-long] Bayerische Vorgeschichtsblätter 
\item[BBasil-long] Bollettino storico della Basilicata 
\item[BBelgRom-long] Bulletin de l'Institut historique belge de Rome 
\item[BBolsena-long] Bollettino di studi e ricerche. Biblioteca comunale di Bolsena 
\item[BBrByzSt-long] Bulletin of British Byzantine Studies 
\item[BCamuno-long] Bullettino del Centro Camuno di studi preistorici 
\item[BCASic-long] Beni culturali e ambientali. Sicilia 
\item[BCercleNum-long] Bulletin du Cercle d'études numismatiques 
\item[BCH-long] Bulletin de correspondance hellénique 
\item[BCircNumNap-long] Bollettino del Circolo numismatico napoletano 
\item[BCl-long] Bollettino dei classici 
\item[BClevMus-long] The Bulletin of The Cleveland Museum of Art 
\item[BCom-long] Bullettino della Commissione archeologica comunale di Roma 
\item[BCord-long] Bulletin d'information de l'Association internationale pour l'étude de la mosaïque antique 
\item[BdA-long] Bollettino d'arte 
\item[BdE-long] Bibliothèque d'études, Institut français d'archéologie orientale, Kairo 
\item[BdEC-long] Bibliothèque d'études coptes, Institut français d'archéologie orientale, Kairo 
\item[BdI-long] Bullettino dell'Instituto di corrispondenza archeologica 
\item[BDirRom-long] Bullettino dell'Istituto di diritto romano »Vittorio Scialoja« 
\item[BeazleyAddenda2-long] T. H. Carpenter (Hrsg.), Beazley Addenda \textsuperscript{2}(Oxford 1989) %*Abweichung!
\item[BeazleyPara-long] J. D. Beazley, Paralipomena. Additions to Attic Black-figure Vase-painters and to Attic Red-figure Vase-painters (Oxford 1971) %*Abweichung!
\item[BEcAntNimes-long] Bulletin (annuel) de l'École antique de Nîmes %*Abweichung!
\item[BediKart-long] Bedi Kartlisa. Revue de kartvélologie 
\item[BEFAR-long] Bibliothèque des Écoles françaises d'Athènes et de Rome 
\item[BeitrESkAr-long] Beiträge zur Erschließung hellenistischer und kaiserzeitlicher Skulptur und Architektur 
\item[BeitrNamF-long] Beiträge zur Namenforschung 
\item[BeitrSudanF-long] Beiträge zur Sudanforschung 
\item[BelArt-long] Bellas artes (Madrid) 
\item[Belleten-long] Belleten. Türk Tarih Kurumu 
\item[Benacus-long] Benàcus. Museo archeologico della Val Tenesi 
\item[BerBayDenkmPfl-long] Bericht der Bayerischen Bodendenkmalpflege 
\item[BerDFG-long] Bericht der Deutschen Forschungsgemeinschaft 
\item[BerlBeitrArchaeom-long] Berliner Beiträge zur Archäometrie %*Abweichung!
\item[BerlBlVFruehGesch-long] Berliner Blätter für Vor- und Frühgeschichte %*Abweichung!
\item[BerlJbVFruehGesch-long] Berliner Jahrbuch für Vor- und Frühgeschichte %*Abweichung!
\item[BerlMus-long] Berliner Museen 
\item[BerlNumZ-long] Berliner numismatische Zeitschrift 
\item[BerOudhBod-long] Berichten van de Rijksdienst voor het oudheidkundig bodemonderzoek 
\item[BerRGK-long] Bericht der Römisch-Germanischen Kommission 
\item[BerVerhLeipz-long] Berichte über die Verhandlungen der Sächsischen Akademie der Wissenschaften zu Leipzig 
\item[Berytus-long] Berytus. Archaeological Studies 
\item[BEspA-long] Boletín. Asociación española de amigos de la arqueología 
\item[BEspOr-long] Boletín de la Asociación española de orientalistas 
\item[BEtOr-long] Bulletin d'études orientales 
\item[BFilGrPadova-long] Bollettino dell'Istituto di filologia greca, Università di Padova 
\item[BFilLingSic-long] Bollettino. Centro di studi filologici e linguistici siciliani 
\item[BFlegr-long] Bollettino flegreo. Rivista di storia, arte e scienze 
\item[BFoligno-long] Bollettino storico della città di Foligno 
\item[BHarvMus-long] Harvard University Art Museums Bulletin 
\item[BIasos-long] Bollettino dell'Associazione Iasos di Caria 
\item[BibAr-long] The Biblical Archaeologist. The American School of Oriental Research New Haven 
\item[BiblClOr-long] Bibliotheca classica orientalis 
\item[BiblSymb-long] Bibliographie zur Symbolik, Ikonographie und Mythologie. Internationales Referateorgan 
\item[BIBulg-long] Izvestija na Archeologičeskija institut. Bulletin de l’Institut d’archéologie 
\item[BICS-long] Bulletin of the Institute of Classical Studies of the University of London 
\item[BIFAO-long] Bulletin de l'Institut français d'archéologie orientale 
\item[BInfCentumcellae-long] Bollettino di informazioni. Associazione archeologica Centumcellae 
\item[BInfCESDAE-long] Bollettino di informazioni del Centro di studi e documentazione sull'area elima 
\item[BiogrZbor-long] Biogradski zbornik 
\item[BiOr-long] Bibliotheca orientalis 
\item[BIstOrvieto-long] Bollettino dell'Istituto storico artistico orvietano 
\item[BJaen-long] Boletín del Instituto de estudios giennenses %*Abweichung!
\item[BJb-long] Bonner Jahrbücher des Rheinischen Landesmuseums in Bonn 
\item[BJerus-long] Bulletin. The Hebrew University, Jerusalem, Rabinowitz Fund 
\item[BLaborMusLouvre-long] Bulletin du laboratoire du Musée du Louvre 
\item[BLazioMerid-long] Bollettino dell'Istituto di storia e di arte del Lazio meridionale 
\item[BLikUm-long] Bulletin Razreda za likovne umjetnosti Hrvatske akademije znanosti i umjetnosti 
\item[BlMueFreundeF-long] Blätter für Münzfreunde und Münzforschung %*Abweichung!
\item[BLugo-long] Boletín de la Comisión provincial de monumentos históricos y artísticos de Lugo 
\item[BMCGreekCoins-long] Catalogue of the Greek Coins in the British Museums %*Abweichung!
\item[BMCOR-long] Catalogue of Oriental Coins in the British Museum I--X 
\item[BMCRE-long] H. Mattingly (u. a.), Coins of the Roman Empire in the British Museum (London 1923--1950; \textsuperscript{2}1975) 
\item[BMCRRI-III-long] H. A. Grueber, Coins of the Roman Republic in the British Museum I--III (London 1910) %*Abweichung!
\item[BMetrMus-long] The Metropolitan Museum of Art Bulletin 
\item[BMon-long] Bulletin monumental 
\item[BMonMusPont-long] Bollettino. Monumenti, musei e gallerie pontificie 
\item[BMQ-long] The British Museum Quarterly 
\item[BMQNSuppl-long] The British Museum Quarterly. News Supplement 
\item[BMusBeyrouth-long] Bulletin du Musée de Beyrouth 
\item[BMusBrux-long] Bulletin des Musées royaux d'art et d'histoire, Bruxelles 
\item[BMusCadiz-long] Boletín del Museo de Cádiz 
\item[BMusCivRom-long] Bullettino del Museo della civiltà romana 
\item[BMusFA-long] Bulletin. Museum of Fine Arts, Boston 
\item[BMusHongr-long] Bulletin du Musée hongrois des beaux-arts 
\item[BMusMadr-long] Boletín del Museo arqueológico nacional, Madrid 
\item[BMusMich-long] Bulletin. Museums of Art and Archaeology, University of Michigan 
\item[BMusMonaco-long] Bulletin du Musée d'anthropologie préhistorique de Monaco 
\item[BMusPadova-long] Bollettino del Museo civico di Padova 
\item[BMusPBelArt-long] Boletín del Museo provincial de bellas artes 
\item[BMusRom-long] Bollettino dei Musei comunali di Roma 
\item[BMusVars-long] Bulletin du Musée national de Varsovie 
\item[BMusZaragoza-long] Museo de Zaragoza. Boletín 
\item[BNumParis-long] Bulletin de la Société française de numismatique 
\item[BNumRoma-long] Bollettino di numismatica 
\item[Bogazkoey-Hattusa-long] Boğazköy-Hattuša. Ergebnisse der Ausgrabungen %*Abweichung!
\item[Bolskan-long] Bolskan. Revista de arqueología del Instituto de estudios Altoaragoneses, revista de arqueología Oscense 
\item[BonnHVg-long] Bonner Hefte zur Vorgeschichte 
\item[BOntMus-long] Bulletin of the Royal Ontario Museum of Archaeology, University of Toronto 
\item[Boreas-long] Boreas. Münstersche Beiträge zur Archäologie 
\item[BoreasUpps-long] Boreas. Uppsala Studies in Ancient Mediterranean and Near Eastern Civilization 
\item[BPeintRom-long] Bulletin de liaison. Centre d'études des peintures murales romaines 
\item[BPI-long] Bullettino di paletnologia italiana 
\item[BPrehistAlp-long] Bulletin d'études préhistoriques alpines %*Abweichung!
\item[BProAvent-long] Bulletin de l'Association Pro Aventico 
\item[BProvidence-long] Bulletin of the Rhode Island School of Design. Museum Notes 
\item[BracAug-long] Bracara Augusta. Revista cultural da Câmara municipal de Braga 
\item[BracaraAugusta-long] Bracara Augusta. Revista cultural da Câmara municipal de Braga 
\item[BremABl-long] Bremer archäologische Blätter 
\item[BRest-long] Bollettino dell'Istituto centrale del restauro 
\item[Brigantium-long] Brigantium. Museo arqueolóxico e histórico 
\item[BrMusYearbook-long] The British Museum Yearbook 
\item[BSA-long] The Annual of the British School at Athens 
\item[BSAA-long] Boletín del Seminario de estudios de arte y arqueología, Universidad de Valladolid 
\item[BSFE-long] Bulletin de la Société française d'égyptologie 
\item[BSiena-long] Bullettino senese di storia patria 
\item[BSOAS-long] Bulletin of the School of Oriental and African Studies (London) 
\item[BSocAChamp-long] Bulletin de la Société archéologique champenoise 
\item[BSocBiblReinach-long] Bulletin de liaison de la Société des amis de la Bibliothèque Salomon Reinach 
\item[BSocNumRom-long] Buletinul Societăţii numismatice române 
\item[BSR-long] Papers of the British School at Rome 
\item[BStLat-long] Bollettino di studi latini 
\item[BStorArt-long] Bollettino della Unione storia ed arte 
\item[BTextilAnc-long] Bulletin du Centre international d'étude des textiles anciens 
\item[BTorino-long] Bollettino della Società piemontese di archeologia e belle arti 
\item[BTravTun-long] Bulletin des travaux de l'Institut national du patrimoine. Comptes rendus 
\item[BudReg-long] Budapest régiségei 
\item[BulletinGetty-long] Bulletin. J. Paul Getty Museum of Art 
\item[BulletinNorthampton-long] Bulletin. Smith College Museum of Art 
\item[BVallad-long] Boletín del Seminario de estudios de arte y arqueología, Universidad de Valladolid 
\item[BVitoria-long] Boletín de la Institución »Sancho el Sabio« 
\item[BWaltersArtGal-long] The Walters Art Gallery Bulletin 
\item[BWPr-long] Winckelmannsprogramm der Archäologischen Gesellschaft zu Berlin 
\item[Byzantina-long] Βυζαντινά. Επιστημονικόν Όργανον Κέντρου Βυζαντινών Ερευνών Φιλοσοφικής Σχολής Αριστοτελείου Πανεπιστημίου Θεσσαλονίκης
\item[ByzF-long] Byzantinische Forschungen. Internationale Zeitschrift für Byzantinistik 
\item[ByzJb-long] Byzantinisch-neugriechische Jahrbücher 
\item[ByzZ-long] Byzantinische Zeitschrift 
\item[BZ-long] Biblische Zeitschrift 
\item[CAA-long] Corpus antiquitatum Aegyptiacarum 
\item[CAD-long] The Assyrian Dictionary of the Oriental Institute of the University of Chicago 
\item[CadA-long] Cadernos de arqueologia 
\item[Caesaraugusta-long] Caesaraugusta. Publicaciones del Seminario de Arqueología y Numismática Aragonesas 
\item[Caesarodunum-long] Caesarodunum. Bulletin de l'Institut d'études latines et du Centre de recherches A. Piganiol 
\item[CAH-long] The Cambridge Ancient History 
\item[CahArmeeRom-long] Cahiers du Groupe de recherches sur l'armée romaine et les provinces 
\item[CahASubaqu-long] Cahiers d'archéologie subaquatique 
\item[CahByrsa-long] Cahiers de Byrsa 
\item[CahCEC-long] Cahier. Centre d'études chypriotes 
\item[CahCerEg-long] Cahiers de la céramique égyptienne 
\item[CahDelFrIran-long] Cahiers de la Délégation française en Iran 
\item[CahGlotz-long] Cahiers du Centre Gustave-Glotz. Revue reconnue par le CNRS 
\item[CahKarnak-long] Cahiers de Karnak 
\item[CahLig-long] Cahiers ligures de préhistoire et de protohistoire 
\item[CahMariemont-long] Les cahiers de Mariemont. Bulletin du Musée royal de Mariemont 
\item[CahMusChampollion-long] Cahiers du Musée Champollion. Histoire et archéologie 
\item[CahPEg-long] Cahiers de recherches de l'Institut de papyrologie et d'égyptologie de Lille. Sociétés urbaines en Égypte et au Soudan 
\item[CahRhod-long] Cahiers rhodaniens 
\item[CahTun-long] Cahiers de Tunisie 
\item[CalifStClAnt-long] California Studies in Classical Antiquity 
\item[CambrAJ-long] Cambridge Archaeological Journal 
\item[CArch-long] Cahiers archéologiques 
\item[CarinthiaI-long] Carinthia I. Geschichtliche und volkskundliche Beiträge zur Heimatkunde Kärntens %*Abweichung!
\item[CarnuntumJb-long] Carnuntum-Jahrbuch. Zeitschrift für Archäologie und Kulturgeschichte des Donauraumes 
\item[Carpica-long] CarpicaMuzeuljudeeandeistorieiartBacu 
\item[Carrobbio-long] Il Carrobbio. Rivista di studi bolognesi 
\item[CAT-long] Ch. W. Clairmont, Classical Attic Tombstones (Kilchberg 1993--1995) 
\item[CE-long] Cuadernos emeritenses 
\item[CEDAC-long] CEDAC. Bulletin. Centre d'études et de documentation archéologique de la conservation de Cartage 
\item[CEFR-long] Collection de l'École française de Rome 
\item[Celticum-long] Celticum. Supplément à Ogam 
\item[CercA-long] Cercetriarheologice 
\item[CercNum-long] Cercetări numismatice. Muzeul naţional de istorie 
\item[Chiron-long] Chiron. Mitteilungen der Kommission für Alte Geschichte und Epigraphik des Deutschen Archäologischen Instituts 
\item[ChronEg-long] Chronique d'Égypte 
\item[CIA-long] Corpus inscriptionum Atticarum 
\item[CIE-long] Corpus inscriptionum Etruscarum 
\item[CIG-long] Corpus inscriptionum Graecarum 
\item[CIH-long] Corpus inscriptionum Semiticarum. Pars quarta. Inscriptiones himyariticas et sabaeas continens 
\item[CIL-long] Corpus inscriptionum Latinarum 
\item[CincArtB-long] The Cincinnati Art Museum Bulletin 
\item[CIS-long] Corpus inscriptionum Semiticarum 
\item[CIstAMilano-long] Contributi dell'Istituto di archeologia. Pubblicazioni dell'Università cattolica del Sacro Cuore, Milano 
\item[CivClCr-long] Civiltà classica e cristiana 
\item[CivPad-long] Civiltà padana. Archeologia e storia del territorio 
\item[ClAnt-long] Classical Antiquity 
\item[ClevStHistArt-long] Cleveland Studies in the History of Art 
\item[Clio-long] Clio. Revista do Centro de história da Universidade Lisboa 
\item[ClIre-long] Classics Ireland 
\item[ClJ-long] The Classical Journal 
\item[ClMediaev-long] Classica et mediaevalia. Revue danoise de philologie et d'histoire 
\item[ClPhil-long] Classical Philology 
\item[ClQ-long] The Classical Quarterly 
\item[ClR-long] The Classical Review 
\item[ClRh-long] Clara Rhodos 
\item[CMatAOr-long] Contributi e materiali di archeologia orientale 
\item[CMGr-long] Convegni di studi sulla Magna Grecia 
\item[CMS-long] Corpus der minoischen und mykenischen Siegel 
\item[CoinHoards-long] Coin Hoards. The Royal Numismatic Society, London %*Abweichung!
\item[ColloquiSod-long] Colloqui del Sodalizio 
\item[CommunicAHung-long] Communicationes archaeologicae hungaricae 
\item[Complutum-long] Complutum. Publicaciones del Departamento de prehistoria de la Universidad complutense de Madrid 
\item[Conoscenze-long] Conoscenze. Rivista annuale della Soprintendenza archeologica e per i beni ambientali, architettonici, artistici e storici del Molise 
\item[Corduba-long] Corduba archaeologica 
\item[Corinth-long] Corinth. Results of Excavations Conducted by the American School of Classical Studies at Athens 
\item[CRAI-long] Académie des inscriptions et belles-lettres. Comptes rendus des séances de l'Académie 
\item[CretAnt-long] Creta antica. Rivista annuale di studi archeologici, storici ed epigrafici 
\item[CretSt-long] Cretan Studies 
\item[CronA-long] Cronache di archeologia 
\item[CronErcol-long] Cronache ercolanesi. Bollettino del Centro internazionale per lo studio dei papiri ercolanesi 
\item[CronPomp-long] Cronache pompeiane 
\item[CRPetersbourg-long] Compte-rendu de la Commission impériale archéologique, St. Pétersbourg %*Abweichung!
\item[CSE-long] Corpus speculorum Etruscorum 
\item[CSIR-long] Corpus signorum Imperii Romani 
\item[CSSpecPisa-long] Contributi della Scuola di specializzazione in archeologia dell'Università degli studi di Pisa 
\item[CuadAMed-long] Cuadernos de arqueología mediterránea 
\item[CuadArquitRom-long] Cuadernos de arquitectura romana 
\item[CuadCastellon-long] Cuadernos de prehistoria y arqueología castellonense 
\item[CuadCat-long] Cuaderni catanesi di studi classici e medievali 
\item[CuadFilCl-long] Cuadernos de filología clásica. Facultad de letras y filosofía, Universidad de Madrid 
\item[CuadGallegos-long] Cuadernos de estudios gallegos 
\item[CuadGranada-long] Cuadernos de prehistoria de la Universidad de Granada 
\item[CuadNavarra-long] Cuadernos de arqueología de la Universidad de Navarra 
\item[CuadPrehistA-long] Cuadernos de prehistoria y arqueología. Universidad autónoma de Madrid 
\item[CuadRom-long] Cuadernos de trabajos de la Escuela española de historia y arqueología en Roma 
\item[CuPaUAM-long] Cuadernos de prehistoria y arqueología. Universidad autónoma de Madrid 
\item[CVA-long] Corpus vasorum antiquorum 
\item[CZero-long] Cota Zero. Revista d'arqueologia i ciencia 
\item[DAA-long] Denkmäler antiker Architektur 
\item[Dacia-long] Dacia. Revue d'archéologie et d'histoire ancienne 
\item[DACL-long] Dictionnaire d'archéologie chrétienne et de liturgie 
\item[Dacoromania-long] Dacoromania. Jahrbuch für östliche Latinität 
\item[DaF-long] Damaszener Forschungen 
\item[Daidalos-long] Daidalos. Studi e ricerche del Dipartimento di scienze del mondo antico 
\item[DAIGeschDok-long] Das Deutsche Archäologische Institut. Geschichte und Dokumente 
\item[DaM-long] Damaszener Mitteilungen 
\item[Daremberg-Saglio-long] Dictionnaire des antiquités grecques et romaines d'après les textes et les monuments. Ouvrage rédigé sous la direction de Ch. Daremberg et E. Saglio %*Abweichung!
\item[DebrecMuzEvk-long] A Debreceni Déri múzeum évkönyve 
\item[Dedalo-long] Dédalo. Revista de arte e arqueologia %*Abweichung!
\item[Delos-long] Exploration archéologique de Délos faite par l'École française d'Athènes %*Abweichung!
\item[DeltChrA-long] Δελτίον της Χριστιανικής Αρχαιολογικής Εταιρείας 
\item[Demircihueyuek-long] Demircihüyük. Die Ergebnisse der Ausgrabungen 1975--1978 %*Abweichung!
\item[DeMuseus-long] De museus. Quaderns de museologia i museografia 
\item[DenkmPflBadWuert-long] Denkmalpflege in Baden-Württemberg %*Abweichung!
\item[DenkschrWien-long] Österreichische Akademie der Wissenschaften, Philosophisch-Historische Klasse. Denkschriften 
\item[Diadora-long] Diadora. Glasilo arheoloskog muzeja u Zadru 
\item[DialA-long] Dialoghi di archeologia 
\item[DialHistAnc-long] Dialogues d'histoire ancienne 
\item[Dike-long] Dike. Rivista di storia del diritto greco ed ellenistico. Università degli studi di Milano. Facoltà di giurisprudenza 
\item[Dioniso-long] Dioniso. Annale della Fondazione INDA, Istituto nazionale del dramma antico 
\item[DiskAB-long] Diskussionen zur archäologischen Bauforschung 
\item[DKuDenkmPfl-long] Deutsche Kunst und Denkmalpflege 
\item[DLZ-long] Deutsche Literaturzeitung für Kritik der internationalen Wissenschaft 
\item[DNP-long] Der Neue Pauly. Enzyklopädie der Antike 
\item[DocAlb-long] Documenta Albana 
\item[DocALouv-long] Documents d'archéologie régionale. Université catholique de Louvain 
\item[DocAMerid-long] Documents d'archéologie méridionale 
\item[DocEmRom-long] Documenti. Istituto per i beni artistici, culturali, naturali della regione Emilia-Romagna 
\item[Dodone-long] Δωδώνη 
\item[DOP-long] Dumbarton Oaks Papers 
\item[DossAlet-long] Les dossiers du Centre régional archéologique d'Alet 
\item[DossAParis-long] Les dossiers d'archéologie 
\item[Dura-Europos-long] The Excavations at Dura-Europos Conducted by Yale University and the French Academy of Inscriptions and Letters 
\item[EAA-long] Enciclopedia dell'arte antica classica e orientale 
\item[EAE-long] Excavaciones arqueológicas en España 
\item[EastWest-long] East and West 
\item[EcAntNimes-long] École antique de Nîmes. Bulletin annuel 
\item[EchosCl-long] Echos du monde classique. Classical Views 
\item[eDAI-F-long] e-Forschungsberichte des Deutschen Archäologischen Instituts 
\item[eDAI-J-long] e-Jahresberichte des Deutschen Archäologischen Instituts 
\item[EgA-long] Egyptian Archaeology. The Bulletin of the Egypt Exploration Society 
\item[Egnatia-long] Εγνατία. Επιστημονική Επετηρίδα της Φιλοσοφικής Σχολής, Αριστοτέλειο Πανεπιστήμιο Θεσσαλονίκης, Τμήμα Ιστορίας και Αρχαιολογίας 
\item[EgVicOr-long] Egitto e Vicino Oriente 
\item[Eikasmos-long] Εικασμός. Quaderni bolognesi di filologia classica 
\item[Eirene-long] Eirene. Studia Graeca et Latina 
\item[Elenchos-long] Elenchos. Rivista di studi sul pensiero antico 
\item[Ellenika-long] Ελληνικά. Φιλολογικόν, Ιστορικόν και Λαογραφικόν Περιοδικόν Σύγγραμμα 
\item[Emerita-long] Emerita. Revista de linguistica y filología clasica 
\item[EmPrerom-long] Emilia preromana 
\item[Empuries-long] Empúries. Revista de prehistòria, arqueologia i etnologia %*Abweichung!
\item[Enalia-long] Ενάλια 
\item[EnaliaAnn-long] Enalia. Annual. English Edition of the Hellenic Institute of Marine Archaeology 
\item[Enchoria-long] Enchoria. Zeitschrift für Demotistik und Koptologie 
\item[Eos-long] Eos. Commentarii Societatis philologae Polonorum 
\item[EpetBoiotMel-long] Επετηρίς της Εταιρείας Βοιωτικών Μελετών 
\item[EpetByzSpud-long] Επετηρίς της Εταιρείας Βυζαντινών Σπουδών 
\item[EpetKyklMel-long] Επετηρίς της Εταιρείας Κυκλαδικών Μελετών 
\item[EphemDac-long] Ephemeris Dacoromana. Annuario della Scuola Romena di Roma 
\item[EphemNapoc-long] Ephemeris Napocensis 
\item[EpigrAnat-long] Epigraphica Anatolica. Zeitschrift für Epigraphik und historische Geographie Anatoliens 
\item[EpistEpetAth-long] Επιστημονική Επετηρίς της Φιλοσοφικής Σχολής του Πανεπιστημίου Αθηνών 
\item[EpistEpetPolytThess-long] Επιστημονική Επετηρίδα της Πολυτεχνικής Σχολής, Αριστοτέλειο Πανεπιστήμιο Θεσσαλονίκης, Τμήμα Αρχιτεκτόνων 
\item[EpistEpetThess-long] Επιστημονική Επετηρίδα της Φιλοσοφικής Σχολής του Πανεπιστημίου Θεσσαλονίκης 
\item[EPRO-long] Études préliminaires aux religions orientales dans l'empire romain 
\item[Eranos-long] Eranos. Acta philologica Suecana 
\item[EranosJb-long] Eranos-Jahrbuch 
\item[Eretria-long] Eretria. Fouilles et recherches 
\item[Eretz-Israel-long] Eretz-Israel. Archaeological, Historical and Geographical Studies 
\item[Ergon-long] Το Έργον της Αρχαιολογικής Εταιρείας 
\item[ESA-long] Eurasia septentrionalis antiqua 
\item[EspacioHist-long] Espacio, tiempo y forma. Revista de la Facultad de geografia e historia. Serie 2, Historia antigua 
\item[EstMadr-long] Estudios de prehistoria y arqueología madrileñas 
\item[EstZaragoza-long] Estudios del Seminario de prehistoria, arqueología e historia antigua de la Facultad de filosofía y letras de Zaragoza 
\item[EtACl-long] Études d'archéologie classique 
\item[EtCl-long] Les études classiques. Revue trimestrielle de recherche et d'enseignement 
\item[EtClAix-long] Études classiques. Faculté des lettres et sciences humaines d'Aix 
\item[EtCret-long] Études crétoises 
\item[Ethnos-long] Ethnos. Revista do Instituto português de arqueología, história e etnografia 
\item[EtP-long] Études de papyrologie 
\item[EtPezenas-long] Études sur Pézenas et l'Hérault %*Abweichung!
\item[EtrSt-long] Etruscan Studies. Journal of the Etruscan Foundation 
\item[Etruscans-long] Etruscans. Bulletin of the Etruscan Foundation 
\item[EtTrav-long] Études et travaux. Studia i prace. Travaux du Centre d'archéologie méditerranéenne de l'Académie des sciences polonaise 
\item[Eulimene-long] Ευλιμένη (Μεσογειακή Αρχαιολογική Εταιρεία) 
\item[Eunomia-long] Eunomia. Ephemeridis Listy filologické supplementum 
\item[Euphrosyne-long] Euphrosyne. Revista de filologia clássica 
\item[EurAnt-long] Eurasia antiqua 
\item[EurRHist-long] European Review of History. Revue européenne d'histoire 
\item[Eutopia-long] Eutopia. Commentarii novi de antiquitatibus totius Europae 
\item[EVP-long] J. D. Beazley, Etruscan Vase Painting (Oxford 1947) 
\item[ExcIsr-long] Excavations and Surveys in Israel 
\item[Expedition-long] Expedition. The Magazine of Archaeology, Anthropology 
\item[ExtremA-long] Extremadura arqueológica 
\item[FA-long] Fasti archaeologici 
\item[FAAK-long] Forschungen zur Archäologie Außereuropäischer Kulturen 
\item[Faenza-long] Faenza. Bollettino del Museo internazionale delle ceramiche in Faenza. Rivista bimestrale di studi storici e di tecnica dell'arte ceramica 
\item[FAVA-long] Forschungen zur Allgemeinen und Vergleichenden Archäologie 
\item[Faventia-long] Faventia. Departement de clássiques, Facultat de letres, Universitat autónoma de Barcelona 
\item[FBerBadWuert-long] Forschungen und Berichte zur Vor- und Frühgeschichte in Baden-Württemberg %*Abweichung!
\item[FdC-long] Fouilles de Conimbriga. Publiées sous la direction de J. Alarcão et R. Etienne 
\item[FdD-long] Fouilles de Delphes 
\item[FdX-long] Fouilles de Xanthos 
\item[FeddersenWierde-long] Feddersen Wierde. Die Ergebnisse der Ausgrabung der vorgeschichtlichen Wurt Feddersen Wierde bei Bremerhaven in den Jahren 1955 bis 1963 %*Abweichung!
\item[FelRav-long] Felix Ravenna 
\item[FGrHist-long] F. Jacoby, Die Fragmente der griechischen Historiker 
\item[FHG-long] Fragmenta historicorum Graecorum 
\item[FiA-long] Forschungen in Augst 
\item[FichEpigr-long] Ficheiro epigráfico. Suplemento de »Conimbriga« 
\item[FiE-long] Forschungen in Ephesos 
\item[FIFAO-long] Fouilles de l'Institut français d'archéologie orientale du Caire 
\item[Figlina-long] Figlina. Documents du Laboratoire de céramologie de Lyon 
\item[Florentia-long] Florentia. Studi di archeologia 
\item[FlorIl-long] Florentia Iliberritana. Revista de estudios de antigüedad clásica 
\item[FMRD-long] Die Fundmünzen der römischen Zeit in Deutschland 
\item[FMROe-long] Fundmünzen der römischen Zeit in Österreich %*Abweichung!
\item[FoggArtMusAcqu-long] Fogg Art Museum. Acquisitions 
\item[FolA-long] Folia archaeologica 
\item[FolOr-long] Folia orientalia 
\item[Fonaments-long] Fonaments. Prehistòria i mon antic als Paisos Catalans 
\item[Fondamenti-long] Fondamenti. Rivista quadrimestrale di cultura 
\item[FontAPos-long] Fontes archaeologici Posnanienses 
\item[Fontes-long] Fontes. Rivista di filologia, iconografia e storia della tradizione classica 
\item[Forlimpopoli-long] Forlimpopoli. Documenti e studi 
\item[Fornvaennen-long] Fornvännen. Tidskrift för svensk antikvarisk forskning %*Abweichung!
\item[Forum-long] Forum. Revue du Groupe d'archéologie antique 
\item[FR-long] A. Furtwängler – K. Reichhold, Griechische Vasenmalerei (München 1900--1925) 
\item[FruehMitAltSt-long] Frühmittelalterliche Studien. Jahrbuch des Instituts für Frühmittelalterforschung der Universität Münster %*Abweichung!
\item[FuAusgrTrier-long] Funde und Ausgrabungen im Bezirk Trier 
\item[FuB-long] Forschungen und Berichte. Staatliche Museen zu Berlin 
\item[FuBerBadWuert-long] Fundberichte aus Baden-Württemberg %*Abweichung!
\item[FuBerHessen-long] Fundberichte aus Hessen 
\item[FuBerOe-long] Fundberichte aus Österreich %*Abweichung!
\item[FuBerSchwab-long] Fundberichte aus Schwaben 
\item[FuF-long] Forschungen und Fortschritte 
\item[FuWien-long] Fundort Wien. Berichte zur Archäologie 
\item[GacNum-long] Gaceta numismática 
\item[Gades-long] Gades. Revista del Colegio universitario de filosofía y letras 
\item[Gallaecia-long] Gallaecia. Publicación del Departamento de prehistoria y arqueología 
\item[Gallia-long] Gallia. Fouilles et monuments archéologiques en France metropolitaine 
\item[GalliaInf-long] Gallia informations. Préhistoire et histoire 
\item[GalliaInfAReg-long] Gallia informations. L'archéologie des régions 
\item[GalliaPrehist-long] Gallia préhistoire. Archéologie de la France préhistorique 
\item[GaR-long] Greece and Rome 
\item[GazBA-long] Gazette des beaux-arts 
\item[Genava-long] Genava. Revue d'histoire de l'art et d'archéologie 
\item[GeoAnt-long] Geographia antiqua. Rivista di geografia storica del mondo antico e di storia della geografia 
\item[Germania-long] Germania. Anzeiger der Römisch-Germanischen Kommission des Deutschen Archäologischen Instituts 
\item[Gesta-long] Gesta. International Center of Medieval Art 
\item[GettyMusJ-long] The J. Paul Getty Museum Journal 
\item[GFA-long] Göttinger Forum für Altertumswissenschaft 
\item[GGA-long] Göttingische Gelehrte Anzeigen 
\item[GiornFilFerr-long] Giornale filologico ferrarese 
\item[GiornItFil-long] Giornale italiano di filologia 
\item[GiornStorLun-long] Giornale storico della Lunigiana e del territorio lucense 
\item[GiRoccPalermo-long] Giglio di roccia 
\item[Gladius-long] Gladius. Estudios sobre armas antiguas, armamento, arte militar y vida cultural en Oriente y Occidente 
\item[GlasAJ-long] Glasgow Archaeological Journal 
\item[GlasBeograd-long] GlasnikSrpskoarheolokodrutvo 
\item[GlasSarajevo-long] Glasnik Zemaljskog muzeja Bosne i Hercegovie u Sarajevu. Arheologija 
\item[Glotta-long] Glotta. Zeitschrift für griechische und lateinische Sprache 
\item[Gnomon-long] Gnomon. Kritische Zeitschrift für die gesamte klassische Altertumswissenschaft 
\item[GodDepA-long] Godišnik na Departament archeologija 
\item[GodMuzPlov-long] Godišnik na Archeologičeski muzej Plovdiv. Annuaire du Musée archéologique Plovdiv 
\item[GodMuzSof-long] Godišnik na Nacionalnija archeologičeski muzej. Annuaire du Musée national archéologique (Sofia) 
\item[GodZborSkopje-long] Godišen zbornik na Filozofskiot fakultet na Universitetot vo Skopje 
\item[GorLet-long] Goriški letnik. Zbornik Goriškega muzeja 
\item[GoettMisz-long] Göttinger Miszellen. Beiträge zur ägyptologischen Diskussion %*Abweichung!
\item[GraRaspr-long] Građa i rasprave. Arheološki muzej Istre, Pula 
\item[GrazBeitr-long] Grazer Beiträge. Zeitschrift für die Klassische Altertumswissenschaft 
\item[GrLatOr-long] Graecolatina et orientalia. Zborník Filozofickej fakulty Univerzity Komenského 
\item[GrLatPrag-long] Graecolatina Pragensia. Acta Universitatis Carolinae. Philologica 
\item[GrRomByzSt-long] Greek, Roman and Byzantine Studies 
\item[Gymnasium-long] Gymnasium. Zeitschrift für Kultur der Antike und humanistische Bildung 
\item[Habis-long] Habis. Universidad de Sevilla. Arqueología, filología clásica 
\item[HallWPr-long] Hallisches Winckelmannsprogramm 
\item[Hama-long] Hama. Fouilles et recherches de la Fondation Carlsberg 
\item[HambBeitrA-long] Hamburger Beiträge zur Archäologie 
\item[HambBeitrNum-long] Hamburger Beiträge zur Numismatik 
\item[Handlingar-long] Kungliga vitterhets historie och antikvitets akademiens handlingar. Antikvariska serien 
\item[HarvStClPhil-long] Harvard Studies in Classical Philology 
\item[HarvTheolR-long] The Harvard Theological Review 
\item[HASB-long] Hefte des Archäologischen Seminars der Universität Bern 
\item[HAW-long] Handbuch der Altertumswissenschaften 
\item[HdArch-long] Handbuch der Archäologie 
\item[Head-long] B. V. Head, Historia Numorum. A Manual of Greek Numismatics (Oxford 1887; 1911) 
\item[Helbig-long] W. Helbig, Führer durch die öffentlichen Sammlungen klassischer Altertümer in Rom 
\item[Helike-long] Helike. Universidad nacional de educación a distancia, Centro regional de Elche 
\item[Helikon-long] Helikon. Rivista di tradizione e cultura classica 
\item[Helinium-long] Helinium. Revue consacrée à l'archéologie des Pays-Bas, de la Belgique et du Grand-Duché de Luxembourg 
\item[Helios-long] Helios. A Journal Devoted to Critical and Methodological Studies of Classical Culture, Literature and Society 
\item[HellenikaJb-long] Hellenika. Jahrbuch für die Freunde Griechenlands 
\item[HelvA-long] Helvetia archaeologica 
\item[Hephaistos-long] Hephaistos. Kritische Zeitschrift zur Theorie und Praxis der Archäologie und angrenzender Wissenschaften 
\item[Hermes-long] Hermes. Zeitschrift für klassische Philologie 
\item[Herrscherbild-long] Das römische Herrscherbild 
\item[Hesperia-long] Hesperia. Journal of the American School of Classical Studies at Athens 
\item[Hispania-long] Hispania. Revista española de historia 
\item[HispAnt-long] Hispania antiqua. Revista de historia antigua 
\item[HispAntEpigr-long] Hispania antiqua epigraphica 
\item[HispEpigr-long] Hispania epigraphica 
\item[HistAnthr-long] Historische Anthropologie. Kultur, Gesellschaft, Alltag 
\item[HistArt-long] Histoire de l'art. Bulletin d'information de l'Institut National d'Histoire de l'Art 
\item[Historia-long] Historia. Zeitschrift für Alte Geschichte 
\item[Historica-long] Historica. Academia RSR. Centrul de istorie, filologie şi etnografie din Craiova 
\item[Histria-long] Histria. Les résultats des fouilles 
\item[HistriaA-long] Histria archaeologica 
\item[HistriaAnt-long] Histria antiqua. Casopis Meunarodnog Istraivakog Centra za Arheologiju. Journal of the International Research Centre for Archeology 
\item[HistSprF-long] Historische Sprachforschung 
\item[HKL-long] R. Borger, Handbuch der Keilschriftliteratur (Berlin 1967--1975) 
\item[Horos-long] Ηόρος. Ἔνα Ἀρχαιογνωστικò Περιοδικό 
\item[HSS-long] Harvard Semitic Series 
\item[HuelvaA-long] Huelva arqueológica 
\item[HumBild-long] Humanistische Bildung 
\item[Hyp-long] Hyperboreus. Studia classica 
\item[HZ-long] Historische Zeitschrift 
\item[IA-long] Iberia archaeologica 
\item[Iberia-long] Iberia. Revista della antigüedad 
\item[IEJ-long] Israel Exploration Journal 
\item[IG-long] Inscriptiones Graecae 
\item[IGCH-long] M. Thompson – C. M. Kraay – O. Mørkholm, An Inventory of Greek Coin Hoards (New York 1973) 
\item[IGR-long] Inscriptiones Graecae ad res Romanas pertinentes 
\item[IK-long] Inschriften griechischer Städte aus Kleinasien 
\item[Ilerda-long] Ilerda. Instituto de estudios ilerdenses 
\item[Iliria-long] Iliria. Revistë arkeologjike 
\item[IllinClSt-long] Illinois Classical Studies 
\item[ILN-long] The Illustrated London News 
\item[ILS-long] H. Dessau, Inscriptiones Latinae selectae (Berlin 1892--1916) 
\item[IndexQuad-long] Index. Quaderni camerti di studi romanistici 
\item[IndogermF-long] Indogermanische Forschungen 
\item[IndUnArtB-long] Indiana University Art Museum Bulletin 
\item[InsFulc-long] Insula Fulcheria 
\item[InstNautAQ-long] The Institute of Nautical Archaeology Quarterly 
\item[IntJClTrad-long] International Journal of the Classical Tradition 
\item[IntJNautA-long] International Journal of Nautical Archaeology 
\item[IntZSchauBibelWiss-long] Internationale Zeitschriftenschau für Bibelwissenschaft und Grenzgebiete 
\item[InvLuc-long] Invigilata lucernis 
\item[Ipek-long] Jahrbuch für prähistorische und ethnographische Kunst 
\item[Iran-long] Iran. Journal of the British Institute of Persian Studies 
\item[IrAnt-long] Iranica antiqua 
\item[IsrMusJ-long] The Israel Museum Journal 
\item[IsrMusN-long] The Israel Museum News 
\item[IsrMusStA-long] Israel Museum Studies in Archaeology. An Annual Publication by the Samuel Bronfman Biblical and Archaeological Museum of the Israel Museum, Jerusalem 
\item[IsrNumJ-long] Israel Numismatic Journal 
\item[IstanbAMuezYil-long] Istanbul Arkeoloji Müzeleri yıllığı %*Abweichung!
\item[IstForsch-long] Istanbuler Forschungen 
\item[Isthmia-long] Isthmia. Excavations by the University of Chicago under the Auspices of the American School of Classical Studies at Athens 
\item[IstMitt-long] Istanbuler Mitteilungen 
\item[Italica-long] Italica. Cuadernos de trabajos de la Escuela española de historia y arqueología en Roma 
\item[ItNostr-long] Italia nostra 
\item[IzvBurgas-long] Izvestija na Narodnija muzej Burgas. Bulletin du Musée national de Bourgas 
\item[IzvMuzJuzBalg-long] Izvestija na muzeite ot Južna Bălgarija. Bulletin des musées de la Bulgarie du Sud %*Abweichung
\item[IzvVarna-long] Izvestija na Narodnija muzej Varna 
\item[Jabega-long] Jábega. Revista de la Disputación provincial de Málaga %*Abweichung!
\item[JadrZbor-long] Jadranski zbornik. Prilozi za povijest Istre, Rijeke i Hrvatskog primorja 
\item[JAOS-long] Journal of the American Oriental Society 
\item[JARCE-long] Journal of the American Research Center in Egypt 
\item[JASc-long] Journal of Archaeological Science 
\item[JbAC-long] Jahrbuch für Antike und Christentum 
\item[JbAkMainz-long] Jahrbuch. Akademie der Wissenschaften und der Literatur, Mainz 
\item[JbBadWuert-long] Jahrbuch der Staatlichen Kunstsammlungen in Baden-Württemberg %*Abweichung!
\item[JbBerlMus-long] Jahrbuch der Berliner Museen 
\item[JbBernHistMus-long] Jahrbuch des Bernischen Historischen Museums in Bern 
\item[JberAugst-long] Jahresberichte aus Augst und Kaiseraugst 
\item[JberBasel-long] Jahresbericht der Archäologischen Bodenforschung des Kantons Basel-Stadt 
\item[JberBayDenkmPfl-long] Jahresbericht der Bayerischen Bodendenkmalpflege 
\item[JberProVindon-long] Jahresbericht. Gesellschaft Pro Vindonissa 
\item[JberVgFrankf-long] Jahresbericht des Instituts für Vorgeschichte der Universität Frankfurt a. M. 
\item[JberZuerich-long] Jahresbericht. Schweizerisches Landesmuseum Zürich %*Abweichung!
\item[JbGoett-long] Jahrbuch der Akademie der Wissenschaften in Göttingen %*Abweichung!
\item[JbHambKuSamml-long] Jahrbuch der Hamburger Kunstsammlungen 
\item[JbKHMWien-long] Jahrbuch des Kunsthistorischen Museums Wien 
\item[JbKHSWien-long] Jahrbuch der Kunsthistorischen Sammlungen in Wien 
\item[JbKleinasF-long] Jahrbuch für kleinasiatische Forschung 
\item[JbMuench-long] Bayerische Akademie der Wissenschaften. Jahrbuch %*Abweichung!
\item[JbMusKGHamb-long] Jahrbuch des Museums für Kunst und Gewerbe, Hamburg 
\item[JbMusLinz-long] Jahrbuch des Oberösterreichischen Musealvereins 
\item[JbOeByz-long] Jahrbuch der Österreichischen Byzantinistik %*Abweichung!
\item[JbPreussKul-long] Jahrbuch Preussischer Kulturbesitz 
\item[JbRGZM-long] Jahrbuch des Römisch-Germanischen Zentralmuseums Mainz 
\item[JbSchwUrgesch-long] Jahrbuch der Schweizerischen Gesellschaft für Ur- und Frühgeschichte 
\item[JCS-long] Journal of Cuneiform Studies 
\item[JdI-long] Jahrbuch des Deutschen Archäologischen Instituts 
\item[JEA-long] The Journal of Egyptian Archaeology 
\item[JEChrSt-long] Journal of Early Christian Studies. Journal of the North American Patristics Society 
\item[JEOL-long] Jaarbericht van het Vooraziatisch-Egyptisch Genootschap Ex Oriente Lux 
\item[JewelSt-long] Jewellery Studies 
\item[JFieldA-long] Journal of Field Archaeology 
\item[JGS-long] Journal of Glass Studies 
\item[JHS-long] The Journal of Hellenic Studies 
\item[JIbA-long] Journal of Iberian Archaeology 
\item[JJurP-long] The Journal of Juristic Papyrology 
\item[JKuGesch-long] Journal für Kunstgeschichte 
\item[JMedA-long] Journal of Mediterranean Archaeology 
\item[JMedAnthrA-long] Journal of Mediterranean Anthropology and Archaeology 
\item[JMithrSt-long] Journal of Mithraic Studies 
\item[JNES-long] Journal of Near Eastern Studies 
\item[JNG-long] Jahrbuch für Numismatik und Geldgeschichte 
\item[JPrehistRel-long] Journal of Prehistoric Religion 
\item[JRA-long] Journal of Roman Archaeology 
\item[JRomMilSt-long] Journal of Roman Military Equipment Studies 
\item[JRomPotSt-long] Journal of Roman Pottery Studies 
\item[JRS-long] The Journal of Roman Studies 
\item[JSav-long] Journal des savants 
\item[JSchrVgHalle-long] Jahresschrift für mitteldeutsche Vorgeschichte 
\item[JSS-long] Journal of Semitic Studies 
\item[JTheorA-long] Journal of Theoretical Archaeology 
\item[Jura-long] Iura. Rivista internazionale di diritto romano e antico 
\item[JWaltersArtGal-long] The Journal of the Walters Art Gallery 
\item[JWCI-long] Journal of the Warburg and Courtauld Institutes 
\item[Kadmos-long] Kadmos. Zeitschrift für vor- und frühgriechische Epigraphik 
\item[Kairos-long] Kairos. Zeitschrift für Judaistik und Religionswissenschaft 
\item[Kalapodi-long] Kalapodi. Ergebnisse der Ausgrabungen im Heiligtum der Artemis und des Apollon von Hyampolis in der antiken Phokis 
\item[Kalathos-long] Kalathos. Revista del Seminario de arqueología y etnología Turolense, Universidad de Teruel 
\item[Kalos-long] Kalós. Arte in Sicilia %*Abweichung!
\item[Karthago-long] Karthago. Revue d'archéologie mediterranéenne 
\item[Kemi-long] Kêmi. Revue de philologie et d'archéologie égytiennes et coptes %*Abweichung!
\item[Kenchreai-long] Kenchreai. Eastern Port of Corinth. Results of Investigations by the University of Chicago and Indiana University for the American School of Classical Studies at Athens 
\item[KentAR-long] Kent Archaeological Review 
\item[Keos-long] Keos. Results of Excavations Conducted by the University of Cincinnati under the Auspices of the American School of Classical Studies at Athens 
\item[Kerameikos-long] Kerameikos. Ergebnisse der Ausgrabungen 
\item[Kernos-long] Kernos. Revue internationale et pluridisciplinaire de religion grecque antique 
\item[Klearchos-long] Klearchos. Bollettino dell'Associazione Amici del Museo nazionale di Reggio Calabria 
\item[Kleos-long] Kleos. Estemporaneo di studi e testi sulla fortuna dell'antico 
\item[Klio-long] Klio. Beiträge zur alten Geschichte 
\item[Kodai-long] Kodai. Journal of Ancient History 
\item[KoelnJb-long] Kölner Jahrbuch %*Abweichung!
\item[KoelnMusB-long] Kölner Museums-Bulletin. Berichte und Forschungen aus den Museen der Stadt Köln %*Abweichung!
\item[Kokalos-long] Kώκαλος. Studi pubblicati dall’Istituto di storia antica dell’Università di Palermo %*Abweichung!
\item[KollAVA-long] Kolloquien zur Allgemeinen und Vergleichenden Archäologie 
\item[Kratylos-long] Kratylos. Kritisches Berichts- und Rezensionsorgan für indogermanische und allgemeine Sprachwissenschaft 
\item[KretChron-long] Κρητικά Χρονικά 
\item[KSIA-long] Kratkie soobščenija o dokladach i polevych issledovanijach Instituta archeologii 
\item[KSIAKiev-long] Kratkie soobščenija Instituta archeologii, Kiev 
\item[KST-long] Kazı Sonuçları Toplantısı 
\item[Ktema-long] Ktema. Civilisations de l'Orient, de la Grèce et de Rome antiques 
\item[KuGeschAnz-long] Kunstgeschichtliche Anzeigen 
\item[Kuml-long] Kuml. Årbog for Jysk Arkaeologisk Selskab 
\item[Kunstchronik-long] Kunstchronik. Monatsschrift für Kunstwissenschaft, Museumswesen und Denkmalpflege, Mitteilungsblatt des Verbandes Deutscher Kunsthistoriker 
\item[KuOr-long] Kunst des Orients 
\item[Kush-long] Kush. Journal of the National Corporation for Antiquities and Museums (NCAM) 
\item[KuWeltBerlMus-long] Kunst der Welt in den Berliner Museen 
\item[KypA-long] Κυπριακή Αρχαιολογία. Archaeologia Cypria 
\item[KypSpud-long] Κυπριακαì Σπουδαί 
\item[Labeo-long] Labeo. Rassegna di ritritto romano 
\item[LAe-long] Lexikon der Ägyptologie %*Abweichung!
\item[Laietania-long] Laietania. Estudios d'arqueologia del Maresme 
\item[Lampas-long] Lampas. Tijdschrift voor nederlandse classici 
\item[Lancia-long] Lancia. Revista de prehistoria, arqueología e historia antigua del noreste peninsular 
\item[LandKunVierJBl-long] Landeskundliche Vierteljahrsblätter. Trier 
\item[LangOrAnc-long] Langues orientales anciennes. Philologie et linguistique 
\item[Latinitas-long] Latinitas. Commentarii linguae Latinae excolendae provehendae 
\item[Latomus-long] Latomus. Revue d'études latines 
\item[Laverna-long] Laverna. Beiträge zur Wirtschafts- und Sozialgeschichte der Alten Welt 
\item[LCS-long] A. D. Trendall, The Red-figured Vases of Lucania, Campania and Sicily (Oxford 1967--­1983) 
\item[Levant-long] Levant. Journal of the British School of Archaeology in Jerusalem and the British Institute at Amman for Archaeology and History 
\item[Lexis-long] Lexis. Poetica, retorica e comunicazione nella tradizione classica 
\item[LF-long] Listy filologické 
\item[LibSt-long] Libyan Studies 
\item[LibyaAnt-long] Libya antiqua 
\item[LibycaBServAnt-long] Libyca. Bulletin du Service des antiquités. Archéologie, épigraphie 
\item[LibycaTrav-long] Libyca. Travaux du Laboratoire d'anthropologie et d'archéologie préhistorique du Musée du Bardo 
\item[LSJ-long] G. Liddell – R. Scott – H. S. Jones, A Greek-English Lexikon \textsuperscript{9}(1996); Suppl. (1996) %*Abweichung!
\item[LIMC-long] Lexikon iconographicum mythologiae classicae 
\item[Limesforschungen-long] Limesforschungen. Studien zur Organisation der römischen Reichsgrenze an Rhein und Donau 
\item[Lindos-long] Lindos. Fouilles et recherches 
\item[LingIt-long] Linguistica, epigrafia, filologia italica 
\item[LTUR-long] Lexikon topographicum urbis Romae 
\item[Lucentum-long] Lucentum. Anales de la Universidad de Alicante. Prehistoria, arqueología e historia antigua 
\item[LundAR-long] Lund Archaeological Review 
\item[Lustrum-long] Lustrum. Internationale Forschungsberichte aus dem Bereich des klassischen Altertums 
\item[Lykia-long] Lykia. Anadolu-akdeniz kültürleri 
\item[MacActaA-long] Macedoniae acta archaeologica 
\item[Maecenas-long] Maecenas. Studi sul mondo classico 
\item[MAGesGraz-long] Mitteilungen der Archäologischen Gesellschaft Graz 
\item[MAGesStei-long] Mitteilungen der Archäologischen Gesellschaft Steiermark 
\item[Maia-long] Maia. Rivista di letterature classiche 
\item[Mainake-long] Mainake. Estudios de arqueología Malagueña 
\item[MAInstUngAk-long] Mitteilungen des Archäologischen Instituts der Ungarischen Akademie der Wissenschaften 
\item[MainzZ-long] Mainzer Zeitschrift 
\item[MakedNasl-long] Makedonsko nasledstvo. Spisanie za arheologija, istorija, istorija na umetnosta i etnologija 
\item[Makedonika-long] Μακεδονικά. Σύγγραμμα Περιοδικόν της Εταιρείας Μακεδονικών Σπουδών 
\item[MAMA-long] Monumenta Asiae Minoris antiqua. Publications of the American Society for Archaeological Research in Asia Minor 
\item[MAnthrWien-long] Mitteilungen der Anthropologischen Gesellschaft in Wien 
\item[MAR-long] Monumenta artis Romanae 
\item[MarbWPr-long] Marburger Winckelmann-Programm 
\item[Marche-long] Le Marche. Archeologia, storia, territorio 
\item[Mari-long] Mari. Annales de recherches interdisciplinaires 
\item[Marisia-long] Marisia. Studii şi materiale. Arheologie, istorie, etnografie 
\item[MarNero-long] Il Mar Nero. Annali di archeologia e storia 
\item[Marsyas-long] Marsyas. Studies in the History of Art 
\item[MascaJ-long] Masca Journal. Museum Applied Science Center for Archaeology, University of Pennsylvania 
\item[MascaP-long] Masca Research Papers in Science and Archaeology 
\item[Mastia-long] Mastia. Revista del Museo arqueológico municipal de Cartagena 
\item[MatABSSR-long] Materialy po archeologii BSSR 
\item[MatASevPri-long] Materialy po archeologii severnogo Pričernomor’ja 
\item[MatCercA-long] Materiale şi cercetări arheologice 
\item[MatIsslA-long] Materialy i issledovanija po archeologii SSSR 
\item[MatStar-long] Materiały starożytne 
\item[MatStarWczes-long] Materiały starożytne i wczesnośredniowieczne 
\item[MatTestiCl-long] Materiali e discussioni per l'analisi dei testi classici 
\item[MatWczes-long] Materiały wczesnośredniowieczne 
\item[MAVA-long] Materialien zur Allgemeinen und Vergleichenden Archäologie 
\item[MB-long] Madrider Beiträge 
\item[MBAH-long] Marburger Beiträge zur antiken Handels-, Wirtschafts- und Sozialgeschichte, vor Jahrgang 2009 Münstersche Beiträge zur antiken Handelsgeschichte 
\item[MBlVFruehGesch-long] Mitteilungsblatt für Vor- und Frühgeschichte %*Abweichung!
\item[MDAIK-long] Mitteilungen des Deutschen Archäologischen Instituts, Abteilung Kairo 
\item[MDAVerb-long] Mitteilungen des Deutschen Archäologen-Verbandes e.V. 
\item[MdI-long] Mitteilungen des Deutschen Archäologischen Instituts 
\item[MDOG-long] Mitteilungen der Deutschen Orient-Gesellschaft zu Berlin 
\item[Meander-long] Meander. Miesięcznik poświęcony kulturze świata starożytnego 
\item[MedA-long] Mediterranean Archaeology 
\item[MedAnt-long] Mediterraneo antico. Economie, società, culture 
\item[MeddelGlypt-long] Meddelelser fra Ny Carlsberg Glyptotek 
\item[MeddelLund-long] Meddelanden från Lunds universtitets historiska museum 
\item[MeddelThor-long] Meddelelser fra Thorvaldsens Museum 
\item[MededRom-long] Mededelingen van het Nederlands instituut te Rome 
\item[MedelhavsMusB-long] Medelhavsmuseet. Bulletin 
\item[MedHistR-long] Mediterranean Historical Review 
\item[MedievA-long] Medieval Archaeology. Journal of the Society for Medieval Archaeology 
\item[MediSec-long] Medicina nei secoli. Arte e scienza 
\item[MEFRA-long] Mélanges de l'École française de Rome. Antiquité 
\item[MelBeyrouth-long] Mélanges de l'Université Saint-Joseph 
\item[MelCasaVelazquez-long] Mélanges de la Casa de Velázquez. Antiquité et moyen âge 
\item[MemAcInscr-long] Mémoires de l'Académie des inscriptions et belles-lettres 
\item[MemAmAc-long] Memoirs of the American Academy in Rome 
\item[MemAnt-long] Memoria antiquitatis. Acta Musei Petrodavensis. Revista Muzeului arheologic Piatra Neamţ 
\item[MemAntFr-long] Mémoires de la Société nationale des antiquaires de France 
\item[MemBarcelA-long] Memoria. Universidad de Barcelona, Instituto de arqueología y prehistoria 
\item[MemBologna-long] Atti de la Accademia delle scienze dell'Istituto di Bologna. Memorie 
\item[MemHistAnt-long] Memorias de historia antigua (Universidad de Oviedo) 
\item[MemInstNatFr-long] Mémoires de l'Institut national de France 
\item[MemLinc-long] Atti dell'Accademia nazionale dei Lincei, Classe di scienze morali, storiche e filologiche. Memorie 
\item[MemNap-long] Memorie dell'Accademia di archeologia, lettere e belle arti di Napoli 
\item[MemPontAc-long] Atti della Pontificia accademia romana di archeologia. Memorie 
\item[MemStor-long] Memoria storica. Rivista del Centro studi storici terni 
\item[MemStorFriuli-long] Memorie storiche forogiuliesi 
\item[Merida-long] Mérida. Ciudad y patrimonio %*Abweichung!
\item[MeridaMem-long] Mérida. Excavaciones arqueológicas. Memoria %*Abweichung!
\item[Meroitica-long] Meroitica. Schriften zur altsudanesischen Geschichte und Archäologie 
\item[Mesopotamia-long] Mesopotamia. Rivista di archeologia 
\item[Messana-long] Messana. Rassegna di studi filologici, linguistici e storici 
\item[Metis-long] Métis. Revue d'anthropologie du monde grec ancien %*Abweichung!
\item[MetrMusJ-long] Metropolitan Museum Journal 
\item[MetrMusSt-long] Metropolitan Museum Studies 
\item[MF-long] Madrider Forschungen 
\item[MFruehChrOe-long] Mitteilungen zur frühchristlichen Archäologie in Österreich %*Abweichung!
\item[Milet-long] Milet. Ergebnisse der Ausgrabungen und Untersuchungen seit dem Jahr 1899 
\item[MilForsch-long] Milesische Forschungen 
\item[MinEpigrP-long] Minima epigraphica et papyrologica. Taccuini della cattedra e del laboratorio di epigrafia e papirologia giuridica dell'Università degli studi di Catanzaro »Magna Graecia« 
\item[Minerva-long] Minerva. Revista de filología clásica 
\item[Minos-long] Minos. Revista de filología egea 
\item[MInstWasser-long] Mitteilungen. Leichtweiss-Institut für Wasserbau der Technischen Universität Braunschweig 
\item[MIO-long] Mitteilungen des Instituts für Orientforschung 
\item[MiscCrAnt-long] Miscellanea di studi di letteratura christiana antica 
\item[MiscStStor-long] Miscellanea di studi storici 
\item[MitChrA-long] Mitteilungen zur christlichen Archäologie 
\item[MKT-long] Menschen – Kulturen – Traditionen. Studien aus den Forschungsclustern des Deutschen Archäologischen Instituts 
\item[MKuHistFlorenz-long] Mitteilungen des Kunsthistorischen Institutes in Florenz 
\item[MKul-long] Mitteilungen zur Kulturkunde 
\item[MM-long] Madrider Mitteilungen 
\item[Mnemosyne-long] Mnemosyne. A Journal of Classical Studies 
\item[MOeNumGes-long] Mitteilungen der Österreichischen Numismatischen Gesellschaft %*Abweichung!
\item[MonAnt-long] Monumenti antichi 
\item[MonInst-long] Monumenti inediti pubblicati dall'Istituto Archeologico 
\item[MonPiot-long] Monuments et mémoires. Fondation E. Piot 
\item[MonPitt-long] Monumenti della pittura antica scoperti in Italia 
\item[Mozia-long] Mozia. Rapporto preliminare della missione congiunta con la Soprintendenza alle antichità della Sicilia occidentale 
\item[MPraehistKomWien-long] Mitteilungen der Prähistorischen Kommission der Österreichischen Akademie der Wissenschaften %*Abweichung!
\item[MSAtene-long] Monografie della Scuola archeologica di Atene e delle missioni italiane in Oriente 
\item[MSchliemann-long] Mitteilungen aus dem Heinrich-Schliemann-Museum Ankershagen 
\item[MSchwUrFruehGesch-long] Mitteilungsblatt der Schweizerischen Gesellschaft für Ur- und Frühgeschichte %*Abweichung!
\item[MSpaetAByz-long] Mitteilungen zur spätantiken Archäologie und byzantinischen Kunstgeschichte %*Abweichung!
\item[MueJb-long] Münchner Jahrbuch der bildenden Kunst %*Abweichung!
\item[MuenchBeitrVFG-long] Münchner Beiträge zur Vor- und Frühgeschichte %*Abweichung!
\item[MuenchStSprWiss-long] Münchener Studien zur Sprachwissenschaft %*Abweichung!
\item[MuM-long] Münzen und Medaillen 
\item[MusAfr-long] Museum Africum. West African Journal of Classical and Related Studies 
\item[MusBenaki-long] Μουσείο Μπενάκη 
\item[MusCrit-long] Museum criticum 
\item[Muse-long] Muse. Annual of the Museum of Art and Archaeology, University of Missouri, Columbia 
\item[Museon-long] Le Muséon. Revue d'études orientales %*Abweichung!
\item[MuseumUnesco-long] Museum. A Quarterly Review Published by UNESCO 
\item[MusFerr-long] Musei Ferraresi. Bollettino annuale 
\item[MusGalIt-long] Musei e gallerie d'Italia 
\item[MusHaaretz-long] Museum Haaretz, Tel Aviv. Yearbook 
\item[MusHelv-long] Museum Helveticum 
\item[MusKoeln-long] Museen in Köln. Bulletin %*Abweichung!
\item[MusNotAmNumSoc-long] Museum Notes. The American Numismatic Society 
\item[MusPontevedra-long] El Museo de Pontevedra 
\item[MusRiv-long] Museo in rivista. Notiziario dei Musei civici di Pavia 
\item[MusTusc-long] Museum Tusculanum. København 
\item[MuzEvkSzeged-long] A Móra Ferenc múzeum évkönyve 
\item[MuzNat-long] Muzeul naţional. Bucureşti
\item[MuzPamKul-long] Muzei i pametnici na kulturata 
\item[NachrArbUWA-long] Nachrichtenblatt Arbeitskreis Unterwasserarchäologie 
\item[NapNobil-long] Napoli nobilissima 
\item[NassAnn-long] Nassauische Annalen 
\item[NAWG-long] Nachrichten der Akademie der Wissenschaften in Göttingen. Philologisch-Historische Klasse 
\item[NBWorcArtMus-long] News Bulletin and Calendar. Worcester Art Museum 
\item[NEphemSemEpigr-long] Neue Ephemeris für semitische Epigraphik 
\item[NewsletterAthen-long] Newsletter of the Netherlands Institute at Athens 
\item[NewsletterPotTech-long] Newsletter. Department of Pottery Technology, University of Leiden 
\item[NGWG-long] Nachrichten von der Gesellschaft der Wissenschaften zu Göttingen. Philologisch-Historische Klasse 
\item[NigCl-long] Nigeria and the Classics 
\item[Nikephoros-long] Nikephoros. Zeitschrift für Sport und Kultur im Altertum 
\item[Nin-long] Nin. Journal of Gender Studies in Antiquity 
\item[NNM-long] American Numismatic Society. Numismatic Notes and Monographs 
\item[NomChron-long] Νομισματικά Χρονικά. Περιοδική Έκδωσις της Ελληνικής Νομισματικής Εταιρείας 
\item[Norba-long] Norba. Revista de arte, geografía e historia 
\item[NordNumArs-long] Nordisk numismatisk årsskrift. Utgiven av Kungliga vitterhets historie och antikvitets akademien i samarbete med Nordisk numismatisk union 
\item[NotABerg-long] Notizie archeologiche bergomensi. Periodico di archeologia del Civico museo archeologico di Bergamo 
\item[NotAHisp-long] Noticiario arqueológico hispánico. Arqueología 
\item[NotAHispPrehistoria-long] Noticiario arqueológico hispánico. Prehistoria 
\item[NotAllumiere-long] Notiziario. Museo civico, Associazione archeologica, Allumiere 
\item[NotALomb-long] Notiziario. Soprintendenza archeologica della Lombardia 
\item[NotMilano-long] Notizie dal chiostro del monastero maggiore. Rassegna di studi del Civico museo archeologico e del Civico gabinetto numismatico di Milano 
\item[NouvClio-long] La nouvelle Clio 
\item[Novaensia-long] Novaensia. Badania Ekspedycji archeologicznej Uniwersytetu warszawskiego w Novae 
\item[NSc-long] Notizie degli scavi di antichità 
\item[NStFan-long] Nuovi studi fanesi 
\item[NubChr-long] Nubia christiana 
\item[NubLet-long] Nubian Letters 
\item[NueBlA-long] Nürnberger Blätter zur Archäologie. Publikationsreihe des Bildungszentrums der Stadt Nürnberg, Fachbereich Archäologie %*Abweichung!
\item[NumAntCl-long] Numismatica e antichità classiche. Quaderni ticinesi 
\item[Numantia-long] Numantia. Investigaciones arqueológicas en Castilla y León 
\item[NumChron-long] The Numismatic Chronicle. The Journal of the Royal Society 
\item[Numen-long] Numen. International Review for the History of Religions 
\item[NumEpigr-long] Numizmatika i epigrafika 
\item[Numisma-long] Numisma. Revista de la Sociedad iberoamericana de estudios numismáticos 
\item[NumismaticaRom-long] Numismatica. Periodico di cultura e di informazione numismatica 
\item[Numizmaticar-long] Numizmatičar. Casopis za anticki i stari jugoslavenski novac %*Abweichung!
\item[Nummus-long] Nummus. Sociedade portuguesa de numismatica 
\item[NumZ-long] Numismatische Zeitschrift 
\item[NuovDidask-long] Nuovo Didaskaleion 
\item[OccasPublClSt-long] Occasional Publications in Classical Studies 
\item[OccOr-long] Occident and Orient 
\item[OGIS-long] W. Dittenberger, Orientis Graeci inscriptiones selectae (Leipzig 1903--­1905) 
\item[OeJh-long] Jahreshefte des Österreichischen Archäologischen Institutes in Wien %*Abweichung!
\item[OF-long] Olympische Forschungen 
\item[Offa-long] Offa. Berichte und Mitteilungen zur Urgeschiche, Frühgeschichte und Mittelalterarchäologie 
\item[Ogam-long] Ogam. Bulletin des Amis de la tradition celtique 
\item[Oikumene-long] Oikumene. Studia ad historiam antiquam classicam et orientalem pertinentia 
\item[OIP-long] Oriental Institute Publications 
\item[Olba-long] Olba. Mersin Üniversitesi Kilikia Arkeolojisini Araştırma Merkezi yayınları 
\item[OlBer-long] Bericht über die Ausgrabungen in Olympia 
\item[Olympia-long] Olympia. Die Ergebnisse der von dem Deutschen Reich veranstalteten Ausgrabung (Berlin 1890--­1897) 
\item[Olynthus-long] Excavations at Olynthus 
\item[OLZ-long] Orientalistische Literaturzeitung 
\item[OpArch-long] Skrifter utgivna av Svenska institutet i Rom. Opuscula archaeologica 
\item[OpAth-long] Opuscula Atheniensia 
\item[OpFin-long] Opuscula Instituti Romani Finlandiae 
\item[OpPomp-long] Opuscula Pompeiana 
\item[OpRom-long] Opuscula Romana 
\item[Opus-long] Opus. Rivista internazionale per la storia economica e sociale dell’antichità 
\item[Or-long] Orientalia (Pontificio Istituto biblico) 
\item[OrA-long] Orient-Archäologie 
\item[OrAnt-long] Oriens antiquus. Rivista del Centro per le antichità e la storia dell’arte del Vicino Oriente 
\item[OrbTerr-long] Orbis terrarum. Internationale Zeitschrift für historische Geographie der Alten Welt 
\item[OrChr-long] Oriens christianus 
\item[OrChrPer-long] Orientalia christiana periodica 
\item[Ordona-long] Ordona. Rapports et études 
\item[Orient-long] Orient. The Reports of the Society for Near Eastern Studies in Japan 
\item[Origini-long] Origini. Preistoria e protostoria delle civiltà antiche 
\item[Orizzonti-long] Orizzonti. Rassegna di archeologia 
\item[Orpheus-long] Orpheus. Rivista di umanità classica e cristiana 
\item[OrpheusThracSt-long] Orpheus. Journal of Indo-European, Palaeo-Balkan and Thracian Studies 
\item[OrSu-long] Orientalia Suecana. An International Journal of Indological, Iranian, Semitic, Turkic Studies 
\item[OsjZbor-long] Osjekizbornik 
\item[OstbGrenzm-long] Ostbairische Grenzmarken. Passauer Jahrbuch für Geschichte, Kunst und Volkskunde 
\item[Ostraka-long] Ostraka. Rivista di antichità 
\item[OudhMeded-long] Oudheidkundige mededelingen uit het Rijksmuseum van oudheden te Leiden 
\item[OxfJA-long] Oxford Journal of Archaeology 
\item[OxfStPhilos-long] Oxford Studies in Ancient Philosophy 
\item[Pact-long] Pact. Revue du Groupe européen d’études pour les techniques physiques, chimiques et mathématiques appliquées à l’archéologie 
\item[Padusa-long] Padusa. Bolletino del Centro polesano di studi storici, archeologici ed etnografici, Rovigo 
\item[PagA-long] Pagine di archeologia. Studi e materiali 
\item[PAI-long] Polevye archeologičeskie issledovanija 
\item[Paideuma-long] Paideuma. Mitteilungen zur Kulturkunde 
\item[Palaeohistoria-long] Palaeohistoria. Acta et communicationes Instituti bio-archaeologici universitatis Groninganae 
\item[Paleohispanica-long] Paleohispánica. Revista sobre lenguas y culturas de la Hispania antigua %*Abweichung!
\item[Palladio-long] Palladio. Rivista di storia dell’architettura 
\item[Pallas-long] Pallas. Revue d’études antiques 
\item[Palmet-long] Palmet. Sadberk Hanım Müzesi yıllığı 
\item[PamA-long] Památky archeologické 
\item[Pan-long] Pan. Studi dell’Istituto di filologia latina 
\item[PapBilb-long] Papeles Bilbilitanos 
\item[Papyri-long] Papyri. Bollettino del Museo del papiro 
\item[Parthica-long] Parthica. Incontri di culture nel mondo antico 
\item[Partenope-long] Partenope. Rivista trimestrale di cultura napoletana 
\item[PAS-long] Prähistorische Archäologie in Südosteuropa 
\item[PBF-long] Prähistorische Bronzefunde 
\item[Pegasus-long] Pegasus. Berliner Beiträge zum Nachleben der Antike 
\item[PEQ-long] Palestine Exploration Quarterly 
\item[Peristil-long] Peristil. Zbornik radova za povijest umjetnosti 
\item[Persica-long] Persica. Jaarboek van het Genootschap Nederland-Iran, Stichting voor culturele betrekkingen 
\item[Peuce-long] Peuce. Studii şi comunicări de istorie veche, arheologie şi numismatică 
\item[PF-long] Pergamenische Forschungen 
\item[Pharos-long] Pharos. Journal of the Netherlands Institute at Athens 
\item[Philologus-long] Philologus. Zeitschrift für das klassische Altertum 
\item[Phoenix-long] Phoenix. The Journal of the Classical Association of Canada 
\item[PhoenixExOrLux-long] Phoenix. Bulletin uitgegeven door het Vooraziatisch-Egyptisch Genootschap »Ex Oriente Lux« 
\item[Phoibos-long] Phoibos. Bulletin du Cercle de philologie classique et orientale de l’Université libre de Bruxelles 
\item[Phronesis-long] Phronesis. A Journal for Ancient Philosophy 
\item[Picus-long] Picus. Studi e ricerche sulle Marche nell’antichità 
\item[PIR-long] Prosopographia Imperii Romani 
\item[PKG-long] Propyläen Kunstgeschichte 
\item[Platon-long] Πλάτον. Δελτίον της Εταιρείας Ελλήνων Φιλολόγων 
\item[PLup-long] Papyrologica Lupiensia 
\item[PolAMed-long] Polish Archaeology in the Mediterranean 
\item[Polemon-long] Πολέμων. Αρχαιολογικόν Περιοδικόν 
\item[Polis-long] Polis. Revista de ideas y formas políticas de la antigüedad clásica 
\item[PompHercStab-long] Pompeii, Herculaneum, Stabiae. Bolletino. Associazione internazionale Amici di Pompei 
\item[Pontica-long] Pontica. Studii şi materiale de istorie, arheologie şi muzeografie, Constanţa 
\item[Portugalia-long] Portugália. Revista do Instituto de arqueologia da Faculdade de letras da Universidade do Porto 
\item[PP-long] La parola del passato 
\item[PPM-long] Pompei: Pitture e mosaici. Enciclopedia dell’arte antica classica e orientale 
\item[PraceA-long] Prace archeologiczne 
\item[PraceMatLodz-long] Prace i materialy Muzeum archeologicznego i etnograficznego w Łodzi %*Abweichung!
\item[Prakt-long] Πρακτικά της εν Αθήναις Αρχαιολογικής Εταιρείας 
\item[PraktAkAth-long] Πρακτικά της Ακαδημίας Αθηνών 
\item[PreistAlp-long] Preistoria alpina. Museo tridentino di scienze naturali 
\item[PriloziZagreb-long] Prilozi Instituta za arheologiju u Zagrebu 
\item[PrincViana-long] Principe di Viana 
\item[ProblIsk-long] Problemi na izkustvoto. Trimesecno spisanie za estetika, teorija, istorija i kritika na izkustvoto 
\item[ProcAfrClAss-long] The Proceedings of the African Classical Associations 
\item[ProcCambrPhilSoc-long] Proceedings of the Cambridge Philological Society 
\item[ProcDanInstAth-long] Proceedings of the Danish Institute at Athens 
\item[ProcPrehistSoc-long] Proceedings of the Prehistoric Society 
\item[Prometheus-long] Prometheus. Rivista quadrimestrale di studi classici 
\item[ProspAQuad-long] Prospezioni archeologiche. Quaderni 
\item[Prospettiva-long] Prospettiva. Rivista di storia dell’arte antica e moderna 
\item[Prospezioni-long] Prospezioni. Bollettino di informazioni scientifiche 
\item[ProvHist-long] Provence historique 
\item[ProvLucca-long] La provincia di Lucca 
\item[PublInstTTMeneses-long] Publicaciones de la Institución »Tello Téllez de Meneses« 
\item[Pulpudeva-long] Pulpudeva. Semaines philippopolitaines de l’histoire et de la culture thrace 
\item[Puteoli-long] Puteoli. Studi di storia antica 
\item[Pyrenae-long] Pyrenae. Crónica arqueológica 
\item[PZ-long] Prähistorische Zeitschrift 
\item[QDAP-long] The Quarterly of the Department of Antiquities in Palestine 
\item[QuadABarcel-long] Quaderns d’arqueologia i història de la ciutat (Barcelona) 
\item[QuadACagl-long] Quaderni. Soprintendenza archeologica per la provincie di Cagliari e Oristano 
\item[QuadACal-long] Quaderni del Dipartimento delle arti, Università della Calabria 
\item[QuadALibya-long] Quaderni di archeologia della Libia 
\item[QuadAMant-long] Quaderni di archeologia del Mantovano 
\item[QuadAMess-long] Quaderni di archeologia, Università di Messina 
\item[QuadAOst-long] Quaderni del Gruppo archeologico ostigliese 
\item[QuadAPiem-long] Quaderni della Soprintendenza archeologica del Piemonte 
\item[QuadAquil-long] Quaderni aquileiesi 
\item[QuadAReggio-long] Quaderni d’archeologia reggiana 
\item[QuadAVen-long] Quaderni di archeologia del Veneto 
\item[QuadCast-long] Quaderns de prehistòria i arqueologia de Castelló 
\item[QuadCat-long] Quaderni catanesi di studi classici e medievali 
\item[QuadChieti-long] Quaderni dell’Istituto di archeologia e storia antica, Università di Chieti 
\item[QuadErb-long] Quaderni erbesi. Civico museo archeologico di Erba 
\item[QuadFriulA-long] Quaderni friulani di archeologia 
\item[QuadGerico-long] Quaderni di Gerico 
\item[QuadIstFilGr-long] Quaderni dell’Istituto di filologia greca 
\item[QuadIstLat-long] Quaderni dell’Istituto di lingua e letteratura latina, Università di Roma 
\item[QuadLecce-long] Quaderni. Istituto di lingue e letterature classiche, Facoltà di magistero, Università degli studi, Lecce 
\item[QuadMusPontecorvo-long] Museo civico Pontecorvo. Quaderni 
\item[QuadMusSalinas-long] Quaderni del Museo archeologico regionale Antonino Salinas 
\item[QuadProtost-long] Quaderni di protostoria. Università di Perugia 
\item[QuadStLun-long] Quaderni. Centro studi lunensi 
\item[QuadStor-long] Quaderni di storia 
\item[QuadStorici-long] Quaderni storici 
\item[QuadUrbin-long] Quaderni urbinati di cultura classica 
\item[Quaternaria-long] Quaternaria. Storia naturale e culturale del quaternario 
\item[RA-long] Revue archéologique 
\item[RAArtLouv-long] Revue des archéologues et historiens d’art de Louvain 
\item[RAC-long] Reallexikon für Antike und Christentum 
\item[RACFr-long] Revue archéologique du Centre de la France 
\item[RAComo-long] Rivista archeologica dell’antica provincia e diocesi di Como 
\item[RACr-long] Rivista di archeologia cristiana 
\item[RadAkZadar-long] Radovi Zavoda za povijesne znanosti, Hrvatska akademija znanosti i umjetnosti u Zadru 
\item[Radiocarbon-long] Radiocarbon. An International Journal of Cosmogenic Isotope Research 
\item[RadSplit-long] Radovi. Razdio povijesnih znanosti 
\item[RAE-long] Revue archéologique de l’Est et du Centre-Est 
\item[Raggi-long] Raggi. Zeitschrift für Kunstgeschichte und Archäologie 
\item[RAMadrid-long] Revista de arqueología 
\item[RANarb-long] Revue archéologique de Narbonnaise 
\item[RAPon-long] Revista d’arqueologia de Ponent 
\item[RapWyk-long] Raporty wykopaliskowe 
\item[RArchBiblMus-long] Revista de archivos, bibliotecas y museos 
\item[RArcheom-long] Revue d’archéométrie 
\item[RArtMus-long] La revue des arts. Musées de France 
\item[RassAPiomb-long] Rassegna di archeologia. Associazione archeologica piombinese 
\item[RassLazio-long] Rassegna del Lazio 
\item[RassStorSalern-long] Rassegna storica salernitana 
\item[RassVolt-long] Rassegna volterrana 
\item[RAssyr-long] Revue d’assyriologie et d’archéologie orientale 
\item[Ratiariensia-long] Ratiariensia. Studi e materiali mesici e danubiani 
\item[RAtlMed-long] Revista atlántica-mediterránea de prehistoria y arqueología social 
\item[RavStRic-long] Ravenna studi e ricerche 
\item[Raydan-long] Raydan. Journal of the Ancient Yemeni Antiquities and Epigraphy 
\item[RB-long] Revue biblique 
\item[RBelgNum-long] Revue belge de numismatique et de sigillographie 
\item[RBelgPhilHist-long] Revue belge de philologie et d’histoire 
\item[RBK-long] Reallexikon zur byzantinischen Kunst 
\item[RCulClMedioev-long] Rivista di cultura classica e medioevale 
\item[RdA-long] Rivista di archeologia 
\item[RDAC-long] Report of the Department of Antiquities, Cyprus 
\item[RdE-long] Revue d’égyptologie (Kairo) 
\item[RDroitsAnt-long] Revue internationale des droits de l’antiquité 
\item[RE-long] Paulys Realencyclopädie der classischen Altertumswissenschaft 
\item[REA-long] Revue des études anciennes 
\item[REByz-long] Revue des études byzantines 
\item[RecConstantine-long] Recueil des notices et memoires de la Société archéologique du département de Constantine 
\item[RechACrac-long] Recherches archéologiques. Institut d’archéologie de l’Université de Cracovie 
\item[RechAlb-long] Recherches albanologiques 
\item[RecMusAlcoi-long] Recerques del Museu d’Alcoi 
\item[RecTrav-long] Recueil de travaux relatifs à philologie et l’archéologie égyptiennes et assyriennes 
\item[REG-long] Revue des études grecques 
\item[ReiCretActa-long] Rei Cretariae Romanae Fautorum acta 
\item[ReiCretCommunic-long] Rei Cretariae Romanae Fautores. Communicationes 
\item[REL-long] Revue des études latines 
\item[Rema-long] Ρήμα. Mitteilungen zur indogermanischen, vornehmlich indo-iranischen Wortkunde sowie zur holothetischen Sprachtheorie 
\item[RendBologna-long] Atti della Accademia delle scienze dell’Istituto di Bologna. Rendiconti 
\item[RendIstLomb-long] Rendiconti. Classe di lettere e scienze morali e storiche, Istituto lombardo, Accademia di scienze e lettere 
\item[RendLinc-long] Rendiconti dell’Accademia nazionale dei Lincei, Classe di scienze morali, storiche e filologiche 
\item[RendNap-long] Rendiconti della Accademia di archeologia, lettere e belle arti, Napoli 
\item[RendPontAc-long] Rendiconti. Atti della Pontificia accademia romana di archeologia 
\item[RepMalta-long] Report on the Working of the Museum Department, Malta, Department of Information 
\item[Reppal-long] Reppal. Revue des études phénicienne-puniques et des antiquités libyques 
\item[RepSocLibSt-long] Annual Report. The Society for Libyan Studies 
\item[RES-long] Répertoire d’épigraphie sémitique (Paris 1900--1950) 
\item[REstIber-long] Revista de estudios ibéricos 
\item[REtArm-long] Revue des études arméniennes 
\item[RevEg-long] Revue égyptologique (Paris) 
\item[RFil-long] Rivista di filologia e di istruzione classica 
\item[RGeorgCauc-long] Revue des études géorgiennes et caucasiennes 
\item[RGF-long] Römisch-Germanische Forschungen 
\item[RGuimar-long] Revista de Guimarães 
\item[RHA-long] Revue hittite et asianique 
\item[RheinMusBonn-long] Das Rheinische Landesmuseum Bonn. Berichte aus der Arbeit des Museums 
\item[RHistArmees-long] Revue historique des armées %*Abweichung!
\item[RHistRel-long] Revue de l’histoire des religions 
\item[RhM-long] Rheinisches Museum für Philologie 
\item[RIA-long] Rivista dell’Istituto nazionale d’archeologia e storia dell’arte 
\item[RIC-long] H. Mattingly – E. A. Sydenham, The Roman Imperial Coinage 
\item[RicEgAntCopt-long] Ricerche di egittologia e di antichità copte 
\item[RicognA-long] Ricognizioni archeologiche 
\item[RicStBrindisi-long] Ricerche e studi. Museo Francesco Ribezzo, Brindisi 
\item[RIngIntem-long] Rivista Ingauna e Intemelia 
\item[RItNum-long] Rivista italiana di numismatica e scienze affini 
\item[RlA-long] Reallexikon der Assyriologie und vorderasiatischen Archäologie 
\item[RLouvre-long] Revue du Louvre. La revue des musées de France 
\item[RM-long] Mitteilungen des Deutschen Archäologischen Instituts, Römische Abteilung 
\item[RNum-long] Revue numismatique 
\item[RoczMuzWarsz-long] Rocznik Muzeum narodowego w Warszawie 
\item[RoemOe-long] Römisches Österreich. Jahresschrift der Österreichischen Gesellschaft für Archäologie %*Abweichung!
\item[RoemQSchr-long] Römische Quartalschrift für christliche Altertumskunde und Kirchengeschichte %*Abweichung!
\item[Romanobarbarica-long] Romanobarbarica. Contributi allo studio dei rapporti culturali tra il mondo latino e mondo barbarico 
\item[RomGens-long] Romana gens. Bollettino dell’Associazione archeologica romana 
\item[RoscherML-long] W. H. Roscher, Ausführliches Lexikon der griechischen und römischen Mythologie %*Abweichung!
\item[RossA-long] Rossijskaja archeologija 
\item[RPC-long] Roman Provincial Coinage 
\item[RPhil-long] Revue de philologie, de littérature et d’histoire anciennes 
\item[RPortA-long] Revista portuguesa de arqueologia 
\item[RPorto-long] Revista da Facultade de letras. Serie de historia. Universidade do Porto 
\item[RRC-long] M. Crawford, Roman Republican Coinage (London 1974) 
\item[RSaintonge-long] Revue de la Saintonge et de l’Aunis 
\item[RScPreist-long] Rivista di scienze preistoriche 
\item[RSO-long] Rivista degli studi orientali 
\item[RStBiz-long] Rivista di studi bizantini e neoellenici 
\item[RStCl-long] Rivista di studi classici 
\item[RStFen-long] Rivista di studi fenici 
\item[RStLig-long] Rivista di studi liguri 
\item[RStMarch-long] Rivista di studi marchigiani 
\item[RStorAnt-long] Rivista storica dell’antichità 
\item[RStorCal-long] Rivista storica calabrese 
\item[RStPomp-long] Rivista di studi pompeiani 
\item[RStPun-long] Rivista di studi punici 
\item[RTopAnt-long] Rivista di topografia antica 
\item[Rudiae-long] Rudiae. Ricerche sul mondo classico 
\item[SaalbJb-long] Saalburg-Jahrbuch. Bericht des Saalburg-Museums 
\item[SaarBeitr-long] Saarbrücker Beiträge zur Altertumskunde 
\item[SaarStMat-long] Saarbrücker Studien und Materialien zur Altertumskunde 
\item[Sacer-long] Sacer. Bollettino della Associazione storica saccarese 
\item[Saeculum-long] Saeculum. Jahrbuch für Universalgeschichte 
\item[SAGA-long] Studien zur Archäologie und Geschichte Altägyptens 
\item[SaggiFen-long] Saggi fenici 
\item[Saguntum-long] Saguntum. Papeles del Laboratorio de arqueología de Valencia 
\item[Saitabi-long] Saitabi. Noticiario de historia, arte y arqueología de Levante 
\item[SAK-long] Studien zur altägyptischen Kultur 
\item[Salduie-long] Salduie. Estudios de prehistoria y arqueología 
\item[Samothrace-long] Samothrace. Excavations Conducted by the Institute of Fine Arts of New York University 
\item[Sandalion-long] Sandalion. Quaderni di cultura classica, cristiana e medievale 
\item[Sardis-long] Sardis. Publications of the American Society for the Excavation of Sardis 
\item[Sargetia-long] Sargetia. Acta Musei regionalis Devensis 
\item[SarkSt-long] Sarkophag-Studien 
\item[Sautuola-long] Sautuola. Revista del Instituto de prehistoria y arqueologia Sautuola 
\item[Savaria-long] Savaria. Bulletin der Museen des Komitats Vas 
\item[SBBerlin-long] Sitzungsberichte der Deutschen Akademie der Wissenschaften zu Berlin. Klasse für Sprache, Literatur und Kunst 
\item[SBLeipzig-long] Sitzungsberichte der Sächsischen Akademie der Wissenschaften zu Leipzig 
\item[SBMuenchen-long] Bayerische Akademie der Wissenschaften. Philosophisch-Historische Klasse. Sitzungsberichte %*Abweichung!
\item[SborBrno-long] Sborník prací Filozofické fakulty Brněnské univerzity. M, Rada archeologická 
\item[SBWien-long] Sitzungsberichte. Österreichische Akademie der Wissenschaften 
\item[ScAnt-long] Scienze dell’antichità. Storia, archeologia, antropologia 
\item[SCE-long] The Swedish Cyprus Expedition 
\item[SchildStei-long] Schild von Steier. Beiträge zur Steirischen Vor- und Frühgeschichte und Münzkunde 
\item[Scholia-long] Scholia. Natal Studies in Classical Antiquity 
\item[SchwMueBl-long] Schweizer Münzblätter %*Abweichung!
\item[SchwNumRu-long] Schweizerische numismatische Rundschau 
\item[ScrCiv-long] Scrittura e civiltà 
\item[ScrClIsr-long] Scripta classica Israelica. Yearbook of the Israel Society for the Promotion of Classical Studies 
\item[ScrHieros-long] Scripta Hierosolymitana. Publications of the Hebrew University, Jerusalem 
\item[ScrMed-long] Scripta mediterranea. Bulletin of the Society for Mediterranean Studies, Toronto 
\item[SDAIK-long] Sonderschriften des Deutschen Archäologischen Instituts, Abteilung Kairo 
\item[SEG-long] Supplementum epigraphicum Graecum 
\item[SeminRom-long] Seminari romani di cultura greca 
\item[Semitica-long] Semitica. Cahiers publiés par l’Institut d’études sémitiques du College de France. Avec le concours du Centre national de la recherche scientifique 
\item[SetubalA-long] Setúbal arqueológica 
\item[Sibrium-long] Sibrium. Collana di studi e documentazioni 
\item[SicA-long] Sicilia archeologica 
\item[SicGymn-long] Siculorum gymnasium 
\item[SIG-long] W. Dittenberger, Sylloge inscriptionum Graecarum (Leipzig 1915--­1924) 
\item[Sileno-long] Sileno. Rivista di studi classici e cristiani 
\item[SilkRoadArtA-long] Silk Road Art and Archaeology. Journal of the Institute of Silk Road Studies, Kamakura 
\item[SIMA-long] Studies in Mediterranean Archaeology 
\item[Simblos-long] Simblos. Scritti di storia antica 
\item[Skyllis-long] Skyllis. Zeitschrift für Unterwasserarchäologie 
\item[SlovA-long] Slovenská archeológia 
\item[SlovNum-long] Slovenská numizmatika 
\item[SMEA-long] Studi micenei ed egeo-anatolici 
\item[SNG-long] Sylloge nummorum Graecorum 
\item[SocGeoAOran-long] (Bulletin trimestriel de la) Société de géographie et d’archéologie (de la province) d’Oran 
\item[SoobErmit-long] Soobščenija Gosudarstvennogo Ėrmitaža 
\item[SoobMuzMoskva-long] Soobščenija Gosudarstvennogo muzeja izobrazitel’nych isskustv imeni A. S. Puškina 
\item[SovA-long] Sovetskaja archeologija 
\item[Spal-long] Spal. Revista de prehistoria y arqueología de la Universidad de Sevilla 
\item[SpNov-long] Specimina nova dissertationum ex Instituto historico Universitatis Quinqueecclesiensis de Iano Pannonio nominatae 
\item[Spoletium-long] Spoletium. Rivista di arte, storia, cultura 
\item[StA-long] Studia archaeologica 
\item[Stadion-long] Stadion. Internationale Zeitschrift für Geschichte des Sports 
\item[StaedelJb-long] Städel-Jahrbuch %*Abweichung!
\item[StAeg-long] Studia Aegyptiaca. Budapest 
\item[StAlb-long] Studia Albanica 
\item[StAnt-long] Studi di antichità. Università di Lecce 
\item[Starinar-long] StarinarArheoloki institut Beograd 
\item[StAWarsz-long] Studia archeologiczne. Uniwersytet Warszawski, Instytut archeologii 
\item[StBiFranc-long] Studium biblicum Franciscanum. Liber annuus 
\item[StBitont-long] Studi bitontini 
\item[StBoT-long] Studien zu den Bogazköy-Texten 
\item[StCercIstorV-long] Studii şi cercetări de istorie veche şi arheologie 
\item[StCercNum-long] Studii şi cercetări de numismatică 
\item[StCl-long] Studii clasice. Societatea de studii clasice din Republica socialistă Romănia 
\item[StClOr-long] Studi classici e orientali 
\item[StDocA-long] Studi e documenti di archeologia 
\item[StDocHistIur-long] Studia et documenta historiae et iuris 
\item[StEbla-long] Studi eblaiti 
\item[StEgAntPun-long] Studi di egittologia e di antichità puniche 
\item[SteMat-long] Studi e materiali. Soprintendenza ai beni archeologici per la Toscana 
\item[StEpigrLing-long] Studi epigrafici e linguistici sul Vicino Oriente antico 
\item[StEtr-long] Studi etruschi 
\item[StGenu-long] Studi genuensi 
\item[StHist-long] Studia historica. Historia antigua 
\item[StiftHambKuSamml-long] Stiftung zur Förderung der Hamburgischen Kunstsammlungen. Erwerbungen 
\item[StItFilCl-long] Studi italiani di filologia classica 
\item[StLatIt-long] Studi latini e italiani 
\item[StMagreb-long] Studi magrebini 
\item[StMatStorRel-long] Studi e materiali di storia delle religioni 
\item[StOliv-long] Studia Oliveriana 
\item[StOr-long] Studia orientalia, Helsinki 
\item[StOrCr-long] Studi sull’Oriente cristiano 
\item[StP-long] Studia papyrologica 
\item[StrennaRom-long] Strenna dei romanisti 
\item[StRom-long] Studi romani 
\item[StRomagn-long] Studi romagnoli 
\item[StSalent-long] Studi salentini 
\item[StSard-long] Studi sardi 
\item[StStorRel-long] Studi storico-religiosi 
\item[StTardoant-long] Studi tardoantichi 
\item[StTrentStor-long] Studi trentini di scienze storiche. Sezione seconda 
\item[StTroica-long] Studia Troica 
\item[StUrbin-long] Studi urbinati. B, Scienze umane. 3, Linguistica, letteratura, arte 
\item[Sumer-long] Sumer. A Journal of Archaeology (and History) in Iraq 
\item[SylvaMala-long] Sylva Mala. Bollettino del Centro di studi archeologici di Boscoreale, Boscotrecase e Trecase %*Abweichung!
\item[SymbOslo-long] Symbolae Osloenses 
\item[Syria-long] Syria. Revue d’art oriental et d’archéologie 
\item[SyrMesopSt-long] Syro-Mesopotamian Studies 
\item[TAD-long] Türk arkeoloji dergisi 
\item[Talanta-long] Τάλαντα. Proceedings of the Dutch Archaeological and Historical Society 
\item[TAM-long] E. Kalinka (Hrsg.), Tituli Asiae Minoris (Wien 1901--­1941) 
\item[Taras-long] Taras. Rivista di archeologia 
\item[Tarsus-long] Excavations at Gözlü Kule, Tarsus 
\item[TAVO-long] Tübinger Atlas des Vorderen Orients 
\item[TeherF-long] Teheraner Forschungen 
\item[Teiresias-long] Teiresias. A Review and Continuing Bibliography of Boiotian Studies 
\item[TelAvivJA-long] Tel Aviv. Journal of the Institute of Archaeology of Tel Aviv University 
\item[TerraAntBalc-long] Acta Associationis internationalis »Terra antiqua balcanica« 
\item[TerraVolsci-long] Terra dei Volsci. Annali del Museo archeologico di Frosinone 
\item[Teruel-long] Teruel. Instituto de estudios turolenses 
\item[TextilAnc-long] Textiles anciens. Bulletin du Centre international d’étude des textiles anciens 
\item[TheolRu-long] Theologische Rundschau 
\item[ThesCRA-long] Thesaurus Cultus et Rituum Antiquorum 
\item[Thessalika-long] Θεσσαλικά 
\item[Thessalonike-long] Η Θεσσαλονίκη 
\item[Thieme-Becker-long] U. Thieme – F. Becker (Hrsg.), Allgemeines Lexikon der bildenden Künstler %*Abweichung!
\item[ThrakChron-long] Θρακικά Χρονικά 
\item[ThrakEp-long] Θρακική Επετηρίδα 
\item[TIB-long] Tabula Imperii Byzantini 
\item[Tibiscus-long] Tibiscus. Istorie, arheologie. Muzeul Banatului Timişoara 
\item[TiLeon-long] Tierras de León 
\item[Tiryns-long] Tiryns. Forschungen und Berichte 
\item[TMA-long] Tijdschrift voor Mediterrane archeologie 
\item[Topoi-long] Τόποι. Orient – Occident 
\item[Torretta-long] La torretta. Rivista quadrimestrale a cura della Biblioteca comunale di Blera 
\item[TourOrleOr-long] La Tour de l’Orle-d’Or, Semur-en-Auxois 
\item[TrabAntrEtn-long] Trabalhos de antropologia e etnologia 
\item[TrabArq-long] Trabalhos de arqueologia 
\item[TrabAssArqPort-long] Trabalhos da associação dos arqueólogos portugueses 
\item[TrabNavarra-long] Trabajos de arqueología de Navarra 
\item[TrabPrehist-long] Trabajos de prehistoria 
\item[Traditio-long] Traditio. Studies in Ancient and Medieval History, Thought and Religion 
\item[TransactAmPhilAss-long] Transactions and Proceedings of the American Philological Association 
\item[TransactAmPhilosSoc-long] Transactions of the American Philosophical Society 
\item[TransactLond-long] Transactions of the London and Middlesex Archaeological Society 
\item[TravMem-long] Travaux et mémoires. Centre de recherche d’histoire et civilisation byzantine, Paris 
\item[TravToulouse-long] Travaux de l’Institut d’art préhistorique, Université de Toulouse – Le Mirail 
\item[TreMonet-long] Trésors monétaires 
\item[TribArq-long] Tribuna d’arqueologia 
\item[TrudyErmit-long] Trudy Gosudarstvennogo Ėrmitaža 
\item[TrWPr-long] Trierer Winckelmannsprogramme 
\item[TrZ-long] Trierer Zeitschrift für Geschichte und Kunst des Trierer Landes und seiner Nachbargebiete 
\item[TTKY-long] Türk Tarih Kurumu yayınları 
\item[TueBA-Ar-long] Türkiye Bilimler Akademisi arkeoloji dergisi %*Abweichung!
\item[Tyche-long] Tyche. Beiträge zur Alten Geschichte, Papyrologie und Epigraphik 
\item[UF-long] Ugarit-Forschungen. Internationales Jahrbuch für die Altertumskunde Syrien-Palästinas 
\item[UPA-long] Universitätsforschungen zur Prähistorischen Archäologie 
\item[LUrbe-long] L’Urbe. Rivista romana %*Abweichung!
\item[UrSchw-long] Ur-Schweiz. La Suisse primitive 
\item[UVB-long] Vorläufiger Bericht über die von dem Deutschen Archäologischen Institut und der Deutschen Orient-Gesellschaft aus den Mitteln der Deutschen Forschungsgemeinschaft unternommenen Ausgrabungen in Uruk-Warka 
\item[VarSpom-long] Varstvo spomenikov 
\item[VDI-long] Vestnik drevnej istorii 
\item[Vekove-long] Vekove. Dvumesečno spisanie. Bălgarsko istoričesko družestvo 
\item[Veleia-long] Veleia. Revista de prehistoria, historia antigua, arqueología y filología clásicas 
\item[VenArt-long] Venezia arti. Bolletino del Dipartimento di storia e critica delle arti dell’Università di Venezia 
\item[VerAmstMeded-long] Mededelingenblad. Vereniging van Vrienden Allard Pierson Museum 
\item[Verbanus-long] Verbanus. Rassegna per la cultura, l’arte, la storia del lago 
\item[VeteraChr-long] Vetera christianorum 
\item[VGesVind-long] Veröffentlichungen der Gesellschaft Pro Vindonissa 
\item[Vichiana-long] Vichiana. Rassegna di studi filologici e storici 
\item[VicOr-long] Vicino Oriente 
\item[VigChr-long] Vigiliae christianae 
\item[Viminacium-long] Viminacium. Zbornik radova Narodnog muzeja 
\item[VisRel-long] Visible Religion 
\item[Vitudurum-long] Beiträge zum römischen Oberwinterthur – Vitudurum. Ausgrabungen im Unteren Bühl 
\item[VivScyl-long] Vivarium Scyllacense. Bollettino dell’Istituto di studi su Cassiodoro e sul medioevo in Calabria 
\item[VizVrem-long] Vizantijskij vremennik 
\item[VjesAMuzZagreb-long] Vjesnik Arheološkog muzeja u Zagrebu 
\item[VjesDal-long] Vjesnik za arheologiju i historiju dalmatinsku. Bulletin d’archéologie et d’histoire dalmates 
\item[Wad-al-Hayara-long] Wad-al-Hayara. Revista de estudios de la Institución provincial de cultura »Marqués de Santillana« de Guadalajara 
\item[WeltGesch-long] Die Welt als Geschichte 
\item[WiadA-long] Wiadomości archeologiczne. Bulletin archéologique polonais 
\item[WissMBosn-long] Wissenschaftliche Mitteilungen des Bosnischen Landesmuseums, A. Archäologie 
\item[WissZBerl-long] Wissenschaftliche Zeitschrift der Humboldt-Universität zu Berlin. Gesellschafts- und sprachwissenschaftliche Reihe 
\item[WissZHalle-long] Wissenschaftliche Zeitschrift. Martin-Luther-Universität Halle-Wittenberg 
\item[WissZJena-long] Wissenschaftliche Zeitschrift der Friedrich-Schiller-Universität Jena 
\item[WissZRostock-long] Wissenschaftliche Zeitschrift der Universität Rostock 
\item[WO-long] Die Welt des Orients. Wissenschaftliche Beiträge zur Kunde des Morgenlandes 
\item[WorldA-long] World Archaeology 
\item[WSt-long] Wiener Studien 
\item[WuerzbJb-long] Würzburger Jahrbücher für die Altertumswissenschaft %*Abweichung!
\item[WVDOG-long] Wissenschaftliche Veröffentlichungen der Deutschen Orient-Gesellschaft 
\item[WZKM-long] Wiener Zeitschrift für die Kunde des Morgenlandes 
\item[Xenia-long] Xenia. Semestrale di antichità 
\item[XeniaAnt-long] Xenia antiqua 
\item[XeniaKonst-long] Xenia. Konstanzer althistorische Vorträge und Forschungen 
\item[YaleClSt-long] Yale Classical Studies 
\item[YaleUnivB-long] Yale University Art Gallery Bulletin 
\item[ZA-long] Zeitschrift für Assyriologie und vorderasiatische Archäologie 
\item[ZAAK-long] Zeitschrift für Archäologie Außereuropäischer Kulturen 
\item[ZAeS-long] Zeitschrift für ägyptische Sprache und Altertumskunde %*Abweichung!
\item[ZAKSSchriften-long] Schriften des Zentrums für Archäologie und Kulturgeschichte des Schwarzmeerraumes 
\item[ZAntChr-long] Zeitschrift für antikes Christentum 
\item[ZAW-long] Zeitschrift für die alttestamentliche Wissenschaft 
\item[ZborMuzBeograd-long] Zbornik Narodnog muzeja Beograd 
\item[ZborRadBeograd-long] Zbornik radova Vizantološkog instituta. Recueil des travaux de l’Institut d’études byzantines, Beograd 
\item[ZborZadar-long] Zbornik Instituta za historijske nauke u Zadru 
\item[ZDMG-long] Zeitschrift der Deutschen Morgenländischen Gesellschaft 
\item[ZDPV-long] Zeitschrift des Deutschen Palästina-Vereins 
\item[Zephyrus-long] Zephyrus. Revista de prehistoria y arqueología 
\item[ZEthn-long] Zeitschrift für Ethnologie der Deutschen Gesellschaft für Völkerkunde und der Berliner Gesellschaft für Anthropologie, Ethnologie und Urgeschichte 
\item[ZfA-long] Zeitschrift für Archäologie 
\item[ZfNum-long] Zeitschrift für Numismatik 
\item[ZivaAnt-long] Živa antika. Antiquité vivante 
\item[ZKuGesch-long] Zeitschrift für Kunstgeschichte 
\item[ZNW-long] Zeitschrift für die neutestamentliche Wissenschaft und die Kunde der älteren Kirche 
\item[ZPE-long] Zeitschrift für Papyrologie und Epigraphik 
\item[ZSav-long] Zeitschrift der Savigny-Stiftung für Rechtsgeschichte. Romanistische Abteilung 
\item[ZSchwA-long] Zeitschrift für Schweizerische Archäologie und Kunstgeschichte 
\item[ZVerglSprF-long] Zeitschrift für vergleichende Sprachforschung 
\end{description}
\end{footnotesize}
%\end{multicols}


 \changes{v1.1}{2015/07/06}{Erstellung der Liste mit Abkürzungen}



%\section{Umsetzung}
%\label{driver}
%|archaeologie| besteht aus einem Bibliographieformat (|bbx|) und einem Zitierformat (|cbx|). 
%%%
%
%\subsection{archaeologie.bbx}
%%|archaeologie| baut auf dem |standard|-Stil von |biblatex| auf, der entsprechend geladen werden muss.
%%	\begin{macrocode}
%%\ProvidesFile{archaeologie.bbx}%
%%               [2015/09/15 v1.0  archaeologie -- %
%%                biblatex fuer Archaeologen, Historiker und Philologen, bbx-Datei]
%%\RequireBibliographyStyle{standard}
%%	\end{macro}
%
%
%\subsection{archaeologie.cbx}

 \changes{v0.1}{2015/06/04}{Started Project}
 \changes{v1}{2015/09/15}{First public version}
\PrintChanges
\PrintIndex


\end{document}
