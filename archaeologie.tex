% archaeologie --%
% biblatex for archaeologists, 
% historians and philologists
% Copyright (c) 2017 Lukas C. Bossert | Johannes Friedl
%  
% This work may be distributed and/or modified under the
% conditions of the LaTeX Project Public License, either version 1.3
% of this license or (at your option) any later version.
% The latest version of this license is in
%   http://www.latex-project.org/lppl.txt
% and version 1.3 or later is part of all distributions of LaTeX
% version 2005/12/01 or later.
% !TEX program = lualatex
\documentclass[a4paper,
10pt,
greek,
french,
spanish,
italian,
ngerman,
english,
]{ltxdoc}
\usepackage{iftex}
\ifPDFTeX
  \usepackage[utf8]{inputenc}
  \usepackage[T1]{fontenc}
  \usepackage{lmodern}
\else
    \ifXeTeX
    \usepackage{fontspec}
		\def\compiler{\hologo{XeLaTeX}}
  \else 
	  \usepackage{fontspec}
    \usepackage{luatextra}
    \usepackage{luaotfload}
	\def\compiler{\hologo{LuaLaTeX}}
\fi
  \defaultfontfeatures{Ligatures=TeX}
\fi
\listfiles
\usepackage{libertine}
\defaultfontfeatures[AnonymousPro]
  {
    Extension      = .ttf                       ,
    BoldFont       = AnonymousPro-Bold          ,
    ItalicFont     = AnonymousPro-BoldItalic    ,
    BoldItalicFont = AnonymousPro-Italic        ,
    UprightFont    = AnonymousPro-Regular       ,
  }
\setmonofont[Scale= MatchUppercase]{AnonymousPro}
\setmainfont[Numbers = {Monospaced, OldStyle}]{Libertinus Serif}
\setsansfont{Libertinus Sans}


\usepackage{xr-hyper}
\usepackage{url}
\usepackage{xspace}


\usepackage[
	backend=biber,
%	backref,
	style=archaeologie,
	lstpublishers,
	lstlocations,
	lstabbrv,
%	initials=false,
]{biblatex}
\renewcommand\bibfont{\normalfont\footnotesize}
\usepackage{metalogo}
\usepackage{hologo}
\usepackage{babel}
\usepackage{coolthms}


\usepackage{chngcntr}

\usepackage[
  autostyle=true,%
]{csquotes}
\usepackage{multicol}
  \setlength{\columnsep}{1.5cm}
  \setlength{\columnseprule}{0.2pt}

\usepackage{xcolor}
\definecolor{codeblue}{RGB}{0,65,137}
\definecolor{codegreen}{RGB}{147,193,26}
\definecolor{codegray}{rgb}{0.5,0.5,0.5}
\definecolor{codepurple}{rgb}{0.58,0,0.82}
\definecolor{backcolour}{rgb}{0.95,0.95,0.92}


\usepackage[ 
	headsepline, 
	footsepline,
%	plainfootsepline, 
%markcase=upper, 
automark, 
draft=false,
]{scrlayer-scrpage} 
\pagestyle{scrheadings}
\clearscrheadfoot
\ihead{\normalfont\footnotesize \texttt{bib}\LaTeX-style \texttt{archaeologie \archaeologieversion} \copyright\ by Lukas C. Bossert | Johannes Friedl}%
\rofoot{\normalfont\footnotesize  \textbf{\sffamily \thepage}}
\lofoot{\normalfont\footnotesize  \href{http://www.biblatex-archaeologie.de}{biblatex-archaeologie.de}}

\usepackage[%
%  flushmargin, %
%  marginal,
  ragged,%
  hang, %
  bottom%
]{footmisc}



\usepackage{enumitem}
\setlength{\parindent}{0pt}
\setlength{\parskip}{6pt plus 2pt minus 2pt}
\setenumerate[1]{label=(\alph*),leftmargin=*,nolistsep,parsep=\parskip}
\usepackage{changepage}
\makeindex

\defbibheading{empty}{}
\addbibresource{archaeologie-examples.bib}
\usepackage{caption}

\usepackage[%
  skins,%
  listings,%
  breakable,%
]{tcolorbox}
\lstMakeShortInline{|}

\newtcblisting[
  auto counter,
  list inside=bibexample,
%  number within=subsection,
  crefname={Example}{Examples}
]{bibexample}[2][]{%
  listing only,
  breakable,
  top=0.5pt,
  bottom=0.5pt,
  colback=codegreen!10,
  colframe=codegreen,
    left=5pt,
    right=5pt,
    sharp corners,
  boxrule=0pt,
  bottomrule=2pt,
  toprule=2pt,
  enhanced jigsaw,
  listing options={%style=tcblatex,
    numbers=left,
    numberstyle=\small\color{codeblue},
    moredelim={[is][keywordstyle]{@@}{@@}},
        basicstyle=\small\ttfamily,
    breaklines=true,
    breakautoindent=false,
    breakindent=0pt,
    escapeinside={{*@}{@*}},
  },%
  lefttitle=0pt,
  coltitle=black,
  colbacktitle=codegreen!20,
  fonttitle=\bfseries\sffamily\footnotesize,
  title={Example \thetcbcounter:  #2}, 
  #1,%  
  borderline north={1pt}{14.4pt}{codegreen,dashed},
}

\newtcblisting[
auto counter,
]{code}{%
    listing only,
    breakable,
    top=0.2pt,
    bottom=0.2pt,
    colback=codegreen!10,
    colframe=codegreen,
    left=5pt,
    right=5pt,
      sharp corners,
    boxrule=0pt,
    bottomrule=0pt,
    toprule=0pt,
    enhanced jigsaw,
    listing options={%style=tcblatex,
%        numbers=left,
        numberstyle=\tiny\color{codeblue},
        moredelim={[is][keywordstyle]{@@}{@@}},
        basicstyle=\small\ttfamily,
        breaklines=true,
        breakautoindent=false,
        breakindent=0pt,
        escapeinside={{*@}{@*}},
    },%
    lefttitle=0pt,
    coltitle=codeblue,
    colbacktitle=codegreen!10,
%    fonttitle=\bfseries\footnotesize,
%    title={Example \thetcbcounter:  #2}, 
%   #1,%  
%    borderline north={1pt}{14.4pt}{codegreen,dashed},
}

\newtcolorbox{bibbox}[1]{
      breakable,
      top=5pt,
      bottom=5pt,
      colback=codeblue!10,
      colframe=codeblue,
      left=5pt,
      right=5pt,
        sharp corners,
      boxrule=0pt,
      bottomrule=2pt,
      toprule=2pt,
      enhanced jigsaw,
        lefttitle=0pt,
        coltitle=black,
          fontupper=\small,%\ttfamily,
        colbacktitle=codeblue!20,
        fonttitle=\bfseries\footnotesize,
  title={\Cref{#1}},
        borderline north={1pt}{14.4pt}{codeblue,dashed},
}


\newtcolorbox{marker}[1][]{
enhanced,
  before skip=2mm,after skip=3mm,
  boxrule=0.4pt,left=5mm,right=2mm,top=1mm,bottom=1mm,
  colback=backcolour,
  colframe=yellow!20!black,
  sharp corners,rounded corners=southeast,arc is angular,arc=3mm,
  underlay={%
    \path[fill=tcbcol@back!80!black] ([yshift=3mm]interior.south east)--++(-0.4,-0.1)--++(0.1,-0.2);
    \path[draw=tcbcol@frame,shorten <=-0.05mm,shorten >=-0.05mm] ([yshift=3mm]interior.south east)--++(-0.4,-0.1)--++(0.1,-0.2);
    \path[fill=red!50!black,draw=none] (interior.south west) rectangle node[white]{\Huge\bfseries !} ([xshift=4mm]interior.north west);
    },
  drop fuzzy shadow,#1
  }
  
\newtcolorbox{examplebox}[1][]{
examplebox,
}

\tcbset{examplebox/.style={%
              boxrule=0pt,
              bottomrule=2pt,
              toprule=2pt,
 colframe=codeblue,
  colback=codegreen!10,
   coltitle=codegreen!10,%  coltitle=codeblue,
  bicolor,
      sharp corners,
 fontupper=\small\ttfamily,
  colbacklower=codeblue!10,
  fonttitle=\sffamily\bfseries,
  }}



\newtcblisting{example}{%
    before skip=\baselineskip,
examplebox,
breakable,
%  sidebyside,
listing and text,
}

\newcommand{\printbib}[2][5em]{%
\begingroup
\begin{bibbox}{#2}
\begin{refsection}
\setlength{\labwidthsameline}{#1} 
\nocite{#2}
\printbibliography[heading=none]
\end{refsection}
\end{bibbox}
\endgroup
}

\newcommand{\printbiball}[2][5em]{%
\begingroup
\setlength{\labwidthsameline}{#1} 
\begin{bibbox}{#2}
\begin{itemize}
\begin{refsection}
\begin{footnotesize}
\nocite{#2}%
\item[English:]{\printbibliography[heading=none]}
\item[German:]\foreignlanguage{ngerman}{\printbibliography[heading=none]}
\item[Italian:]\foreignlanguage{italian}{\printbibliography[heading=none]}
\item[French:]\foreignlanguage{french}{\printbibliography[heading=none]}
\item[Spanish:]\foreignlanguage{spanish}{\printbibliography[heading=none]}
\end{footnotesize}
\end{refsection}
\end{itemize}%
\end{bibbox}
\endgroup
}




\newcommand\DAI{Deutsches Archäologisches Institut\xspace}




\usepackage{hyperxmp}
\usepackage{hyperref}
\hypersetup{					% setup the hyperref-package options
  unicode       = true,
	pdftitle      = {bib\LaTeX-archaeologie},	% 	- title (PDF meta)
	pdfsubject    = {This citation-style covers the citation and bibliography rules of the \DAI. 
                   Various options are available to change and adjust the outcome according to one's own preferences. 
                   The style is compatible with the English, German, Italian, Spanish and French languages, since all bibstrings used are defined in each language.},% 	- subject (PDF meta)
	pdfauthor      = {Lukas C. Bossert, Johannes Friedl},	% 	- author (PDF meta)
	pdfauthortitle = {},
	pdfcopyright   = {This work may be distributed and/or modified under the
                    conditions of the LaTeX Project Public License, either version 1.3
                    of this license or (at your option) any later version.},
	pdfhighlight   = /N,
	pdfdisplaydoctitle = true,
	pdfdate        = {\the\year-\the\month-\the\day}
	pdflang        = {en},
	pdfcaptionwriter = {Lukas C. Bossert},
	pdfkeywords    = {biblatex, archaeology, humanities},
%	pdfproducer={XeLaTeX},
	pdflicenseurl  = {http://www.latex-project.org/lppl.txt},
	plainpages     = false,			% 	- 
  colorlinks     = true, %Colours links instead of ugly boxes
  urlcolor       = codeblue, %Colour for external hyperlinks
  linkcolor      = codeblue, %Colour of internal links
  citecolor      = black, %Colour of citations
  linktoc        = page,
  pdfborder      = {0 0 0},			% 	-
	breaklinks     = true,			% 	- allow line break inside links
	bookmarksnumbered  = true,		%
	bookmarksopenlevel = 4,
	bookmarksopen  = true,		%
	final          = true
}
\usepackage{bookmark}

\crefformat{lstlisting}{#2example\ #1#3}
\Crefformat{lstlisting}{#2Example #1#3}
\crefmultiformat{lstlisting}{#2examples #1#3}{; #2#1#3}{; #2#1#3}{; #2#1#3}
\Crefmultiformat{lstlisting}{#2Examples #1#3}{; #2#1#3}{; #2#1#3}{; #2#1#3}
\Crefrangeformat{lstlisting}{#3Examples #1#4--#5#2#6}
\crefrangeformat{lstlisting}{#3examples #1#4--#5#2#6}

\hyphenation{}
\externaldocument{archaeologie-ger}[archaeologie-ger.pdf]% <- full or relative path

% \makeatletter
% \patchcmd{\@mpfootnotetext}%
%   {\color@begingroup}
%   {\color@begingroup\toggletrue{blx@footnote}}
%   {}
%   {}
% \makeatother
 
 
\begin{document}
\pdfbookmark[1]{Titlepage}{title}
\title{\texttt{archaeologie} -- \\\texttt{bib\LaTeX} for archaeologists\footnote{Also very handy for (ancient) History or Classics, too.
For further information about the code visit \href{http://www.biblatex-archaeologie.de}{biblatex-archaeologie.de}: 
Comments and criticisms are welcome.
We thank especially \href{https://tex.stackexchange.com/users/35864/moewe}{moewe} and \href{https://tex.stackexchange.com/users/2478/herbert}{herbert} for their great help with the code.%
}}
\author{Lukas C. Bossert\\{\small \href{mailto:info@biblatex-archaeologie.de}{info@biblatex-archaeologie.de}} 
\and Johannes Friedl}
\date{Version: \archaeologieversion{} (\archaeologiedate)} 
\maketitle

\begin{abstract}
\noindent This citation-style covers the citation and bibliography rules of the \DAI (DAI). 
Various options are available to change and adjust the outcome according to one's own preferences. 
The style is compatible with the English, German, Italian, Spanish and French languages, since all |bibstrings| used are defined in each language.
\bigskip

\noindent For a short introduction in German see   \href{pdfdeu.biblatex-archaeologie.de}{\textbf{online}} or   \href{file:archaeologie-ger.pdf}{\textbf{local}}.
\end{abstract}


\begin{multicols}{2}
\footnotesize\parskip=0mm \tableofcontents
\end{multicols}

\section{Installation}
|archaeologie| is part of the distributions MiK\TeX \footnote{Website: \url{http://www.miktex.org}.} 
and \TeX Live\footnote{Website: \url{http://www.tug.org/texlive}.}~-- thus, you
can easily install it using the respective package manager. 
If you would like to
install |archaeologie| manually, do the following:
Download the folder |archaeologie| with all relevant files from the CTAN-server\footnote{\url{https://www.ctan.org/pkg/archaeologie}} and copy the content of the |zip|-file to the \texttt{\$LOCALTEXMF} directory of
 your system.\footnote{If you don't know what that is, have a look at
\url{http://www.tex.ac.uk/cgi-bin/texfaq2html?label=tds} or 
\url{http://mirror.ctan.org/tds/tds.html}.} 
Refresh your filename database.\footnote{ 
Here is some additional information from the UK \TeX\ FAQ:
\begin{itemize}[nosep,after=\vspace{-\baselineskip} ]
  \item \href{%
    http://www.tex.ac.uk/cgi-bin/texfaq2html?label=install-where}{%
    Where to install packages}
  \item \href{%
    http://www.tex.ac.uk/cgi-bin/texfaq2html?label=inst-wlcf}{%
    Installing files \enquote{where \LaTeX /TeX\ can find them}}
  \item \href{%
    http://www.tex.ac.uk/cgi-bin/texfaq2html?label=privinst}{%
    \enquote{Private} installations of files}
\end{itemize}
}
%%introduction from biblatex-dw copied and applied. might to be rewritten.

\section{Usage}
 \DescribeMacro{archaeologie}  The name of the bib\LaTeX-style is  |archaeologie| which has to be activated in the preamble. 

\begin{code}
\usepackage[style=archaeologie,%
          *@\meta{further options}@*]{biblatex}
\addbibresource*@\marg{|bib|-file.bib}@*
\end{code}

Without any further options the style follows the rules of the \DAI. 
No additional settings are needed,
but you can change the outcome by using some options which are explained below.\footnote{For an easy and unproblematic compiling we suggest to use \hologo{XeLaTeX} or  \hologo{LuaTeX}.}

At the end of your document you can write the command |\printbibliography| to print 
the bibliography. 
Since |archaeologie| supports different citations of various texts such as those of ancient authors and  modern scholars we suggest  having them listed in separate bibliographies. 
Further information can be found below   (\cref{bibliographie}).

\section{Overview}\label{overview}
There follows a quick overview of possible options of the style |archaeologie|. 
Contrary to the alphabetically ordered description later (\cref{options-description}) they here are listed by topic.
Furthermore you can -- at your own risk -- also use the conventional |bib|\LaTeX-options relating to indent, etc. 
For that please see the excellent documentation of  |bib|\LaTeX.

\subsection{Preamble options}\label{preamble_options}
\subsubsection{Additional bibliographies and macro lists}
\DescribeMacro{bibancient}A separate bibliography-file is loaded, in which round about 
600 ancient authors and works are listed and can be cited right away; cf. \cref{bibancient}.

\DescribeMacro{bibcorpora}
A separate bibliography-file is loaded, in which the common corpora for ancient studies are stored cf. \cref{bibcorpora}.
Additionally this activates the bibliography |archaeologie-lstabbrv.bib|.

\DescribeMacro{lstabbrv}
Activates the additional bibliography file |archaeologie-lstabbrv.bib|.
It provides a list of journals and series according to the abbreviations of the 
\DAI which can be used as |@String| macros in bibliography entries; cf. \cref{abbrv}. 

\DescribeMacro{lstlocations}
Activates the additional bibliography file |archaeologie-lstlocations.bib|
with |@String| macros of locations which can be used to automatically print out their correct exonym in the selected language; cf. \cref{lstlocations}.

\DescribeMacro{lstpublishers}
Activates the additional bibliography file |archaeologie-lstpublishers.bib| with |@String| macros 
of several publishers which can be used to easily print out their names; cf. \cref{lstpublishers}.

\subsubsection{Notation of names}
\DescribeMacro{bibfullname}
In the bibliography full names of authors and/or editors are shown; cf. \cref{bibfullname}.

\DescribeMacro{citeauthorformat}\archversion{2.3.6}
You can chose how the name of authors or editors are displayed within your text when they are cited with \cs{citeauthor}\marg{bibtex-key}.
You can chose between the options 
\meta{initials}, 
\meta{full}, 
\meta{family}, 
\meta{firstfulltheninitials},
\meta{firstinitialsthenfamily},
\meta{firstfullthenfamily}; 
cf. \cref{citeauthorformat}.



\DescribeMacro{initials}
First names are abbreviated keeping digraphs and trigraphs (this is default); cf. \cref{initials}.

\DescribeMacro{scshape}
Cited names are shown with small capital letters cf. \cref{scshape}.
Bibliography entries with |option={ancient}| or |option={frgancient}| (\cref{ancient,frgancient}) are not affected by this option.

\subsubsection{Manner of citing}

\DescribeMacro{edby}
Switches \enquote{ed.}/\enquote{Hrsg.} to \enquote{ed. by}/\enquote{hrsg. v.}; cf. \cref{edby}.

\DescribeMacro{inreferences}
Each bibliography entry which is an |@Inreference| is fully referenced according to the special rules of the 
\DAI for manuals and encyclopaedias; cf. \cref{inreferences}.

\DescribeMacro{noabbrv}
By default the short titles of journals and series (|shortjournal| and |shortseries|) are shown in the bibliography.
With this option full titles are printed instead (|journaltitle| and |series|); cf. \cref{noabbrevs}.

\DescribeMacro{publisher}
All locations and the publisher is shown. 
It also changes the format of the edition and the first print; cf. \cref{publisher}.

\DescribeMacro{seenote}
By default |archaeologie| prints author-year-system. 
With this option you can change it to a different outcome (but still according to the rules of the \DAI). 
So the first citation will be a full citation and all the following citations will refer to the first full citation; cf. \cref{seenote}

\DescribeMacro{translation}
Original title, translator and original language are shown in the bibliography. 
Setting a bibliography entry to |option={ancient}| this behaviour is default; cf. \cref{translation}. 

\DescribeMacro{yearinparens}
The year is shown in parentheses; cf. \cref{yearinparens}.

\DescribeMacro{yearseries}
Switches the order of series and year; cf. \cref{yearseries}.
 
\subsubsection{Global bibliography settings}

\DescribeMacro{width}
|width={value}| defines the bibliography width between label and reference; cf. \cref{width}.

\DescribeMacro{counter}
Reveals at the end of each reference a summary of citations in the text; cf. \cref{counter}.

\subsection{Entry Options}
A single bibliography entry can contain a value in its |options|-field.
Depending on the option it changes the behaviour of how that entry is cited; cf. \cref{options-bibentry,beispiele}. 
Beside their distinct properties all of these options have in common that the separating comma between citation and page record is missing. 
Actually this concerns citation of ancient texts and corpora where usually the |shorthand|-field is printed in citations.

\DescribeMacro{ancient}
The entry is an ancient source (e.\,g. Cicero, Plutarch, etc); cf. \cref{ancient}.

\DescribeMacro{frgancient}
The entry is a fragmentary ancient source (e.\,g. Festus); cf. \cref{frgancient}.

\DescribeMacro{corpus} 
Only the |shorthand|-field is printed.
This is needed especially for corpora of inscriptions or coins (CIL, AE, RIC, etc.); cf. \cref{corpus}.

\DescribeMacro{uniqueme} 
In cases there are different translations of an ancient work you can decide which one is the standard translation and which ones should be made unique by displaying the translator/series/editor; cf. \cref{uniqueme}.

\changes{v1.1}{2015/06/04}{New options added in summary.}

\subsection{Cite commands}
\label{cite-commands}
\subsubsection{cite and cites}
\DescribeMacro{\cite}
As always citing is done with \cs{cite}:
\begin{code}
\cite*@\oarg{prenote}\oarg{postnote}\marg{bibtex-key}%@*
\end{code}

\meta{prenote} sets a short preliminary note (e.\,g. \enquote{Vgl.}) and \meta{postnote} is usually used for page numbers.
If only one optional argument is used then it is \oarg{postnote}.
\begin{code}
\cite*@\oarg{postnote}\marg{bibtex-key}%@*
\end{code}
The \meta{bibtex-key} corresponds to the key from the bibliography file.

\begin{example}
Public space is part of a city says \cite{Osland2016}.
\end{example}

\DescribeMacro{\cites}
If one wants to cite several authors or works a very convenient way is the following using the \cs{cites}-command:
\begin{code}
\cites(pre-prenote)(post-postnote)
  *@\oarg{prenote}\oarg{postnote}\marg{bibtex-key}@*%
  *@\oarg{prenote}\oarg{postnote}\marg{bibtex-key}@*%
  *@\oarg{prenote}\oarg{postnote}\marg{bibtex-key}\ldots@*
\end{code}
\begin{example}
Public space is part of a city say \cites(cf.)(){Osland2016}{Evangelidis2014}.
\end{example}
 
 \subsubsection{parencite and parencites}
\DescribeMacro{\parencite}
Sometimes a citation has to be put in parentheses. 
Therefore we implemented the command \cs{parencite}:
\begin{code}
\parencite*@\oarg{postnote}\marg{bibtex-key}%@*
\end{code} 
This cite command takes care of the correct corresponding parentheses and brackets.
Especially in |@Inreference| citations the parentheses  change to (square) brackets.
The example shown in \cref{faq:inreference} makes it clear.
\begin{example}
Public space is part of a city \parencite{Osland2016}.
\end{example}

\DescribeMacro{\parencites}
Of course there is also the possibility to cite several authors/works in parentheses.
This is done with \cs{parencites}:
\begin{code}
\parencites(pre-prenote)(post-postnote)%
*@\oarg{prenote}\oarg{postnote}\marg{bibtex-key}@*%
*@\oarg{prenote}\oarg{postnote}\marg{bibtex-key}@*%
*@\oarg{prenote}\oarg{postnote}\marg{bibtex-key}\ldots@*
\end{code}
\begin{example}
Public space is part of a city \parencites(cf.)(){Osland2016}{Evangelidis2014}.
\end{example}

\subsubsection{textcite and textcites}
\DescribeMacro{\textcite}
Beside the listed \cs{cite} commands above there is a third way of citing:
\cs{textcite} is useful if the author should be mentioned in the text and
the remaining components such as year and page will immediately follow in parentheses. 
\begin{code}
\textcite*@\oarg{postnote}\marg{bibtex-key}%@*
\end{code} 

\begin{example}
Public space is part of a city says \textcite{Osland2016}.
\end{example}

\DescribeMacro{\textcites}
And again there is also a \cs{textcites} in case of several authors: 
\begin{code}
\textcites(pre-prenote)(post-postnote)%
  *@\oarg{prenote}\oarg{postnote}\marg{bibtex-key}@*%
  *@\oarg{prenote}\oarg{postnote}\marg{bibtex-key}@*%
  *@\oarg{prenote}\oarg{postnote}\marg{bibtex-key}\ldots@*
\end{code}
\begin{example}
Public space is part of a city say \textcites{Osland2016}[cf.][]{Evangelidis2014}.
\end{example}
Just be aware that using \oarg{prenote} may be give a odd sounding sentence depending what you use \oarg{prenote} for.

\subsubsection{footcite}
 \DescribeMacro{\footcite}
 Beside the listed \cs{cite} commands above there are more possibilities citing:
 There is also the possibility to put the citation into a footnote at once with \cs{footcite}:
 \begin{code}
\footcite*@\oarg{prenote}\oarg{postnote}\marg{bibtex-key}@*
\end{code}
\begin{example}
Public space is part of a city.\footcite{Osland2016}
\end{example}
This is the equivalent to |\footnote{\cite{Osland2016}.}| but it saves you a lot of time typing.
 \DescribeMacro{\footcites} And there is as well \cs{footcites}:
\begin{example}
Public space is part of a city.\footcites(cf.)(){Osland2016}{Evangelidis2014}
\end{example}
 
 \subsubsection{smartcite and smartcites}

\DescribeMacro{\smartcite}
And there is also a clever way citing with \cs{smartcite}.
\cs{smartcite} depends on its environment it is used in. If it is your normal text it behaves like \cs{footcite} and will print the citation within a footnote.
If it is already within a footnote it will be handled like \cs{cite}. 
 is a clever 
\begin{code}
\smartcite*@\oarg{postnote}\marg{bibtex-key}%@*
\end{code} 

\begin{example}
Public space is part of a city.\smartcite{Osland2016} 
And sometimes more than that.\footnote{\smartcite[cf.][]{Evangelidis2014}.}
\end{example}


\DescribeMacro{\smartcites}
And again there is also a \cs{smartcites} in case of several authors: 
\begin{code}
\smartcites(pre-prenote)(post-postnote)%
  *@\oarg{prenote}\oarg{postnote}\marg{bibtex-key}@*%
  *@\oarg{prenote}\oarg{postnote}\marg{bibtex-key}@*%
  *@\oarg{prenote}\oarg{postnote}\marg{bibtex-key}\ldots@*
\end{code}
\begin{example}
Public space is part of a city.\smartcites{Osland2016}{Evangelidis2014} 
And sometimes more than that.\footnote{\smartcites{Osland2016}[cf.][]{Evangelidis2014}.}
\end{example}

\subsubsection{autocite}

\DescribeMacro{\autocite}
With \cs{autocite} there is a flexible way of citing. 
We set up \cs{autocite} as \cs{footcite} by default.
If you want to change it you can also write in the preamble e.\,g. |autocite=inline|.
\begin{code}
\autocite*@\oarg{prenote}\oarg{postnote}\marg{bibtex-key}%@*
\end{code} 

\begin{example}
Public space is part of a city.\autocite{Osland2016} 
\end{example}

\subsubsection{fullcite and footfullcite}
\DescribeMacro{\fullcite}\DescribeMacro{\footfullcite}
With \cs{fullcite} and \cs{footfullcite} you can print the complete entry in your current text.
\begin{code}
\fullcite*@\oarg{prenote}\oarg{postnote}\marg{bibtex-key}@*
\footfullcite*@\oarg{prenote}\oarg{postnote}\marg{bibtex-key}@*
\end{code} 

\begin{example}
Public space is part of a city.\footfullcite{Osland2016}
As can be read in \fullcite{Evangelidis2014}
\end{example}


\subsubsection{citeauthor and citetitle}

\DescribeMacro{\citeauthor}
\DescribeMacro{\citetitle}\label{citeauthor}%
Furthermore and in addition to the ›normal‹ \cs{cite}-commands one can also cite only the author or the work title in the text and in the footnotes.
\begin{code}
\citeauthor*@\oarg{prenote}\oarg{postnote}\marg{bibtex-key}%@*
\end{code} 
  and for the works 
\begin{code}
\citetitle*@\oarg{prenote}\oarg{postnote}\marg{bibtex-key}%@*
\end{code} 

\begin{example}
Public space is part of a city says \citeauthor{Osland2016} in \citetitle{Osland2016}.
\end{example}
For further information cf. \cref{citeauthorformat}.

Note that the |\citetitle|-command works differently with ancient works (for all those who have  |option={ancient}|). 
First the field |origtitle| will be shown if this field is empty it will show its |title| instead. 

\DescribeMacro{\citetitle*}
Sometimes you don’t need the year of publication but still want the published title. 
Then |\citetitle*| is what you need:
\archversion{2.3.4} 
\begin{example}
In 2016  \citeauthor{Osland2016} says in \citetitle*{Osland2016} 
public space is part of a city.
\end{example}

Note that the |\citetitle*|-command will always show the field |title| without the year of publication it doesn’t matter if there is |option={ancient}|.

\subsubsection{citetranslator}
\DescribeMacro{\citetranslator}\label{citetranslator}%
\archversion{2.3.0} 
Addionally there is also a \cs{cite}-command which can be used to print the translator of an (ancient) publication.
\begin{code}
\citetranslator*@\marg{bibtex-key}%@*
\end{code}
This will print the name(s) of the translator according to the chosen |citeauthorformat|.

\begin{example}
\blockcquote[12,25,1]{Cic:Att}{Aber dein Heim ist das Forum.} (\citetranslator{Cic:Att})
\end{example}
\archversion{2.3.3} 
If there is no translator given it will name you as translator using a |bibstring| with \enquote{own translation}.
\begin{example}
\blockcquote[3,62,1]{Artem}{The Agora means confusion and uproar because of the people that are gathered there.} (\citetranslator{Artem})
\end{example}
You can change the |bibstring| \enquote{own translation} and replace it with anything you like, 
at least it is done in the preamble of your document.
\begin{code}
\DefineBibliographyStrings{english}{%
  owntranslation = {by me},%
}
\end{code}



\DescribeMacro{\citetranslator*}\archversion{2.3.0}
\begin{code}
\citetranslator* *@\marg{bibtex-key}%@*
\end{code} 

The starred version prints also the informtion from which language the text has been translated.

\begin{example}
\blockcquote[12,25,1]{Cic:Att}{Aber dein Heim ist das Forum.} (\citetranslator*{Cic:Att})
\end{example}



\subsection{Entries with @String}\label{string}
The citation rules of the \DAI instruct to abbreviate journals and series according to a given list.\footnote{\url{http://www.dainst.org/documents/10180/70593/03_Liste+abzukürzender+Zeitschriften_quer.pdf} <2016-06-06>}
For this purpose we provide a list with bibliography macros which refer to these abbreviations. 
These abbreviations can be included by loading package option |lstabbrv| (cf. \cref{abbrv}). 
Besides there are two further lists with |@String| macros cf. \cref{lstpublishers,lstlocations}.

\DescribeMacro{@String} The style |archaeologie| respects the guidelines of the \DAI 
and is therefore compatible with the given abbreviations of journals and series.
To minimize the susceptibility to errors and to omit unnecessary typing of sometimes very long journal titles |archaeologie| works with so-called |@Strings|.
The advantage of these |@Strings| is that several bibliography entries can be defined by one globally given value. 
The |@String| is loaded at begin of the |bib|-file, therefore all |@Strings| have to be previous to all other bibliography entries.
 
To use this offer of simplification the following bibliography fields should be field with such a a |@String|: |journaltitle| and |shortjournal|,
(|series| and |shortseries|.
In \cref{abbrv-lists} there is a list with all the abbreviations given by the \DAI, 
in which the |@String| (with endings |-short| for |shortjournal| or |shortseries|) are listed in the left column.  
An |@String| has to be written \emph{without} any curly brackets.\footnote{If you use \emph{JabRef} in its non-coding window, 
then you have to write \#|AyasofyaMuezYil|\#. 
JabRef converts this internally to a |@String| and omits the \# in the coding window. 
\emph{BibDesk} provides such conversion as well by pressing
  %|cmd-R| 
\LKeyStrg + \LKey{R}
 which enables direct BibTeX typing without enclosing curly brackets.}


An example shows how to use it:

\begin{bibexample}[label=Koyunlu1990]{{@}Article\{Koyunlu1990,…\}}
@Article{Koyunlu1990,
  author       = {Koyunlu, A.},
  title        = {Die Bodenbelage und der Errichtungsort der Hagia Sophia},
  journaltitle = AyasofyaMuezYil,  %@String used
  shortjournal = AyasofyaMuezYil-short,  %@String used
  volume       = {11},
  pages        = {147--156},
  year         = {1990},
}
\end{bibexample}


That article appeared within a rather unusual journal, 
which should be abbreviated with ›AyasofyaMüzYıl‹.

To save  time  looking for the special character and insert ›ı‹ manually 
it is written in the |@String| with an ›i‹ (for further information see \cref{abbrv-lists}) but will be replaced after compiling with the correct character:

\printbib[6em]{Koyunlu1990}


Whether using the provided abbreviation list with |lstabbrv| or filling the |journaltitle| and |shortjournal| fields manually, 
|archaeologie| uses by default short titles if defined.
The default embedding of such abbreviations can be switched off, of course.
In case you use the package option |noabbrv| in the preamble (see \cref{noabbrevs}), then the output changes as follows:
\begin{bibbox}{Koyunlu1990}\footnotesize
\parbox[t]{2cm}{Koyunlu 1990} \parbox[t]{9cm}{A. Koyunlu, 
Die Bodenbelage und der Errichtungsort der Hagia Sophia, {\color{red}Ayasofia Müzesi yıllığı. Annual of Ayasofya Museum} 11, 1990, 147–156}
\end{bibbox}
Nevertheless the advantage of our abbreviation list lies in the possibility of creating a separate bibliography 
with all the abbreviations of used journal titles and series (see \cref{bibliographie}) without being prone to citation differences and typing errors.

However, if a journal or a series is \emph{not} included in the list (\cref{abbrv-lists}) 
then this journal/series will \emph{not} be abbreviated and converted to full title in curly brackets in the respective field e.\,g. |journaltitle=|\marg{title of the journal} or |series=|\marg{name of the series}. 
Therefore the field content will not be printed. 
At least |biblatex-biber| gives a warning in its log which can be checked.\footnote{For example something like \enquote{\enquote{journaltitle} in entry \meta{entrykey} cannot be null, deleting it}.}
For the following examples we use |@String| whenever it is appropriate and possible.

Lastly we want to point out that |@Strings| can also be used partly as following shows:

\begin{code}
@Incollection{Mundt2015,
  ...
  location     = Berlin #{ and Boston}, %@String partly used
  ...
}
\end{code}

Each time you want to leave the |@String| environment and enter the curly bracket environment (and reverse) make use of a hash \# to concatenate elements.  

\section{Details of optional preferences}\label{options-description}
In the following  we give a more detailed insight into the various options of |archaeologie| 
and show their results on the bases of concrete examples.
Changes made by these options are {\color{red}coloured in red}.

\subsection{Preamble options}\label{options-preamble}
Optional preferences in the preamble are loaded within the package |bib|\LaTeX:
\begin{code}
\usepackage[%        
    backend=biber,  % activates biber (default; 
                    % but will give an error if not done)
    style=archaeologie,   % loads the style *@|archaeologie|@*
    inreferencesasfullcite=true,    % option *@|inreferencesasfullcite|@* is loaded
    lstabbrv              % option *@|lstabbrv|@* is loaded as well
    ]{biblatex}
\end{code}
In this example the style |archaeologie| is loaded with options |inreferencesasfullcite| and |lstabbrv|. 
Now, manual entries don't appear in author-year-style anymore and journal/series |@string|-macros are enabled.
By the way, it doesn't matter if you write |inreferencesasfullcite| or |inreferencesasfullcite=true|.

Each of the listed options is disabled by default even if we strongly recommend their use in particular the additional bibliographies and |@String| lists. 
All remaining options are rather a matter of taste.

Despite to the overview section (\cref{overview}) the following list is arranged in alphabetic order.

\subsubsection{bibancient}\label{bibancient}
\DescribeMacro{bibancient}
In case of citing ancient authors and their works you can do it with common \cs{cite}-commands.
Exclusively for this case we included a modification that respects the different citation of ancient authors and works.
With the option |bibancient| you load an additional bibliography called |archaeologie-bibancient.bib| in which we inserted almost 600 ancient authors and works with their abbreviation according to The New Pauly/Thesaurus Linguae Latinae.
For the complete list of those see \cref{list-bibancient}.
Using these pre-sets is recommended because it will guarantee a high level of consistency and minimize error-proneness.

You can cite the authors or works with their |bibtex-key| which you find in bold in the left column of the list. 
Authors and works are separated in the |bibtex-key| by a colon.
The entry on the right marks the |shorthand| which will be printed in your paper.

Let us make it clear with an example:

With the loaded option |bibancient| you can cite like this:  


\begin{example}
\footnote{\cite[3,2,5--7]{Apul:met}.}
\end{example}


The corresponding bibliography-entry looks like this
\begin{bibexample}[label=Apul:met]{{@}Book\{Apul:met,…\}}
@Book{Apul:met,
  author      = {Apuleius Madaurensis, Lucius},
  title       = {metamorphoses},
  shorthand   = {Apul. met.},
  shortauthor = {Apuleius},
  keywords    = {ancient},
  options     = {ancient},
}
\end{bibexample}
All entries in the mentioned additional bibliography contain the line |keywords = {ancient}|.
With that you can print all ancient authors in a separated bibliography by typing:
\begin{code}
\printbibliography[keyword=ancient]
\end{code}
\printbib{Apul:met}

\begin{refsection}
By means of the |bibtexkey| (e.g. |Apul:met|) you can also cite only authors or titles like this: 


\begin{example}
\citeauthor{Apul:met} remarks in \citetitle{Apul:met} ...
\end{example}

\end{refsection}


\changes{v1.5}{2016/05/31}{Extra Bibliographie}

\subsubsection{bibcorpora}\label{bibcorpora}
\DescribeMacro{bibcorpora}
This loads an additional bibliography which contains the most important corpora for ancient studies, so you can cite them right away in your document without creating a new bibentry by yourself. 
These corpora are listed in \cref{list-bibcorpora}. 
Advantage and usage mostly correspond to |bibancient|, 
so have a look at \cref{bibancient} for details.



\subsubsection{lstabbrv}\label{abbrv}
\DescribeMacro{lstabbrv}
If you want to benefit from the above mentioned method with |@String| (cf. \cref{string}) 
you have to activate the option called |lstabbrv| (list of abbreviations) in the preamble.
Once activated the additional bibliography |archaeologie-lstabbrv.bib| is loaded. 
In this bibliography all abbreviations listed in \cref{abbrv-lists} are stored; 
for further details of usage see \cref{string}.

\subsubsection{lstlocations}\label{lstlocations}
\DescribeMacro{lstlocations}
This loads an additional bibliography with |@Strings| of locations used to print out their correct exonym in the selected language. 
In that case you are not forced to change location spelling when switching the language. 
(Otherwise it is necessary to adjust location names like \emph{Rome} to \emph{Rom} or \emph{Roma} 
in your potentially multiple-used bibliography each time you change the language of your scientific text).
For details on the locations list, cf. \cref{list-locations}.

\subsubsection{lstpublishers}\label{lstpublishers}
\DescribeMacro{lstpublishers}
Activates the additional bibliography file |archaeologie-lstpublishers.bib| with |@Strings| 
of publishers which can be used to print out their correct name. 
Benefits are similar to the other lists mentioned above and in \cref{string}.
For the list, cf. \cref{list-publishers}




\subsubsection{seenote}\label{seenote}
\DescribeMacro{seenote}
Even if author-year-citation seems to be commonly accepted in Ancient Studies in the meantime you may want to use a traditional citation style. 
For this purpose you can switch to the other allowed citation rule by the \DAI
which works like this:
If you cite a work for the first time in a footnote |archaeologie| will print a full cite which contains all bibliography elements.
Henceforward each following citation is printed as short cite and will additionally refer to the footnote where the first cite was done.
Bibliography entries with |options={ancient}| are excluded from this speciality and are cited as always.

You can use the cite-commands \cs{cite(s)} and \cs{parencite(s)} but \cs{textcite(s)} 
will behave like \cs{cite(s)} because |seenote| actually just checks for occurrences in footnotes and does not refer to cites in running text.

We give an example:
\begin{tcolorbox}[examplebox] 
This is the first citation.|\footnote{\cite{Ball2013}.}|
This is one in between.|\footnote{anything in here.}|
And this is the third footnote and the second citation.|\footnote{\cite[470]{Ball2013}.}|
\tcblower
This is the first footnote.\footnote{L. F. Ball – J. J. Dobbins, Pompeii Forum Project. Current thinking on the Pompeii Forum, 117/3, 2013, 461–492.}
This is one in between.\footnote{anything in here.}
And this is the third footnote and the second citation.\footnote{Ball – Dobbins loc. cit. (see n. 1) 470.}
\end{tcolorbox}



\changes{v1.5}{2016/05/31}{Rückverweis}

\subsubsection{biblabel}\label{biblabel}
\DescribeMacro{biblabel=bold}
\DescribeMacro{biblabel=parens}
\DescribeMacro{biblabel=brackets}
You can set the ›biblabel‹ in bold, parens or brackets. \archversion{2.3.7}
This only applied to their appearence in the bibliography.
The style in footnotes are not being touched by |biblabel|. 


%%% Example


\subsubsection{eventdatelanguage}\label{eventdatelanguage}
\DescribeMacro{eventdatelanguage}
\archversion{2.3.7}

%%% Example


\subsubsection{translation}\label{translation}
\DescribeMacro{translation}
Once this option is activated the original title, the original language and the translator of the work are printed (|origtitle|, |origlanguage|, |translator|).
For ancient texts and fragments (|options={ancient}| or |options={frgancient}|) this is default, 
so they will always be printed with original title, language and translator.

An example will clarify matters:
The bibliographical entry |Lefebvre2011| contains following fields:
\begin{bibexample}[label=Lefebvre2011]{{@}Book\{Lefebvre2011,…\}}
@Book{Lefebvre2011,
  author       = {Lefebvre,Henri},
  title        = {The Production of Space},
  publisher    = {Blackwell Publishing Ltd},
  location     = {Maien, MA and Oxford and Victoria},
  year         = {2011},
  edition      = {30},
  origlocation = {Oxford},
  origyear     = {1991},
  origtitle    = {La production de l’espace},
  origlanguage = {french},
  translator   = {Donald Nicholson-Smith},
}
\end{bibexample}

The bibliography result is:
\printbib[6em]{Lefebvre2011}

By activating option |translation| it will change to:

\begin{bibbox}{Lefebvre2011}\footnotesize
\parbox[t]{2cm}{Lefebvre 2011} \parbox[t]{9cm}{H. Lefebvre,  The Production of Space, 
{\color{red} La production de l’espace, trans. from French by D. Nicholson-Smith} \textsuperscript{30}(Oxford 1991; repr. Maien, MA 2011)}
\end{bibbox}
 
However, it works not only with entries like |@Book| but also with e.\,g. |@Article|:

\begin{bibexample}[label=Lefebvre1977]{{@}Article\{Lefebvre1977,…\}}
@Article{Lefebvre1977,
  author       = {Lefebvre, Henri},
  title        = {Die Produktion des städtischen Raums},
  journaltitle = {ARCH+},
  volume       = {34},
  pages        = {52--57},
  year         = {1977},
  translator   = {Franz Hiss and Hans-Ulrich Wegener},
  origlanguage = {french},
  number       = {9},
  origtitle    = {Introduction à l'espace urbain},
}
\end{bibexample}

Once again the bibliography entry alters:

\begin{bibbox}{Lefebvre1977}\footnotesize
\parbox[t]{2cm}{Lefebvre 1977} \parbox[t]{9cm}{H. Lefebvre, 
Die Produktion des stätischen Raums, \emph{Introduction à l’espace urbain}, 
{\color{red} trans. from French by F. Hiss -- H.-U. Wegener}, ARCH+ 34/9, 1977, 52–57}
\end{bibbox}

\subsubsection{inreferencesasfullcite}\label{inreferences}
\DescribeMacro{inreferencesasfullcite}  
There is the possibility to cite inreferences in the footnote as a full citations.
It is only required that the bibliography-entry is an |@Inreference|  (cf. \cref{inreference}).
 
Another example makes it clear: 
\begin{bibexample}[label=Nieddu1995]{{@}Inreference\{Nieddu1995,…\}}
@Inreference{Nieddu1995,
  author    = {Nieddu, Giuseppe},
  title     = {Dei Consentes},
  booktitle = LTUR-short,
  pages     = {9\psq},
  year      = {1995},
  volume    = {2},
}
\end{bibexample}

There are two ways to display this entry:
 \begin{enumerate}
 \item by default it will give:  
 \begin{example}
\footnote{\cite{Nieddu1995}.}
 \end{example}
 \item with the option |inreferencesasfullcite| it will change:
 \begin{tcolorbox}[examplebox]
 |\footnote{\cite{Nieddu1995}.}|
 \tcblower
\footnote{LTUR 2 (1995) 9\,f. s. v. Dei Consentes (G. Nieddu).}
 \end{tcolorbox}
  \end{enumerate}

If the \oarg{postnote} is defined with the columns/page number, (e.\,g. |\cite[9]{Nieddu1995}|), 
then it will change the position for the \oarg{postnote}:
\begin{enumerate} 
 \item by default it will give: 
 \begin{example}
\footnote{\cite[9]{Nieddu1995}.}
 \end{example}
 \item with the option |inreferencesasfullcite| it will change again:
  \begin{tcolorbox}[examplebox]
|\footnote{\cite[9]{Nieddu1995}.}|
 \tcblower
\footnote{LTUR 2 (1995) 9 s. v. Dei Consentes (G. Nieddu).}
 \end{tcolorbox}
  \end{enumerate}



\begin{marker}
    Activating |inreferencesasfullcite=true| causes the cited |@Inreference| entry to be automatically \emph{omitted} in the (final) bibliography,
because the entry is fully cited before in the footnotes (also stated in the \DAI rules).
\end{marker}

If the option is not used (|inreferencesasfullcite=false|) the entry will look like this in the bibliography:

\printbib[6em]{Nieddu1995} 

\subsubsection{yearseries}\label{yearseries}
\DescribeMacro{yearseries}
The option |yearseries| leads to a different position of the fields |series| and |number|.
The |series| of a |@Book| or |@Collection| is now printed \emph{after} the year.
An example with an |@Incollection| demonstrates the effect of this option:
 
\begin{bibexample}[label=Mundt2015]{{@}Incollection\{Mundt2015,…\}}
@Incollection{Mundt2015,
  author       = {Mundt, Felix},
  title        = {Der Mensch, das Licht und die Stadt},
  subtitle     = {Rhetorische Theorie und Praxis antiker und humanistischer Städtebeschreibung},
  pages        = {179--206},
  editor       = {Therese Fuhrer and Felix Mundt and Jan Stenger},
  booktitle    = {Cityscaping},
  booksubtitle = {Constructing and Modelling Images of the City},
  publisher    = WdG,
  location     = Berlin #{ and Boston}, %@String partly used
  year         = {2015},
  series       = Philologus-long #{ Supplement},
  number       = {3},
  shortseries  = Philologus-short #{ Suppl.},
}
\end{bibexample}

Without any option activated it will look like this:
\printbib[5em]{Mundt2015}
 
By activating |yearseries| it will change to:
\begin{bibbox}{Mundt2015}\footnotesize
\parbox[t]{1.7cm}{Mundt 2015} \parbox[t]{9cm}{F. Mundt, Der Mensch, das Licht und die Stadt. Rhetorische Theorie und Praxis antiker und humanistischer Städtebeschreibung, in: T. Fuhrer -- F. Mundt -- J. Stenger (ed.), Cityscaping. Constructing and Modelling Images of the City (Berlin 2015) {\color{red}Philologus Suppl. 3,} 179–206}
\end{bibbox}

\subsubsection{citeauthorformat}\label{citeauthorformat}\archversion{2.3.6}
\DescribeMacro{citeauthorformat}
\DescribeMacro{=initials}
\DescribeMacro{=full}
\DescribeMacro{=family}
\DescribeMacro{=firstfulltheninitials}
\DescribeMacro{=firstinitialsthenfamily}
\DescribeMacro{=firstfullthenfamily}
Every time you mention authors in the running text it is possible to cite them 
directly with their names (\cs{citeauthor}\marg{bibtex-key}) or their works  (\cs{citetitle}\marg{bibtex-key});
this has the benefit that they will be linked to your bibliography (cf. \cref{citeauthor}).

By default the author's name is printed with abbreviated first name\footnote{Usually only the first letter, but setting the option |initials| to true it might change (cf. \cref{initials}).} and last name.
If you prefer to have full names printed (in running text, not in the bibliography!) switch on the option |citeauthorformat=full|.
If you want in contrast the authors to be shorten to their last names use |citeauthorformat=family|.

The following example illustrates it:

\begin{bibexample}[label=Boehmer1985]{{@}Article\{Boehmer1985,…\}}
@Article{Boehmer1985,
  author       = {Boehmer, Rainer Michael and Wrede, Nadja},
  title        = {Astragalspiele in und um Warka},
  journaltitle = BaM,
  shortjournal = BaM-short,
  volume       = {16},
  pages        = {399--404},
  year         = {1985},
}
\end{bibexample}

Let's assume you would like to write something like that and
after compiling it will look like this, 
because the default is set  |citeauthorformat=initials| 

\begin{refsection}
\begin{example}
... , this is also shown by \citeauthor{Boehmer1985} 
 in their latest article \citetitle{Boehmer1985}.
 \end{example}



Or you can change it using the settings in the preamble:

\begin{enumerate}
\item\label{name:full} 
\begin{tcolorbox}[examplebox]
 |citeauthorformat=full| 
 \tcblower
\ldots , this is also shown by {\color{red}Rainer Michael Boehmer and Nadja Wrede} in their latest article \emph{Astragalspiele in und um Warka} (1985).
\end{tcolorbox}
\item\label{name:family}
\begin{tcolorbox}[examplebox]
 |citeauthorformat=family| 
 \tcblower
\ldots , this is also shown by {\color{red}Boehmer and  Wrede} in their latest article \emph{Astragalspiele in und um Warka} (1985).
\end{tcolorbox}
\item\label{name:firstfulltheninitials}
\begin{tcolorbox}[examplebox]
 |citeauthorformat=firstfulltheninitials|
 \tcblower
\ldots , this is also shown by \textcolor{red}{Rainer Michael Boehmer  and Nadja Wrede} in their latest article \emph{Astragalspiele in und um Warka} (1985). 
\textcolor{red}{R. M. Boehmer  and N. Wrede} argue that \ldots
\end{tcolorbox}
  If you use |citeauthorformat=firstfulltheninitials| the first citation will look like \ref{name:firstfull}, but after that that all following citations of |\citeauthor{Boehmer1985}| will change to the default behaviour and show the initials.
  
\item\label{name:firstfulltheninitials}
\begin{tcolorbox}[examplebox]
 |citeauthorformat=firstfullthenfamily|
 \tcblower
\ldots , this is also shown by \textcolor{red}{Rainer Michael Boehmer  and Nadja Wrede} in their latest article \emph{Astragalspiele in und um Warka} (1985). 
\textcolor{red}{Boehmer  and Wrede} argue that \ldots
\end{tcolorbox}
 If you use |citeauthorformat=firstfullthenfamily| the first citation will look like \ref{name:firstfull}, but after that that all following citations of |\citeauthor{Boehmer1985}| will change and only show the family name(s), \ref{name:family}.


\item\label{name:firstinitialsthenfamily}
\begin{tcolorbox}[examplebox]
 |citeauthorformat=firstinitialsthenfamily|
 \tcblower
\ldots , this is also shown by \textcolor{red}{R. M. Boehmer  and N. Wrede} in their latest article \emph{Astragalspiele in und um Warka} (1985). 
\textcolor{red}{Boehmer  and Wrede} argue that \ldots
\end{tcolorbox}

 If you use |citeauthorformat=firstinitialsthenfamily| the first citation will look like |citeauthorformat=initials|, but after that that all following citations of |\citeauthor{Boehmer1985}| will change and only show the family name(s), \ref{name:family}.
\end{enumerate}
\end{refsection}

To complete this example,
here is the appearence of the entry in a bibliography:
\printbib[9.5em]{Boehmer1985}


Two things are left to mention: 
\begin{enumerate}
\item Citing an author in a footnote will start again with a first mention and then continue writing the name depending on your chosen option.

\item There is a slightly different behavior if you use  \cs{citeauthor} or \cs{citetitle}  with ancient authors and work titles (|options={ancient}|).
Instead of printing the field |author| which contains usually the full ancient name the field |shortauthor| is considered in which you can record the more common name of the ancient author.
Ancient work titles will be printed without the year in parentheses. 
Both are demonstrated in the following example: Based on the bibliography entry
\end{enumerate}
\begin{bibexample}[label=Quint:inst]{{@}Book\{Quint:inst,…\}}
@Book{Quint:inst,
  author       = {Fabius Quintilianus, Marcus},
  title        = {Ausbildung des Redners},
  subtitle     = {Institutio oratoria},
  publisher    = WBG,
  location     = {Darmstadt},
  year         = {2015},
  edition      = {6},
  origlanguage = {latin},
  translator   = {Rahn, Helmut},
  shorthand    = {Quint. inst.},
  shortauthor  = {Quintilian},
  keywords     = {ancient},
  options      = {ancient},
}
\end{bibexample}

and the following statement we obtain the result:

\begin{refsection}
\begin{example}
... and \citeauthor{Quint:inst} names in \citetitle{Quint:inst} the  necessary physical qualities of an orator, too.
\end{example}
\end{refsection}

And again the bibliography for |@Book{Quint:inst}|:
\printbib[5em]{Quint:inst}

\subsubsection{yearinparens}\label{yearinparens}
\DescribeMacro{yearinparens}%
As the options name evokes the publication year of the cited entries 
(|year| or year from |date|) will be put in parentheses,
in footnotes as well as in the bibliography. 
The ›Klammerregel‹ (correct alternation of different brackets) will be respected.

In the case of a common entry which will be shown like this
\begin{example}
\footnote{\cite[475]{Ball2013}.}
\end{example}


now we get

\begin{tcolorbox}[examplebox] 
|\footnote{\cite[475]{Ball2013}.}|
\tcblower
\footnote{Ball – Dobbins {\color{red}(}2013{\color{red})}, 475.}
\end{tcolorbox}


\subsubsection{scshape}\label{scshape}
\DescribeMacro{scshape}
You can also change the look of your citations. 
With |scshape| author names are set to small capitals---in footnotes and in the bibliography.

Entries without |author| or |editor| setting but with a defined
 |label| (\cref{unknown}) are excluded from this option
because |label| is not an author name but a self-defined expression with varying purposes.
Further excluded are ancient authors (|options={ancient}| or |options={frgancient}|).

By default---to quote the just established example |\@Article{Ball2013}|---we have again by default:

\begin{example}
\footnote{\cite[475]{Ball2013}.}
\end{example}

But with |schape| it will turn into:

\begin{tcolorbox}[examplebox] 
|\footnote{\cite[475]{Ball2013}.}|
\tcblower
\footnote{{\scshape {\color{red}Ball – Dobbins}} 2013, 475.}
\end{tcolorbox}

And since the entry looks like this

\begin{bibexample}[label=Ball2013]{{@}Article\{Ball2013,…\}}
@Article{Ball2013,
author       = {Larry F. Ball and John J. Dobbins},
title        = {Pompeii Forum Project},
subtitle     = {Current Thinking on the Pompeii Forum},
journaltitle = AJA,
shortjournal = AJA-short,
volume       = {117},
pages        = {461--492},
year         = {2013},
doi          = {10.3764/aja.117.3.0461},
jstor       = {10.3764/aja.117.3.0461},
number       = {3},
}
\end{bibexample}


the output in the bibliography changes from:

\printbib[8em]{Ball2013}

to 

\begin{bibbox}{Ball2013}\footnotesize
\parbox[t]{3cm}{{\scshape \color{red}Ball – Dobbins} 2013}\parbox[t]{8.5cm}{%
L. F. Ball – J. J. Dobbins, Pompeii Forum Project. Current Thinking on the Pompeii Forum, AJA 117/3, 2013, 461–492,\newline
...}
\end{bibbox}

\subsubsection{bibfullname}\label{bibfullname}
\DescribeMacro{bibfullname}
This will show the full name of an author and/or editor in the bibliography. 
By default first names are abbreviated.

Without any options the entry

\begin{bibexample}[label=Osland2016]{{@}Article\{Osland2016,…\}}
@Article{Osland2016,
  author       = {Osland, Daniel},
  title        = {Abuse or Reuse?},
  subtitle     = {Public Space in Late Antique Emerita},
  journaltitle = AJA,
  shortjournal = AJA-short,
  volume       = {120},
  pages        = {67--97},
  year         = {2016},
  jstor        = {10.3764/aja.120.1.0067},
  number       = {1},
  zenon        = {001454110},
}
\end{bibexample}

looks like

\printbib[6em]{Osland2016}

and with |bibfullname| it will change to:

\begin{bibbox}{Osland2016}\footnotesize
\parbox[t]{2cm}{Osland 2016} \parbox[t]{9cm}{{\color{red}Daniel} Osland, Abuse or Reuse? Public Space in Late Antique Emerita, AJA 120/ 1, 2016, 67–97,\\
...}
\end{bibbox}

\subsubsection{noabbrv}\label{noabbrevs}
\DescribeMacro{noabbrv}
According to the guidelines of the \DAI journal titles and series have to be abbreviated.
Therefore the fields |shortjournal| or |shortseries| will be considered. 
If you like to have printed full names of journals and series instead you can switch on the option |noabbrv|.
For an example see \cref{string,Koyunlu1990}.

\subsubsection{publisher}\label{publisher}
\DescribeMacro{publisher} 
Once activated all locations and the publisher are printed. 
This will lead to a different output of the edition which will be right in front of the year.
In case of reprint or second edition the first edition |origyear| will be put in square brackets after the year.

\begin{bibexample}[label=Emme2013]{{@}Book\{Emme2013,…\}}
@Book{Emme2013,
  author    = {Burkhard Emme},
  title     = {Peristyl und Polis},
  subtitle  = {Entwicklung und Funktionen öffentlicher griechischer Hofanlagen},
  publisher = WdG,
  location  = Berlin #{ and New York}, %@String partly used
  year      = {2013},
  series    = {Urban Spaces},
  number    = {1},
}
\end{bibexample}
\begin{refsection}\end{refsection}

Default settings produce:
\printbib[5em]{Emme2013}
 
By activating option |publisher| you obtain:

\begin{bibbox}{Emme2013}\footnotesize
\parbox[t]{1.7cm}{Emme 2013} \parbox[t]{9.4cm}{B. Emme, Peristyl und Polis. Entwicklung und Funktionen öffentlicher griechischer Hofanlagen, Urban Spaces 1 (Berlin {\color{red} – New York: Walter de Gruyter} 2013)}
\end{bibbox}
 
And here a more detailed example with |origlocation|, |origyear| and |origpublisher|:
\begin{bibexample}[label=Neufert2002]{{@}Book\{Neufert2002,…\}}
@Book{Neufert2002,
  author       = {Neufert, Ernst},
  editor       = {Neufert, Peter and Neufert, Cornelius and Neff, Ludwig and Franken, Corinna},
  title        = {Bauentwurfslehre},
  subtitle     = {Grundlagen, Normen, Vorschriften ...},
  publisher    = VT, %@String used
  origpublisher = {Mann},
  location     = {Wiesbaden},
  year         = {2002},
  edition      = {37},
  origlocation = Berlin, %@String used
  origyear     = {1936},
}
\end{bibexample}
 
\printbib[5.5em]{Neufert2002}

\begin{bibbox}{Neufert2002}\footnotesize
\parbox[t]{1.7cm}{Neufert  2002} \parbox[t]{9.4cm}{%
E. Neufert, Bauentwurfslehre. Grundlagen, Normen, Vorschriften ... Ed. by Peter Neufert – Cornelius Neufert – Ludwig Neff  – Corinna Franken {\color{red} (Wiesbaden: Vieweg \textsuperscript{3}2002 [Berlin: Mann 1936])}}
\end{bibbox}
 
\subsubsection{edby}\label{edby}
\DescribeMacro{edby}
This option gives you a different output and position of editors: 
Instead of embedding ›(ed.)‹/›(Hrsg.)‹ right after the editor name
 it places the adjunct ›ed. by‹/›hrsg. v.‹ behind the editor. 
 Furthermore, in case of |@Incollections| and |@Inproceedings| editor names and book title switch their positions as it is shown below.

\begin{bibexample}[label=Wulf-Rheidt2013]{{@}Inproceedings\{Wulf-Rheidt2013,…\}}
@Inproceedings{Wulf-Rheidt2013,
  author       = {Wulf-Rheidt, Ulrike},
  title        = {Der Palast auf dem Palatin -- Zentrum im Zentrum},
  subtitle     = {Geplanter Herrschersitz oder Produkt eines langen Entwicklungsprozesses?},
  pages        = {277--289},
  editor       = {Dally, Ortwin and Fless, Friederike and Haensch, Rudolf and Pirson, Felix and Sievers, Susanne},
  booktitle    = {Politische Räume in vormodernen Gesellschaften},
  booksubtitle = {Gestaltung – Wahrnehmung – Funktion},
  location     = {Rahden/Westf\adddot},
  publisher    = VML,    %@String used
  year         = {2013},
  venue        = Berlin,   %@String used
  eventdate    = {2009-11-18/2009-11-22},
  eventtitle   = {Internationale Tagung des DAI und des DFG-Exzellenzclusters TOPOI},
  zenon       = {001371402},
  number       = {6},
  series       = MKT,    %@String used
  shortseries  = MKT-short,  %@String used
}
\end{bibexample}

Without option |edby| this |@Inproceedings| would look like:
 
\printbib[7em]{Wulf-Rheidt2013}

But activating |edby| it changes to:

\begin{bibbox}{Wulf-Rheidt2013}\footnotesize
\parbox[t]{2.3cm}{Wulf-Rheidt 2013} \parbox[t]{9cm}{%
U. Wulf-Rheidt, Der Palast auf dem Palatin – Zentrum im Zentrum. Geplanter Herrschersitz oder Produkt eines langen Entwicklungsprozesses?, in:  {\color{red}Politische Räume in vormodernen Gesellschaften. Gestaltung – Wahrnehmung – Funktion, ed. by O. Dally – F. Fless – R. Haensch – F. Pirson – S. Sievers}. Internationale Tagung des DAI und des DFG-Exzellenzclusters TOPOI Berlin November 18–22, 2009, MKT 6 (Rahden/Westf. 2013) 277–289}
\end{bibbox}
 
 
\subsubsection{width}\label{width}
\DescribeMacro{width}
|width| controls the width between label (which consists usually of |author| and |year|) and reference in the bibliography, pre-defined as |4em|.
If you wish to have it bigger or smaller you can change it to every length you would like to have:

|width =| \meta{length}

\meta{length} stands for the length you want (e.\,g. |3em|, |7pt| or |4cm|), you can even do |-1em|; 
then there is no indent at all.

\subsubsection{counter}\label{counter}
\DescribeMacro{counter} 
If you like to know how many times you cited an author or work then use this option called |counter|.\footnote{The idea is based on \href{http://tex.stackexchange.com/a/14159/98739}{tex.stackexchange.com/a/14159/98739} and has been modified.} 
Depending on the language you chose in the preamble of your document the information will be given in German (|ngerman|) or in English (if not |ngerman|).

\begin{bibbox}{Boehm2001}\footnotesize
\parbox[t]{3cm}{Böhm – Eickstedt 2001} \parbox[t]{8cm}{%
S. Böhm – K.-V. v. Eickstedt (Hrsg.), Ithake. Festschrift Jörg Schäfer (Würzburg 2001)  $\vert$  {\scshape  wurde 1-mal zitiert.}}
\end{bibbox}

If there has been no citation in the text (but maybe a \cs{citeauthor} or \cs{citetitle}):
\begin{bibbox}{Boehm2001}\footnotesize
\parbox[t]{3cm}{Böhm – Eickstedt 2001} \parbox[t]{8cm}{%
S. Böhm – K.-V. v. Eickstedt (Hrsg.), Ithake. Festschrift Jörg Schäfer (Würzburg 2001)  $\vert$  {\scshape  wurde {\color{red}{keinmal}} zitiert.}}
\end{bibbox} 

For all languages besides German:
\begin{bibbox}{Boehm2001}\footnotesize
\parbox[t]{3cm}{Böhm – Eickstedt 2001} \parbox[t]{8cm}{%
S. Böhm – K.-V. v. Eickstedt (ed.), Ithake. Festschrift Jörg Schäfer (Würzburg 2001) $\vert$  {\scshape cited {{\color{red}{not once}}}.}}
\end{bibbox}
  
If there has been only one citation:
\begin{bibbox}{Boehm2001}\footnotesize
\parbox[t]{3cm}{Böhm – Eickstedt 2001} \parbox[t]{8cm}{%
S. Böhm – K.-V. v. Eickstedt (ed.), Ithake. Festschrift Jörg Schäfer (Würzburg 2001) $\vert$  {\scshape cited 1 time.}}
\end{bibbox}

If there has been more than one citation:
\begin{bibbox}{Boehm2001}\footnotesize
\parbox[t]{3cm}{Böhm – Eickstedt 2001} \parbox[t]{8cm}{%
S. Böhm – K.-V. v. Eickstedt (ed.), Ithake. Festschrift Jörg Schäfer (Würzburg 2001) $\vert$  {\scshape cited 3 times.}}
\end{bibbox}

Note that |biblatex| provides a related option |backref| which lists per reference every page 
that contains the cited reference. But having a different goal that option doesn't support counts. 
 
\subsubsection{initials}\label{initials}
\DescribeMacro{initials} 
First names are abbreviated keeping digraphs and trigraphs instead of simple first letter initials.\footnote{The code for this option was taken from \href{http://tex.stackexchange.com/a/295486/98739}{tex.stackexchange.com/a/295486/98739} and has been modified.}
Also see the warning when printing an index; cf. \cref{initials:index}.
First names starting with |Ph..., Chr..., Ch..., Th..., St...| are abbreviated to these digraphs and trigraphs. For example in the bibliography or when you use \cs{citeauthor}\marg{bibtex-key}.

Citing these two authors will show their digraphs and trigraphs:

\begin{example}
\citeauthor{Mann2011} and \citeauthor{Hufschmid2010}
\end{example}



This option is set to default, but you can --- of course --- deactivate it with |initials=false| in the preamble.

With this default option you don’t have to change anything in your bibliographical-data.
However  you can switch off the option |initials| and do the abbreviation manually:

\begin{code}
author = {family=Mann, given=Christian, given-i={Chr}}

author = {family=Hufschmid, given=Thomas, given-i={Th}}
\end{code}


\subsection{Bibliography entry options}\label{options-bibentry}
\subsubsection{ancient}\label{ancient}
\DescribeMacro{ancient}
This option was found in an excellent |bib|\LaTeX-style called  |geschichtsfrkl| ( by Jonathan Zachhuber),\footnote{\href{https://www.ctan.org/pkg/geschichtsfrkl}{www.ctan.org/pkg/geschichtsfrkl}} 
so after some modifications we adopted and included it into |archaeologie|.
 
If you intend to cite ancient authors we strongly recommend this option to you 
because it enables you to cite ancient texts in the style archaeologists and 
historians are used to \emph{plus} you get entirely supported bibliography referencing.

Let us have a look at an example:
\begin{bibexample}[label=Cic:Att]{{@}Book\{Cic:Att,…\}}
@Book{Cic:Att,
  author       = {Tullius Cicero, Marcus},
  editor       = {Kasten, Helmut},
  title        = {Atticus-Briefe},
  publisher    = AWi,   %@String used
  location     = {Düsseldorf and Zürich},
  year         = {1980},
  series       = {Tusculum Bücherei},
  edition      = {3},
  origyear     = {1959},
  origtitle    = {epistulae ad Atticum},
  origlanguage = {latin},
  translator   = {Kasten, Helmut},
  shorthand    = {Cic. Att.},
  shortauthor  = {Cicero},
  keywords     = {ancient},
  options      = {ancient}, %!!
}
\end{bibexample}

Instead of applying |author| and |year| as labels the option |ancient| takes the field |shorthand| into account.

So you write in your text and it will be printed as

\begin{example}
\footnote{\cite[1, 3,3]{Cic:Att}.}
\end{example} 

Equally the field |shorthand| is used as a label in the bibliography instead of an author-year label:
\printbib{Cic:Att}

Take notice how |options={ancient}| treats editor and translator in this example. 
While the ancient author is mentioned first, translator and editor are put behind the title, 
in this case even united due to fact that editor and translator are identical.

This works not only with |@Book| but also with those ancient texts which are part of an |@Incollection| employing them similarly to the entries defined as |@Book|.

Another example:
\begin{bibexample}[label=Cic:Sest]{{@}Inbook\{Cic:Sest,…\}}
@Inbook{Cic:Sest,
  author       = {Tullius Cicero, Marcus},
  title        = {Rede für P.\ Sestius},
  booktitle    = {Die politischen Reden},
  year         = {1993},
  editor       = {Fuhrmann, Manfred},
  volume       = {2},
  publisher    = AWi,    %@String used
  pages        = {110--185},
  origlanguage = {latin},
  series       = {Sammlung Tusculum},
  location     = Munich,      %@String used
  intranslator = {Fuhrmann, Manfred},
  keywords     = {ancient},
  options      = {ancient},
  origtitle    = {pro P. Sestio},
  shortauthor  = {Cicero},
  shorthand    = {Cic. Sest.},
}
\end{bibexample}

Even support for several languages is taken into account as this bibliography entries show: 
\printbiball[4em]{Cic:Sest}

\subsubsection{frgancient}\label{frgancient}
\DescribeMacro{frgancient}
When dealing with fragments of ancient texts often it is important to cite the edition respective to the editor.
If you want to cite such a collection of fragments use |options={frgancient}|.
In this case the editor (|shorteditor| or |editor|) will be put in after the \marg{postnote}. 

\begin{bibexample}[label=Fest]{{@}Book\{Fest,…\}}
@Book{Fest,
  author      = {Pompeius Festus, {\relax Sex}tus},
  editor      = {Lindsay, Wallace Martin},
  title       = {De verborum significatu quae supersunt cum Pauli epitome},
  publisher   = {Teubner},
  location    = Leipzig,     %@String used
  year        = {1965},
  series      = {Bibliotheca scriptorum et Grecorum et Romanorum Teubnerina},
  origyear    = {1913},
  shorthand   = {Fest.},
  shortauthor = {Festus},
  keywords    = {ancient},
  options     = {frgancient},
  shorteditor = {L},
}
\end{bibexample}

When you cite this entry in this example the field  |shorteditor| will be shown.
\begin{example}
\footnote{\cite[3]{Fest}.}
\end{example}

In the bibliography the reference differentiates slightly from |options = {ancient}| because of the missing ancient author of fragment collections, 
so the editors name is printed in the first place:
\printbib[3em]{Fest}

\subsubsection{uniqueme}\label{uniqueme}
\DescribeMacro{uniqueme} 
Let us stick to ancient works for the option |uniqueme|:
Sometimes you cite different kinds of translations of one ancient author.
Despite you have several books the abbreviation of the ancient work is still the same and will be shown without a difference. 
For these cases we introduced an option with which you can decide which translation should be cited in the common citation and which ones should be enhanced with e.g. the translator, the series, or the editor.

Let us give you an example with the most famous ancient architect \citeauthor{Vitr}:
Since he is so famous there are variant translations of his text.
The German translation should be the standard referenced translation;
the bibliographical entry looks like this:
\begin{bibexample}[label=Vitr]{{@}Book\{Vitr,…\}}
@Book{Vitr,
  author        = {Vitruvius},
  title         = {Zehn Bücher über Architektur},
  publisher     = WBG, %@String used
  location      = {Darmstadt},
  year          = {2008},
  edition       = {6},
  origyear      = {1964},
  origtitle     = {De architectura},
  origlanguage  = {latin},
  translator    = {Fensterbusch, Curt},
  shorthand     = {Vitr.},
  shortauthor   = {Vitruv},
  keywords      = {ancient},
  options       = {ancient},
  sortshorthand = {Vitr.},
}
\end{bibexample}

When we cite \citeauthor{Vitr} we get the following result:
\begin{example}
\footnote{\cite[1,1,2]{Vitr}.}
\end{example}
Now let us assume we want to compare that with other, different translations / text editions. 
First the english standard translation (\cref{Vitr:Loeb}):
\begin{bibexample}[label=Vitr:Loeb]{{@}Book\{Vitr:Loeb,…\}}
@Book{Vitr:Loeb,
  author        = {Vitruvius},
  title         = {On Architecture},
  publisher     = HUP, %@String used
  location      = {Cambridge and London},
  year          = {1983},
  series        = {Loeb Classical Library},
  number        = {251},
  origyear      = {1931},
  origtitle     = {De architectura},
  origlanguage  = {latin},
  shorthand     = {Vitr.},
  shortauthor   = {Vitruv},
  keywords      = {ancient},
  options       = {ancient,uniqueme},
  shortseries   = {Loeb},
  sortshorthand = {Vitr. Loeb},
}
\end{bibexample}
Then there is also the French most common translation (\cref{Vitr:Saliou}):
\begin{bibexample}[label=Vitr:Saliou]{{@}Book\{Vitr:Saliou,…\}}
@Book{Vitr:Saliou,
  author        = {Vitruvius},
  title         = {De L'Architecture},
  publisher     = {Les belles lettres},
  location      = Paris, %@String used
  volumes        = {10},
  series        = {Collection des Universités de France},
  origtitle     = {De architectura},
  origlanguage  = {latin},
  translator    = {Saliou, Catherine},
  shorthand     = {Vitr.},
  shortauthor   = {Vitruv},
  keywords      = {ancient},
  options       = {ancient,uniqueme},
  date          = {1986/2009},
  sortshorthand = {Vitr. Saliou},
}
\end{bibexample}
And an old one (\cref{Vitr:Krohn}):
\begin{bibexample}[label=Vitr:Krohn]{{@}Book\{Vitr:Krohn,…\}}
@Book{Vitr:Krohn,
  author        = {Vitruvius},
  editor        = {Krohn, Franz},
  publisher     = {Teubner},
  location      = Leipzig, %@String used
  origtitle     = {Vitruvii De architectura libri decem},
  origlanguage  = {latin},
  shorthand     = {Vitr.},
  shortauthor   = {Vitruv},
  keywords      = {ancient},
  options       = {ancient,uniqueme},
  date          = {1912},
  sortshorthand = {Vitr. Krohn},
}
\end{bibexample}

And last there is a translation published in a section of a book (\cref{Vitr:Fischer}):
\begin{bibexample}[label=Vitr:Fischer]{{@}Book\{Vitr:Fischer,…\}}
@Inbook{Vitr:Fischer,
  author       = {Vitruvius},
  booktitle    = {Vitruv NEU oder Was ist Architektur?},
  year         = {2010},
  editor       = {Fischer, Günther},
  publisher    = {Birkhäuser},
  pages        = {70--76. 92--95. 132\psq},
  origlanguage = {latin},
  location     = Berlin, %@String used
  intranslator = {Fischer, Günther},
  keywords     = {ancient},
  options      = {ancient,uniqueme},
  origtitle    = {De architectura},
  origyear     = {2009},
  shorthand    = {Vitr.},
  sortshorhand = {Vitr. Fischer},
}
\end{bibexample}

If you cite these works you do as usual, you will get:
\begin{example}
\footnote{\cite[1,1,2]{Vitr}; see as well \cite{Vitr:Loeb}; with changed translation in \cite{Vitr:Krohn} and in \cite{Vitr:Saliou}; in \cite{Vitr:Fischer} it is kept untranslated.}
\end{example}


\begin{marker}
Notice that the field |sortshorthand| should be filled with the sorting order you would like to have in the bibliography for the varient works by \citeauthor{Vitr}, see below.
\end{marker}

\printbib[6em]{Vitr,Vitr:Fischer,Vitr:Krohn,Vitr:Loeb,Vitr:Saliou}


\subsubsection{corpus}\label{corpus}
\DescribeMacro{corpus}
There are some corpora (e.\,g. inscriptions, minted coins, vases, etc.) which are usually cited with a common abbreviation. 
Those abbreviations are typed in in the field |shorthand|.
To cite such a corpus with its abbreviation you have to write an additional |options={corpus}| in the bibliographical entry.
Now you can cite it as usual with the \oarg{prenote} or \oarg{postnote} you like to have.

This example shows the behaviour with a corpus that is common for Latin epigraphy:
\begin{bibexample}[label=CIL]{{@}Book\{CIL,…\}}
@Book{CIL,
  title     = CIL, %@String used
  location  = Berlin, %@String used
  year      = {1863--},
  shorthand = CIL-short,  %@String used
  keywords  = {corpus}, %!
  options   = {corpus},
}
\end{bibexample}

It will be cited with and you see the result right away:
\begin{example}
\footnote{\cite[06, 01234]{CIL}.}
\end{example}

The field |shorthand| will be used for bibliography reference,
where it is listed as label.
\printbib{CIL}

If you also set something like |keywords={corpus}| then you can make a separate bibliography with all the corpora cf. \cref{bibliographie}.

\changes{v1.1}{2015/06/15}{Modifikation der Option |corpus|.}


\section{Examples of entry types}\label{beispiele}
The style |archaeologie| defines several so-called bibliographical drivers,
which allow you to cite different kind of works.
Below we describe how they are working and which fields you should fill out.
For the examples we don't use any further options which are described above.

\subsection{Type \texttt{@Book}}\label{book}
\DescribeMacro{@Book}
\DescribeMacro{@Collection}%\footnote{Der Typ |@collection| entspricht hier dem Typ |@Book|.}
Let’s start with an easy example: 
a book or a collection (both are treated equally).
You can use the following fields:
\begin{description}
\item[mandatory:] 
|author|/|editor|, 
|title|, |subtitle|, |titleaddon|,
|location|, |year|,
\item[optional:]
|maintitle|, |mainsubtitle|, |maintitleaddon|, |volume|, 
|publisher|, |series|, |number|, |edition|, 
|origyear|, |origlocation|, |origpublisher|, 
|translator|, |origlanguage|,
|related|, |relatedtype|,
|doi|, |url|, |urldate|, |eprint|, |eprinttype|, |note|, |pubstate|, 
 \end{description}
 
 
An entry of a book might look as this in your |bib|-file:
\begin{bibexample}[label=Mann2011]{{@}Book\{Mann2011,…\}}
@Book{Mann2011,
  author    = {Mann, Christian},
  title     = {\enquote{Um keinen Kranz, um das Leben kämpfen wir!}},
  subtitle  = {Gladiatoren im Osten des Römischen Reiches und die Frage der Romanisierung},
  publisher = {Verlag Antike},
  location  = Berlin,   %@String used
  year      = {2011},
  series    = {Studien zur Alten Geschichte},
  number    = {14},
}
\end{bibexample}

A citation in a footnote is done like this and will be printed as you see:
\begin{example}
\footnote{\cite[Vgl.][142--144]{Mann2011}.}
\end{example} 
\printbib[5em]{Mann2011}

\subsubsection{›Festschrift‹, commemorative volume, catalogue etc.}
To mark that the book or collection is a so-called ›Festschrift‹/ commemorative volume, 
or an exhibition or auction catalogue you need an additional note to make it clear.
We suggest using  the field |titleaddon| or if it is a |@Incollection| or |@Inproceedings| you can use the field |booktitleaddon| (for papers in collections see \cref{inbook}).
\begin{bibexample}[label=Boehm2001]{{@}Book\{Boehm2001,…\}}
@Book{Boehm2001,
  editor     = {Böhm, Stephanie and Eickstedt, Klaus-Valtin von},
  title      = {Ithake},
  publisher  = {Ergon-Verlag},
  location   = {Würzburg},
  year       = {2001},
  titleaddon = {Festschrift Jörg Schäfer},
}
\end{bibexample}

\printbib[9.5em]{Boehm2001}
 
\subsubsection{Translated book}
If you cite a translated book you can link it to the original book and let display the translator as well as the original language. 
To obtain this fill the fields |related| and |relatedtype| (for further guidance see the information about reviews in \cref{review}).

This example will clarify matters:
The first edition of \citetitle*{Zanker2009} by \citeauthor*{Zanker2009} has been published in 1987 (|origyear|), but by now it has been released in its 5th edition.

\begin{bibexample}[label=Zanker2009]{{@}Book\{Zanker2009,…\}}
@Book{Zanker2009,
  author        = {Zanker, Paul},
  title         = {Augustus und die Macht der Bilder},
  publisher     = CHB,   %@String used
  location      = Munich,  %@String used
  year          = {2009},
  edition       = {5},
  origlocation  = Leipzig, %@String used
  origyear      = {1987},
  zenon        = {000250713},
  language      = {german},
  origpublisher = {Koehler \& Amelang},
}
\end{bibexample}

In \citeyear{Zanker1988}, one year later after the first edition (|origyear|) the book was translated by A. H. Shapiro:
\begin{bibexample}[label=Zanker1988]{{@}Book\{Zanker1988,…\}}
@Book{Zanker1988,
  author      = {Zanker, Paul},
  title       = {The Power of Images in the Age of Augustus},
  publisher   = UMP,    %@String used
  location    = {Ann Arbor},
  year        = {1988},
  series      = {Jerome Lectures},
  number      = {16},
  translator  = {Shapiro, Alan H.},
  language    = {english},
  related     = {Zanker2009},
  relatedtype = {translationof},
}
\end{bibexample}

The translated book |Zanker1988| is connected with the book |Zanker2009| via the field |related = |\marg{bibtex-key} 
and the relation got specified with the field |relatedtype|, in this case it is a translation so |={translationof}|.

You don't have to cite |Zanker2009| to have the information visible in the bibliography. 
It will be included automatically. For the following bibliography we only use |\cite{Zanker1988}|:

\printbiball[5em]{Zanker1988}


\subsubsection{Multiple volumes of a monograph (cf. \cref{mvbook})}
It may be the case that you have to cite a book which consists of several volumes:
usually there is a volume with text and one volume with plates.
To cite  e.\,g. the second volume in particular you can do the following.
Let’s assume this is the bibliography entry:
\begin{bibexample}[label=MacDonald1986]{{@}Book\{MacDonald1986,…\}}
@Book{MacDonald1986,
  author    = {MacDonald, William L.},
  title     = {An urban Appraisal},
  publisher = YUP,    %@String used
  location  = {New Haven and }# London, %@String used
  year      = {1986},
  maintitle = {The Architecture of the Roman Empire},
  volume    = {2},
  series    = {Yale Publications in the History of Art},
  number    = {35},
}
\end{bibexample}
In the bibliography the main title of the monograph (|maintitle|)
and the title of the book (|title|) are shown separately  so the volume  (|volume|) 
appears before the title of the book
\printbib[6.5em]{MacDonald1986}


\subsection{Type \texttt{@Inbook / @Incollection}}\label{inbook}
\DescribeMacro{@Incollection}\DescribeMacro{@Inbook}
Single entries/chapters of a collection are cited best when they are set up as  |@Incollection| or |@Inbook|.

\begin{description}
\item[mandatory:] 
|author|, |title|, |subtitle|, |titleaddon|,
|editor|,  |booktitle|, |booksubtitle|, |booktitleaddon|,
|location|, |year|, |pages|, 
\item[optional:]
|maintitle|, |mainsubtitle|, |maintitleaddon|, |volume|, 
|publisher|, |series|, |number|, |edition|, 
|origyear|, |origlocation|, |origpublisher|, 
|translator|, |origlanguage|,
|related|, |relatedtype|,
|doi|, |url|, |urldate|, |eprint|, |eprinttype|, |note|, |pubstate|, 
 \end{description}
 
 
 
This following example clarify matters:
 \begin{bibexample}[label=Carter2014]{{@}Incollection\{Carter2014,…\}}
@Incollection{Carter2014,
  author    = {Carter, Michael J. and Edmondson, Jonathan},
  title     = {Spectacle in Rome, Italy, and the Provinces},
  pages     = {537--558},
  editor    = {Bruun, Christer and Edmondson, Jonathan},
  booktitle = {The Oxford Handbook of Roman Epigraphy},
  publisher = OUP,    %@String used
  location  = {Oxford},
  year      = {2014},
}
\end{bibexample}

\printbib[12em]{Carter2014}

You can also have contributions to a ›Festschrift‹ etc. set up as |@Incollection|,
but then notice the additional information in |booktitleaddon|.
\begin{bibexample}[label=Hoelscher2001]{{@}Incollection\{Hoelscher2001,…\}}
@Incollection{Hoelscher2001,
  author         = {Hölscher, Tonio},
  title          = {Schatzhäuser -- Banketthäuser?},
  pages          = {143--152},
  editor         = {Böhm, Stephanie and Eickstedt, Klaus-Valtin von},
  booktitle      = {Ithake},
  publisher      = {Ergon-Verlag},
  location       = {Würzburg},
  year           = {2001},
  booktitleaddon = {Festschrift Jörg Schäfer},
}
\end{bibexample}
In the bibliography it will look like:
\printbib[6em]{Hoelscher2001}

\subsubsection{Short series}
Some books or collections are part of a small series (not an ongoing series).
This book is part of the series called abbreviated \emph{MemAmAc}.
Have a look:
\begin{bibexample}[label=Fentress2003]{{@}Incollection\{Fentress2003,…\}}
@Incollection{Fentress2003,
  author       = {Fentress, Elizabeth and John Bodel and Adam Rabinowitz and Rabun Taylor},
  title        = {Cosa in the Republic and Early Empire},
  pages        = {13--62},
  editor       = {Fentress, Elizabeth},
  booktitle    = {An Intermittent Town},
  booksubtitle = {Excavations 1991--1997},
  publisher    = UMP,    %@String used
  location     = {Ann Arbor, Mich.},
  year         = {2003},
  volume       = {5},
  series       = MemAmAc,    %@String used
  number       = {2},
  maintitle    = {Cosa},
  shortseries  = MemAmAc-short,    %@String used
}
\end{bibexample}
As we can see it is the fifth volume (|volume|) of the series with the main title 
\emph{Cosa} (|maintitle|) but has an individual title (|title|) which is
\emph{Cosa in the Republic and Early Empire}, furthermore it is the second book (|number|) 
of the series  \emph{MemAmAc} (|series|).

Notice the different language-based behaviour of ›et. al.‹ for more than two authors/editors.
\printbiball[7.5em]{Fentress2003}

 
\subsubsection{Inventory catalogue}
The output of an inventory catalogue changes slightly compared to collections or something similar. 
The title is omitted and therefore there is no comma after the author’s name.
We provide two examples so you see the difference.
\begin{bibexample}[label=Kohlmeyer1983]{{@}Inbook\{Kohlmeyer1983,…\}}
@Inbook{Kohlmeyer1983,
  author       = {K. Kohlmeyer},
  booktitle    = {Tierbilder aus vier Jahrtausenden},
  year         = {1983},
  editor       = {U. Gehrig},
  booksubtitle = {Antiken der Sammlung Mildenberg},
  pages        = {20 Nr. 9},
  location     = Mainz, %@String used
}
\end{bibexample}
and the second example

\begin{bibexample}[label=Parlasca1969]{{@}Inbook\{Parlasca1969,…\}}
@Inbook{Parlasca1969,
  author    = {K. Parlasca},
  booktitle = {Helbig},
  year      = {1969},
  volume    = {3},
  edition   = {4},
  pages     = {98\psq\ Nr. 2176},
  location  = Tuebingen, %@String used
}
\end{bibexample}


\printbib[6em]{Kohlmeyer1983}
and
\printbiball[6em]{Parlasca1969}

\subsubsection{Section of Monograph}
\archversion{2.3.4}
In some cases there are several parts of a monograph which are written by variant people (e.\,g. excavation reports). 
Then the book is technically not an edited one by an editor but (mainly) written by one (book)author. 

The difference to an edited volume is that in the bibliography   there is no |(ed.)|, cf. \cref{Ganzert1984}.


\begin{bibexample}[label=Ganzert1984]{{@}Inbook\{Ganzert1984,…\}}
@Inbook{Ganzert1984,
author = {Herz, Peter},
title = {Gaius Caesar und Artavasdes},
bookauthor = {Ganzert, Joachim},
booktitle = {Das Kenotaph für Gaius Caesar in Limyra},
booksubtitle = {Architektur und Bauornamentik},
pages = {118--126},
publisher = {Wasmuth},
location = Tuebingen,
year = {1984},
zenon = {000042874},
series = IstForsch,
shortseries = IstForsch-short,
number = {35}
}
\end{bibexample}

\printbib[6em]{Ganzert1984}


\subsection{Type \texttt{@MvBook}}\label{mvbook}
\DescribeMacro{@MvBook}\archversion{2.3.4} If you have a small series consistend of several volumes and you want to have them listed together you should read the following.

Using the entry type |@MvBook| (Multivolume) for the main entry you can have several volumes connected to it (children).
For example we have \citetitle{EAOR} with seven volumes. 
The main entry looks like this

\begin{bibexample}[label=EAOR]{{@}MvBook\{EAOR,…\}}
@Mvbook{EAOR,
  title       = {Epigrafia anfiteatrale dell'Occidente Romano},
  date        = {1988/2009},
  editor      = {Patrizia Sabbatini Tumolesi},
  keywords    = {corpus},
  location    = Rome, %@String used
  options     = {corpus},
  publisher   = EQ, %@String used
  related     = {EAOR-01,EAOR-02,EAOR-03,EAOR-04,EAOR-05,EAOR-06,EAOR-07},
  relatedtype = {multivolume},
  series      = {Vetera. Richerche di storia, epigrafia e antichità},
  shorthand   = {EAOR I--VII},
  volumes     = {7},
}
\end{bibexample}

Two children belonging to \cref{EAOR} look like

\begin{bibexample}[label=EAOR-01]{{@}MvBook\{EAOR-01,…\}}
@Book{EAOR-01,
  title     = {Roma},
  year      = {1988},
  editor    = {Tumolesi, Patrizia Sabbatini},
  volume    = {1},
  keywords  = {corpus},
  options   = {corpus,skipbib},
  shorthand = {EAOR I},
  crossref = {EAOR},
}
\end{bibexample}

\begin{bibexample}[label=EAOR-02]{{@}MvBook\{EAOR-02,…\}}
@Book{EAOR-02,
  title     = {Regiones Italiae VI--XI},
  year      = {1989},
  editor    = {Gregori, Gian Luca},
  volume    = {2},
  keywords  = {corpus},
  options   = {corpus,skipbib},
  shorthand = {EAOR II},
  crossref = {EAOR},
}
\end{bibexample}
It is important that you have |skipbib| in the field |options|!

The advantage is that you cite either |EAOR| then you will all of the volumes listed in the bibliography 
or you can only cite individual volumes in your text.
In this case (e.g. citing |EAOR-01| and |EAOR-02|) it looks like this in the bibliography 
\printbib[5em]{EAOR-01,EAOR-02}


\subsection{Type \texttt{@Article}}\label{article}
\DescribeMacro{@Article} This is probably the most common type 
because you find detailed information about specific topics in articles.

\begin{description}
\item[mandatory:] 
|author|, |title|, |subtitle|, |titleaddon|,
|journaltitle|, |shortjournal|, |volume|, |number|, |issue|
|year|, |pages|, 
\item[optional:]
|translator|, |origlanguage|,
|related|, |relatedtype|,
|doi|, |url|, |urldate|, |eprint|, |eprinttype|, |note|, |pubstate|, 
 \end{description}

Here we have an example which will explain the (required)  fields:
\begin{bibexample}[label=Evangelidis2014]{{@}Article\{Evangelidis2014,…\}}
@Article{Evangelidis2014,
  author       = {Evangelidis, Vasilis},
  title        = {Agoras {and} Fora},
  subtitle     = {Developments in the Central Public Space of the Cities of Greece during the {Roman} Period},
  journaltitle = BSA,    %@String used
  shortjournal = BSA-short,    %@String used
  volume       = {109},
  pages        = {335--356},
  year         = {2014},
  doi          = {10.1017/s006824541400015x},
}
\end{bibexample}
In line 5 and 6 you can also write the full or abbreviated journal title in the fields |journaltitle| or |shorttitle| (e.\,g. |journaltitle = {British School of Athens}|, |shortjournal = {BSA}|), but we chose to use a |@String| (cf. \cref{list-bibancient,string}) again.
\printbib[6.5em]{Evangelidis2014}


\subsection{Type \texttt{@Proceedings}}\label{proceedings}
\DescribeMacro{@Proceedings}
Similar to a collection but still different in the bibliographical output are proceedings.
Therefore we recommend  using the type |@Proceedings|.
The difference lies in the additional mandatory fields |venue|, |eventdate| and |eventtitle|. 
Everything else is like the type |@Book|.

\begin{description}
\item[mandatory:] 
%|author|/
|editor|, 
|title|, |subtitle|, |titleaddon|,
|venue|, |eventdate|, |eventtitle|,
|location|, |year|
\item[optional:]
|maintitle|, |mainsubtitle|, |maintitleaddon|, |volume|, 
|publisher|, |series|, |number|, |edition|, 
|origyear|, |origlocation|, |origpublisher|, 
|translator|, |origlanguage|,
|related|, |relatedtype|,
|doi|, |url|, |urldate|, |eprint|, |eprinttype|, |note|, |pubstate|, 
 \end{description}
 
An example:
 \begin{bibexample}[label=Kurapkat2014]{{@}Proceedings\{Kurapkat2014,…\}}
@Proceedings{Kurapkat2014,
  title        = {Die Architektur des Weges},
  year         = {2014},
  editor       = {Kurapkat, Dietmar and Schneider, Peter I. and Wulf-Rheidt, Ulrike},
  subtitle     = {Gestaltete Bewegung im gebauten Raum},
  eventtitle   = {Kolloquium Architekturreferat des DAI},
  eventdate    = {2012-02-08/2012-02-11},
  venue        = Berlin,     %@String used
  series       = DiskAB,    %@String used
  number       = {11},
  organization = {Architekturreferat des DAI},
  publisher    = {Schnell + Steiner},
  location     = Regensburg,     %@String used
  shortseries  = DiskAB-short,    %@String used
}
\end{bibexample}
%\iffalse
With |venue| we specify the place where the proceeding took place 
(e.\,g. \emph{Berlin} -- |location| is where the book was printed and is connected 
to the |publisher| e.\,g. \emph{Regensburg}),
|eventtitle| is used for a special title of the proceeding (e.\,g. \emph{Kolloquium Architekturreferat des DAI}),
|eventtitle| gives the date(range) when the proceeding was hold and has to be typed in in the format YYYY-MM-DD, 
a range has to be separated with a |/| (e.\,g.  \emph{2012-02-08/2012-02-11}).

In the bibliography the information of these additional fields will be used (of course) as this, notice how the output of the date changes according to the chosen language.
\printbiball[7.5em]{Kurapkat2014}

\subsection{Type \texttt{@Inproceedings}}\label{inproceedings}
\DescribeMacro{@Inproceedings}
This entry type works similar to |@Procecedings| and |@Incollection| and therefore the needed fields are straightforward to use:

\begin{description}
\item[mandatory:] 
|author|, |title|, |subtitle|, |titleaddon|,
|editor|,  |booktitle|, |booksubtitle|, |booktitleaddon|,
|venue|, |eventdate|, |eventtitle|,
|location|, |year|, |pages|, 
\item[optional:]
|maintitle|, |mainsubtitle|, |maintitleaddon|, |volume|, 
|publisher|, |series|, |number|, |edition|, 
|origyear|, |origlocation|, |origpublisher|, 
|translator|, |origlanguage|,
|related|, |relatedtype|,
|doi|, |url|, |urldate|, |eprint|, |eprinttype|, |note|, |pubstate|, 
\end{description}
 
 
 
 \begin{bibexample}[label=Torelli1991]{{@}Inproceedings\{Torelli1991,…\}}
@Inproceedings{Torelli1991,
  author     = {Torelli, Mario},
  title      = {Il \enquote{diribitorium} di Alba Fucens e il \enquote{campus} eroico di Herdonia},
  pages      = {39--63},
  editor     = {Mertens, Josef},
  booktitle  = {Comunitá indigene e problemi della romanizzazione nell’Italia centro\--meri\-dionale (IV--III sec. a.C.)},
  location   = Brussels,     %@String used
  publisher  = {Institut Historique Belge de Rome},
  year       = {1991},
  venue      = Rome #{, Academia Belgica},    %@String used
  eventdate  = {1990-02-01/1990-02-03},
  eventtitle = {Actes du Colloque International Organisé à l'Occasion du 50. Anniversaire de l'Academia Belgica et du 40. Anniversaire des Fouilles Belges en Italie},
  hyphenate  = {italian},
  language   = {italian},
  number     = {29},
  series     = {Études de philologie, d'archéologie et d'histoire anciennes},
  shorttitle = {Il \enquote{diribitorium}},
}
\end{bibexample}
It will be printed as:
 
\printbiball[5em]{Torelli1991}

 \subsection{Type \texttt{@Reference}}\label{reference}
 \DescribeMacro{@Reference}
 This entry type can be used especially for references if you want to cite it as whole or if you need to relate to a reference. 
We provide an example below---cf. \cref{inreference}

You don’t need to fill out many fields to have a working entry:
\begin{description}
\item[mandatory:] |title|, |shorthand|,
\item[optional:] 
 |editor|, |subtitle|, |titleaddon|,
 |location|, |year|
|maintitle|, |mainsubtitle|, |maintitleaddon|,
|related|, |relatedtype|,
|publisher|, |series|, |number|, |edition|, |volume|,
|doi|, |url|, |urldate|, |eprint|, |eprinttype|, |note|, |pubstate|, 
\end{description}

And so a complete entry is quite small:
\begin{bibexample}[label=LIMC]{{@}Reference\{LIMC,…\}}
@Reference{LIMC,
  title     = LIMC,
  keywords  = {corpus},
  options   = {corpus},
  shorthand = LIMC-short,
}
\end{bibexample}
 
But you can also have it  more detailed  like this one:
\begin{bibexample}[label=Lexikon-der-Technik]{{@}Reference\{Lexikon-der-Technik,…\}}
@Reference{Lexikon-der-Technik,
  editor    = {Otto Lueger},
  title     = {Lexikon der gesamten Technik und ihrer Hilfswissenschaften},
  date      = {1904/1920},
  edition   = {2},
  location  = Stuttgart,   %@String used
  keywords  = {corpus},
  shorthand = {Lexikon d. T.},
}
\end{bibexample}

 \subsection{Type \texttt{@Inreference}}\label{inreference}
 \DescribeMacro{@Inreference}
Besides a whole reference you can also -- and which is more likely -- cite only an entry of it via the type  |@Inreference|.

Let’s clarify matters with an example:
\begin{bibexample}[label=Neils1994]{{@}Inreference\{Neils1994,…\}}
@Inreference{Neils1994,
  author    = {Neils, Jenifer},
  title     = {Theseus},
  booktitle = LIMC-short,    %@String used
  pages     = {922--951},
  year      = {1994},
  volume    = {7.1},
  keywords  = {lexikon},
}
\end{bibexample}
You can cite this entry with any of the provided \cs{cite}-commands above---cf. \cref{cite-commands,faq:inreference}.

But for the final display of the entry  you have two possibilities:
 \begin{enumerate}
\item\label{inreference:a} 

If you want it in the default style ›author-year‹, so it will have a label and is referenced 
to your final bibliography, then you don’t have to do anything.
In the bibliography it will look like
\printbib{Neils1994}

\item\label{inreference:b} 
The \DAI has a special rule for inreferences in the footnote.
Then the output will be like:\\
|reference volume (year) pages s.v. title (author)| \\
and there will be no reference to the bibliography since the entry is fully described in the footnote.
\DescribeMacro{inreferencesasfullcite} If you prefer this method you have to use the preamble option called  |inreferencesasfullcite|---cf. \cref{inreferences}
Then it will look like this in the footnote:
\begin{examplemanual}
\footnote{LIMC 7.1 (1994) 922--951 s. v. Theseus (J. Neils).}
\end{examplemanual}
When you have the \oarg{postnote} filled out in a citation which belongs to an |@Inreference| then it won’t be printed in the end of the citation.
The |postnote| \oarg{930 Nr. 283} will be printed instead of the |pages|:
\begin{example}
\footnote{\cite[vgl.][930 Nr. 283]{Neils1994}.}
\end{example} 
\end{enumerate}

As mentioned above it is  advantageous to relate entries such as  an |@Inreference| with its |@Reference|. 
And since not all references have a ›canonical‹ abbreviation (e.\,g. RE, LIMC, DNP, LTUR, LÄ, etc.) it might be necessary to define a |shorthand|.
This is shown in the example below.
 
\begin{bibexample}[label=Weinbrenner1914]{{@}Inreference\{Weinbrenner1914,…\}}
@Inreference{Weinbrenner1914,
  author    = {Weinbrenner},
  title     = {Rennbahn},
  booktitle = {Lexikon d. T.},
  pages     = {636--637},
  year      = {1914},
  related   = {Lexikon-der-Technik},
  volume    = {9},
  number    = {2},
}
\end{bibexample}
As you see this entry is related to |Lexikon-der-Technik| which is described above and has a |shorthand = {Lexikon d. T.}|
You just need to make sure that  |booktitle| of the |@Inreference| and the |shorthand| of 
the  |@Reference| are equal so the title can be referenced properly in the bibliography.

For the following bibliography result we just typed (assuming that the entries were already cited in text):

\begin{refsection}
\nocite{Lexikon-der-Technik,Weinbrenner1914,LTUR,Neils1994}
\setlength{\labwidthsameline}{5em} 
\begin{example}
\printbibliography[keyword=corpus,title={Corpora}]
\printbibliography[notkeyword=corpus]
\end{example}
\end{refsection}


 
\subsection{Type \texttt{@Review}}\label{review}
\DescribeMacro{@Review}
Reviews in journals are best cited when they are edited as a |@Review|.
For a full citation of a review you have to name the reviewed work in detail.
The following example will show an easy way to combine the review with the reviewed work.

\begin{description}
\item[mandatory:] 
|author|, |title|, |subtitle|, |titleaddon|,
|journaltitle|, |shortjournal|, |volume|, |number|, |issue|
|year|, |pages|, 
|related|, |relatedtype|,
\item[optional:]
|translator|, |origlanguage|,
|doi|, |url|, |urldate|, |eprint|, |eprinttype|, |note|, |pubstate|, 
 \end{description}

What you need are two separate entries: one as a |@Review| the other is a |@Book|, |@Collection|, |@Proceedings| or something else.

First we have the reviewed work:
\begin{bibexample}[label=Welch2007]{{@}Book\{Welch2007,…\}}
@Book{Welch2007,
  author    = {Welch, Katherine E.},
  title     = {The {Roman} Amphitheatre},
  subtitle  = {From its Origins to the Colosseum},
  publisher = CUP,    %@String used
  location  = {Cambridge and New York},
  year      = {2007},
}
\end{bibexample}
then the review itself:
\begin{bibexample}[label=Bell2011]{{@}Review\{Bell2011,…\}}
@Review{Bell2011,
  author       = {Bell, Sinclair},
  number       = {1},
  pages        = {1--4},
  volume       = {115},
  journaltitle = AJA,    %@String used
  shortjournal = AJA-short,    %@String used
  related      = {Welch2007},
  relatedtype  = {reviewof},
  year         = {2011},
  publisher    = {Archaeological Institute of America},
}
\end{bibexample}
You maybe noticed that the review (|Bell2011|) is connected to the entry |Welch2007| with the field |related| in line 8.
In addition we not only need a connected work but also to qualify the relation:
This is done in line 9 with |relatedtype = {reviewof}|.
This so-called |bibstring| is especially for reviews and contains the language based correct abbreviation for \emph{Review of} or e.\,g. \emph{Rez. zu}.
You don’t have to type in all relevant information of the reviewed work in the entry of the review, 
since they will be inserted automatically and dynamically with the  |related|-function. 
So whenever settings in the reviewed work are changed the print of the review will be automatically adjusted. 
Furthermore, even if the review is cited, the reviewed work won't be listed in the bibliography until it is explicitly cited in the text.
\printbib[4em]{Bell2011}


\subsubsection{Reviews with an individual title}
Some reviews are more detailed then others and  have their own title which should be displayed in the bibliography.
In these cases you can use the field |title| the other things stay the same.

The following entry is an example which also reviews the book by \citeauthor*{Welch2007},  but with an individual title:
\begin{bibexample}[label=Hufschmid2010]{{@}Review\{Hufschmid2010,…\}}
@Review{Hufschmid2010,
  author       = {Hufschmid, Thomas},
  title        = {Von Caesars \emph{theatron kynegetikon} zum \emph{amphitheatrum novum} Vespasians},
  pages        = {487--504},
  volume       = {23},
  journaltitle = JRA,    %@String used
  shortjournal = JRA-short,    %@String used
  related      = {Welch2007},
  relatedtype  = {reviewof},
  year         = {2010},
}
\end{bibexample}
In the bibliography there will be first the individual title followed by the information of the reviewed work.

\printbib[6em]{Hufschmid2010}


\subsubsection{multiple reviewed works in one review}
Some reviews analyse several works in the same article. 
This makes no big difference for the citing or editing process.

In his review
\citeauthor{Taylor2008} not only describes the book called
 \citetitle{Welch2007} by \citeauthor{Welch2007}, 
 but at the same time compares it with \citeauthor{Sear2006}s \citetitle{Sear2006}.

The entry of the first analysed book \citetitle{Welch2007} has been described in \cref{review}.
The entry of the second reviewed book is this:
\begin{bibexample}[label=Sear2006]{{@}Book\{Sear2006,…\}}
@Book{Sear2006,
  author     = {Sear, Frank},
  title      = {Roman Theatres},
  subtitle   = {An Architectural Study},
  publisher  = OUP,    %@String used
  location   = {Oxford},
  year       = {2006},
  series     = {Oxford Monographs on Classical Archaeology},
}
\end{bibexample}

The entry of the review looks like this:
\begin{bibexample}[label=Taylor2008]{{@}Review\{Taylor2008,…\}}
@Review{Taylor2008,
  author       = {Taylor, Rabun},
  number       = {3},
  pages        = {443--445},
  volume       = {67},
  journaltitle = {Journal of the Society of Architectural Historians},
  related      = {Sear2006,Welch2007},
  relatedtype  = {reviewof},
  year         = {2008},
}
\end{bibexample}
In the field |related| you can have several \meta{bibtex-keys} which have to be separated by a comma (see line 7).

Subsequently, all information is gathered in the bibliography:
\printbib[5em]{Taylor2008}


 \subsection{Type \texttt{@Thesis}}\label{thesis}
MA and PhD theses, which are not published as a monograph or such, can be cited when they are edited as |@Thesis|.
It is important to differentiate between an entry referring to an MA or a PhD thesis;
this can be done by |type=|\marg{|phdthesis|} or
 \marg{|mathesis|}. 
You also have to define the  |institution=|\marg{university}.
 
\begin{description}
\item[mandatory:] 
|author|,
|title|, |subtitle|, |titleaddon|,
|type|, |institution|,
|year|,
\item[optional:]
|doi|, |url|, |urldate|, |eprint|, |eprinttype|, |note|, |pubstate|, 
 \end{description}
 

Here is an example:
\begin{bibexample}[label=Arnolds2005]{{@}Thesis\{Arnolds2005,…\}}
@Thesis{Arnolds2005,
  author      = {Markus Arnolds},
  title       = {Funktionen republikanischer und frühkaiserzeitlicher Forumsbasiliken in Italien},
  type        = {phdthesis},
  institution = {Ruprecht-Karls-Universität zu Heidelberg},
  urn      = {urn:nbn:de:bsz:16-heidok-74406},
  date        = {2005},
}
\end{bibexample}

Here is the entry in the bibliography:

\printbib[5.5em]{Arnolds2005}

 
 \changes{v1.1}{2015/06/04}{Umsetzung von |@thesis| im Stil.}

 \subsection{Type \texttt{@Talk}}\label{talk}
For (oral) given papers e.\,g. at a colloquium or a proceeding we created a new entry type called ›@Talk‹.

\begin{description}
\item[mandatory:] 
|author|,
|title|, |subtitle|, |titleaddon|,
|date|,
|venue|,
|institution|,
|eventtitle|,
|eventdate|,
\item[optional:]
|doi|, |url|, |urldate|, |eprint|, |eprinttype|, |note|, |pubstate|, 
 \end{description}
 
Here is an example for a  paper  given in Berlin in 2015:
\begin{bibexample}[label=Bergmann2015]{{@}Talk\{Bergmann2015,…\}}
@Talk{Bergmann2015,
  author      = {Bergmann, Birgit},
  title       = {\enquote{An exciting find}},
  date        = {2015-04-27},
  subtitle    = {Neues zum Forums-Fries der Praedia Iuliae Felicis},
  titleaddon  = {(Pompeii II, 4)},
  url         = {https://www.antikezentrum.hu-berlin.de/de/veranstaltungskalender/bibergmann},
  urldate     = {2016-05-14},
  eventtitle  = {Kolloquium der Klassischen Archäologie},
  institution = {Freie Universität Berlin},
  venue       = Berlin,
}
\end{bibexample}
The bibliography will show the entry as:

\printbib[7em]{Bergmann2015}
  
\changes{v1.5}{2016/05/31}{Rückverweis}


 \section{Bibliography}\label{bibliographie}
 \DescribeMacro{\printbibliography}
As long as you don’t use the option\DescribeMacro{seenote} |seenote|---for 
which a final bibliography is not needed---you will need to print your cited entries in a bibliography 
at a certain place in your document.
It can be useful to differentiate your bibliography and divide it e.\,g. into a bibliography 
with ancient authors and one with modern scholars.
Additionally you can have a bibliography with the |shorthand| shortcuts or all abbreviated journal titles, etc.

How the different bibliographies can be set up is explained now:
Let’s assume you want to have a bibliography with the ancient authors and one with modern scholars.
Since the entries of the ancient authors have the field |keyword={ancient}| (or should have) this is done quite easy.

But first we define the heading of the whole  bibliography:

\begin{refsection}
    \nocite{*}
    \renewcommand\bibfont{\normalfont\footnotesize}
    \setlength{\labwidthsameline}{6em} 
\begin{example}
\printbibheading[%
  heading=bibliography,%
  %heading=bibnumbered,% if you want it numbered
  title={Bibliography}] %heading for bibliography
\end{example}
You can give any title you would like to give (|title = |\marg{any title}).

The next step is to set up the bibliography for the ancient authors.

\setlength{\labwidthsameline}{6.5em} 
\begin{example}
\printbibliography[%
  keyword=ancient,%
  heading=subbibliography,
  %heading=subbibnumbered,% if you want it numbered
  title={Ancient authors and works}]
\end{example}
We tell the bibliography just to contain the entries which have |ancient| in the field |keywords| (line 2).


Finally the bibliography for modern scholars.
This time we exclude all entries which have |ancient| or |corpus| in the field |keywords|. 
That’s it.
(Don't be surprised about the line |notkeyword=corpus| which excludes entries with special |shorthand| labels, a further bibliography part with all the |shorthands| is described below.).

\setlength{\labwidthsameline}{7em} 
\begin{example}
\printbibliography[%
  notkeyword=ancient,%
  notkeyword=corpus,%
  heading=subbibliography,
  %heading=subbibnumbered,% if you want it numbered
  title={Secondary literature}]
\end{example}



You can create as many bibliographies as you wish each with another keyword if you like.
Or you can make a bibliography with all the |shorthands| used in your text---for that we use |keyword= {corpus}| (line 2):
Now the bibliography only lists the used entries which have |corpus| in the field |keywords|:
\begin{example}
\printbibliography[%
keyword={corpus},
heading=subbibliography,
title={Abbreviation of corpora}]
\label{bib:corpus}
\end{example}

\begin{marker}
 If you want to separate author-year labels from |shorthand| labels in your bibliography,  
 you should ensure that bibliography entries which contain a |shorthand| denomination 
are set with a keyword either |ancient|, |corpus| or something else, to guarantee that there is 
no bibliographical shortcut wrongly sorted in the bibliography.
\end{marker}


Furthermore you can have a bibliography for all the abbreviated journal titles and series to have the abbreviation and its long form.
For journals it works like this:
\begin{example}
\printbiblist[%
  heading=subbibliography,
  title={Abbreviation of journals}]{shortjournal}
\end{example}

For series it is done like this:
\begin{example}
\printbiblist[%
  heading=subbibliography,
  title={Abbreviation of series}]{shortseries}
\end{example}
\end{refsection}


\section{FAQ: For  Ancient (scholars of high) Quality}
\subsection{Following pages}
\DescribeMacro{\psq} If you have to cite two following pages there is the macro \cs{psq} which is best to you since it is also controlled by the set up language.
Just write the first page in the \oarg{postnote} and then the \cs{psq}---cf. \cref{Parlasca1969,Vitr:Fischer}.

\subsection{Online referencing}
\DescribeMacro{eprint}\DescribeMacro{eprinttype} If you have a paper or book which is available 
in the internet via a permanent link there are different possibilities for referring to it---see also the
 \href{http://tug.ctan.org/macros/latex/exptl/biblatex/doc/biblatex.pdf}{|bib|\LaTeX-documentation chapter 3.11.7}.\footnote{You can also use as an |eprinttype|: |arXiv|, |pubmed|, |hdl|, |googlebooks|.
}

\DescribeMacro{jstor}
 If you can refer to the on-line platform \href{www.jstor.org}{jstor} then you need the individual number for the article---cf. \cref{Ball2013,Osland2016}:

\begin{code}
jstor = *@\marg{jstor-number}@*
\end{code}  

\DescribeMacro{urn}
For all papers referable via an ›urn‹ (\emph{Uniform Resource Name}), which have been registered at the German National Library---cf. \cref{Arnolds2005}.

\begin{code}
urn = *@\marg{urn:xxx}@*
\end{code}  

\DescribeMacro{zenon}
This eprint-form is especially designed for the OPAC (Online Public Access Catalogue) of the \DAI.
All bibliographical entries in this OPAC can be referred to via this link.
You only need to insert the individual Zenon-number of the entry, e.\,g. \emph{http://zenon.dainst.org/Record/001371402} where \emph{001371402} is the individual number---cf. \cref{Osland2016,Wulf-Rheidt2013,Ganzert1984.Zanker2009}.

This option is set to |false| by default.
\begin{code}
zenon = *@\marg{Zenon-number}@*
\end{code}      

\DescribeMacro{arachne}
There is the object-database of the \DAI which is called iDAI.objects arachne. 
You can cite the items with their entity-IDs 


This option is set to |false| by default.
\begin{code}
arachne = *@\marg{entity-number}@*
\end{code}      



\DescribeMacro{doi} In addition you can also refer to a document via its |doi|-number---cf. \cref{Ball2013,Evangelidis2014} 
\begin{code}
doi = *@\marg{doi-number}@*
\end{code}  

If you want or don’t want to have the online references printed you can enable or disable the fields with |jstor=false|, 
|urn=false|,
|zenon=true|,
|doi=false| etc. as preamble option.

\subsection{Brackets (with @Inreference)}\label{faq:inreference}
Ensure you stick to the correct order of parentheses and brackets.
The rule says that within a pair of parentheses you have to use square brackets.
Citing |@Inreferences| can lead easily to a false behaviour when you put it in parentheses.
This is an example how it should not be done:
\begin{tcolorbox}[examplebox]
|(\cite[vgl.][930 Nr. 283]{Neils1994}).| 
\tcblower
(vgl. LIMC 7.1 (1994) 930 Nr. 283 s. v. Theseus (J. Neils)). \textcolor{red}{\textbf{WRONG!!}}
\end{tcolorbox}
 
\DescribeMacro{\parencite} \DescribeMacro{\parencites}
This example shows that the correct order of parentheses has not been followed.
Especially when you activated the preamble option  |inreferencesasfullcite=true| \DescribeMacro{inreferencesasfullcite} you should use \cs{parencite}\marg{bibtex-key} instead (\cref{cite-commands}), then the correct order of parentheses and square brackets is given:
\begin{tcolorbox}[examplebox]
|\parencite[vgl.][930 Nr. 283]{Neils1994}.| 
\tcblower
(vgl. LIMC 7.1 [1994] 930 Nr. 283 s. v. Theseus [J. Neils]). \textcolor{green!50!black}{\textbf{CORRECT!}}
\end{tcolorbox}

\subsection{Unknown work}\label{unknown}
Sometimes it is impossible to ascertain the author or editor but you still want to cite them.
If you come along such a paper you can define a |label| which will be used for citing and sorting.
This is not connected to an entry type -- it can be used with any work.
In the following example we use an entry type |@Article|:
\begin{bibexample}[label=Cosa1949]{{@}Article\{Cosa1949,…\}}
@Article{Cosa1949,
  title        = {Cosa},
  subtitle     = {Republican Colony in Etruria},
  journaltitle = ClJ,
  shortjournal = ClJ-short,
  volume       = {45},
  pages        = {141--149},
  year         = {1949},
  label        = {Cosa},
  number       = {1},
}
\end{bibexample}
 The |label| (line 9) can be defined as one wishes; in this case we chose it analogous to the title: |label = {Cosa}|.
When you cite such an anonymous work it will be done like all the others,
 It will look like this:
\begin{example}
\footnote{\cite[Vgl.][145--147]{Cosa1949}.}
\end{example}
and be printed in the bibliography like this:
\printbib{Cosa1949}

\subsection{Publication status}
Sometimes you know the author of an article or a book, proceedings etc. and you get a proof of the work beforehand.
You can also cite this version of the work and provide information about the publication status in the field |pubstate|.
This field is usable with all entry types provided (except |@Talk|).
There are some predefined publication statuses which you are recommended to use since they are translated into the used language:

\DescribeMacro{inpreparation}
Typoscript is prepared for your publication. 
\begin{code}
pubstate = {inpreparation}
\end{code}

\DescribeMacro{submitted}
Typoscript has been submitted.
\begin{code}
pubstate = {submitted}
\end{code}

\DescribeMacro{forthcoming}
Typoscript has been accepted by the journal.
\begin{code}
pubstate = {forthcoming}
\end{code}

\DescribeMacro{inpress}
Typescript has been edited and you have a proof version of it.
\begin{code}
pubstate = {inpress}
\end{code}

\DescribeMacro{prepublished}
Article has been published in an (online) preversion.
\begin{code}
pubstate = {prepublished}
\end{code}


This following example is about an article which was accepted by the journal and so we use |pubstate = {forthcoming}|:
\begin{bibexample}[label=Doe:forthcoming]{{@}Article\{Doe:forthcoming,…\}}
@Article{Doe:forthcoming,
  author       = {Doe, John},
  title        = {My Ideas about Antiquity},
  subtitle     = {And Some Lines for a Subtitle},
  journaltitle = {Journal of Ancient Thoughts},
  shortjournal = {JAT},
  volume       = {12},
  pubstate     = {forthcoming},
}
\end{bibexample}

\printbib[8em]{Doe:forthcoming}


\subsection{Print the used options}
If you want to print the options  of the |bib|\LaTeX -style |archaeologie| you used in your document you can use the command \cs{archaeologieoptions}.
It will list all options used:
\begin{otherlanguage}{ngerman}
\begin{example}
\archaeologieoptions
\end{example}
\end{otherlanguage}
\begin{example}
\archaeologieoptions
\end{example}
If you do not want to have the text in the beginning (which is defined in English and in German) you can get rid of them with the optional argument  \cs{archaeologieoptions}\texttt{[plain]}
\begin{example}
\archaeologieoptions[plain]
\end{example}
\begin{otherlanguage}{ngerman}
\begin{example}
\archaeologieoptions[plain]
\end{example}
\end{otherlanguage}

You can print the version of this style with |\archaeologieversion| 
\begin{example}
\archaeologieversion
\end{example}
or the date of this version with |\archaeologiedate|.
\begin{example}
\archaeologiedate
\end{example}

\subsection{Print the cited authors of secondary literature}
In case you want to have an index about the authors you cited in your text,
you can do that quite easily.
We coded it that way so authors of ancient sources (e.\,g. Cicero) will be omitted in that index (when these entries have |options={ancient}| or |options={frgancient}|).

First you have to activate indexing in the package |biblatex|:
\begin{code}
\usepackage[style=archaeologie,%
          indexing=cite,
          *@\meta{further options}@*]{biblatex}
\end{code}
|cite| will enable indexing in citations only, 
you can also do |bib| which will enable indexing in the bibliography only.
Or |true| so in citations and in the bibliography (|false| is the default setting).

Then you have to load a package for indexing,
we suggest using the package |imakeidx| since you can create multiple indexes with it.
If you have only one index which will be used for the authors of secondary literature you can pass several options to it.
 \begin{code}
\usepackage{imakeidx}
\makeindex[%
  title=Index of  authors,
  columns=3,
]
\end{code}   

Now you only need to place \cs{printindex} whre you want to have the index.
\begin{marker}
If you have the option \texttt{initials} activated your index will have some issues with names that are being shortend automatically.\label{initials:index}
We are working on it, see the issue on \href{https://github.com/LukasCBossert/biblatex-archaeologie/issues/97}{GitHub} and on \href{http://tex.stackexchange.com/q/330971/98739}{\TeX.sx} for more information.\footnote{So far we can only offer a work-around:
Go into your \texttt{.bbl}-file which is in the same folder as your \texttt{.tex}-file.
Replace |family=\{\{|\meta{Ch / Chr / Ph / St / Th}|\}| with |family=\{|\meta{Ch / Chr / Ph / St / Th}.
Then compile your document one more time. You will have the initials in the text, 
and the index looks fine. 
Be aware that after running \texttt{biber} you have to repeat this step.}
\end{marker}


\subsection{Variant ways of entries in year/date-field}
Sometimes you have a range of years of a publication because it is maybe a sequence of volumes.
Let us take as example the \citetitle{LTUR}:
\begin{bibexample}[label=LTUR]{{@}Reference\{LTUR,…\}}
@Reference{LTUR,
  editor    = {Steinby, Eva Margareta},
  title     = LTUR, %@String used
  date      = {1993/2000},
  publisher = EQ, %@String used
  location  = Rome, %@String used
  keywords  = {corpus},
  options   = {corpus},
  shorthand = LTUR-short, %@String used
}
\end{bibexample}
\printbib[4em]{LTUR}

Since this reference-series is completed we define the first and last year by using |date = {1993/2000}|. Be sure \emph{not}  using field |year| but instead |date| since |year| cannot cope with date ranges dependably.

Let us take a look at another example:
\begin{bibexample}[label=DeVisscher1951-1952]{{@}Article\{DeVisscher1951-1952,…\}}
@Article{DeVisscher1951-1952,
  author       = {de Visscher, Fernand and Mertens, Joseph},
  title        = {Les puits du Forum d'Alba Fucense},
  journaltitle = BCom, %@String used
  shortjournal = BCom-short, %@String used
  volume       = {74},
  pages        = {3--13},
  date         = {1951/1952},
}
\end{bibexample}
This time the article appeared in an issue which covers two years (\citedate{DeVisscher1951-1952}) and we want them to appear in the label.
That is why we have to use the field |date| instead of |year|. 
Note that if the date range consists out of only two sequent years, the delimiter will be a slash |/|.
If the range is bigger than two years it will be an em-dash |–|.
\printbib[11em]{DeVisscher1951-1952}
By the way, if you want to cite only the year of publication use |\citedate|\marg{bibtex-key} and not |\citeyear|\marg{bibtex-key} since the field |year| will only give you the first year.


%\begin{multicols}{1}
{\footnotesize
%\lstlistoflistings
\tcblistof[\section]{bibexample}{List of Examples}}
%\end{multicols}
\section{Additional bibliography with ancient authors and works}\label{list-bibancient}
The bibliography |archaeologie-bibancient.bib| is filled with ancient authors, works and their abbreviation according to  The New Pauly.
The bold entry on the left is the |bibtex-key|.\footnote{If you think the list should be enlarged, let us know the entries.}
All the entries have the fields  |keywords={ancient}|, |options={ancient}|.

\begin{multicols}{2}
  \begin{footnotesize}
\begin{description}[%
				%style=multiline,
				style=nextline,
				leftmargin=2cm,
				%font=\normalfont\bfseries
				]
\item[Acc] \textbf{Acc.}\newline Accius\newline 
\item[AchTat] \textbf{Ach. Tat.}\newline Achilleus Tatios\newline 
\item[Ael:Ep] \textbf{Ael. Ep.}\newline Aelian\newline \emph{Epistulae rusticae}
\item[Ael:Fr] \textbf{Ael. Fr.}\newline Aelian\newline \emph{Fragmenta}
\item[Ael:NA] \textbf{Ael. NA}\newline Aelian\newline \emph{de natura animalium}
\item[Ael:Tact] \textbf{Ael. Tact}\newline {Aelianus Tacticus}\newline \emph{Tactica (De instruendis aciebus)}
\item[Ael:VA] \textbf{Ael. VH}\newline Aelian\newline \emph{varia historia}
\item[AelArist:Or] \textbf{Ael. Arist. Or.}\newline {Aelius Aristides}\newline \emph{Orationes}
\item[AelArist:Technrhet] \textbf{Ael. Arist. Techn. rhet.}\newline {Aelius Aristides}\newline \emph{Technai rhetorikai}
\item[AenTact] \textbf{Aen. Tact.}\newline {Aineias Taktikos}\newline \emph{Poliorketika}
\item[Aet] \textbf{Aet.}\newline Aetios\newline 
\item[Aischin:Ctes] \textbf{Aischin. Ctes.}\newline Aischines\newline \emph{in Ctesiphontum}
\item[Aischin:epist] \textbf{Aischin. epist.}\newline Aischines\newline \emph{epistulae}
\item[Aischin:leg] \textbf{Aischin. leg.}\newline Aischines\newline \emph{de falsa legatione}
\item[Aischin:orat] \textbf{Aischin. orat.}\newline Aischines\newline \emph{Orationes}
\item[Aischin:Tim] \textbf{Aischin. Tim.}\newline Aischines\newline \emph{in Timarchum}
\item[Aischyl:Ag] \textbf{Aischyl. Ag.}\newline Aischylos\newline \emph{Agamemnon}
\item[Aischyl:Choeph] \textbf{Aischyl. Choeph.}\newline Aischylos\newline \emph{Choephoroi}
\item[Aischyl:Eum] \textbf{Aischyl. Eum.}\newline Aischylos\newline \emph{Eumenides}
\item[Aischyl:Pers] \textbf{Aischyl. Pers.}\newline Aischylos\newline \emph{Persae}
\item[Aischyl:Prom] \textbf{Aischyl. Prom.}\newline Aischylos\newline \emph{Prometheus}
\item[Aischyl:Sept] \textbf{Aischyl. Sept.}\newline Aischylos\newline \emph{Septem adversus Thebas}
\item[Aischyl:Suppl] \textbf{Aischyl. Suppl.}\newline Aischylos\newline \emph{Supplices (Hiketides)}
\item[Aisop] \textbf{Aisop.}\newline Aisopos\newline 
\item[AlexAphr] \textbf{Alex. Aphr.}\newline Alexandros von Aphrodisias\newline 
\item[Alk] \textbf{Alk.}\newline Alkaios\newline 
\item[Ambr:epist] \textbf{Ambr. epist.}\newline Ambrosius\newline \emph{epistulae}
\item[Ambr:excSat] \textbf{Ambr. exc. Sat.}\newline Ambrosius\newline \emph{de excessu fratris (Satyri)}
\item[Ambr:obitTheod] \textbf{Ambr. obit. Theod.}\newline Ambrosius\newline \emph{de obitu Theodosii}
\item[Ambr:obitValent] \textbf{Ambr. obit. Valent.}\newline Ambrosius\newline \emph{de obitu Valentiniani (iunioris)}
\item[Ambr:off] \textbf{Ambr. off.}\newline Ambrosius\newline \emph{de officiis magistrorum}
\item[Ambr:paenit] \textbf{Ambr. paenit.}\newline Ambrosius\newline \emph{de paenitentia}
\item[Amm] \textbf{Amm.}\newline Ammianus Marcellinus\newline 
\item[Anakr] \textbf{Anakr.}\newline Anakreon\newline 
\item[Anaxag] \textbf{Anaxag.}\newline Anaxagoras\newline 
\item[Anaximand] \textbf{Anaximand.}\newline Anaximandros\newline 
\item[Anaximen] \textbf{Anaximen.}\newline Anaximenes\newline 
\item[And] \textbf{And.}\newline Andokides\newline 
\item[AnthGr] \textbf{Anth. Gr.}\newline Antologia Graeca\newline 
\item[AnthLat] \textbf{Anth. Lat.}\newline Antologia Latina\newline 
\item[AnthPal] \textbf{Anth. Pal.}\newline Antologia Palatina\newline 
\item[AnthPlan] \textbf{Anth. Plan.}\newline Anthologia Planudea\newline 
\item[Antiph] \textbf{Antiph.}\newline Antiphon\newline 
\item[Antisth] \textbf{Antisth.}\newline Antisthenes\newline 
\item[Apg] \textbf{Apg.}\newline \newline \emph{Apostelgeschichte}
\item[Apk] \textbf{Apk.}\newline \newline \emph{Apokalypse}
\item[Apollod:bibl] \textbf{Apollod. bibl.}\newline Apollodoros\newline \emph{bibliotheke}
\item[Apollod:epit] \textbf{Apollod. epit.}\newline Apollodoros\newline \emph{epitome}
\item[ApollRhod] \textbf{Apoll. Rhod.}\newline Apollonios Rhodios\newline \emph{Argonautica}
\item[App:Celt] \textbf{App. Celt.}\newline Appian\newline \emph{Celtica}
\item[App:civ] \textbf{App. civ.}\newline Appian\newline \emph{bella civilia}
\item[App:Hann] \textbf{App. Hann.}\newline Appian\newline \emph{Hannibalica}
\item[App:Ib] \textbf{App. Ib.}\newline Appian\newline \emph{Iberica}
\item[App:Ill] \textbf{App. Ill.}\newline Appian\newline \emph{Illyrica}
\item[App:It] \textbf{App. It.}\newline Appian\newline \emph{Italica}
\item[App:Lib] \textbf{App. Lib.}\newline Appian\newline \emph{Libyca}
\item[App:Mac] \textbf{App. Mac.}\newline Appian\newline \emph{Macedonica}
\item[App:Mithr] \textbf{App. Mithr.}\newline Appian\newline \emph{Mithridatius}
\item[App:pr] \textbf{App. pr.}\newline Appian\newline \emph{proemium}
\item[App:reg] \textbf{App. reg.}\newline Appian\newline \emph{regia}
\item[App:Rom] \textbf{App. Rom.}\newline Appian\newline \emph{historia romana}
\item[App:Samn] \textbf{App. Samn.}\newline Appian\newline \emph{Samnitica}
\item[App:Sic] \textbf{App. Sic.}\newline Appian\newline \emph{Sicula}
\item[App:Syr] \textbf{App. Syr.}\newline Appian\newline \emph{Syriaca}
\item[Apul:Flor] \textbf{Apul. Flor.}\newline Apuleius\newline \emph{florida}
\item[Apul:met] \textbf{Apul. met.}\newline Apuleius\newline \emph{metamorphoses}
\item[Arat] \textbf{Arat.}\newline Aratos\newline 
\item[Archil] \textbf{Archil.}\newline Archilochos\newline 
\item[Archim] \textbf{Archim.}\newline Archimedes\newline 
\item[Archyt] \textbf{Archyt.}\newline Archytas\newline 
\item[Aristeid] \textbf{Aristeid.}\newline Ailios Aristeides\newline 
\item[Aristoph:Ach] \textbf{Aristoph. Ach.}\newline Aristophanes\newline \emph{Acharnenses}
\item[Aristoph:Av] \textbf{Aristoph. Av.}\newline Aristophanes\newline \emph{Aves (Ornithes)}
\item[Aristoph:Eccl] \textbf{Aristoph. Eccl.}\newline Aristophanes\newline \emph{ Ecclesiazusae}
\item[Aristoph:Equ] \textbf{Aristoph. Equ.}\newline Aristophanes\newline \emph{Equites (Hippeis)}
\item[Aristoph:Lys] \textbf{Aristoph. Lys.}\newline Aristophanes\newline \emph{Lysistrata}
\item[Aristoph:Nub] \textbf{Aristoph. Nub.}\newline Aristophanes\newline \emph{Nubes (Nephelai)}
\item[Aristoph:Pax] \textbf{Aristoph. Pax}\newline Aristophanes\newline \emph{Pax (Eirene)}
\item[Aristoph:Plut] \textbf{Aristoph. Plut.}\newline Aristophanes\newline \emph{Plutus}
\item[Aristoph:Ran] \textbf{Aristoph. Ran.}\newline Aristophanes\newline \emph{Ranae (Batrachoi)}
\item[Aristoph:Thesm] \textbf{Aristoph. Thesm.}\newline Aristophanes\newline \emph{Thesmophoriazusae}
\item[Aristoph:Vesp] \textbf{Aristoph. Vesp.}\newline Aristophanes\newline \emph{Vespae (Sphekes)}
\item[Aristot:an] \textbf{Aristot. an.}\newline Aristoteles\newline \emph{de anima (peri psyches)}
\item[Aristot:anpost] \textbf{Aristot. an post.}\newline Aristoteles\newline \emph{analytica posteriora}
\item[Aristot:anpr] \textbf{Aristot. an pr.}\newline Aristoteles\newline \emph{analytica priora}
\item[Aristot:Athpol] \textbf{Aristot. Ath. pol.}\newline Aristoteles\newline \emph{Athenaion politeia}
\item[Aristot:cael] \textbf{Aristot. cael.}\newline Aristoteles\newline \emph{de caelo}
\item[Aristot:cat] \textbf{Aristot. cat.}\newline Aristoteles\newline \emph{categoriae}
\item[Aristot:col] \textbf{Aristot. col.}\newline Aristoteles\newline \emph{de coloribus (περὶ χρωμάτων)}
\item[Aristot:div] \textbf{Aristot. div.}\newline Aristoteles\newline \emph{de divinatione (peri mantikes)}
\item[Aristot:EthEud] \textbf{Aristot. Eth. Eud.}\newline Aristoteles\newline \emph{ ethica Eudemia}
\item[Aristot:ethNic] \textbf{Aristot. eth. Nic.}\newline Aristoteles\newline \emph{ethica Nicomachea}
\item[Aristot:genan] \textbf{Aristot. gen. an.}\newline Aristoteles\newline \emph{de generatione animalium}
\item[Aristot:gencorr] \textbf{Aristot. gen. corr.}\newline Aristoteles\newline \emph{de generatione et corruptione}
\item[Aristot:hist] \textbf{Aristot. hist.}\newline Aristoteles\newline \emph{historia animalium}
\item[Aristot:metaph] \textbf{Aristot. metaph.}\newline Aristoteles\newline \emph{metaphysica}
\item[Aristot:meteor] \textbf{Aristot. meteor.}\newline Aristoteles\newline \emph{meteorologica}
\item[Aristot:mir] \textbf{Aristot. mir.}\newline Aristoteles\newline \emph{mirabilia}
\item[Aristot:mmor] \textbf{Aristot. m. mor.}\newline Aristoteles\newline \emph{magna moralia}
\item[Aristot:motan] \textbf{Aristot. mot. an.}\newline Aristoteles\newline \emph{de motu animalium}
\item[Aristot:mund] \textbf{Aristot. mund.}\newline Aristoteles\newline \emph{de mundo (peri kosmu)}
\item[Aristot:oec] \textbf{Aristot. oec.}\newline Aristoteles\newline \emph{oeconomica}
\item[Aristot:partan] \textbf{Aristot. part. an.}\newline Aristoteles\newline \emph{de partibus animalium}
\item[Aristot:phgn] \textbf{Aristot. phgn.}\newline Aristoteles\newline \emph{physiognomica}
\item[Aristot:phys] \textbf{Aristot. phys.}\newline Aristoteles\newline \emph{physica}
\item[Aristot:poet] \textbf{Aristot. poet.}\newline Aristoteles\newline \emph{poetica}
\item[Aristot:pol] \textbf{Aristot. pol.}\newline Aristoteles\newline \emph{politica}
\item[Aristot:probl] \textbf{Aristot. probl.}\newline Aristoteles\newline \emph{problemata}
\item[Aristot:rhet] \textbf{Aristot. rhet.}\newline Aristoteles\newline \emph{rhetorica}
\item[Aristot:rhetAlex] \textbf{Aristot. rhet. Alex.}\newline Aristoteles\newline \emph{rhetorica ad Alexandrum}
\item[Aristot:sens] \textbf{Aristot. sens.}\newline Aristoteles\newline \emph{de sensu (περὶ αἰσθήσεως)}
\item[Aristot:somn] \textbf{Aristot. somn.}\newline Aristoteles\newline \emph{de somno et vigilia (περὶ ὔπνου καὶ ἐγρηγόρσεως)}
\item[Aristot:sophel] \textbf{Aristot. soph. el.}\newline Aristoteles\newline \emph{sophistici elenchi}
\item[Aristot:spir] \textbf{Aristot. spir.}\newline Aristoteles\newline \emph{de spiritu}
\item[Aristot:top] \textbf{Aristot. top.}\newline Aristoteles\newline \emph{topica}
\item[Arnob] \textbf{Arnob.}\newline Arnobius\newline \emph{adversus nationes}
\item[Arr:an] \textbf{Arr. an.}\newline Arrian\newline \emph{Anabasis}
\item[Arr:cyn] \textbf{Arr. cyn.}\newline Arrian\newline \emph{Cynegeticus}
\item[Arr:Ind] \textbf{Arr. Ind.}\newline Arrian\newline \emph{Indica}
\item[Arr:perpE] \textbf{Arr. per p. E.}\newline Arrian\newline \emph{Periplus ponti Euxeni}
\item[Arr:succ] \textbf{Arr. succ.}\newline Arrian\newline \emph{Istoria successorum Alexandri}
\item[Arr:tact] \textbf{Arr. tact.}\newline Arrian\newline \emph{tactica}
\item[Artem] \textbf{Artem.}\newline Artemidor\newline \emph{Onirocriticon}
\item[Athan:adConst] \textbf{Athan. ad Const.}\newline Athanasios\newline \emph{Apologia ad Constantium}
\item[Athan:cAr] \textbf{Athan. c. Ar.}\newline Athanasios\newline \emph{Apologia contra Arianos}
\item[Athan:fuga] \textbf{Athan. fuga}\newline Athanasios\newline \emph{Apologia de fuga sua}
\item[Athan:histAr] \textbf{Athan. hist. Ar.}\newline Athanasios\newline \emph{Istoria Arianorum ad monachos}
\item[Athen] \textbf{Athen.}\newline Athenaios\newline 
\item[Aug:civ] \textbf{Aug. civ.}\newline Augustinus\newline \emph{de civitate dei}
\item[Aug:conf] \textbf{Aug. conf.}\newline Augustinus\newline \emph{confessiones}
\item[Aug:doctchrist] \textbf{Aug. doct.christ.}\newline Augustinus\newline \emph{de doctrina christiana}
\item[Aug:epist] \textbf{Aug. epist.}\newline Augustinus\newline \emph{epistulae}
\item[Aug:retract] \textbf{Aug. retract.}\newline Augustinus\newline \emph{retractationes}
\item[Aug:serm] \textbf{Aug. serm.}\newline Augustinus\newline \emph{sermones}
\item[Aug:soliloq] \textbf{Aug. soliloq.}\newline Augustinus\newline \emph{soliloquia}
\item[Aug:trin] \textbf{Aug. trin.}\newline Augustinus\newline \emph{de trinitate}
\item[Aur:Vict] \textbf{Aur. Vict.}\newline Aurelius Victor\newline 
\item[Auson:Mos] \textbf{Auson. Mos.}\newline Ausonius\newline \emph{Mosella}
\item[Auson:urb] \textbf{Auson. urb.}\newline Ausonius\newline \emph{ordo nobilium urbium}
\item[Avien] \textbf{Avien.}\newline Avienus\newline 
\item[Babr] \textbf{Babr.}\newline Babrios\newline 
\item[Bakchyl] \textbf{Bakchyl.}\newline Bakchylides\newline 
\item[Boeth] \textbf{Boeth.}\newline Boethius\newline 
\item[Caes:BellAfr] \textbf{Caes. Bell. Afr.}\newline Caesar\newline \emph{Bellum Africum (Ps.-Caes.)}
\item[Caes:BellAlex] \textbf{Caes. Bell. Alex.}\newline Caesar\newline \emph{Bellum Alexandrinum (Ps.-Caes.)}
\item[Caes:BellHisp] \textbf{Caes. Bell. Hisp.}\newline Caesar\newline \emph{Bellum Hispaniense (Ps.-Caes.)}
\item[Caes:civ] \textbf{Caes. civ.}\newline Caesar\newline \emph{de bello civili}
\item[Caes:Gall] \textbf{Caes. Gall.}\newline Caesar\newline \emph{de bello Gallico}
\item[Calp:ecl] \textbf{Calp. ecl.}\newline Calpurnius\newline \emph{Bucolica vel Eclogae}
\item[Cass:Dio] \textbf{Cass. Dio}\newline Cassius Dio\newline 
\item[Cassiod:inst] \textbf{Cassiod. inst.}\newline Cassiodorus\newline \emph{institutiones variae}
\item[Cato:agr] \textbf{Cato agr.}\newline Cato\newline \emph{de agri cultura}
\item[Cato:Orig] \textbf{Cato Orig.}\newline Cato\newline \emph{origines}
\item[Catull] \textbf{Catull.}\newline Catullus\newline \emph{carmina}
\item[Cels] \textbf{Cels.}\newline Cornelius Celsus\newline \emph{De medicina}
\item[Cels:artes] \textbf{Cels. artes}\newline Celsus\newline 
\item[Cels:Dig] \textbf{Cels. Dig.}\newline Celsus\newline \emph{Digesta}
\item[Cens] \textbf{Cens.}\newline Censorinus\newline \emph{de die natali}
\item[Cic:ac1] \textbf{Cic. ac. 1}\newline Cicero\newline \emph{Academicorum posteriorum liber 1}
\item[Cic:ac2] \textbf{Cic. ac. 2}\newline Cicero\newline \emph{Lucullus sive Academicorum priorum liber 2}
\item[Cic:adBrut] \textbf{Cic. ad. Brut.}\newline Cicero\newline \emph{epistulae ad Brutum}
\item[Cic:adQfr] \textbf{Cic. ad Q. fr.}\newline Cicero\newline \emph{ad Quintum fratrem}
\item[Cic:Arat] \textbf{Cic. Arat.}\newline Cicero\newline \emph{Aratea}
\item[Cic:Arch] \textbf{Cic. Arch.}\newline Cicero\newline \emph{pro Archia poeta}
\item[Cic:Att] \textbf{Cic. Att.}\newline Cicero\newline \emph{epistulae ad Atticum}
\item[Cic:Balb] \textbf{Cic. Balb.}\newline Cicero\newline \emph{pro L. Balbo}
\item[Cic:Brut] \textbf{Cic. Brut.}\newline Cicero\newline \emph{Brutus}
\item[Cic:Caecin] \textbf{Cic. Caecin.}\newline Cicero\newline \emph{pro A. Caecina}
\item[Cic:Cael] \textbf{Cic. Cael.}\newline Cicero\newline \emph{pro M. Caelio}
\item[Cic:Catil] \textbf{Cic. Catil.}\newline Cicero\newline \emph{in Catilinam}
\item[Cic:Cato] \textbf{Cic. Cato}\newline Cicero\newline \emph{Cato maior de senectute}
\item[Cic:Cluent] \textbf{Cic. Cluent.}\newline Cicero\newline \emph{pro A. Cluentio}
\item[Cic:Deiot] \textbf{Cic. Deiot.}\newline Cicero\newline \emph{pro rege Deiotaro}
\item[Cic:deorat] \textbf{Cic. de orat.}\newline Cicero\newline \emph{de oratore}
\item[Cic:div] \textbf{Cic. div.}\newline Cicero\newline \emph{de divinatione}
\item[Cic:divinCaec] \textbf{Cic. div. in Caec.}\newline Cicero\newline \emph{divinatio in Q. Caecilium}
\item[Cic:fam] \textbf{Cic. fam.}\newline Cicero\newline \emph{epistulae ad familiares}
\item[Cic:fat] \textbf{Cic. fat.}\newline Cicero\newline \emph{de fato}
\item[Cic:fin] \textbf{Cic. fin.}\newline Cicero\newline \emph{de finibus bonorum et malorum}
\item[Cic:Flacc] \textbf{Cic. Flacc.}\newline Cicero\newline \emph{pro L. Valerio Flacco}
\item[Cic:Font] \textbf{Cic. Font.}\newline Cicero\newline \emph{pro M. Fonteio}
\item[Cic:harresp] \textbf{Cic. har. resp.}\newline Cicero\newline \emph{de haruspicum responso}
\item[Cic:inv] \textbf{Cic. inv.}\newline Cicero\newline \emph{de inventione}
\item[Cic:Lael] \textbf{Cic. Lael.}\newline Cicero\newline \emph{Laelius de amicitia}
\item[Cic:leg] \textbf{Cic. leg.}\newline Cicero\newline \emph{de legibus}
\item[Cic:legagr] \textbf{Cic. leg. agr.}\newline Cicero\newline \emph{de lege agraria}
\item[Cic:Lig] \textbf{Cic. Lig.}\newline Cicero\newline \emph{pro Q. Ligario}
\item[Cic:Manil] \textbf{Cic. Manil.}\newline Cicero\newline \emph{pro lege Manilia (de imperio Cn. Pompei)}
\item[Cic:Marcell] \textbf{Cic. Marcell.}\newline Cicero\newline \emph{pro M. Marcello}
\item[Cic:Mil] \textbf{Cic. Mil.}\newline Cicero\newline \emph{pro T. Annio Milone}
\item[Cic:Mur] \textbf{Cic. Mur.}\newline Cicero\newline \emph{pro L. Murena}
\item[Cic:natdeor] \textbf{Cic. nat. deor.}\newline Cicero\newline \emph{de natura deorum}
\item[Cic:off] \textbf{Cic. off.}\newline Cicero\newline \emph{de officiis}
\item[Cic:optgen] \textbf{Cic. opt. gen.}\newline Cicero\newline \emph{de optimo genere oratorum}
\item[Cic:orat] \textbf{Cic. orat.}\newline Cicero\newline \emph{orator}
\item[Cic:parad] \textbf{Cic. parad.}\newline Cicero\newline \emph{paradoxa}
\item[Cic:part] \textbf{Cic. part.}\newline Cicero\newline \emph{partitiones oratoriae}
\item[Cic:Phil] \textbf{Cic. Phil.}\newline Cicero\newline \emph{in M. Antonium orationes Philippicae}
\item[Cic:Pis] \textbf{Cic. Pis.}\newline Cicero\newline \emph{in L. Pisonem}
\item[Cic:Planc] \textbf{Cic. Planc.}\newline Cicero\newline \emph{pro Cn. Plancio}
\item[Cic:predadQuir] \textbf{Cic. p.red. ad Quir.}\newline Cicero\newline \emph{oratio post reditum ad Quirites}
\item[Cic:predinsen] \textbf{Cic. p.red. in sen.}\newline Cicero\newline \emph{oratio post reditum in senatu}
\item[Cic:QRosc] \textbf{Cic. Q. Rosc.}\newline Cicero\newline \emph{pro Q. Roscio comoedo}
\item[Cic:Quinct] \textbf{Cic. Quinct.}\newline Cicero\newline \emph{pro P. Quinctio}
\item[Cic:Rabperd] \textbf{Cic. Rab. perd.}\newline Cicero\newline \emph{pro C. Rabirioperduellionis reo}
\item[Cic:Rabpost] \textbf{Cic. Rab post.}\newline Cicero\newline \emph{pro C. Rabirio Postumo}
\item[Cic:rep] \textbf{Cic. rep.}\newline Cicero\newline \emph{de re publica}
\item[Cic:Scaur] \textbf{Cic. Scaur.}\newline Cicero\newline \emph{pro M. Aemilio Scauro}
\item[Cic:Sest] \textbf{Cic. Sest.}\newline Cicero\newline \emph{pro P. Sestio}
\item[Cic:SRosc] \textbf{Cic. S. Rosc.}\newline Cicero\newline \emph{pro Sex. Roscio Amerino}
\item[Cic:Sull] \textbf{Cic. Sull.}\newline Cicero\newline \emph{pro P. Sulla}
\item[Cic:Tim] \textbf{Cic. Tim.}\newline Cicero\newline \emph{Timaeus}
\item[Cic:top] \textbf{Cic. top.}\newline Cicero\newline \emph{topica}
\item[Cic:Tull] \textbf{Cic. Tull.}\newline Cicero\newline \emph{pro M.Tullio}
\item[Cic:Tusc] \textbf{Cic. Tusc.}\newline Cicero\newline \emph{Tusculanae disputationes}
\item[Cic:Vatin] \textbf{Cic. Vatin.}\newline Cicero\newline \emph{in P. Vatinium testem interrogatio}
\item[Cic:Verr12] \textbf{Cic. Verr. 1,2}\newline Cicero\newline \emph{in Verrem actio prima, secunda}
\item[Claud:carm] \textbf{Claud. carm.}\newline Claudius Claudianus\newline \emph{carmina}
\item[Claud:raptPros] \textbf{Claud. rapt. Pros.}\newline Claudius Claudianus\newline \emph{de raptu Proserpinae}
\item[CodIust] \textbf{ Cod. Iust.}\newline \newline \emph{Corpus Iuris Civilis, Codex Iustinianus}
\item[Colum] \textbf{Colum.}\newline Columella\newline 
\item[Curt] \textbf{Curt.}\newline Curtius Rufus\newline \emph{historiae Alexandri Magni}
\item[Cypr] \textbf{Cypr.}\newline Cyprianus\newline 
\item[Demokr] \textbf{Demokr.}\newline Demokritos\newline 
\item[Demosth:or] \textbf{Demosth. or.}\newline Demosthenes\newline \emph{or.}
\item[Dig] \textbf{ Dig.}\newline \newline \emph{Corpus Iuris Civilis, Digesta (Pandekten.}
\item[Diod] \textbf{Diod.}\newline Diodorus Siculus\newline 
\item[Diog:Laert] \textbf{Diog. Laert.}\newline Diogenes Laertios\newline 
\item[Dion:Chrys] \textbf{Dion Chrys.}\newline Dion Chrysostomos\newline \emph{Orationes}
\item[Dion:Halant] \textbf{Dion. Hal. ant.}\newline Dionysios Halicarnassos\newline \emph{antiquitates Romanae}
\item[Dion:Halcomp] \textbf{Dion. Hal. comp.}\newline Dionysios Halicarnassos\newline \emph{de compositione verborum}
\item[Dion:Halrhet] \textbf{Dion. Hal. rhet.}\newline Dionysios Halicarnassos\newline \emph{ars rhetorica       .}
\item[Don] \textbf{Don.}\newline Donatus grammaticus\newline 
\item[Drac] \textbf{Drac.}\newline Dracontius\newline 
\item[Dt] \textbf{Dt}\newline Deuteronomium (= 5. Mose)\newline 
\item[Emp] \textbf{Emp.}\newline Empedokles\newline 
\item[Enn:ann] \textbf{Enn. ann.}\newline Ennius\newline \emph{Annales (Skutsch, 1985)}
\item[Enn:sat] \textbf{Enn. sat.}\newline Ennius\newline \emph{Saturae, (Vahlen, 21928)}
\item[Enn:scaen] \textbf{Enn. scaen.}\newline Ennius\newline \emph{fragmenta scenica (Vahlen, 21928)}
\item[Ennod] \textbf{Ennod.}\newline Ennodius\newline 
\item[Eph] \textbf{Eph}\newline Epheserbrief\newline 
\item[Epik] \textbf{Epik.}\newline Epikuros\newline 
\item[Epikt] \textbf{Epikt.}\newline Epiktetos\newline 
\item[Etym:m] \textbf{Etym. m.}\newline Etymologicum magnum (Gaisford)\newline 
\item[Eukl:elem] \textbf{Eukl. elem.}\newline Euklid\newline \emph{elem.}
\item[Eur:Alc] \textbf{Eur. Alc.}\newline Euripides\newline \emph{Alcestis}
\item[Eur:Andr] \textbf{Eur. Andr.}\newline Euripides\newline \emph{Andromache}
\item[Eur:Bacch] \textbf{Eur. Bacch.}\newline Euripides\newline \emph{Bacchae}
\item[Eur:Cycl] \textbf{Eur. Cycl.}\newline Euripides\newline \emph{Cyclops}
\item[Eur:El] \textbf{Eur. El.}\newline Euripides\newline \emph{Electra}
\item[Eur:Hec] \textbf{Eur. Hec.}\newline Euripides\newline \emph{Hecuba (Hekabe)}
\item[Eur:Hel] \textbf{Eur. Hel.}\newline Euripides\newline \emph{Helena}
\item[Eur:Heracl] \textbf{Eur. Heracl.}\newline Euripides\newline \emph{Heraclidae}
\item[Eur:Herc] \textbf{Eur. Herc.}\newline Euripides\newline \emph{Hercules (furens)}
\item[Eur:Hipp] \textbf{Eur. Hipp.}\newline Euripides\newline \emph{Hippolytos}
\item[Eur:Ion] \textbf{Eur. Ion}\newline Euripides\newline \emph{Ion}
\item[Eur:IphA] \textbf{Eur. Iph. A.}\newline Euripides\newline \emph{Iphigenia Aulidensis}
\item[Eur:IphT] \textbf{Eur. Iph. T.}\newline Euripides\newline \emph{Iphigenia Taurica}
\item[Eur:Med] \textbf{Eur. Med.}\newline Euripides\newline \emph{Medea}
\item[Eur:Or] \textbf{Eur. Or.}\newline Euripides\newline \emph{Orestes}
\item[Eur:Phoen] \textbf{Eur. Phoen.}\newline Euripides\newline \emph{Phoenissae}
\item[Eur:Rhes] \textbf{Eur. Rhes.}\newline Euripides\newline \emph{Rhesus}
\item[Eur:Suppl] \textbf{Eur. Suppl.}\newline Euripides\newline \emph{Supplices (ἱκέτιδες)}
\item[Eur:Tro] \textbf{Eur. Tro.}\newline Euripides\newline \emph{Troades}
\item[Eus:DeEv] \textbf{Eus. De. Ev.}\newline Eusebios\newline \emph{Demonstratio Evangelica}
\item[Eus:HE] \textbf{Eus. HE}\newline Eusebios\newline \emph{Historia Ecclesiastica}
\item[Eus:On] \textbf{Eus. On.}\newline Eusebios\newline \emph{Onomastikon}
\item[Eus:PrEv] \textbf{Eus. Pr. Ev.}\newline Eusebios\newline \emph{Praeparatio Evangelica}
\item[Eust] \textbf{Eust.}\newline Eustathios\newline 
\item[Eutr] \textbf{Eutr.}\newline Eutropius\newline 
\item[Fest] \textbf{Fest.}\newline Festus\newline 
\item[Flor:epit] \textbf{Flor.  epit.}\newline Florus\newline \emph{epitoma de Tito Livio}
\item[Frontin:aqustrat] \textbf{Frontin. aqu. strat.}\newline Frontinus\newline \emph{de aquae ductu urbis Romae strategemata}
\item[Fulg] \textbf{Fulg.}\newline Fulgentius Afer\newline 
\item[Gal] \textbf{Gal.}\newline Galenos\newline 
\item[Galbr] \textbf{Gal}\newline Galaterbrief\newline 
\item[Gell] \textbf{Gell.}\newline Gellius\newline \emph{noctes Atticae}
\item[Gorg] \textbf{Gorg.}\newline Gorgias\newline 
\item[GregM:dial] \textbf{Greg. M. dial.}\newline Gregorius Magnus\newline \emph{dialogi (de miraculis patrum Italicorum)}
\item[GregM:epist] \textbf{Greg. M. epist.}\newline Gregorius Magnus\newline \emph{epistulae}
\item[GregM:past] \textbf{Greg. M. past.}\newline Gregorius Magnus\newline \emph{regula pastoralis}
\item[GregNaz:epist] \textbf{Greg. Naz. epist.}\newline Gregorius Nazianzienus\newline \emph{epistulae}
\item[GregNaz:or] \textbf{Greg. Naz. or.}\newline Gregorius Nazianzienus\newline \emph{orationes}
\item[GregNyss] \textbf{Greg. Nyss.}\newline Gregorius Nyssenus\newline 
\item[GregTur:Franc] \textbf{Greg. Tur. Franc.}\newline Gregorius von Tours\newline \emph{historia Francorum}
\item[GregTur:Mart] \textbf{Greg. Tur. Mart.}\newline Gregorius von Tours\newline \emph{de virtutibus Martini}
\item[GregTur:vitpatr] \textbf{Greg. Tur. vit. patr.}\newline Gregorius von Tours\newline \emph{de vita patrum}
\item[Hdt] \textbf{Hdt.}\newline Herodot\newline 
\item[Hekat] \textbf{Hekat.}\newline Hekataios\newline 
\item[Herakl] \textbf{Herakl.}\newline Herakleitos\newline 
\item[Herakl:Pont] \textbf{Herakl. Pont.}\newline Herakleides Pontikos\newline 
\item[Herm:Trism] \textbf{Herm. Trism.}\newline Hermes Trismegistos\newline 
\item[Herodian] \textbf{Herodian.}\newline Herodian\newline \emph{ab excessu divi Marci}
\item[Hes:cat] \textbf{Hes. cat.}\newline Hesiodos\newline \emph{Catalogues feminarum (ἠοίαι)}
\item[Hes:erg] \textbf{Hes. erg.}\newline Hesiodos\newline \emph{opera et dies (ἔργα καὶ ἡμέραι)}
\item[Hes:scut] \textbf{Hes. scut.}\newline Hesiodos\newline \emph{scutum (ἀσπίς)}
\item[Hes:theog] \textbf{Hes. theog.}\newline Hesiodos\newline \emph{Theogonia}
\item[Hesych] \textbf{Hesych.}\newline Hesychios\newline 
\item[Hier:chron] \textbf{Hier. chron.}\newline Hieronymos\newline \emph{chronicon}
\item[Hier:comminEz] \textbf{Hier. comm. in Ez}\newline Hieronymos\newline \emph{commentaria in Ezechielem (PL 25)}
\item[Hier:epist] \textbf{Hier. epist.}\newline Hieronymos\newline \emph{epistulae}
\item[Hier:On] \textbf{Hier. On.}\newline Hieronymos\newline \emph{Onomastikon}
\item[Hier:virill] \textbf{Hier. vir. ill.}\newline Hieronymos\newline \emph{de viris illustribus}
\item[Hippokr] \textbf{Hippokr.}\newline Hippokrates\newline 
\item[HL] \textbf{HL}\newline Hohelied\newline 
\item[Hom:h] \textbf{Hom. h.}\newline Homer\newline \emph{Hymni Homeri}
\item[Hom:Il] \textbf{Hom. Il.}\newline Homer\newline \emph{Ilias}
\item[Hom:Od] \textbf{Hom. Od.}\newline Homer\newline \emph{Odyssee}
\item[Hor:ars] \textbf{Hor. ars}\newline Horaz\newline \emph{ars poetica}
\item[Hor:carm] \textbf{Hor. carm.}\newline Horaz\newline \emph{carmina}
\item[Hor:carmsaec] \textbf{Hor. carm. saec.}\newline Horaz\newline \emph{carmen saeculare}
\item[Hor:epist] \textbf{Hor. epist.}\newline Horaz\newline \emph{epistulae}
\item[Hor:epod] \textbf{Hor. epod.}\newline Horaz\newline \emph{epodi}
\item[Hor:sat] \textbf{Hor. sat.}\newline Horaz\newline \emph{saturae (sermones)}
\item[Hyg:astr] \textbf{Hyg. astr.}\newline Hygin\newline \emph{astronomica}
\item[Hyg:fab] \textbf{Hyg. fab.}\newline Hygin\newline \emph{fabulae}
\item[Iambl:demyst] \textbf{Iambl. de myst.}\newline Iamblichos\newline \emph{de mysteriis}
\item[Iambl:protr] \textbf{Iambl. protr.}\newline Iamblichos\newline \emph{protrepticus in philosophiam}
\item[Iambl:vP] \textbf{Iambl. v. P.}\newline Iamblichos\newline \emph{de vita Pythagorica}
\item[InstIust] \textbf{ Inst. Iust.}\newline \newline \emph{Corpus Iuris Civilis, Institutiones}
\item[Ios:AntIud] \textbf{Ios. Ant. Iud.}\newline Iosephos (Flavius Iosephus)\newline \emph{antiquitates Iudaicae}
\item[Ios:BellIud] \textbf{Ios. Bell. Iud.}\newline Iosephos (Flavius Iosephus)\newline \emph{bellum Iudaicum}
\item[Ios:cAp] \textbf{Ios. c. Ap.}\newline Iosephos (Flavius Iosephus)\newline \emph{contra Apionem}
\item[Ios:vita] \textbf{Ios. vita}\newline Iosephos (Flavius Iosephus)\newline \emph{de sua vita}
\item[Isid:nat] \textbf{Isid. nat.}\newline Isidorus von Sevilla\newline \emph{de natura rerum}
\item[Isid:orig] \textbf{Isid. orig.}\newline Isidorus von Sevilla\newline \emph{origines (etymologiae)}
\item[Isokr:or] \textbf{Isokr. or.}\newline Isokrates\newline \emph{orationes}
\item[ItinAnton] \textbf{ Itin. Anton.}\newline \newline \emph{Itinerarium Antonini}
\item[Iul:epist] \textbf{Iul. epist.}\newline Iulianos\newline \emph{epistulae}
\item[Iul:inGal] \textbf{Iul. in Gal.}\newline Iulianos\newline \emph{in Galilaeos}
\item[Iul:mis] \textbf{Iul. mis.}\newline Iulianos\newline \emph{Misopogon}
\item[Iul:or] \textbf{Iul. or.}\newline Iulianos\newline \emph{orationes}
\item[Iul:symp] \textbf{Iul. symp.}\newline Iulianos\newline \emph{symposion}
\item[IulVict:Rhet] \textbf{Iul. Vict. Rhet.}\newline C. Iulius Victor\newline \emph{ars rhetorica}
\item[Iust] \textbf{Iust.}\newline Iustinus\newline \emph{epitoma historiarum Philippicarum}
\item[Iuv:sat] \textbf{Iuv.}\newline Iuvenal\newline \emph{saturae}
\item[Iuvenc] \textbf{Iuvenc.}\newline Iuvencus\newline \emph{evangelia}
\item[Jo] \textbf{Jo}\newline Johannes\newline 
\item[Kall:epigr] \textbf{Kall. epigr.}\newline Kallimachos\newline \emph{Epigrammata}
\item[Kall:frg] \textbf{Kall. frg.}\newline Kallimachos\newline \emph{fragmentum}
\item[Kall:h] \textbf{Kall. h.}\newline Kallimachos\newline \emph{hymni}
\item[Lact:inst] \textbf{Lact. inst.}\newline Lactantius\newline \emph{divinae institutiones}
\item[Lact:ira] \textbf{Lact. ira}\newline Lactantius\newline \emph{de ira dei}
\item[Lact:mortpers] \textbf{Lact. mort. pers.}\newline Lactantius\newline \emph{de mortibus persecutorum}
\item[Lact:opif] \textbf{Lact. opif.}\newline Lactantius\newline \emph{de opificio dei}
\item[Liv] \textbf{Liv.}\newline Livius\newline \emph{ab urbe condita}
\item[Lk] \textbf{Lk}\newline Lukas\newline 
\item[Lucan] \textbf{Lucan.}\newline Lucanus\newline \emph{bellum civile}
\item[Lucil] \textbf{Lucil.}\newline Lucilius\newline \emph{saturae}
\item[Lucr] \textbf{Lucr.}\newline Lukrez\newline \emph{de rerum natura}
\item[Lukian] \textbf{Lukian.}\newline Lukianos\newline 
\item[LXX] \textbf{LXX}\newline Septuaginta\newline 
\item[Lyd:mag] \textbf{Lyd. mag.}\newline Lydos\newline \emph{de magistratibus}
\item[Lyd:mens] \textbf{Lyd. mens.}\newline Lydos\newline \emph{de mensibus}
\item[Lykurg] \textbf{Lykurg.}\newline Lykurgos\newline 
\item[Lys] \textbf{Lys.}\newline Lysias\newline 
\item[Macr:Sat] \textbf{Macr. Sat.}\newline Macrobius\newline \emph{Saturnalia}
\item[Manil] \textbf{Manil.}\newline Manilius\newline \emph{astronomica}
\item[Mart] \textbf{Mart.}\newline Martial\newline \emph{Epigrammata}
\item[Mart:spect] \textbf{Mart. spect.}\newline Martial\newline \emph{spectacula}
\item[MartCap] \textbf{Mart. Cap.}\newline Martianus Capella\newline 
\item[MarVictorin] \textbf{Mar. Victorin.}\newline Marius Victorinus\newline 
\item[Maur] \textbf{M. Aur.}\newline Marcus Aurelius Antonius Augustus\newline 
\item[Mel] \textbf{Mel.}\newline Pomponius Mela\newline 
\item[Men:Dysk] \textbf{Men. Dysk.}\newline Menandros\newline \emph{Dyskolo.}
\item[Men:Epitr] \textbf{Men. Epitr.}\newline Menandros\newline \emph{Epitrepontes}
\item[Men:Sam] \textbf{Men. Sam.}\newline Menandros\newline \emph{Samia}
\item[Mimn] \textbf{Mimn.}\newline Mimnermos\newline 
\item[MinFel] \textbf{Min. Fel.}\newline Oct. Minucius Felix\newline 
\item[Mk] \textbf{Mk}\newline Markus\newline 
\item[Mosch] \textbf{Mosch.}\newline Moschos\newline 
\item[Mt] \textbf{Mt}\newline Matthäus\newline 
\item[Naev] \textbf{Naev.}\newline Naevius\newline 
\item[Nep:Att] \textbf{Nep. Att.}\newline Cornelius Nepos\newline \emph{Atticus}
\item[Nep:Hann] \textbf{Nep. Hann.}\newline Cornelius Nepos\newline \emph{Hannibal}
\item[Nik:Alex] \textbf{Nik. Alex.}\newline Nikander\newline \emph{Alexipharmaka}
\item[Nik:Ther] \textbf{Nik. Ther.}\newline Nikander\newline \emph{Theriaka}
\item[Nm] \textbf{Nm}\newline Numeri (= 4. Mose)\newline 
\item[Non] \textbf{Non.}\newline Nonius\newline 
\item[Nonn:Dion] \textbf{Nonn.  Dion.}\newline {Nonnos von Panopolis}\newline \emph{Dion.}
\item[Nov] \textbf{ Nov.}\newline \newline \emph{Corpus Iuris Civilis, Leges Novellae}
\item[Orig] \textbf{Orig.}\newline Origines\newline 
\item[Oros] \textbf{Oros..}\newline Orosius\newline 
\item[Ov:am] \textbf{Ov. am.}\newline Ovid\newline \emph{amores}
\item[Ov:ars] \textbf{Ov. ars}\newline Ovid\newline \emph{ars amatoria}
\item[Ov:epist] \textbf{Ov. epist.}\newline Ovid\newline \emph{epistulae (heroides)}
\item[Ov:fast] \textbf{Ov. fast.}\newline Ovid\newline \emph{fasti}
\item[Ov:Ib] \textbf{Ov. Ib.}\newline Ovid\newline \emph{Ibis}
\item[Ov:medic] \textbf{Ov. medic.}\newline Ovid\newline \emph{medicamina faciei femineae}
\item[Ov:met] \textbf{Ov. met.}\newline Ovid\newline \emph{metamorphoses}
\item[Ov:Pont] \textbf{Ov. Pont.}\newline Ovid\newline \emph{epistulae ex Ponto}
\item[Ov:rem] \textbf{Ov. rem.}\newline Ovid\newline \emph{remedia amoris}
\item[Ov:trist] \textbf{Ov. trist.}\newline Ovid\newline \emph{tristia}
\item[Papin] \textbf{Papin.}\newline Aemilius Papinianus\newline 
\item[Passmart] \textbf{ Pass. mart.}\newline \newline \emph{Passiones martyrum}
\item[Paus] \textbf{Paus.}\newline Pausanias\newline 
\item[Pers] \textbf{Pers.}\newline Persius\newline \emph{saturae}
\item[Petron] \textbf{Petron.}\newline Petronius\newline \emph{satyrica}
\item[Phaedr] \textbf{Phaedr.}\newline Phaedrus\newline \emph{fabulae}
\item[Pind:frg] \textbf{Pind. frg.}\newline Pindar\newline \emph{fragmentum}
\item[Pind:I] \textbf{Pind. I.}\newline Pindar\newline \emph{Isthmien}
\item[Pind:N] \textbf{Pind. N.}\newline Pindar\newline \emph{Nemeen}
\item[Pind:O] \textbf{Pind. O.}\newline Pindar\newline \emph{Olympien}
\item[Pind:P] \textbf{Pind. P.}\newline Pindar\newline \emph{Pythien}
\item[Plat:Alk1] \textbf{Plat. Alk. 1}\newline Platon\newline \emph{Alkibiades 1}
\item[Plat:Alk2] \textbf{Plat. Alk. 2}\newline Platon\newline \emph{Alkibiades 2}
\item[Plat:apol] \textbf{Plat. apol.}\newline Platon\newline \emph{apologia}
\item[Plat:Ax] \textbf{Plat. Ax.}\newline Platon\newline \emph{Axiochos}
\item[Plat:Charm] \textbf{Plat. Charm.}\newline Platon\newline \emph{Charmides}
\item[Plat:def] \textbf{Plat. def.}\newline Platon\newline \emph{Definitiones (ὅροι)}
\item[Plat:Dem] \textbf{Plat. Dem.}\newline Platon\newline \emph{Demodokos}
\item[Plat:epin] \textbf{Plat. epin.}\newline Platon\newline \emph{epinomis}
\item[Plat:epist] \textbf{Plat. epist.}\newline Platon\newline \emph{epistulae}
\item[Plat:eras] \textbf{Plat. eras.}\newline Platon\newline \emph{erastae}
\item[Plat:Eryx] \textbf{Plat. Eryx.}\newline Platon\newline \emph{Eryxias}
\item[Plat:Euthyd] \textbf{Plat. Euthyd.}\newline Platon\newline \emph{Euthydemos}
\item[Plat:Euthyphr] \textbf{Plat. Euthyphr.}\newline Platon\newline \emph{Euthyphron}
\item[Plat:Gorg] \textbf{Plat. Gorg.}\newline Platon\newline \emph{Gorgias}
\item[Plat:Hipparch] \textbf{Plat. Hipparch.}\newline Platon\newline \emph{Hipparchos}
\item[Plat:Hippmai] \textbf{Plat. Hipp. mai.}\newline Platon\newline \emph{Hippias maior}
\item[Plat:Hippmin] \textbf{Plat. Hipp. min.}\newline Platon\newline \emph{Hippias minor}
\item[Plat:Ion] \textbf{Plat. Ion}\newline Platon\newline \emph{Ion}
\item[Plat:Kleit] \textbf{Plat. Kleit.}\newline Platon\newline \emph{Kleitophon}
\item[Plat:Krat] \textbf{Plat. Krat.}\newline Platon\newline \emph{Kratylos}
\item[Plat:Krit] \textbf{Plat. Krit.}\newline Platon\newline \emph{Kriton}
\item[Plat:Kritias] \textbf{Plat. Kritias}\newline Platon\newline \emph{Kritias}
\item[Plat:Lach] \textbf{Plat. Lach.}\newline Platon\newline \emph{Laches}
\item[Plat:leg] \textbf{Plat. leg.}\newline Platon\newline \emph{leges (νόμοι)}
\item[Plat:Lys] \textbf{Plat. Lys.}\newline Platon\newline \emph{Lysis}
\item[Plat:Men] \textbf{Plat. Men.}\newline Platon\newline \emph{Menon}
\item[Plat:Min] \textbf{Plat. Min.}\newline Platon\newline \emph{Minos}
\item[Plat:Mx] \textbf{Plat. Mx.}\newline Platon\newline \emph{Menexenos}
\item[Plat:Parm] \textbf{Plat. Parm.}\newline Platon\newline \emph{Parmenides}
\item[Plat:Phaid] \textbf{Plat. Phaid.}\newline Platon\newline \emph{Phaidon}
\item[Plat:Phaidr] \textbf{Plat. Phaidr.}\newline Platon\newline \emph{Phaidros}
\item[Plat:Phil] \textbf{Plat. Phil.}\newline Platon\newline \emph{Philebos}
\item[Plat:polit] \textbf{Plat. polit.}\newline Platon\newline \emph{politicus}
\item[Plat:Prot] \textbf{Plat. Prot.}\newline Platon\newline \emph{Protagoras}
\item[Plat:rep] \textbf{Plat. rep.}\newline Platon\newline \emph{de re publica (πολιτεία)}
\item[Plat:Sis] \textbf{Plat. Sis.}\newline Platon\newline \emph{Sisyphos}
\item[Plat:soph] \textbf{Plat. soph.}\newline Platon\newline \emph{sophista}
\item[Plat:symp] \textbf{Plat. symp.}\newline Platon\newline \emph{symposium}
\item[Plat:Thg] \textbf{Plat. Thg.}\newline Platon\newline \emph{Theages}
\item[Plat:Tht] \textbf{Plat. Tht.}\newline Platon\newline \emph{Theaitetos}
\item[Plat:Tim] \textbf{Plat. Tim.}\newline Platon\newline \emph{Timaios}
\item[Plaut:Amph] \textbf{Plaut. Amph.}\newline Plautus\newline \emph{Amphitruo}
\item[Plaut:Asin] \textbf{Plaut. Asin.}\newline Plautus\newline \emph{Asinaria}
\item[Plaut:Aul] \textbf{Plaut. Aul.}\newline Plautus\newline \emph{Aulularia}
\item[Plaut:Bacch] \textbf{Plaut. Bacch.}\newline Plautus\newline \emph{Bacchides}
\item[Plaut:Cist] \textbf{Plaut. Cist.}\newline Plautus\newline \emph{Cistellaria}
\item[Plaut:Curc] \textbf{Plaut. Curc.}\newline Plautus\newline \emph{Curculio}
\item[Plaut:Epid] \textbf{Plaut. Epid.}\newline Plautus\newline \emph{Epidicus}
\item[Plaut:Men] \textbf{Plaut. Men.}\newline Plautus\newline \emph{Menaechmi}
\item[Plaut:Merc] \textbf{Plaut. Merc.}\newline Plautus\newline \emph{Mercator}
\item[Plaut:Mil] \textbf{Plaut. Mil.}\newline Plautus\newline \emph{Miles gloriosus}
\item[Plaut:Most] \textbf{Plaut. Most.}\newline Plautus\newline \emph{Mostellaria}
\item[Plaut:Poen] \textbf{Plaut. Poen.}\newline Plautus\newline \emph{Poenulus}
\item[Plaut:Pseud] \textbf{Plaut. Pseud.}\newline Plautus\newline \emph{Pseudolus}
\item[Plaut:Rud] \textbf{Plaut. Rud.}\newline Plautus\newline \emph{Rudens}
\item[Plaut:Stich] \textbf{Plaut. Stich.}\newline Plautus\newline \emph{Stichus}
\item[Plaut:Trin] \textbf{Plaut. Trin.}\newline Plautus\newline \emph{Trinummus}
\item[Plaut:Truc] \textbf{Plaut. Truc.}\newline Plautus\newline \emph{Truculentus}
\item[Plaut:Vid] \textbf{Plaut. Vid.}\newline Plautus\newline \emph{Vidularia}
\item[Plin:epist] \textbf{Plin. epist.}\newline Plinius minor\newline \emph{Epistulae}
\item[Plin:nat] \textbf{Plin. nat.}\newline Plinius maior\newline \emph{naturalis historia}
\item[Plin:paneg] \textbf{Plin. paneg.}\newline Plinius minor\newline \emph{panegyricus}
\item[Plot] \textbf{Plot.}\newline Plotinos\newline 
\item[Plut] \textbf{Plut.}\newline Plutarch\newline \emph{vitae parallelae}
\item[Plut:am] \textbf{Plut. am.}\newline Plutarch\newline \emph{Amatorius (erotikos)}
\item[Plut:dePythOr] \textbf{Plut. de Pyth.Or.}\newline Plutarch\newline \emph{de Pythiae oraculis}
\item[Plut:Is] \textbf{Plut. Is.}\newline Plutarch\newline \emph{de Iside et Osiride}
\item[Plut:mor] \textbf{Plut. mor.}\newline Plutarch\newline \emph{moralia}
\item[Plut:quGr] \textbf{Plut. qu. Gr.}\newline Plutarch\newline \emph{quaestiones Graecae}
\item[Plut:quR] \textbf{Plut. qu. R.}\newline Plutarch\newline \emph{quaestiones Romanae}
\item[Plut:symp] \textbf{Plut. symp.}\newline Plutarch\newline \emph{quaestiones convivales}
\item[Pol] \textbf{Pol.}\newline Polybios\newline 
\item[Poll] \textbf{Poll.}\newline Pollux\newline 
\item[Porph] \textbf{Porph.}\newline Porphyrios\newline 
\item[Poseid] \textbf{Poseid.}\newline Poseidonios\newline 
\item[Priap] \textbf{Priap.}\newline Priapea\newline 
\item[Prisc] \textbf{Prisc.}\newline Priscianus\newline 
\item[Prok:aed] \textbf{Prok. aed.}\newline Prokop\newline \emph{de aedificiis (περὶ κτισμάτων)}
\item[Prok:BG] \textbf{Prok. BG}\newline Prokop\newline \emph{bellum Gothicum}
\item[Prok:BP] \textbf{Prok. BP}\newline Prokop\newline \emph{bellum Persicum}
\item[Prok:BV] \textbf{Prok. BV}\newline Prokop\newline \emph{bellum Vandalicum}
\item[Prok:HA] \textbf{Prok. HA}\newline Prokop\newline \emph{historia arcana}
\item[Prokl] \textbf{Prokl.}\newline Proklos\newline 
\item[Prop] \textbf{Prop.}\newline Properz\newline \emph{elegiae}
\item[Prud] \textbf{Prud.}\newline Prudentius\newline 
\item[Quint:decl] \textbf{Quint. decl.}\newline Quintilian\newline \emph{declamationes minores}
\item[Quint:inst] \textbf{Quint. inst.}\newline Quintilian\newline \emph{institutio oratoria}
\item[RgestdivAug] \textbf{ R. Gest. div. Aug.}\newline \newline \emph{Res gestae divi Augusti}
\item[Rhet:Her] \textbf{Rhet. Her.}\newline Rhetorica ad Herennium\newline 
\item[Roem] \textbf{Röm}\newline Römerbrief\newline 
\item[RutNam] \textbf{Rut. Nam.}\newline Rutilius Claudius Namatianus\newline \emph{de reditu suo}
\item[Sall:Catil] \textbf{Sall. Catil.}\newline Sallust\newline \emph{de coniuratione Catilinae}
\item[Sall:epist] \textbf{Sall. epist.}\newline Sallust\newline \emph{epistulae ad Caesarem}
\item[Sall:hist] \textbf{Sall. hist.}\newline Sallust\newline \emph{historiae}
\item[Sall:Iug] \textbf{Sall. Iug.}\newline Sallust\newline \emph{de bello Iugurthino}
\item[Semp] \textbf{S. Emp.}\newline Sextus Empiricus\newline 
\item[Sen:apocol] \textbf{Sen. apocol.}\newline Seneca minor\newline \emph{divi Claudii apocolocynthosis}
\item[Sen:benef] \textbf{Sen. benef.}\newline Seneca minor\newline \emph{de beneficiis}
\item[Sen:clem] \textbf{Sen. clem.}\newline Seneca minor\newline \emph{de clementia}
\item[Sen:contr] \textbf{Sen. contr.}\newline Seneca maior\newline \emph{controversiae}
\item[Sen:dial] \textbf{Sen. dial.}\newline Seneca minor\newline \emph{dialogi}
\item[Sen:epist] \textbf{Sen. epist.}\newline Seneca minor\newline \emph{epistulae morales ad Lucilium}
\item[Sen:Hercf] \textbf{Sen. Herc. f.}\newline Seneca minor\newline \emph{Hercules furens}
\item[Sen:HercO] \textbf{ Herc. O.}\newline Seneca minor\newline \emph{Hercules Oetaeus}
\item[Sen:Med] \textbf{Sen. Med.}\newline Seneca minor\newline \emph{Medea}
\item[Sen:nat] \textbf{Sen. nat.}\newline Seneca minor\newline \emph{naturales quaestiones}
\item[Sen:Oed] \textbf{Sen. Oed.}\newline Seneca minor\newline \emph{Oedipus}
\item[Sen:Phaedr] \textbf{Sen. Phaedr.}\newline Seneca minor\newline \emph{Phaedra}
\item[Sen:Phoen] \textbf{Sen. Phoen.}\newline Seneca minor\newline \emph{Phoenissae}
\item[Sen:suas] \textbf{Sen. suas.}\newline Seneca maior\newline \emph{suasoriae}
\item[Sen:Thy] \textbf{Sen. Thy.}\newline Seneca minor\newline \emph{Thyestes}
\item[Sen:Tro] \textbf{Sen. Tro.}\newline Seneca minor\newline \emph{Troades}
\item[Serv:Aen] \textbf{Serv. Aen.}\newline Servius\newline \emph{commentarius in Vergilii Aeneida}
\item[Serv:auct] \textbf{Serv. auct.}\newline Servius auctus Danielis\newline 
\item[Serv:ecl] \textbf{Serv. ecl.}\newline Servius\newline \emph{commentarius in Vergilii eclogas}
\item[Serv:georg] \textbf{Serv. georg.}\newline Servius\newline \emph{commentarius in Vergilii georgica}
\item[Sil] \textbf{Sil.}\newline Silius Italicus\newline \emph{Punica}
\item[Sim] \textbf{Sim.}\newline Simonides\newline 
\item[Simpl] \textbf{Simpl.}\newline Simplikios\newline 
\item[Sol] \textbf{Sol.}\newline Solon\newline 
\item[Soph:Ai] \textbf{Soph. Ai.}\newline Sophokles\newline \emph{Aias}
\item[Soph:Ant] \textbf{Soph. Ant.}\newline Sophokles\newline \emph{Antigone}
\item[Soph:El] \textbf{Soph. El.}\newline Sophokles\newline \emph{Electra}
\item[Soph:OidK] \textbf{Soph. Oid. K.}\newline Sophokles\newline \emph{Oedipus Coloneus}
\item[Soph:OidT] \textbf{Soph. Oid. T.}\newline Sophokles\newline \emph{Oedipus Rex (Oedipus Tyrannus)}
\item[Soph:Phil] \textbf{Soph. Phil.}\newline Sophokles\newline \emph{Philoctetes}
\item[Stat:Ach] \textbf{Stat. Ach.}\newline Statius\newline \emph{Achilleis}
\item[Stat:silv] \textbf{Stat. silv.}\newline Statius\newline \emph{silvae}
\item[Stat:Theb] \textbf{Stat. Theb.}\newline Statius\newline \emph{Thebais}
\item[Stesich] \textbf{Stesich.}\newline Stesichoros\newline 
\item[Stob] \textbf{Stob.}\newline Stobaios\newline 
\item[Strab] \textbf{Strab.}\newline Strabon\newline 
\item[Suet:Aug] \textbf{Suet. Aug.}\newline Sueton\newline \emph{divus Augustus}
\item[Suet:Cal] \textbf{Suet. Cal.}\newline Sueton\newline \emph{Caligula}
\item[Suet:Claud] \textbf{Suet. Claud.}\newline Sueton\newline \emph{divus Claudius}
\item[Suet:Dom] \textbf{Suet. Dom.}\newline Sueton\newline \emph{Domitianus}
\item[Suet:gramm] \textbf{Suet. gramm.}\newline Sueton\newline \emph{de grammaticis}
\item[Suet:Iul] \textbf{Suet. Iul.}\newline Sueton\newline \emph{divus Iulius}
\item[Suet:Tib] \textbf{Suet. Tib.}\newline Sueton\newline \emph{divus Tiberius}
\item[Suet:Tit] \textbf{Suet. Tit.}\newline Sueton\newline \emph{divus Titus}
\item[Suet:Vesp] \textbf{Suet. Vesp.}\newline Sueton\newline \emph{divus Vespasianus}
\item[Suet:Vit] \textbf{Suet. Vit.}\newline Sueton\newline \emph{Vitellius}
\item[Symm:epist] \textbf{Symm. epist.}\newline Symmachus\newline \emph{epistulae}
\item[Symm:or] \textbf{Symm. or.}\newline Symmachus\newline \emph{orationes}
\item[Symm:rel] \textbf{Symm. rel.}\newline Symmachus\newline \emph{relationes}
\item[Synes:epist] \textbf{Synes. Epist.}\newline epist.\newline \emph{epistulae}
\item[Tac:Agr] \textbf{Tac. Agr.}\newline Tacitus\newline \emph{Agricola}
\item[Tac:ann] \textbf{Tac. ann.}\newline Tacitus\newline \emph{annales}
\item[Tac:dial] \textbf{Tac. dial.}\newline Tacitus\newline \emph{dialogus de oratoribus}
\item[Tac:Germ] \textbf{Tac. Germ.}\newline Tacitus\newline \emph{Germania}
\item[Tac:hist] \textbf{Tac. hist.}\newline Tacitus\newline \emph{historiae}
\item[Ter:Ad] \textbf{Ter. Ad.}\newline Terenz\newline \emph{Adelphoe}
\item[Ter:Andr] \textbf{Ter. Andr.}\newline Terenz\newline \emph{Andria}
\item[Ter:Eun] \textbf{Ter. Eun.}\newline Terenz\newline \emph{Eunuchus}
\item[Ter:Haut] \textbf{Ter. Haut.}\newline Terenz\newline \emph{Heautontimorumenos}
\item[Ter:Hec] \textbf{Ter. Hec.}\newline Terenz\newline \emph{Hecyra}
\item[Ter:Maur] \textbf{Ter. Maur.}\newline Terentianus Maurus\newline 
\item[Ter:Phorm] \textbf{Ter. Phorm.}\newline Terenz\newline \emph{Phormio}
\item[Tert:apol] \textbf{Tert. apol.}\newline Tertullianus\newline \emph{apologeticum}
\item[Tert:nat] \textbf{Tert. nat.}\newline Tertullianus\newline \emph{ad nationes}
\item[Theokr] \textbf{Theokr.}\newline Theokritos\newline 
\item[Thgn] \textbf{Thgn.}\newline Theognis\newline 
\item[Thuk] \textbf{Thuk.}\newline Thukydides\newline 
\item[Tib] \textbf{Tib.}\newline Tibullus\newline \emph{elegiae}
\item[Ulp:reg] \textbf{Ulp.  Reg.}\newline Ulpianus\newline \emph{(Ulpiani regulae)}
\item[ValFl] \textbf{Val. Fl.}\newline Valerius Flaccus\newline \emph{Argonautica}
\item[ValMax] \textbf{Val. Max.}\newline Valerius Maximus\newline \emph{facta et dicta memorabilia}
\item[Varro:ling] \textbf{Varro ling.}\newline Varro\newline \emph{de lingua Latina}
\item[Varro:Men] \textbf{Varro Men.}\newline Varro\newline \emph{saturae Menippeae}
\item[Varro:rust] \textbf{Varro rust.}\newline Varro\newline \emph{res rusticae}
\item[Vell] \textbf{Vell. }\newline Velleius Paterculus\newline \emph{historiae Romanae}
\item[Ven:Fort] \textbf{Ven. Fort.}\newline Venantius Fortunatus\newline 
\item[Verg:Aen] \textbf{Verg. Aen.}\newline Vergilius\newline \emph{Aeneis}
\item[Verg:catal] \textbf{Verg. catal.}\newline Vergilius\newline \emph{catalepton}
\item[Verg:ecl] \textbf{Verg. ecl.}\newline Vergilius\newline \emph{eclogae}
\item[Verg:georg] \textbf{Verg. georg.}\newline Vergilius\newline \emph{georgica}
\item[Vitr] \textbf{Vitr.}\newline Vitruv\newline \emph{de architectura}
\item[Vulg] \textbf{ Vulg.}\newline \newline \emph{Vulgata}
\item[Xen:an] \textbf{Xen. an.}\newline Xenophon\newline \emph{anabasis}
\item[Xen:apol] \textbf{Xen. apol.}\newline Xenophon\newline \emph{apologia}
\item[Xen:Athpol] \textbf{Xen. Ath. pol.}\newline Xenophon\newline \emph{Athenaion politeia}
\item[Xen:hell] \textbf{Xen. hell.}\newline Xenophon\newline \emph{hellenica}
\item[Xen:kyn] \textbf{Xen. kyn.}\newline Xenophon\newline \emph{cynegeticus}
\item[Xen:Kyr] \textbf{Xen. Kyr.}\newline Xenophon\newline \emph{Cyrupaideia}
\item[Xen:mem] \textbf{Xen. mem.}\newline Xenophon\newline \emph{memorabilia (Apomnemoneumata)}
\item[Xen:oec] \textbf{Xen. oec.}\newline Xenophon\newline \emph{oeconomicus}
\item[Xen:symp] \textbf{Xen. symp.}\newline Xenophon\newline \emph{symposium}
\item[Xenophan] \textbf{Xenophan.}\newline Xenophanes\newline 
\item[Zen] \textbf{Zen.}\newline Zenon\newline 
\item[Zenob] \textbf{Zenob.}\newline Zenobios\newline 
\item[Zenod] \textbf{Zenod.}\newline Zenodotos\newline 
\item[Zos] \textbf{Zos.}\newline Zosimos\newline 


\end{description}
\end{footnotesize}

\end{multicols}
 \changes{v1.5}{2016/05/31}{Antike Bibliographie}


\section{Additional bibliography with corpora}\label{list-bibcorpora}
List of corpora in ancient studies |archaeologie-bibcorpora.bib|.
This activates the other additional bibliography |archaeologie-lstabbrv.bib| automatically.
The bold entry on the left is the |bibtex-key|.\footnote{If you think the list should be enlarged, let us know the entries.}
All the entries have |keywords={corpus}|, |options={corpus}|.
\begin{multicols}{2}
  \begin{footnotesize}
\begin{description}[%
			%	style=multiline,
				style=nextline,
				leftmargin=1.5cm,
				font=\normalfont]
\item[ABV] Attic Black-figure Vase-painters
\item[AE] L'année épigraphique 
\item[AHw] Akkadisches Handwörterbuch
\item[ARV2] Attic Red-figure Vase-painters
\item[CAD] The Assyrian Dictionary of the Oriental Institute of the University of Chicago 
\item[CIL] Corpus inscriptionum Latinarum 
\item[DACL] Dictionnaire d'archéologie chrétienne et de liturgie 
\item[Daremberg-Saglio] Dictionnaire des antiquités grecques et romaines d'après les textes et les monuments. Ouvrage rédigé sous la direction de Ch. Daremberg et E. Saglio 
\item[DNP] Der Neue Pauly. Enzyklopädie der Antike 
\item[EAA] Enciclopedia dell'arte antica classica e orientale 
\item[FGrHist] Die Fragmente der griechischen Historiker
\item[FHG] Fragmenta historicorum Graecorum 
\item[FR] A. Furtwängler – K. Reichhold, Griechische Vasenmalerei (München 1900--1925) 
\item[HAW] Handbuch der Altertumswissenschaften 
\item[HdArch] Handbuch der Archäologie 
\item[Head] B. V. Head, Historia Numorum. A Manual of Greek Numismatics (Oxford 1887; 1911)
\item[Helbig] W. Helbig, Führer durch die öffentlichen Sammlungen klassischer Altertümer in Rom 
\item[IG] Inscriptiones Graecae 
\item[IGR] Inscriptiones Graecae ad res Romanas pertinentes 
\item[IK] Inschriften griechischer Städte aus Kleinasien
\item[ILS] Inscriptiones Latinae selectae
\item[LAe] Lexikon der Ägyptologie
\item[LIMC] Lexikon iconographicum mythologiae classicae
\item[LSJ] G. Liddell – R. Scott – H. S. Jones, A Greek-English Lexikon \textsuperscript{9}(1996); Suppl. (1996)
\item[LTUR] Lexikon topographicum urbis Romae 
\item[PIR] Prosopographia Imperii Romani 
\item[PPM] Pompei: Pitture e mosaici. Enciclopedia dell’arte antica classica e orientale 
\item[PPP] Pitture e Pavimenti di Pompei
\item[RAC] Reallexikon für Antike und Christentum 
\item[RBK] Reallexikon zur byzantinischen Kunst
\item[RE] Paulys Realencyclopädie der classischen Altertumswissenschaft 
\item[RES] Répertoire d’épigraphie sémitique (Paris 1900--1950) 
\item[RIA] Rivista dell’Istituto nazionale d’archeologia e storia dell’arte
\item[RIC] H. Mattingly – E. A. Sydenham, The Roman Imperial Coinage 
\item[RoscherML] W. H. Roscher, Ausführliches Lexikon der griechischen und römischen Mythologie
\item[RPC] Roman Provincial Coinage
\item[RRC] M. Crawford, Roman Republican Coinage (London 1974) 
\item[SEG] Supplementum epigraphicum Graecum 
\item[SIG] Sylloge inscriptionum Graecarum
\item[SNG] Sylloge nummorum Graecorum 
\item[TAM] Tituli Asiae Minoris
\item[ThesCRA] Thesaurus Cultus et Rituum Antiquorum
\item[Thieme-Becker] U. Thieme – F. Becker (Hrsg.), Allgemeines Lexikon der bildenden Künstler
\item[TIB] Tabula Imperii Byzantini
\end{description}
\end{footnotesize}
\end{multicols}
 \changes{v1.5}{2016/05/31}{Antike Bibliographie}

\section{List of locations}\label{list-locations}
\DescribeMacro{lstlocations}If you use the option |lstlocations| it will load an additional bibliography called |archaeologie-lstlocations.bib|.\footnote{If you think the list should be enlarged, let us know the entries.} 
Below you find a list with the available locations in five (at maximum) different languages.
The bold entry on the left is to use for |location=|\meta{location} -- do not put \meta{location} into |{}|.
In these examples we used such |@Strings|:
\cref{Mundt2015,Emme2013,Neufert2002,Wulf-Rheidt2013,Cic:Sest,Fest,CIL,Mann2011,Zanker2009,%
MacDonald1986,Kohlmeyer1983,Parlasca1969,Kurapkat2014,Torelli1991,%
Lexikon-der-Technik,Bergmann2015,LTUR}


\begin{multicols}{3}
\begin{description}\footnotesize
\item[Aix-la-Chapelle] <-- English\newline German: Aachen\newline Italian: \newline Spanish: Aquisgrán\newline French: Aix-la-Chapelle
\item[Athens] <-- English\newline German: Athen\newline Italian: Atene\newline Spanish: Atenas\newline French: Athènes
\item[Augsburg] <-- English\newline German: Augsburg\newline Italian: Augusta\newline Spanish: Ausburgo\newline French: Augsbourg
\item[Basle] <-- English\newline German: Basel\newline Italian: Basilea\newline Spanish: Basilea\newline French: Basel
\item[Berlin] <-- English\newline German: Berlin\newline Italian: Berlino\newline Spanish: Berlín\newline French: Berlin
\item[Brussels] <-- English\newline German: Brüssel\newline Italian: Bruxelles\newline Spanish: Bruselas\newline French: Bruxelles
\item[Cologne] <-- English\newline German: Köln\newline Italian: Colonia\newline Spanish: Colonia\newline French: Cologne
\item[Copenhagen] <-- English\newline German: Kopenhagen\newline Italian: Copenaghen\newline Spanish: Copenhague\newline French: Copenhague
\item[Dresden] <-- English\newline German: Dresden\newline Italian: Desda\newline Spanish: Dresde\newline French: Dresde
\item[Florence] <-- English\newline German: Florenz\newline Italian: Firenze\newline Spanish: Firenze\newline French: Florence
\item[Frankfurt] <-- English\newline German: Frankfurt am Main\newline Italian: Francoforte sul Meno\newline Spanish: Francfort del Meno\newline French: Francfort-sur-le-Main
\item[Freiburg] <-- English\newline German: Freiburg \parentext{i. Breisgau}\newline Italian: Friburgo in Brisgovia\newline Spanish: Friburgo de Brisgovia\newline French: Fribourg-en-Brisgau
\item[Göttingen] <-- English\newline German: Göttingen\newline Italian: Gottinga\newline Spanish: Gotinga\newline French: Gœttingue
\item[Hamburg] <-- English\newline German: Hamburg\newline Italian: Amburgo\newline Spanish: Hamburgo\newline French: Hambourg
\item[Leipzig] <-- English\newline German: Leipzig\newline Italian: Lipsia\newline Spanish: Leipzig\newline French: Leipzig
\item[London] <-- English\newline German: London\newline Italian: Londra\newline Spanish: Londres\newline French: Londres
\item[Louvain] <-- English\newline German: Löwen\newline Italian: Lovanio\newline Spanish: Lovaina\newline French: Louvain
\item[Mainz] <-- English\newline German: Mainz am Rhein\newline Italian: Magonza\newline Spanish: Maguncia\newline French: Mayence
\item[Milano] <-- English\newline German: Mailand\newline Italian: Milano\newline Spanish: Milán\newline French: Milan
\item[Munich] <-- English\newline German: München\newline Italian: Monaco \parentext{di Bavaria}\newline Spanish: Múnich\newline French: Munich
\item[Naples] <-- English\newline German: Neapel\newline Italian: Napoli\newline Spanish: Napoli\newline French: Naples
\item[Paris] <-- English\newline German: Paris\newline Italian: Parigi\newline Spanish: París\newline French: Paris
\item[Regensburg] <-- English\newline German: Regensburg\newline Italian: Ratisbona\newline Spanish: Ratisbona\newline French: Ratisbonne
\item[Rome] <-- English\newline German: Rom\newline Italian: Roma\newline Spanish: Roma\newline French: Rome
\item[Saarbrucken] <-- English\newline German: Saarbrücken\newline Italian: \newline Spanish: Saarbruck\newline French: Saarbruck
\item[Stuttgart] <-- English\newline German: Stuttgart\newline Italian: Stoccardo\newline Spanish: Estútgart\newline French: Stuttgart
\item[Trier] <-- English\newline German: Trier\newline Italian: Treviri\newline Spanish: Trèveris\newline French: Trèves
\item[Tuebingen] <-- English\newline German: Tübingen\newline Italian: Tubinga\newline Spanish: Tubinga\newline French: Tubingue
\item[Vienna] <-- English\newline German: Wien\newline Italian: Vienna\newline Spanish: Viena\newline French: Vienne
\end{description}
\end{multicols}
\section{List of publishers}\label{list-publishers}
Below there is a list with\DescribeMacro{@String} |@String| (in \textbf{bold} letters on the left), which you can use for the field |publisher|.\footnote{If you think the list should be enlarged, let us know the entries.}
In these examples we used such |@String|:
\cref{Mundt2015,Quint:inst,Emme2013,Neufert2002,Wulf-Rheidt2013,Cic:Att,Cic:Sest,Zanker2009,Zanker1988,MacDonald1986,Carter2014,Fentress2003,Welch2007,Sear2006}

\begin{multicols}{2}
\begin{description}\footnotesize
\item[AWi] Artemis \& Winkler
\item[CHB] C.\ H.~Beck
\item[COUP] Cornell University Press
\item[CUP] Cambridge University Press
\item[EdB] L'erma di Bretschneider
\item[EQ] Edizioni Quasar
\item[FZ] Franz Steiner
\item[GLF] Gius. Laterza \& Figli Spa
\item[HUP] Harvard University Press
\item[JHUP] Johns Hopkins University Press
\item[JPGM] J. Paul Getty Museum
\item[MI] Michael Imhof
\item[MIT] MIT Press
\item[OUP] Oxford University Press
\item[PUP] Princeton University Press
\item[PSUP] Pennsylvania State University Press
\item[PvZ] Philip von Zabern
\item[stw] suhrkamp taschenbuch wissenschaft
\item[TopoiB] Topoi. Berliner Studien der Alten Welt
\item[UCP] University of California Press
\item[UMP] University of Michigan Press
\item[UTP] University of Texas Press
\item[UWP] University of Wisconsin Press
\item[VML] Verlag Marie Leidorf
\item[VR] Vandenhoeck \& Ruprecht
\item[VS] Verlag für Sozialwissenschaften
\item[VT] Vieweg+Teubner
\item[WBG] Wissenschaftliche Buchgesellschaft
\item[WdG] Walter de Gruyter
\item[YUP] Yale University Press
\end{description}
\end{multicols}
\section{List of abbreviation according to the \DAI-guidelines}\label{abbrv-lists}
We modified the lists with abbreviations of journals, corpora, etc. to make them usable with |bib|\LaTeX.
Below there are two lists with\DescribeMacro{@String} |@String| (in \textbf{bold} letters on the left), 
one with the abbreviations (\cref{liste-kurz}), the other with the long forms  (\cref{liste-lang}).

We recommend  looking up the journal names in the list and inserting the |@String| in the fields  |journaltitle| and |shortjournal|, or |series| and |shortseries|.


\subsection{Short form} \label{liste-kurz}
\begin{multicols}{2}
\begin{footnotesize}
\begin{description}[%
			%	style=multiline,
				style=nextline,
				leftmargin=3cm,
				font=\normalfont\bfseries]
%\begin{description}
 \item[AA-short] AA 
 \item[AAA-short] AAA 
 \item[AAcque-short] AAcque 
 \item[AAdv-short] AAdv 
 \item[AAJ-short] AAJ 
 \item[AAlpi-short] AAlpi 
 \item[AarbKob-short] AarbKøb %*Abweichung!
 \item[AArchit-short] AArchit 
 \item[AAS-short] AAS 
 \item[AASOR-short] AASOR 
 \item[AAusgrBadWuert-short] AAusgrBadWürt %*Abweichung!
 \item[AAustr-short] AAustr 
 \item[ABADY-short] ABADY 
 \item[AbhBerlin-short] AbhBerlin 
 \item[AbhDuesseldorf-short] AbhDüsseldorf %*Abweichung!
 \item[AbhGoettingen-short] AbhGöttingen %*Abweichung!
 \item[AbhLeipzig-short] AbhLeipzig 
 \item[AbhMainz-short] AbhMainz 
 \item[AbhMuenchen-short] AbhMünchen %*Abweichung!
 \item[ABret-short] ABret 
 \item[Abr-Nahrain-short] Abr-Nahrain 
 \item[ABulg-short] ABulg 
 \item[ABV-short] ABV 
 \item[ACalc-short] ACalc 
 \item[ACamp-short] ACamp 
 \item[ACant-short] ACant 
 \item[AcBibl-short] AcBibl 
 \item[Achse-short] Achse 
 \item[Acme-short] Acme 
 \item[Acontia-short] Acontia 
 \item[ACors-short] ACors 
 \item[ActaAArtHist-short] ActaAArtHist 
 \item[ActaAArtHist-sa-short] ActaAArtHist s.a. %*Abweichung!
 \item[ActaAcAbo-short] ActaAcAbo 
 \item[ActaACarp-short] ActaACarp 
 \item[ActaALov-short] ActaALov 
 \item[ActaALovMono-short] ActaALovMono 
 \item[ActaAntHung-short] ActaAntHung 
 \item[ActaArch-short] ActaArch 
 \item[ActaArchHung-short] ActaArchHung 
 \item[ActaAth-short] ActaAth 
 \item[ActaCl-short] ActaCl 
 \item[ActaClDebrec-short] ActaClDebrec 
 \item[ActaHistDac-short] ActaHistDac 
 \item[ActaHyp-short] ActaHyp 
 \item[ActaInstRomFin-short] ActaInstRomFin 
 \item[ActaMusNapoca-short] ActaMusNapoca 
 \item[ActaMusPorol-short] ActaMusPorol 
 \item[ActaNum-short] ActaNum 
 \item[ActaOr-short] ActaOr 
 \item[ActaOrHung-short] ActaOrHung 
 \item[ActaPhilSocDac-short] ActaPhilSocDac 
 \item[ActaPraehistA-short] ActaPraehistA 
 \item[ActaTorunA-short] ActaTorunA 
 \item[ActaTorunHist-short] ActaTorunHist 
 \item[AD-short] AD 
 \item[ADAIK-short] ADAIK 
 \item[Adalya-short] Adalya 
 \item[ADelt-A-short] ADelt A %*Abweichung!
 \item[ADelt-B-short] ADelt B %*Abweichung!
 \item[ADerg-short] ADerg 
 \item[ADFU-short] ADFU 
 \item[AdI-short] AdI 
 \item[ADOG-short] ADOG 
 \item[Adumatu-short] Adumatu 
 \item[AE-short] AE 
 \item[AeA-short] AeA 
 \item[Aegaeum-short] Aegaeum 
 \item[AegLev-short] ÄgLev %*Abweichung!
 \item[AEmil-short] AEmil 
 \item[AEphem-short] AEphem 
 \item[AeR-short] AeR 
 \item[AErgoMak-short] AErgoMak 
 \item[AErt-short] AErt 
 \item[AEspA-short] AEspA 
 \item[Aevum-short] Aevum 
 \item[AevumAnt-short] AevumAnt 
 \item[AF-short] AF 
 \item[AfO-short] AfO 
 \item[Africa-short] Africa 
 \item[AGD-short] AGD 
 \item[AGeo-short] AGeo 
 \item[Agora-short] Agora 
 \item[AgoraPB-short] AgoraPB 
 \item[AHist-short] AHist 
 \item[AHistStAlex-short] AHistStAlex 
 \item[AHw-short] AHw 
 \item[AiD-short] AiD 
 \item[AInf-short] AInf 
 \item[AIONArch-short] AIONArch 
 \item[AIONFil-short] AIONFil 
 \item[AIONLing-short] AIONLing 
 \item[AIPhOr-short] AIPhOr 
 \item[Aitna-short] Aitna 
 \item[AJA-short] AJA 
 \item[AJahrBay-short] AJahrBay 
 \item[AJPh-short] AJPh 
 \item[AJug-short] AJug 
 \item[AKorrBl-short] AKorrBl 
 \item[AlbaRegia-short] AlbaRegia 
 \item[AlmaMaterSt-short] AlmaMaterSt 
 \item[AlmanachWien-short] AlmanachWien 
 \item[AlonJisrael-short] AlonJisrael 
 \item[Al-Qannis-short] Al-Qanniš
 \item[Altamura-short] Altamura 
 \item[AltoMed-short] AltoMed 
 \item[Alt-Paphos-short] Alt-Paphos 
 \item[AltThuer-short] AltThür %*Abweichung!
 \item[AM-short] AM 
 \item[AMediev-short] AMediev 
 \item[AMethTh-short] AMethTh 
 \item[AMI-short] AMI 
 \item[AMIT-short] AMIT 
 \item[AmJAncHist-short] AmJAncHist 
 \item[AmJNum-short] AmJNum 
 \item[AMold-short] AMold 
 \item[AMosel-short] AMosel 
 \item[AMS-short] AMS 
 \item[AmStP-short] AmStP 
 \item[AMuGS-short] AMuGS 
 \item[ANachr-short] ANachr 
 \item[ANachrBad-short] ANachrBad 
 \item[AnadoluAra-short] AnadoluAra 
 \item[AnadoluKonf-short] AnadoluKonf 
 \item[AnadoluYil-short] AnadoluYıl 
 \item[AnAe-short] AnAe 
 \item[Anagennesis-short] Anagennesis 
 \item[AnalBolland-short] AnalBolland 
 \item[AnalP-short] AnalP 
 \item[AnalRom-short] AnalRom 
 \item[AnArqAnd-short] AnArqAnd 
 \item[Anas-short] Anas 
 \item[AnatA-short] AnatA 
 \item[Anatolia-short] Anatolia 
 \item[ANaturwiss-short] ANaturwiss 
 \item[AncCivScytSib-short] AncCivScytSib 
 \item[AncHistB-short] AncHistB 
 \item[AncInd-short] AncInd 
 \item[AncNearEastSt-short] AncNearEastSt 
 \item[AnCord-short] AnCord 
 \item[AncSoc-short] AncSoc 
 \item[AncW-short] AncW 
 \item[AncWestEast-short] AncWestEast 
 \item[AnDubr-short] AnDubr 
 \item[ANews-short] ANews 
 \item[ANilMoy-short] ANilMoy 
 \item[ANL-short] ANL 
 \item[AnMunFaro-short] AnMunFaro 
 \item[AnMurcia-short] AnMurcia 
 \item[AnnAcEtr-short] AnnAcEtr 
 \item[AnnAcTorino-short] AnnAcTorino 
 \item[AnnAStorAnt-short] AnnAStorAnt 
 \item[AnnBari-short] AnnBari 
 \item[AnnBenac-short] AnnBenac 
 \item[AnnBiblAModena-short] AnnBiblAModena 
 \item[AnnBiblARom-short] AnnBiblARom 
 \item[AnnByzConf-short] AnnByzConf 
 \item[AnnCagl-short] AnnCagl 
 \item[AnnCaglMag-short] AnnCaglMag 
 \item[AnnEconSocCiv-short] AnnEconSocCiv 
 \item[AnnEgBibl-short] AnnEgBibl 
 \item[AnnEth-short] AnnEth 
 \item[AnnFaina-short] AnnFaina 
 \item[AnnHistA-short] AnnHistA 
 \item[AnnHistScSoc-short] AnnHistScSoc 
 \item[AnnIstGiapp-short] AnnIstGiapp 
 \item[AnnIstItNum-short] AnnIstItNum 
 \item[AnnLecce-short] AnnLecce 
 \item[AnnLeedsUnOrSoc-short] AnnLeedsUnOrSoc 
 \item[AnnMacerata-short] AnnMacerata 
 \item[AnnMessMag-short] AnnMessMag 
 \item[AnnMusRov-short] AnnMusRov 
 \item[AnnNap-short] AnnNap 
 \item[AnnNivern-short] AnnNivern 
 \item[AnnNoment-short] AnnNoment 
 \item[AnnOrNap-short] AnnOrNap 
 \item[AnnotNum-short] AnnotNum 
 \item[AnnPerugia-short] AnnPerugia 
 \item[AnnPisa-short] AnnPisa 
 \item[AnnPontAcRom-short] AnnPontAcRom 
 \item[AnnRepBSA-short] AnnRepBSA 
 \item[AnnRepCypr-short] AnnRepCypr 
 \item[AnnRepFoggArtMus-short] AnnRepFoggArtMus 
 \item[AnnSiena-short] AnnSiena 
 \item[AnnuarioAcLinc-short] AnnuarioAcLinc 
 \item[AnnuarioLecce-short] AnnuarioLecce 
 \item[AnnUnBud-short] AnnUnBud 
 \item[AnnWorcArtMus-short] AnnWorcArtMus 
 \item[Anodos-short] Anodos 
 \item[AnOr-short] AnOr 
 \item[ANRW-short] ANRW 
 \item[Anschnitt-short] Anschnitt 
 \item[ANSMusNotes-short] ANSMusNotes 
 \item[AnSt-short] AnSt 
 \item[Antaeus-short] Antaeus 
 \item[AntAfr-short] AntAfr 
 \item[AntChr-short] AntChr 
 \item[AntCl-short] AntCl 
 \item[AnthrAChron-short] AnthrAChron 
 \item[Anthropos-short] Anthropos 
 \item[Antichthon-short] Antichthon 
 \item[AntigCr-short] AntigCr 
 \item[Antipolis-short] Antipolis 
 \item[Antiqua-short] Antiqua 
 \item[Antiquity-short] Antiquity 
 \item[AntJ-short] AntJ 
 \item[AntK-short] AntK 
 \item[AntNat-short] AntNat 
 \item[AntPisa-short] AntPisa 
 \item[AntPl-short] AntPl 
 \item[AntSurv-short] AntSurv 
 \item[AntTard-short] AntTard 
 \item[AnzAW-short] AnzAW 
 \item[AnzWien-short] AnzWien 
 \item[AOAT-short] AOAT 
 \item[AoF-short] AoF 
 \item[AOtkryt-short] AOtkryt 
 \item[APamKiiv-short] APamKiiv 
 \item[APh-short] APh 
 \item[APol-short] APol 
 \item[Apollo-short] Apollo 
 \item[ApolloLond-short] ApolloLond 
 \item[APort-short] APort 
 \item[AppRomFil-short] AppRomFil 
 \item[APregl-short] APregl 
 \item[Apulum-short] Apulum 
 \item[AquiLeg-short] AquiLeg 
 \item[AquilNost-short] AquilNost 
 \item[Aquitania-short] Aquitania 
 \item[ArabAEpigr-short] ArabAEpigr 
 \item[ARadRaspr-short] ARadRaspr 
 \item[ArbFBerSaechs-short] ArbFBerSaechs 
 \item[Archaeographie-short] Archäographie %*Abweichung!
 \item[Archaeologia-short] Archäographie 
 \item[Archaeology-short] Archaeology 
 \item[Archaeometry-short] Archaeometry 
 \item[Archaiognosia-short] Archaiognosia 
 \item[ArchBegriffsGesch-short] ArchBegriffsGesch 
 \item[ArchByzMnem-short] ArchByzMnem 
 \item[ArchCl-short] ArchCl 
 \item[Archeo-short] Archeo 
 \item[ArcheogrTriest-short] ArcheogrTriest 
 \item[ArcheologiaParis-short] ArcheologiaParis 
 \item[ArcheologiaRoma-short] ArcheologiaRoma 
 \item[ArcheologiaWarsz-short] ArcheologiaWarsz 
 \item[ArcheologijaKiiv-short] ArcheologijaKiiv 
 \item[ArcheologijaSof-short] ArcheologijaSof 
 \item[ArchEubMel-short] ArchEubMel 
 \item[ArchHom-short] ArchHom 
 \item[Architectura-short] Architectura 
 \item[Archivi-short] Archivi 
 \item[ArchPF-short] ArchPF 
 \item[ArchPrehistLev-short] ArchPrehistLev 
 \item[ArchRel-short] ArchRel 
 \item[ArchStorCal-short] ArchStorCal 
 \item[ArchStorPugl-short] ArchStorPugl 
 \item[ArchStorRom-short] ArchStorRom 
 \item[ArchStorSicOr-short] ArchStorSicOr 
 \item[ArchStorSir-short] ArchStorSir 
 \item[Arctos-short] Arctos 
 \item[ARepLond-short] ARepLond 
 \item[Argo-short] Argo 
 \item[ArOr-short] ArOr 
 \item[ArOrMono-short] ArOrMono 
 \item[ArOrSuppl-short] ArOrSuppl 
 \item[ARozhl-short] ARozhl 
 \item[ArqBeja-short] ArqBeja 
 \item[Arse-short] Arse 
 \item[ArsGeorg-short] ArsGeorg 
 \item[ArtAntMod-short] ArtAntMod 
 \item[ArtARhone-short] ArtARhône %*Abweichung!
 \item[ArtAs-short] ArtAs 
 \item[ArtB-short] ArtB 
 \item[ArtJ-short] ArtJ 
 \item[ArtLomb-short] ArtLomb 
 \item[ArtMediev-short] ArtMediev 
 \item[ArtVirg-short] ArtVirg 
 \item[ARV2-short] ARV2 
 \item[ASachs-short] ASachs 
 \item[ASAE-short] ASAE 
 \item[ASammlUnZuerch-short] ASammlUnZürch %*Abweichung!
 \item[ASAtene-short] ASAtene 
 \item[ASbor-short] ASbor 
 \item[ASchw-short] ASchw 
 \item[ASoc-short] ASoc 
 \item[ASoloth-short] ASoloth 
 \item[ASR-short] ASR 
 \item[Assaph-short] Assaph 
 \item[AssyrMisc-short] AssyrMisc 
 \item[AST-short] AST 
 \item[ASub-short] ASub 
 \item[ASubacq-short] ASubacq 
 \item[Athenaeum-short] Athenaeum 
 \item[Atiqot-short] Atiqot 
 \item[AtiqotHeb-short] AtiqotHeb 
 \item[Atlal-short] Atlal 
 \item[AttiAcPontan-short] AttiAcPontan 
 \item[AttiAcRov-short] AttiAcRov 
 \item[AttiAcTorino-short] AttiAcTorino 
 \item[AttiCAntCl-short] AttiCAntCl 
 \item[AttiCItRom-short] AttiCItRom 
 \item[AttiMemBologna-short] AttiMemBologna 
 \item[AttiMemDal-short] AttiMemDal 
 \item[AttiMemFirenze-short] AttiMemFirenze 
 \item[AttiMemIstria-short] AttiMemIstria 
 \item[AttiMemMagnaGr-short] AttiMemMagnaGr 
 \item[AttiMemModena-short] AttiMemModena 
 \item[AttiMemTivoli-short] AttiMemTivoli 
 \item[AttiMusTrieste-short] AttiMusTrieste 
 \item[AttiPalermo-short] AttiPalermo 
 \item[AttiRovigno-short] AttiRovigno 
 \item[AttiSocFriuli-short] AttiSocFriuli 
 \item[AttiVenezia-short] AttiVenezia 
 \item[AuA-short] AuA 
 \item[AulaOr-short] AulaOr 
 \item[AusgrFu-short] AusgrFu 
 \item[AusgrFuWestf-short] AusgrFuWestf 
 \item[AustrRom-short] AustrRom 
 \item[AUTerr-short] AUTerr 
 \item[AUWE-short] AUWE 
 \item[AV-short] AV 
 \item[AVen-short] AVen 
 \item[AVes-short] AVes 
 \item[AViva-short] AViva 
 \item[AvP-short] AvP 
 \item[AW-short] AW 
 \item[AyasofyaMuezYil-short] AyasofyaMüzYıl \label{AyasofyaMuezYil-kurz} %*Abweichung!
 \item[AZ-short] AZ 
 \item[Azotea-short] Azotea 
 \item[BA-short] BA 
 \item[Baalbek-short] Baalbek 
 \item[BAAlger-short] BAAlger 
 \item[BABarcel-short] BABarcel 
 \item[BABesch-short] BABesch 
 \item[BAcRHist-short] BAcRHist 
 \item[BACopt-short] BACopt 
 \item[BadFuBer-short] BadFuBer 
 \item[Baetica-short] Baetica 
 \item[BaF-short] BaF 
 \item[BalacaiKoez-short] BalacaiKöz %*Abweichung!
 \item[BalkSt-short] BalkSt 
 \item[BALond-short] BALond 
 \item[BALux-short] BALux 
 \item[BaM-short] BaM 
 \item[BAMaroc-short] BAMaroc 
 \item[BAmSocP-short] BAmSocP 
 \item[BAncOrMus-short] BAncOrMus 
 \item[BAngers-short] BAngers 
 \item[BAngloIsrASoc-short] BAngloIsrASoc 
 \item[BAnnMusFerr-short] BAnnMusFerr 
 \item[BAntFr-short] BAntFr 
 \item[BAntLux-short] BAntLux 
 \item[BAParis-short] BAParis 
 \item[BAProv-short] BAProv 
 \item[BAR-short] BAR 
 \item[BArchAlex-short] BArchAlex 
 \item[BArchit-short] BArchit 
 \item[BARIntSer-short] BARIntSer 
 \item[BASard-short] BASard 
 \item[BAsEspA-short] BAsEspA 
 \item[BAsInst-short] BAsInst 
 \item[BASOR-short] BASOR 
 \item[BAssBude-short] BAssBudé %*Abweichung!
 \item[BAssMosAnt-short] BAssMosAnt 
 \item[BASub-short] BASub 
 \item[BASudEstEur-short] BASudEstEur 
 \item[BATarr-short] BATarr 
 \item[BAur-short] BAur 
 \item[BAVA-short] BAVA 
 \item[BayVgBl-short] BayVgBl 
 \item[BBasil-short] BBasil 
 \item[BBelgRom-short] BBelgRom 
 \item[BBolsena-short] BBolsena 
 \item[BBrByzSt-short] BBrByzSt 
 \item[BCamuno-short] BCamuno 
 \item[BCASic-short] BCASic 
 \item[BCercleNum-short] BCercleNum 
 \item[BCH-short] BCH 
 \item[BCircNumNap-short] BCircNumNap 
 \item[BCl-short] BCl 
 \item[BClevMus-short] BClevMus 
 \item[BCom-short] BCom 
 \item[BCord-short] BCord 
 \item[BdA-short] BdA 
 \item[BdE-short] BdE 
 \item[BdEC-short] BdEC 
 \item[BdI-short] BdI 
 \item[BDirRom-short] BDirRom 
 \item[BeazleyAddenda2-short] Beazley Addenda\textsuperscript{2}%*Abweichung!
 \item[BeazleyPara-short] Beazley, Para. %*Abweichung!
 \item[BEcAntNimes-short] BEcAntNîmes %*Abweichung!
 \item[BediKart-short] BediKart 
 \item[BEFAR-short] BEFAR 
 \item[BeitrESkAr-short] BeitrESkAr 
 \item[BeitrNamF-short] BeitrNamF 
 \item[BeitrSudanF-short] BeitrSudanF 
 \item[BelArt-short] BelArt 
 \item[Belleten-short] Belleten 
 \item[Benacus-short] Benacus 
 \item[BerBayDenkmPfl-short] BerBayDenkmPfl 
 \item[BerDFG-short] BerDFG 
 \item[BerlBeitrArchaeom-short] BerlBeitrArchäom %*Abweichung!
 \item[BerlBlVFruehGesch-short] BerlBlVFrühGesch %*Abweichung!
 \item[BerlJbVFruehGesch-short] BerlJbVFrühGesch %*Abweichung!
 \item[BerlMus-short] BerlMus 
 \item[BerlNumZ-short] BerlNumZ 
 \item[BerOudhBod-short] BerOudhBod 
 \item[BerRGK-short] BerRGK 
 \item[BerVerhLeipz-short] BerVerhLeipz 
 \item[Berytus-short] Berytus 
 \item[BEspA-short] BEspA 
 \item[BEspOr-short] BEspOr 
 \item[BEtOr-short] BEtOr 
 \item[BFilGrPadova-short] BFilGrPadova 
 \item[BFilLingSic-short] BFilLingSic 
 \item[BFlegr-short] BFlegr 
 \item[BFoligno-short] BFoligno 
 \item[BHarvMus-short] BHarvMus 
 \item[BIasos-short] BIasos 
 \item[BibAr-short] BibAr 
 \item[BiblClOr-short] BiblClOr 
 \item[BiblSymb-short] BiblSymb 
 \item[BIBulg-short] BIBulg 
 \item[BICS-short] BICS 
 \item[BIFAO-short] BIFAO 
 \item[BInfCentumcellae-short] BInfCentumcellae 
 \item[BInfCESDAE-short] BInfCESDAE 
 \item[BiogrZbor-short] BiogrZbor 
 \item[BiOr-short] BiOr 
 \item[BIstOrvieto-short] BIstOrvieto 
 \item[BJaen-short] BJaen %*Abweichung!
 \item[BJb-short] BJb 
 \item[BJerus-short] BJerus 
 \item[BLaborMusLouvre-short] BLaborMusLouvre 
 \item[BLazioMerid-short] BLazioMerid 
 \item[BLikUm-short] BLikUm 
 \item[BlMueFreundeF-short] BlMüFreundeF %*Abweichung!
 \item[BLugo-short] BLugo 
 \item[BMCGreekCoins-short] BMC Greek Coins %*Abweichung!
 \item[BMCOR-short] BMCOR 
 \item[BMCRE-short] BMCRE 
 \item[BMCRRI-III-short] BMCRR I--III %*Abweichung!
 \item[BMetrMus-short] BMetrMus 
 \item[BMon-short] BMon 
 \item[BMonMusPont-short] BMonMusPont 
 \item[BMQ-short] BMQ 
 \item[BMQNSuppl-short] BMQNSuppl 
 \item[BMusBeyrouth-short] BMusBeyrouth 
 \item[BMusBrux-short] BMusBruxs 
 \item[BMusCadiz-short] BMusCadiz 
 \item[BMusCivRom-short] BMusCivRom 
 \item[BMusFA-short] BMusFA 
 \item[BMusHongr-short] BMusHongr 
 \item[BMusMadr-short] BMusMadr 
 \item[BMusMich-short] BMusMich 
 \item[BMusMonaco-short] BMusMonaco 
 \item[BMusPadova-short] BMusPadova 
 \item[BMusPBelArt-short] BMusPBelArt 
 \item[BMusRom-short] BMusRom 
 \item[BMusVars-short] BMusVars 
 \item[BMusZaragoza-short] BMusZaragoza 
 \item[BNumParis-short] BNumParis 
 \item[BNumRoma-short] BNumRoma 
 \item[Bogazkoey-Hattusa-short] Boğazköy-Hattuša %*Abweichung!
 \item[Bolskan-short] Bolskan 
 \item[BonnHVg-short] BonnHVg 
 \item[BOntMus-short] BOntMus 
 \item[Boreas-short] Boreas 
 \item[BoreasUpps-short] BoreasUpps 
 \item[BPeintRom-short] BPeintRom 
 \item[BPI-short] BPI 
 \item[BPrehistAlp-short] BPI %*Abweichung!
 \item[BProAvent-short] BProAvent 
 \item[BProvidence-short] BProvidence 
 \item[BracAug-short] BracAug 
 \item[BracaraAugusta-short] BracaraAugusta 
 \item[BremABl-short] BremABl 
 \item[BRest-short] BRest 
 \item[Brigantium-short] Brigantium 
 \item[BrMusYearbook-short] BrMusYearbook 
 \item[BSA-short] BSA 
 \item[BSAA-short] BSAA 
 \item[BSFE-short] BSFE 
 \item[BSiena-short] BSiena 
 \item[BSOAS-short] BSOAS 
 \item[BSocAChamp-short] BSocAChamp 
 \item[BSocBiblReinach-short] BSocBiblReinach 
 \item[BSocNumRom-short] BSocNumRom 
 \item[BSR-short] BSR 
 \item[BStLat-short] BStLat 
 \item[BStorArt-short] BStorArt 
 \item[BTextilAnc-short] BStorArt 
 \item[BTorino-short] BTorino 
 \item[BTravTun-short] BTravTun 
 \item[BudReg-short] BudReg 
 \item[BulletinGetty-short] BulletinGetty 
 \item[BulletinNorthampton-short] BulletinNorthampton 
 \item[BVallad-short] BVallad 
 \item[BVitoria-short] BVitoria 
 \item[BWaltersArtGal-short] BWaltersArtGal 
 \item[BWPr-short] BWPr 
 \item[Byzantina-short] Byzantina
 \item[ByzF-short] ByzF 
 \item[ByzJb-short] ByzJb 
 \item[ByzZ-short] ByzZ 
 \item[BZ-short] BZ 
 \item[CAA-short] CAA 
 \item[CAD-short] CAD 
 \item[CadA-short] CadA 
 \item[Caesaraugusta-short] Caesaraugusta 
 \item[Caesarodunum-short] Caesarodunum 
 \item[CAH-short] CAH 
 \item[CahArmeeRom-short] CahArmeeRom 
 \item[CahASubaqu-short] CahASubaqu 
 \item[CahByrsa-short] CahByrsa 
 \item[CahCEC-short] CahCEC 
 \item[CahCerEg-short] CahCerEg 
 \item[CahDelFrIran-short] CahDelFrIran 
 \item[CahGlotz-short] CahGlotz 
 \item[CahKarnak-short] CahKarnak 
 \item[CahLig-short] CahLig 
 \item[CahMariemont-short] CahMariemont 
 \item[CahMusChampollion-short] CahMusChampollion 
 \item[CahPEg-short] CahPEg 
 \item[CahRhod-short] CahRhod 
 \item[CahTun-short] CahTun 
 \item[CalifStClAnt-short] CalifStClAnt 
 \item[CambrAJ-short] CambrAJ 
 \item[CArch-short] CArch 
 \item[CarinthiaI-short] Carinthia I %*Abweichung!
 \item[CarnuntumJb-short] CarnuntumJb 
 \item[Carpica-short] Carpica 
 \item[Carrobbio-short] Carrobbio 
 \item[CAT-short] CAT 
 \item[CE-short] CE 
 \item[CEDAC-short] CEDAC 
 \item[CEFR-short] CEFR 
 \item[Celticum-short] Celticum 
 \item[CercA-short] CercA 
 \item[CercNum-short] CercNum 
 \item[Chiron-short] Chiron 
 \item[ChronEg-short] ChronEg 
 \item[CIA-short] CIA 
 \item[CIE-short] CIE 
 \item[CIG-short] CIG 
 \item[CIH-short] CIH 
 \item[CIL-short] CIL 
 \item[CincArtB-short] CincArtB 
 \item[CIS-short] CIS 
 \item[CIstAMilano-short] CIstAMilano 
 \item[CivClCr-short] CivClCr 
 \item[CivPad-short] CivPad 
 \item[ClAnt-short] ClAnt 
 \item[ClevStHistArt-short] ClevStHistArt 
 \item[Clio-short] Clio 
 \item[ClIre-short] ClIre 
 \item[ClJ-short] ClJ 
 \item[ClMediaev-short] ClMediaev 
 \item[ClPhil-short] ClPhil 
 \item[ClQ-short] ClQ 
 \item[ClR-short] ClR 
 \item[ClRh-short] ClRh 
 \item[CMatAOr-short] CMatAOr 
 \item[CMGr-short] CMGr 
 \item[CMS-short] CMS 
 \item[CoinHoards-short] Coin Hoards %*Abweichung!
 \item[ColloquiSod-short] ColloquiSod 
 \item[CommunicAHung-short] CommunicAHung 
 \item[Complutum-short] Complutum 
 \item[Conoscenze-short] Conoscenze 
 \item[Corduba-short] Corduba 
 \item[Corinth-short] Corinth 
 \item[CRAI-short] CRAI 
 \item[CretAnt-short] CretAnt 
 \item[CretSt-short] CretSt 
 \item[CronA-short] CronA 
 \item[CronErcol-short] CronErcol 
 \item[CronPomp-short] CronPomp 
 \item[CRPetersbourg-short] CRPetersbourg %*Abweichung!
 \item[CSE-short] CSE 
 \item[CSIR-short] CSIR 
 \item[CSSpecPisa-short] CSSpecPisa 
 \item[CuadAMed-short] CuadAMed 
 \item[CuadArquitRom-short] CuadArquitRom 
 \item[CuadCastellon-short] CuadCastellon 
 \item[CuadCat-short] CuadCat 
 \item[CuadFilCl-short] CuadFilCl 
 \item[CuadGallegos-short] CuadGallegos 
 \item[CuadGranada-short] CuadGranada 
 \item[CuadNavarra-short] CuadNavarra 
 \item[CuadPrehistA-short] CuadPrehistA 
 \item[CuadRom-short] CuadRom 
 \item[CuPaUAM-short] CuPaUAM 
 \item[CVA-short] CVA 
 \item[CZero-short] CZero 
 \item[DAA-short] DAA 
 \item[Dacia-short] Dacia 
 \item[DACL-short] DACL 
 \item[Dacoromania-short] Dacoromania 
 \item[DaF-short] DaF 
 \item[Daidalos-short] Daidalos 
 \item[DAIGeschDok-short] DAIGeschDok 
 \item[DaM-short] DaM 
 \item[Daremberg-Saglio-short] Daremberg -- Saglio %*Abweichung!
 \item[DebrecMuzEvk-short] DebrecMuzEvk 
 \item[Dedalo-short] Dédalo %*Abweichung!
 \item[Delos-short] Delos %*Abweichung!
 \item[DeltChrA-short] DeltChrA 
 \item[Demircihueyuek-short] Demircihüyük %*Abweichung!
 \item[DeMuseus-short] DeMuseus 
 \item[DenkmPflBadWuert-short] DenkmPflBadWürt %*Abweichung!
 \item[DenkschrWien-short] DenkschrWien 
 \item[Diadora-short] Diadora 
 \item[DialA-short] DialA 
 \item[DialHistAnc-short] DialHistAnc 
 \item[Dike-short] Dike 
 \item[Dioniso-short] Dioniso 
 \item[DiskAB-short] DiskAB 
 \item[DKuDenkmPfl-short] DKuDenkmPfl 
 \item[DLZ-short] DLZ 
 \item[DNP-short] DNP 
 \item[DocAlb-short] DocAlb 
 \item[DocALouv-short] DocALouv 
 \item[DocAMerid-short] DocAMerid 
 \item[DocEmRom-short] DocEmRom 
 \item[Dodone-short] Dodone 
 \item[DOP-short] DOP 
 \item[DossAlet-short] DossAlet 
 \item[DossAParis-short] DossAParis 
 \item[Dura-Europos-short] Dura-Europos 
 \item[EAA-short] EAA 
 \item[EAE-short] EAE 
 \item[EastWest-short] EastWest 
 \item[EcAntNimes-short] EcAntNimes 
 \item[EchosCl-short] EchosCl 
 \item[eDAI-F-short] eDAI-F 
 \item[eDAI-J-short] eDAI-J 
 \item[EgA-short] EgA 
 \item[Egnatia-short] Egnatia 
 \item[EgVicOr-short] EgVicOr 
 \item[Eikasmos-short] Eikasmos 
 \item[Eirene-short] Eirene 
 \item[Elenchos-short] Elenchos 
 \item[Ellenika-short] Ellenika 
 \item[Emerita-short] Emerita 
 \item[EmPrerom-short] EmPrerom 
 \item[Empuries-short] Empúries %*Abweichung!
 \item[Enalia-short] Enalia 
 \item[EnaliaAnn-short] EnaliaAnn 
 \item[Enchoria-short] Enchoria 
 \item[Eos-short] Eos 
 \item[EpetBoiotMel-short] EpetBoiotMel 
 \item[EpetByzSpud-short] EpetByzSpud 
 \item[EpetKyklMel-short] EpetKyklMel 
 \item[EphemDac-short] EphemDac 
 \item[EphemNapoc-short] EphemNapoc 
 \item[EpigrAnat-short] EpigrAnat 
 \item[EpistEpetAth-short] EpistEpetAth 
 \item[EpistEpetPolytThess-short] EpistEpetPolytThess 
 \item[EpistEpetThess-short] EpistEpetThess 
 \item[EPRO-short] EPRO 
 \item[Eranos-short] Eranos 
 \item[EranosJb-short] EranosJb 
 \item[Eretria-short] Eretria 
 \item[Eretz-Israel-short] Eretz-Israel 
 \item[Ergon-short] Ergon 
 \item[ESA-short] ESA 
 \item[EspacioHist-short] EspacioHist 
 \item[EstMadr-short] EstMadr 
 \item[EstZaragoza-short] EstZaragoza 
 \item[EtACl-short] EtACl 
 \item[EtCl-short] EtCl 
 \item[EtClAix-short] EtClAix 
 \item[EtCret-short] EtCret 
 \item[Ethnos-short] Ethnos 
 \item[EtP-short] EtP 
 \item[EtPezenas-short] EtPézenas %*Abweichung!
 \item[EtrSt-short] EtrSt 
 \item[Etruscans-short] Etruscans 
 \item[EtTrav-short] EtTrav 
 \item[Eulimene-short] Eulimene 
 \item[Eunomia-short] Eunomia 
 \item[Euphrosyne-short] Euphrosyne 
 \item[EurAnt-short] EurAnt 
 \item[EurRHist-short] EurRHist 
 \item[Eutopia-short] Eutopia 
 \item[EVP-short] EVP 
 \item[ExcIsr-short] ExcIsr 
 \item[Expedition-short] Expedition 
 \item[ExtremA-short] ExtremA 
 \item[FA-short] FA 
 \item[FAAK-short] FAAK 
 \item[Faenza-short] Faenza 
 \item[FAVA-short] FAVA 
 \item[Faventia-short] Faventia 
 \item[FBerBadWuert-short] FBerBadWürt %*Abweichung!
 \item[FdC-short] FdC 
 \item[FdD-short] FdD 
 \item[FdX-short] FdX 
 \item[FeddersenWierde-short] Feddersen Wierde %*Abweichung!
 \item[FelRav-short] FelRav 
 \item[FGrHist-short] FGrHistr 
 \item[FHG-short] FHG 
 \item[FiA-short] FiA 
 \item[FichEpigr-short] FichEpigr 
 \item[FiE-short] FiE 
 \item[FIFAO-short] FIFAO 
 \item[Figlina-short] Figlina 
 \item[Florentia-short] Florentia 
 \item[FlorIl-short] FlorIl 
 \item[FMRD-short] FMRD 
 \item[FMROe-short] FMRÖ %*Abweichung!
 \item[FoggArtMusAcqu-short] FoggArtMusAcqu 
 \item[FolA-short] FolA 
 \item[FolOr-short] FolOr 
 \item[Fonaments-short] Fonaments 
 \item[Fondamenti-short] Fondamenti 
 \item[FontAPos-short] FontAPos 
 \item[Fontes-short] Fontes 
 \item[Forlimpopoli-short] Forlimpopoli 
 \item[Fornvaennen-short] Fornvännen %*Abweichung!
 \item[Forum-short] Forum 
 \item[FR-short] FR 
 \item[FruehMitAltSt-short] FrühMitAltSt %*Abweichung!
 \item[FuAusgrTrier-short] FuAusgrTrier 
 \item[FuB-short] FuB 
 \item[FuBerBadWuert-short] FuBerBadWürt %*Abweichung!
 \item[FuBerHessen-short] FuBerHessen 
 \item[FuBerOe-short] FuBerÖ %*Abweichung!
 \item[FuBerSchwab-short] FuBerSchwab 
 \item[FuF-short] FuF 
 \item[FuWien-short] FuWien 
 \item[GacNum-short] GacNum 
 \item[Gades-short] Gades 
 \item[Gallaecia-short] Gallaecia 
 \item[Gallia-short] Gallia 
 \item[GalliaInf-short] GalliaInf 
 \item[GalliaInfAReg-short] GalliaInfAReg 
 \item[GalliaPrehist-short] GalliaPrehist 
 \item[GaR-short] GaR 
 \item[GazBA-short] GazBA 
 \item[Genava-short] Genava 
 \item[GeoAnt-short] GeoAnt 
 \item[Germania-short] Germania 
 \item[Gesta-short] Gesta 
 \item[GettyMusJ-short] GettyMusJ 
 \item[GFA-short] GFA 
 \item[GGA-short] GGA 
 \item[GiornFilFerr-short] GiornFilFerr 
 \item[GiornItFil-short] GiornItFil 
 \item[GiornStorLun-short] GiornStorLun 
 \item[GiRoccPalermo-short] GiRoccPalermo 
 \item[Gladius-short] Gladius 
 \item[GlasAJ-short] GlasAJ 
 \item[GlasBeograd-short] GlasBeograd 
 \item[GlasSarajevo-short] GlasSarajevo 
 \item[Glotta-short] Glotta 
 \item[Gnomon-short] Gnomon 
 \item[GodDepA-short] GodDepA 
 \item[GodMuzPlov-short] GodMuzPlov 
 \item[GodMuzSof-short] GodMuzSof 
 \item[GodZborSkopje-short] GodZborSkopje 
 \item[GorLet-short] GorLet 
 \item[GoettMisz-short] GöttMisz %*Abweichung!
 \item[GraRaspr-short] GraRaspr 
 \item[GrazBeitr-short] GrazBeitr 
 \item[GrLatOr-short] GrLatOr 
 \item[GrLatPrag-short] GrLatPrag 
 \item[GrRomByzSt-short] GrRomByzSt 
 \item[Gymnasium-short] Gymnasium 
 \item[Habis-short] Habis 
 \item[HallWPr-short] HallWPr 
 \item[Hama-short] Hama 
 \item[HambBeitrA-short] HambBeitrA 
 \item[HambBeitrNum-short] HambBeitrNum 
 \item[Handlingar-short] Handlingar 
 \item[HarvStClPhil-short] HarvStClPhil 
 \item[HarvTheolR-short] HarvTheolR 
 \item[HASB-short] HASB 
 \item[HAW-short] HAW 
 \item[HdArch-short] HdArch 
 \item[Head-short] Head 
 \item[Helbig-short] Helbig 
 \item[Helike-short] Helike 
 \item[Helikon-short] Helikon 
 \item[Helinium-short] Helinium 
 \item[Helios-short] Helios 
 \item[HellenikaJb-short] HellenikaJb 
 \item[HelvA-short] HelvA 
 \item[Hephaistos-short] Hephaistos 
 \item[Hermes-short] Hermes 
 \item[Herrscherbild-short] Herrscherbild 
 \item[Hesperia-short] Hesperia 
 \item[Hispania-short] Hispania 
 \item[HispAnt-short] HispAnt 
 \item[HispAntEpigr-short] HispAntEpigra 
 \item[HispEpigr-short] HispEpigr 
 \item[HistAnthr-short] HistAnthr 
 \item[HistArt-short] HistArt 
 \item[Historia-short] Historia 
 \item[Historica-short] Historica 
 \item[Histria-short] Histria 
 \item[HistriaA-short] HistriaA 
 \item[HistriaAnt-short] HistriaAnt 
 \item[HistSprF-short] HistSprF 
 \item[HKL-short] HKL 
 \item[Horos-short] Horos 
 \item[HSS-short] HSS 
 \item[HuelvaA-short] HuelvaA 
 \item[HumBild-short] HumBild 
 \item[Hyp-short] Hyp 
 \item[HZ-short] HZ 
 \item[IA-short] IA 
 \item[Iberia-short] Iberia 
 \item[IEJ-short] IEJ 
 \item[IG-short] IG 
 \item[IGCH-short] IGCH 
 \item[IGR-short] IGR 
 \item[IK-short] IK 
 \item[Ilerda-short] Ilerda 
 \item[Iliria-short] Iliria 
 \item[IllinClSt-short] IllinClSt 
 \item[ILN-short] ILN 
 \item[ILS-short] ILS 
 \item[IndexQuad-short] IndexQuad 
 \item[IndogermF-short] IndogermF 
 \item[IndUnArtB-short] IndUnArtB 
 \item[InsFulc-short] InsFulc 
 \item[InstNautAQ-short] InstNautAQ 
 \item[IntJClTrad-short] IntJClTrad 
 \item[IntJNautA-short] IntJNautA 
 \item[IntZSchauBibelWiss-short] IntZSchauBibelWiss 
 \item[InvLuc-short] InvLuc 
 \item[Ipek-short] Ipek 
 \item[Iran-short] Iran 
 \item[IrAnt-short] IrAnt 
 \item[IsrMusJ-short] IsrMusJ 
 \item[IsrMusN-short] IsrMusN 
 \item[IsrMusStA-short] IsrMusStA 
 \item[IsrNumJ-short] IsrNumJ 
 \item[IstanbAMuezYil-short] IstanbAMüzYıl %*Abweichung!
 \item[IstForsch-short] IstForsch 
 \item[Isthmia-short] Isthmia 
 \item[IstMitt-short] IstMitt 
 \item[Italica-short] Italica 
 \item[ItNostr-short] ItNostr 
 \item[IzvBurgas-short] IzvBurgas 
 \item[IzvMuzJuzBalg-short] IzvMuzJužBalg %*Abweichung!
 \item[IzvVarna-short] IzvVarna 
 \item[Jabega-short] Jábega %*Abweichung!
 \item[JadrZbor-short] JadrZbor 
 \item[JAOS-short] JAOS 
 \item[JARCE-short] JARCE 
 \item[JASc-short] JASc 
 \item[JbAC-short] JbAC 
 \item[JbAkMainz-short] JbAkMainz 
 \item[JbBadWuert-short] JbBadWürt %*Abweichung!
 \item[JbBerlMus-short] JbBerlMus 
 \item[JbBernHistMus-short] JbBernHistMus 
 \item[JberAugst-short] JberAugst 
 \item[JberBasel-short] JberBasel 
 \item[JberBayDenkmPfl-short] JberBayDenkmPfl 
 \item[JberProVindon-short] JberProVindon 
 \item[JberVgFrankf-short] JberVgFrankf 
 \item[JberZuerich-short] JberZürich %*Abweichung!
 \item[JbGoett-short] JbGött %*Abweichung!
 \item[JbHambKuSamml-short] JbHambKuSamml 
 \item[JbKHMWien-short] JbKHMWien 
 \item[JbKHSWien-short] JbKHSWien 
 \item[JbKleinasF-short] JbKleinasF 
 \item[JbMuench-short] JbMünch %*Abweichung!
 \item[JbMusKGHamb-short] JbMusKGHamb 
 \item[JbMusLinz-short] JbMusLinz 
 \item[JbOeByz-short] JbÖByz %*Abweichung!
 \item[JbPreussKul-short] JbPreussKul 
 \item[JbRGZM-short] JbRGZM 
 \item[JbSchwUrgesch-short] JbSchwUrgesch 
 \item[JCS-short] JCS 
 \item[JdI-short] JdI 
 \item[JEA-short] JEA 
 \item[JEChrSt-short] JEChrSt 
 \item[JEOL-short] JEOL 
 \item[JewelSt-short] JewelSt 
 \item[JFieldA-short] JFieldA 
 \item[JGS-short] JGS 
 \item[JHS-short] JHS 
 \item[JIbA-short] JIbA 
 \item[JJurP-short] JJurP 
 \item[JKuGesch-short] JKuGesch 
 \item[JMedA-short] JMedA 
 \item[JMedAnthrA-short] JMedAnthrA 
 \item[JMithrSt-short] JMithrSt 
 \item[JNES-short] JNES 
 \item[JNG-short] JNG 
 \item[JPrehistRel-short] JPrehistRel 
 \item[JRA-short] JRA 
 \item[JRomMilSt-short] JRomMilSt 
 \item[JRomPotSt-short] JRomPotSt 
 \item[JRS-short] JRS 
 \item[JSav-short] JSav 
 \item[JSchrVgHalle-short] JSchrVgHalle 
 \item[JSS-short] JSS 
 \item[JTheorA-short] JTheorA 
 \item[Jura-short] Jura 
 \item[JWaltersArtGal-short] JWaltersArtGal 
 \item[JWCI-short] JWCI 
 \item[Kadmos-short] Kadmos 
 \item[Kairos-short] Kairos 
 \item[Kalapodi-short] Kalapodi 
 \item[Kalathos-short] Kalathos 
 \item[Kalos-short] Kalós %*Abweichung!
 \item[Karthago-short] Karthago 
 \item[Kemi-short] Kêmi %*Abweichung!
 \item[Kenchreai-short] Kenchreai 
 \item[KentAR-short] KentAR 
 \item[Keos-short] Keos 
 \item[Kerameikos-short] Kerameikos 
 \item[Kernos-short] Kernos 
 \item[Klearchos-short] Klearchos 
 \item[Kleos-short] Kleos 
 \item[Klio-short] Klio 
 \item[Kodai-short] Kodai 
 \item[KoelnJb-short] KölnJb %*Abweichung!
 \item[KoelnMusB-short] KölnMusB %*Abweichung!
 \item[Kokalos-short] Kokalos 
 \item[KollAVA-short] KollAVA 
 \item[Kratylos-short] Kratylos 
 \item[KretChron-short] KretChron 
 \item[KSIA-short] KSIA 
 \item[KSIAKiev-short] KSIAKiev 
 \item[KST-short] KST 
 \item[Ktema-short] Ktema 
 \item[KuGeschAnz-short] KuGeschAnz 
 \item[Kuml-short] Kuml 
 \item[Kunstchronik-short] Kunstchronik 
 \item[KuOr-short] KuOr 
 \item[Kush-short] Kush 
 \item[KuWeltBerlMus-short] KuWeltBerlMus 
 \item[KypA-short] KypA 
 \item[KypSpud-short] KypSpud 
 \item[Labeo-short] Labeo 
 \item[LAe-short] LÄ %*Abweichung!
 \item[Laietania-short] Laietania 
 \item[Lampas-short] Lampas 
 \item[Lancia-short] Lancia 
 \item[LandKunVierJBl-short] LandKunVierJBl 
 \item[LangOrAnc-short] LangOrAnc 
 \item[Latinitas-short] Latinitas 
 \item[Latomus-short] Latomus 
 \item[Laverna-short] Laverna 
 \item[LCS-short] LCS 
 \item[Levant-short] Levant 
 \item[Lexis-short] Lexis 
 \item[LF-short] LF 
 \item[LibSt-short] LibSt 
 \item[LibyaAnt-short] LibyaAnt 
 \item[LibycaBServAnt-short] LibycaBServAnt 
 \item[LibycaTrav-short] LibycaTrav 
 \item[LSJ-short] Liddell -- ­Scott -- Jones %*Abweichung!
 \item[LIMC-short] LIMC 
 \item[Limesforschungen-short] Limesforschungen 
 \item[Lindos-short] Lindos 
 \item[LingIt-short] LingIt 
 \item[LTUR-short] LTUR 
 \item[Lucentum-short] Lucentum 
 \item[LundAR-short] LundAR 
 \item[Lustrum-short] Lustrum 
 \item[Lykia-short] Lykia 
 \item[MacActaA-short] MacActaA 
 \item[Maecenas-short] Maecenas 
 \item[MAGesGraz-short] MAGesGraz 
 \item[MAGesStei-short] MAGesStei 
 \item[Maia-short] Maia 
 \item[Mainake-short] Mainake 
 \item[MAInstUngAk-short] MAInstUngAk 
 \item[MainzZ-short] MainzZ 
 \item[MakedNasl-short] MakedNasl 
 \item[Makedonika-short] Makedonika 
 \item[MAMA-short] MAMA 
 \item[MAnthrWien-short] MAnthrWien 
 \item[MARo-short] MAR 
 \item[MarbWPr-short] MarbWPr 
 \item[Marche-short] Marche 
 \item[Mari-short] Mari 
 \item[Marisia-short] Marisia 
 \item[MarNero-short] MarNero 
 \item[Marsyas-short] Marsyas 
 \item[MascaJ-short] MascaJ 
 \item[MascaP-short] MascaP 
 \item[Mastia-short] Mastia 
 \item[MatABSSR-short] MatABSSR 
 \item[MatASevPri-short] MatASevPri 
 \item[MatCercA-short] MatCercA 
 \item[MatIsslA-short] MatIsslA 
 \item[MatStar-short] MatStar 
 \item[MatStarWczes-short] MatStarWczes 
 \item[MatTestiCl-short] MatTestiCl 
 \item[MatWczes-short] MatWczes 
 \item[MAVA-short] MAVA 
 \item[MB-short] MB 
 \item[MBAH-short] MBAH 
 \item[MBlVFruehGesch-short] MBlVFrühGesch %*Abweichung!
 \item[MDAIK-short] MDAIK 
 \item[MDAVerb-short] MDAVerb 
 \item[MdI-short] MdI 
 \item[MDOG-short] MDOG 
 \item[Meander-short] Meander 
 \item[MedA-short] MedA 
 \item[MedAnt-short] MedAnt 
 \item[MeddelGlypt-short] MeddelGlypt 
 \item[MeddelLund-short] MeddelLund 
 \item[MeddelThor-short] MeddelThor 
 \item[MededRom-short] MededRom 
 \item[MedelhavsMusB-short] MedelhavsMusB 
 \item[MedHistR-short] MedHistR 
 \item[MedievA-short] MedievA 
 \item[MediSec-short] MediSec 
 \item[MEFRA-short] MEFRA 
 \item[MelBeyrouth-short] MelBeyrouth 
 \item[MelCasaVelazquez-short] MelCasaVelazquez 
 \item[MemAcInscr-short] MemAcInscr 
 \item[MemAmAc-short] MemAmAc 
 \item[MemAnt-short] MemAnt 
 \item[MemAntFr-short] MemAntFr 
 \item[MemBarcelA-short] MemBarcelA 
 \item[MemBologna-short] MemBologna 
 \item[MemHistAnt-short] MemHistAnt 
 \item[MemInstNatFr-short] MemInstNatFr 
 \item[MemLinc-short] MemLinc 
 \item[MemNap-short] MemNap 
 \item[MemPontAc-short] MemPontAc 
 \item[MemStor-short] MemStor 
 \item[MemStorFriuli-short] MemStorFriuli 
 \item[Merida-short] Mérida %*Abweichung!
 \item[MeridaMem-short] MéridaMem %*Abweichung!
 \item[Meroitica-short] Meroitica 
 \item[Mesopotamia-short] Mesopotamia 
 \item[Messana-short] Messana 
 \item[Metis-short] Métis %*Abweichung!
 \item[MetrMusJ-short] MetrMusJ 
 \item[MetrMusSt-short] MetrMusSt 
 \item[MF-short] MF 
 \item[MFruehChrOe-short] MFrühChrÖ %*Abweichung!
 \item[Milet-short] Milet 
 \item[MilForsch-short] MilForsch 
 \item[MinEpigrP-short] MinEpigrP 
 \item[Minerva-short] Minerva 
 \item[Minos-short] Minos 
 \item[MInstWasser-short] MInstWasser 
 \item[MIO-short] MIO 
 \item[MiscCrAnt-short] MiscCrAnt 
 \item[MiscStStor-short] MiscStStor 
 \item[MitChrA-short] MitChrA 
 \item[MKT-short] MKT 
 \item[MKuHistFlorenz-short] MKuHistFlorenz 
 \item[MKul-short] MKul 
 \item[MM-short] MM 
 \item[Mnemosyne-short] Mnemosyne 
 \item[MOeNumGes-short] MÖNumGes %*Abweichung!
 \item[MonAnt-short] MonAnt 
 \item[MonInst-short] MonInst 
 \item[MonPiot-short] MonPiot 
 \item[MonPitt-short] MonPitt 
 \item[Mozia-short] Mozia 
 \item[MPraehistKomWien-short] MPraehistKomWien %*Abweichung!
 \item[MSAtene-short] MSAtene 
 \item[MSchliemann-short] MSchliemann 
 \item[MSchwUrFruehGesch-short] MSchwUrFrühGesch %*Abweichung!
 \item[MSpaetAByz-short] MSpätAByz %*Abweichung!
 \item[MueJb-short] MüJb %*Abweichung!
 \item[MuenchBeitrVFG-short] MünchBeitrVFG %*Abweichung!
 \item[MuenchStSprWiss-short] MünchStSprWiss %*Abweichung!
 \item[MuM-short] MuM 
 \item[MusAfr-short] MusAfr 
 \item[MusBenaki-short] MusBenaki 
 \item[MusCrit-short] MusCrit 
 \item[Muse-short] Muse 
 \item[Museon-short] Muséon %*Abweichung!
 \item[MuseumUnesco-short] MuseumUnesco 
 \item[MusFerr-short] MusFerr 
 \item[MusGalIt-short] MusGalIt 
 \item[MusHaaretz-short] MusHaaretz 
 \item[MusHelv-short] MusHelv 
 \item[MusKoeln-short] MusKöln %*Abweichung!
 \item[MusNotAmNumSoc-short] MusNotAmNumSoc 
 \item[MusPontevedra-short] MusPontevedra 
 \item[MusRiv-short] MusRiv 
 \item[MusTusc-short] MusTusc 
 \item[MuzEvkSzeged-short] MuzEvkSzeged 
 \item[MuzNat-short] MuzNaţ
 \item[MuzPamKul-short] MuzPamKul 
 \item[NachrArbUWA-short] NachrArbUWA 
 \item[NapNobil-short] NapNobil 
 \item[NassAnn-short] NassAnn 
 \item[NAWG-short] NAWG 
 \item[NBWorcArtMus-short] NBWorcArtMus 
 \item[NEphemSemEpigr-short] NEphemSemEpigr 
 \item[NewsletterAthen-short] NewsletterAthen 
 \item[NewsletterPotTech-short] NewsletterPotTech 
 \item[NGWG-short] NGWG 
 \item[NigCl-short] NigCl 
 \item[Nikephoros-short] Nikephoros 
 \item[Nin-short] Nin 
 \item[NNM-short] NNM 
 \item[NomChron-short] NomChron 
 \item[Norba-short] Norba 
 \item[NordNumArs-short] NordNumArs 
 \item[NotABerg-short] NotABerg 
 \item[NotAHisp-short] NotAHisp 
 \item[NotAHispPrehistoria-short] NotAHispPrehistoria 
 \item[NotAllumiere-short] NotAllumiere 
 \item[NotALomb-short] NotALomb 
 \item[NotMilano-short] NotMilano 
 \item[NouvClio-short] NouvClio 
 \item[Novaensia-short] Novaensia 
 \item[NSc-short] NSc 
 \item[NStFan-short] NStFan 
 \item[NubChr-short] NubChr 
 \item[NubLet-short] NubLet 
 \item[NueBlA-short] NüBlA %*Abweichung!
 \item[NumAntCl-short] NumAntCl 
 \item[Numantia-short] Numantia 
 \item[NumChron-short] NumChron 
 \item[Numen-short] Numen 
 \item[NumEpigr-short] NumEpigr 
 \item[Numisma-short] Numisma 
 \item[NumismaticaRom-short] NumismaticaRom 
 \item[Numizmaticar-short] Numizmatičar %*Abweichung!
 \item[Nummus-short] Nummus 
 \item[NumZ-short] NumZ 
 \item[NuovDidask-short] NuovDidask 
 \item[OccasPublClSt-short] OccasPublClSt 
 \item[OccOr-short] OccOr 
 \item[OGIS-short] OGIS 
 \item[OeJh-short] ÖJh %*Abweichung!
 \item[OF-short] OF 
 \item[Offa-short] Offa 
 \item[Ogam-short] Ogam 
 \item[Oikumene-short] Oikumene 
 \item[OIP-short] OIP 
 \item[Olba-short] Olba 
 \item[OlBer-short] OlBer 
 \item[Olympia-short] Olympia 
 \item[Olynthus-short] Olynthus 
 \item[OLZ-short] OLZ 
 \item[OpArch-short] OpArch 
 \item[OpAth-short] OpAth 
 \item[OpFin-short] OpFin 
 \item[OpPomp-short] OpPomp 
 \item[OpRom-short] OpRom 
 \item[Opus-short] Opus 
 \item[Or-short] Or 
 \item[OrA-short] OrA 
 \item[OrAnt-short] OrAnt 
 \item[OrbTerr-short] OrbTerr 
 \item[OrChr-short] OrChr 
 \item[OrChrPer-short] OrChrPer 
 \item[Ordona-short] Ordona 
 \item[Orient-short] Orient 
 \item[Origini-short] Origini 
 \item[Orizzonti-short] Orizzonti 
 \item[Orpheus-short] Orpheus 
 \item[OrpheusThracSt-short] OrpheusThracSt 
 \item[OrSu-short] OrSu 
 \item[OsjZbor-short] OsjZbor 
 \item[OstbGrenzm-short] OstbGrenzm 
 \item[Ostraka-short] Ostraka 
 \item[OudhMeded-short] OudhMeded 
 \item[OxfJA-short] OxfJA 
 \item[OxfStPhilos-short] OxfStPhilos 
 \item[Pact-short] Pact 
 \item[Padusa-short] Padusa 
 \item[PagA-short] PagA 
 \item[PAI-short] PAI 
 \item[Paideuma-short] Paideuma 
 \item[Palaeohistoria-short] Palaeohistoria 
 \item[Paleohispanica-short] Paleohispánica %*Abweichung!
 \item[Palladio-short] Palladio 
 \item[Pallas-short] Pallas 
 \item[Palmet-short] Palmet 
 \item[PamA-short] PamA 
 \item[Pan-short] Pan 
 \item[PapBilb-short] PapBilb 
 \item[Papyri-short] Papyri 
 \item[Parthica-short] Parthica 
 \item[Partenope-short] Partenope 
 \item[PAS-short] PAS 
 \item[PBF-short] PBF 
 \item[Pegasus-short] Pegasus 
 \item[PEQ-short] PEQ 
 \item[Peristil-short] Peristil 
 \item[Persica-short] Persica 
 \item[Peuce-short] Peuce 
 \item[PF-short] PF 
 \item[Pharos-short] Pharos 
 \item[Philologus-short] Philologus 
 \item[Phoenix-short] Phoenix 
 \item[PhoenixExOrLux-short] PhoenixExOrLux 
 \item[Phoibos-short] Phoibos 
 \item[Phronesis-short] Phronesis 
 \item[Picus-short] Picus 
 \item[PIR-short] PIR 
 \item[PKG-short] PKG 
 \item[Platon-short] Platon 
 \item[PLup-short] PLup 
 \item[PolAMed-short] PolAMed 
 \item[Polemon-short] Polemon 
 \item[Polis-short] Polis 
 \item[PompHercStab-short] PompHercStab 
 \item[Pontica-short] Pontica 
 \item[Portugalia-short] Portugalia 
 \item[PP-short] PP 
 \item[PPM-short] PPM 
 \item[PraceA-short] PraceA 
 \item[PraceMatLodz-short] PraceMatŁodz %*Abweichung!
 \item[Prakt-short] Prakt 
 \item[PraktAkAth-short] PraktAkAth 
 \item[PreistAlp-short] PreistAlp 
 \item[PriloziZagreb-short] PriloziZagreb 
 \item[PrincViana-short] PrincViana 
 \item[ProblIsk-short] ProblIsk 
 \item[ProcAfrClAss-short] ProcAfrClAss 
 \item[ProcCambrPhilSoc-short] ProcCambrPhilSoc 
 \item[ProcDanInstAth-short] ProcDanInstAth 
 \item[ProcPrehistSoc-short] ProcPrehistSoc 
 \item[Prometheus-short] Prometheus 
 \item[ProspAQuad-short] ProspAQuad 
 \item[Prospettiva-short] Prospettiva 
 \item[Prospezioni-short] Prospezioni 
 \item[ProvHist-short] ProvHist 
 \item[ProvLucca-short] ProvLucca 
 \item[PublInstTTMeneses-short] PublInstTTMeneses 
 \item[Pulpudeva-short] Pulpudeva 
 \item[Puteoli-short] Puteoli 
 \item[Pyrenae-short] Pyrenae 
 \item[PZ-short] PZ 
 \item[QDAP-short] QDAP 
 \item[QuadABarcel-short] QuadABarcel 
 \item[QuadACagl-short] QuadACagl 
 \item[QuadACal-short] QuadACal 
 \item[QuadALibya-short] QuadALibya 
 \item[QuadAMant-short] QuadAMant 
 \item[QuadAMess-short] QuadAMess 
 \item[QuadAOst-short] QuadAOst 
 \item[QuadAPiem-short] QuadAPiem 
 \item[QuadAquil-short] QuadAquil 
 \item[QuadAReggio-short] QuadAReggio 
 \item[QuadAVen-short] QuadAVen 
 \item[QuadCast-short] QuadCast 
 \item[QuadCat-short] QuadCat 
 \item[QuadChieti-short] QuadChieti 
 \item[QuadErb-short] QuadErb 
 \item[QuadFriulA-short] QuadFriulA 
 \item[QuadGerico-short] QuadGerico 
 \item[QuadIstFilGr-short] QuadIstFilGr 
 \item[QuadIstLat-short] QuadIstLat 
 \item[QuadLecce-short] QuadLecce 
 \item[QuadMusPontecorvo-short] QuadMusPontecorvo 
 \item[QuadMusSalinas-short] QuadMusSalinas 
 \item[QuadProtost-short] QuadProtost 
 \item[QuadStLun-short] QuadStLun 
 \item[QuadStor-short] QuadStor 
 \item[QuadStorici-short] QuadStorici 
 \item[QuadUrbin-short] QuadUrbin 
 \item[Quaternaria-short] Quaternaria 
 \item[RA-short] RA 
 \item[RAArtLouv-short] RAArtLouv 
 \item[RAC-short] RAC 
 \item[RACFr-short] RACFr 
 \item[RAComo-short] RAComo 
 \item[RACr-short] RACr 
 \item[RadAkZadar-short] RadAkZadar 
 \item[Radiocarbon-short] Radiocarbon 
 \item[RadSplit-short] RadSplit 
 \item[RAE-short] RAE 
 \item[Raggi-short] Raggi 
 \item[RAMadrid-short] RAMadrid 
 \item[RANarb-short] RANarb 
 \item[RAPon-short] RAPon 
 \item[RapWyk-short] RapWyk 
 \item[RArchBiblMus-short] RArchBiblMus 
 \item[RArcheom-short] RArcheom 
 \item[RArtMus-short] RArtMus 
 \item[RassAPiomb-short] RassAPiomb 
 \item[RassLazio-short] RassLazio 
 \item[RassStorSalern-short] RassStorSalern 
 \item[RassVolt-short] RassVolt 
 \item[RAssyr-short] RAssyr 
 \item[Ratiariensia-short] Ratiariensia 
 \item[RAtlMed-short] RAtlMed 
 \item[RavStRic-short] RavStRic 
 \item[Raydan-short] Raydan 
 \item[RB-short] RB 
 \item[RBelgNum-short] RBelgNum 
 \item[RBelgPhilHist-short] RBelgPhilHist 
 \item[RBK-short] RBK 
 \item[RCulClMedioev-short] RCulClMedioev 
 \item[RdA-short] RdA 
 \item[RDAC-short] RDAC 
 \item[RdE-short] RdE 
 \item[RDroitsAnt-short] RDroitsAnt 
 \item[RE-short] RE 
 \item[REA-short] REA 
 \item[REByz-short] REByz 
 \item[RecConstantine-short] RecConstantine 
 \item[RechACrac-short] RechACrac 
 \item[RechAlb-short] RechAlb 
 \item[RecMusAlcoi-short] RecMusAlcoi 
 \item[RecTrav-short] RecTrav 
 \item[REG-short] REG 
 \item[ReiCretActa-short] ReiCretActa 
 \item[ReiCretCommunic-short] ReiCretCommunic 
 \item[REL-short] REL 
 \item[Rema-short] Rema 
 \item[RendBologna-short] 	RendBologna 
 \item[RendIstLomb-short] RendIstLomb 
 \item[RendLinc-short] RendLinc 
 \item[RendNap-short] RendNap 
 \item[RendPontAc-short] RendPontAc 
 \item[RepMalta-short] RepMalta 
 \item[Reppal-short] Reppal 
 \item[RepSocLibSt-short] RepSocLibSt 
 \item[RES-short] RES 
 \item[REstIber-short] REstIber 
 \item[REtArm-short] REtArm 
 \item[RevEg-short] RevEg 
 \item[RFil-short] RFil 
 \item[RGeorgCauc-short] RGeorgCauc 
 \item[RGF-short] RGF 
 \item[RGuimar-short] RGuimar 
 \item[RHA-short] RHA 
 \item[RheinMusBonn-short] RheinMusBonn 
 \item[RHistArmees-short] RHistArmees %*Abweichung!
 \item[RHistRel-short] RHistRel 
 \item[RhM-short] RhM 
 \item[RIA-short] RIA 
 \item[RIC-short] RIC 
 \item[RicEgAntCopt-short] RicEgAntCopt 
 \item[RicognA-short] RicognA 
 \item[RicStBrindisi-short] RicStBrindisi 
 \item[RIngIntem-short] RIngIntem 
 \item[RItNum-short] RItNum 
 \item[RlA-short] RlA 
 \item[RLouvre-short] RLouvre 
 \item[RM-short] RM 
 \item[RNum-short] RNum 
 \item[RoczMuzWarsz-short] RoczMuzWarsz 
 \item[RoemOe-short] RömÖ %*Abweichung!
 \item[RoemQSchr-short] RömQSchr %*Abweichung!
 \item[Romanobarbarica-short] Romanobarbarica 
 \item[RomGens-short] RomGens 
 \item[RoscherML-short] Roscher, ML %*Abweichung!
 \item[RossA-short] RossA 
 \item[RPC-short] RPC 
 \item[RPhil-short] RPhil 
 \item[RPortA-short] RPortA 
 \item[RPorto-short] RPorto 
 \item[RRC-short] RRC 
 \item[RSaintonge-short] RSaintonge 
 \item[RScPreist-short] RScPreist 
 \item[RSO-short] RSO 
 \item[RStBiz-short] RStBiz 
 \item[RStCl-short] RStCl 
 \item[RStFen-short] RStFen 
 \item[RStLig-short] RStLig 
 \item[RStMarch-short] RStMarch 
 \item[RStorAnt-short] RStorAnt 
 \item[RStorCal-short] RStorCal 
 \item[RStPomp-short] RStPomp 
 \item[RStPun-short] RStPun 
 \item[RTopAnt-short] RTopAnt 
 \item[Rudiae-short] Rudiae 
 \item[SaalbJb-short] SaalbJb 
 \item[SaarBeitr-short] SaarBeitr 
 \item[SaarStMat-short] SaarStMat 
 \item[Sacer-short] Sacer 
 \item[Saeculum-short] Saeculum 
 \item[SAGA-short] SAGA 
 \item[SaggiFen-short] SaggiFen 
 \item[Saguntum-short] Saguntum 
 \item[Saitabi-short] Saitabi 
 \item[SAK-short] SAK 
 \item[Salduie-short] Salduie 
 \item[Samothrace-short] Samothrace 
 \item[Sandalion-short] Sandalion 
 \item[Sardis-short] Sardis 
 \item[Sargetia-short] Sargetia 
 \item[SarkSt-short] SarkSt 
 \item[Sautuola-short] Sautuola 
 \item[Savaria-short] Savaria 
 \item[SBBerlin-short] SBBerlin 
 \item[SBLeipzig-short] SBLeipzig 
 \item[SBMuenchen-short] SBMünchen %*Abweichung!
 \item[SborBrno-short] SborBrno 
 \item[SBWien-short] SBWien 
 \item[ScAnt-short] ScAnt 
 \item[SCE-short] SCE 
 \item[SchildStei-short] SchildStei 
 \item[Scholia-short] Scholia 
 \item[SchwMueBl-short] SchwMüBl %*Abweichung!
 \item[SchwNumRu-short] SchwNumRu 
 \item[ScrCiv-short] ScrCiv 
 \item[ScrClIsr-short] ScrClIsr 
 \item[ScrHieros-short] ScrHieros 
 \item[ScrMed-short] ScrMed 
 \item[SDAIK-short] SDAIK 
 \item[SEG-short] SEG 
 \item[SeminRom-short] SeminRom 
 \item[Semitica-short] Semitica 
 \item[SetubalA-short] SetubalA 
 \item[Sibrium-short] Sibrium 
 \item[SicA-short] SicA 
 \item[SicGymn-short] SicGymn 
 \item[SIG-short] SIG 
 \item[Sileno-short] Sileno 
 \item[SilkRoadArtA-short] SilkRoadArtA 
 \item[SIMA-short] SIMA 
 \item[Simblos-short] Simblos 
 \item[Skyllis-short] Skyllis 
 \item[SlovA-short] SlovA 
 \item[SlovNum-short] SlovNum 
 \item[SMEA-short] SMEA 
 \item[SNG-short] SNG 
 \item[SocGeoAOran-short] SocGeoAOran 
 \item[SoobErmit-short] SoobErmit 
 \item[SoobMuzMoskva-short] SoobMuzMoskva 
 \item[SovA-short] SovA 
 \item[Spal-short] Spal 
 \item[SpNov-short] SpNov 
 \item[Spoletium-short] Spoletium 
 \item[StA-short] StA 
 \item[Stadion-short] Stadion 
 \item[StaedelJb-short] StädelJb %*Abweichung!
 \item[StAeg-short] StAeg 
 \item[StAlb-short] StAlb 
 \item[StAnt-short] StAnt 
 \item[Starinar-short] Starinar 
 \item[StAWarsz-short] StAWarsz 
 \item[StBiFranc-short] StBiFranc 
 \item[StBitont-short] StBitont 
 \item[StBoT-short] StBoT 
 \item[StCercIstorV-short] StCercIstorV 
 \item[StCercNum-short] StCercNum 
 \item[StCl-short] StCl 
 \item[StClOr-short] StClOr 
 \item[StDocA-short] StDocA 
 \item[StDocHistIur-short] StDocHistIur 
 \item[StEbla-short] StEbla 
 \item[StEgAntPun-short] StEgAntPun 
 \item[SteMat-short] SteMat 
 \item[StEpigrLing-short] StEpigrLing 
 \item[StEtr-short] StEtr 
 \item[StGenu-short] StGenu 
 \item[StHist-short] StHist 
 \item[StiftHambKuSamml-short] StiftHambKuSamml 
 \item[StItFilCl-short] StItFilCl 
 \item[StLatIt-short] StLatIt 
 \item[StMagreb-short] StMagreb 
 \item[StMatStorRel-short] StMatStorRel 
 \item[StOliv-short] StOliv 
 \item[StOr-short] StOr 
 \item[StOrCr-short] StOrCr 
 \item[StP-short] StP 
 \item[StrennaRom-short] StrennaRom 
 \item[StRom-short] StRom 
 \item[StRomagn-short] StRomagn 
 \item[StSalent-short] StSalent 
 \item[StSard-short] StSard 
 \item[StStorRel-short] StStorRel 
 \item[StTardoant-short] StTardoant 
 \item[StTrentStor-short] StTrentStor 
 \item[StTroica-short] StTroica 
 \item[StUrbin-short] StUrbin 
 \item[Sumer-short] Sumer 
 \item[SylvaMala-short] Sylva Mala %*Abweichung!
 \item[SymbOslo-short] SymbOslo 
 \item[Syria-short] Syria 
 \item[SyrMesopSt-short] SyrMesopSt 
 \item[TAD-short] TAD 
 \item[Talanta-short] Talanta 
 \item[TAM-short] TAM 
 \item[Taras-short] Taras 
 \item[Tarsus-short] ETarsus 
 \item[TAVO-short] TAVO 
 \item[TeherF-short] TeherF 
 \item[Teiresias-short] Teiresias 
 \item[TelAvivJA-short] TelAvivJA 
 \item[TerraAntBalc-short] TerraAntBalc 
 \item[TerraVolsci-short] TerraVolsci 
 \item[Teruel-short] Teruel 
 \item[TextilAnc-short] TextilAnc 
 \item[TheolRu-short] TheolRu 
 \item[ThesCRA-short] ThesCRA 
 \item[Thessalika-short] Thessalika 
 \item[Thessalonike-short] Thessalonike 
 \item[Thieme-Becker-short] Thieme -- Becker %*Abweichung!
 \item[ThrakChron-short] ThrakChron 
 \item[ThrakEp-short] ThrakEp 
 \item[TIB-short] TIB 
 \item[Tibiscus-short] Tibiscus 
 \item[TiLeon-short] TiLeon 
 \item[Tiryns-short] Tiryns 
 \item[TMA-short] TMA 
 \item[Topoi-short] Topoi 
 \item[Torretta-short] Torretta 
 \item[TourOrleOr-short] TourOrleOr 
 \item[TrabAntrEtn-short] TrabAntrEtn 
 \item[TrabArq-short] TrabArq 
 \item[TrabAssArqPort-short] TrabAssArqPort 
 \item[TrabNavarra-short] TrabNavarra 
 \item[TrabPrehist-short] TrabPrehist 
 \item[Traditio-short] Traditio 
 \item[TransactAmPhilAss-short] TransactAmPhilAss 
 \item[TransactAmPhilosSoc-short] TransactAmPhilosSoc 
 \item[TransactLond-short] TransactLond 
 \item[TravMem-short] TravMem 
 \item[TravToulouse-short] TravToulouse 
 \item[TreMonet-short] TreMonet 
 \item[TribArq-short] TribArq 
 \item[TrudyErmit-short] Trudy 
 \item[TrWPr-short] TrWPr 
 \item[TrZ-short] TrZ 
 \item[TTKY-short] TTKY 
 \item[TueBA-Ar-short] TüBA-Ar %*Abweichung!
 \item[Tyche-short] Tyche 
 \item[UF-short] UF 
 \item[UPA-short] UPA 
 \item[LUrbe-short] L’Urbe %*Abweichung!
 \item[UrSchw-short] UrSchw 
 \item[UVB-short] UVB 
 \item[VarSpom-short] VarSpom 
 \item[VDI-short] VDI 
 \item[Vekove-short] Vekove 
 \item[Veleia-short] Veleia 
 \item[VenArt-short] VenArt 
 \item[VerAmstMeded-short] VerAmstMeded 
 \item[Verbanus-short] Verbanus 
 \item[VeteraChr-short] VeteraChr 
 \item[VGesVind-short] VGesVind 
 \item[Vichiana-short] Vichiana 
 \item[VicOr-short] VicOr 
 \item[VigChr-short] VigChr 
 \item[Viminacium-short] Viminacium 
 \item[VisRel-short] VisRel 
 \item[Vitudurum-short] Vitudurum 
 \item[VivScyl-short] VivScyl 
 \item[VizVrem-short] VizVrem 
 \item[VjesAMuzZagreb-short] VjesAMuzZagreb 
 \item[VjesDal-short] VjesDal 
 \item[Wad-al-Hayara-short] Wad-al-Hayara 
 \item[WeltGesch-short] WeltGesch 
 \item[WiadA-short] WiadA 
 \item[WissMBosn-short] WissMBosn 
 \item[WissZBerl-short] WissZBerl 
 \item[WissZHalle-short] WissZHalle 
 \item[WissZJena-short] WissZJena 
 \item[WissZRostock-short] WissZRostock 
 \item[WO-short] WO 
 \item[WorldA-short] WorldA 
 \item[WSt-short] WSt 
 \item[WuerzbJb-short] WürzbJb %*Abweichung!
 \item[WVDOG-short] WVDOG 
 \item[WZKM-short] WZKM 
 \item[Xenia-short] Xenia 
 \item[XeniaAnt-short] XeniaAnt 
 \item[XeniaKonst-short] XeniaKonst 
 \item[YaleClSt-short] YaleClSt 
 \item[YaleUnivB-short] YaleUnivB 
 \item[ZA-short] ZA 
 \item[ZAAK-short] ZAAK 
 \item[ZAeS-short] ZÄS %*Abweichung!
 \item[ZAKSSchriften-short] ZAKSSchriften 
 \item[ZAntChr-short] ZAntChr 
 \item[ZAW-short] ZAW 
 \item[ZborMuzBeograd-short] ZborMuzBeograd 
 \item[ZborRadBeograd-short] ZborRadBeograd 
 \item[ZborZadar-short] ZborZadar 
 \item[ZDMG-short] ZDMG 
 \item[ZDPV-short] ZDPV 
 \item[Zephyrus-short] Zephyrus 
 \item[ZEthn-short] ZEthn 
 \item[ZfA-short] ZfA 
 \item[ZfNum-short] ZfNum 
 \item[ZivaAnt-short] ZivaAnt 
 \item[ZKuGesch-short] ZKuGesch 
 \item[ZNW-short] ZNW 
 \item[ZPE-short] ZPE 
 \item[ZSav-short] ZSav 
 \item[ZSchwA-short] ZSchwA 
 \item[ZVerglSprF-short] ZVerglSprF
\end{description}
\end{footnotesize}
\end{multicols}


\subsection{Long forms}\label{liste-lang}
\DescribeMacro{noabbrv}
To see the long forms of journal titles or series you have to switch on the option  |noabbrv|.
%\begin{multicols}{1}
\begin{footnotesize}
\begin{description}[%
			%	style=multiline,
				style=nextline,
				leftmargin=3cm,
				%font=\normalfont\bfseries
				]
\item[AA] Archäologischer Anzeiger 
\item[AAA] Αρχαιολογικά Ανάλεκτα εξ Αθηνών 
\item[AAcque] Archeologia delle acque. Rivista semestrale di antropologia, archeologia, etnografia, storia dell'acqua 
\item[AAdv] The Archaeological Advertiser 
\item[AAJ] Annual of the Department of Antiquities of Jordan 
\item[AAlpi] Archeologia delle Alpi 
\item[AarbKob] Aarbøger for nordisk oldkyndighed og historie %*Abweichung!
\item[AArchit] Archeologia dell'architettura 
\item[AAS] Les annales archéologiques arabes syriennes 
\item[AASOR] The Annual of the American Schools of Oriental Research 
\item[AAusgrBadWuert] Archäologische Ausgrabungen in Baden-Württemberg %*Abweichung!
\item[AAustr] Archaeologia Austriaca 
\item[ABADY] Archäologische Berichte aus dem Yemen 
\item[AbhBerlin] Abhandlungen der Deutschen Akademie der Wissenschaften zu Berlin 
\item[AbhDuesseldorf] Abhandlungen der Rheinisch-Westfälischen Akademie der Wissenschaften %*Abweichung!
\item[AbhGoettingen] Abhandlungen der Akademie der Wissenschaften zu Göttingen. Philologisch-Historische Klasse %*Abweichung!
\item[AbhLeipzig] Abhandlungen der Sächsischen Akademie der Wissenschaften zu Leipzig. Philologisch-Historische Klasse 
\item[AbhMainz] Akademie der Wissenschaften und der Literatur in Mainz. Abhandlungen der Geistes- und Sozialwissenschaftlichen Klasse 
\item[AbhMuenchen] Bayerische Akademie der Wissenschaften. Philosophisch-Historische Klasse. Abhandlungen %*Abweichung!
\item[ABret] Archéologie en Bretagne. Bulletin d'information 
\item[Abr-Nahrain] Abr-Nahrain. An Annual Published by the Department of Middle Eastern Studies, University of Melbourne 
\item[ABulg] Archaeologia Bulgarica 
\item[ABV] J. D. Beazley, Attic Black-figure Vase-painters (Oxford 1956) 
\item[ACalc] Archeologia e calcolatori 
\item[ACamp] Archeologia in Campania. Bolletino di informazioni a cura della Soprintendenza archeologica delle province di Napoli e Caserta 
\item[ACant] Archaeologia Cantiana 
\item[AcBibl] Accademie e biblioteche d'Italia 
\item[Achse] Achse, Rad und Wagen. Beiträge zur Geschichte der Landfahrzeuge 
\item[Acme] Acme. Annali della Facoltà di lettere e filosofia dell'Università degli studi di Milano 
\item[Acontia] Acontia. Revista de arqueología 
\item[ACors] Archeologia corsa 
\item[ActaAArtHist] Acta ad archaeologiam et artium historiam pertinentia 
\item[ActaAArtHist-sa] Acta ad archaeologiam et artium historiam pertinentia. Series altera in 8 %*Abweichung!
\item[ActaAcAbo] Acta Academiae Aboensis 
\item[ActaACarp] Acta archaeologica Carpathica 
\item[ActaALov] Acta archaeologica Lovaniensia 
\item[ActaALovMono] Acta archaeologica Lovaniensia. Monographiae 
\item[ActaAntHung] Acta antiqua Academiae scientiarum Hungaricae 
\item[ActaArch] Acta archaeologica. København 
\item[ActaArchHung] Acta archaeologica Academiae scientiarum Hungaricae 
\item[ActaAth] Acta Instituti Atheniensis regni Sueciae 
\item[ActaCl] Acta classica. Proceedings of the Classical Association of South Africa 
\item[ActaClDebrec] Acta classica Universitatis scientiarum Debreceniensis 
\item[ActaHistDac] Acta historica. Societas academica Dacoromana 
\item[ActaHyp] Acta hyperborea. Danish Studies in Classical Archaeology 
\item[ActaInstRomFin] Acta Instituti Romani Finlandiae 
\item[ActaMusNapoca] Acta Musei Napocensis 
\item[ActaMusPorol] Acta Musei Porolissensis 
\item[ActaNum] Acta numismática (Barcelona) 
\item[ActaOr] Acta orientalia (Kopenhagen) 
\item[ActaOrHung] Acta orientalia Academiae scientiarum Hungaricae 
\item[ActaPhilSocDac] Acta philologica. Societas academica Dacoromana 
\item[ActaPraehistA] Acta praehistorica et archaeologica 
\item[ActaTorunA] Acta Universitatis Nicolai Copernici. Archaeologia 
\item[ActaTorunHist] Acta Universitatis Nicolai Copernici. Historia 
\item[AD] Antike Denkmäler 
\item[ADAIK] Abhandlungen des Deutschen Archäologischen Instituts, Abteilung Kairo 
\item[Adalya] Adalya. Annual of the Suna \& Inan Kiraç-Research Institute on Mediterranean Civilizations 
\item[ADelt-A] Αρχαιολογικόν Δελτίον (Μελέτες) %*Abweichung!
\item[ADelt-B] Αρχαιολογικόν Δελτίον (Χρονικά) %*Abweichung!
\item[ADerg] Arkeoloji dergisi. Ege Üniversitesi Edebiyat Fakültesi 
\item[ADFU] Ausgrabungen der Deutschen Forschungsgemeinschaft in Uruk-Warka 
\item[AdI] Annali dell'Instituto di corrispondenza archeologica 
\item[ADOG] Abhandlungen der Deutschen Orient-Gesellschaft 
\item[Adumatu] Adumatu. A Semi-Annual Archeological Refereed Journal on the Arab World 
\item[AE] L'année épigraphique 
\item[AeA] Aegean Archaeology 
\item[Aegaeum] Aegaeum. Annales d'archéologie égéenne de l'Université de Liège 
\item[AegLev] Ägypten und Levante. Egypt and the Levant. Internationale Zeitschrift für ägyptische Archäologie und deren Nachbargebiete %*Abweichung!
\item[AEmil] Archeologia dell'Emilia-Romagna 
\item[AEphem] Αρχαιολογική Eφημερίς 
\item[AeR] Atene e Roma 
\item[AErgoMak] Το Αρχαιολογικό Έργο στη Μακεδονία και Θράκη 
\item[AErt] Archaeologiai értesitő
\item[AEspA] Archivo español de arqueología 
\item[Aevum] Aevum. Rassegna di scienze storiche linguistiche e filologiche 
\item[AevumAnt] Aevum antiquum 
\item[AF] Archäologische Forschungen 
\item[AfO] Archiv für Orientforschung 
\item[Africa] Africa. Institut national d'archéologie et d'art, Tunis 
\item[AGD] Antike Gemmen in deutschen Sammlungen 
\item[AGeo] Archaeologia geographica 
\item[Agora] The Athenian Agora 
\item[AgoraPB] Excavations of the Athenian Agora. Picture Book 
\item[AHist] Arqueologia e história 
\item[AHistStAlex] Archaeological and Historical Studies. The Archaeological Society of Alexandria 
\item[AHw] W. von Soden, Akkadisches Handwörterbuch (Wiesbaden 1965--1981) 
\item[AiD] Archäologie in Deutschland 
\item[AInf] Archäologische Informationen. Mitteilungen zur Ur- und Frühgeschichte 
\item[AIONArch] Annali dell'Istituto universitario orientale di Napoli. Dipartimento di studi del mondo classico e del Mediterraneo antico. Sezione di archeologia e storia antica 
\item[AIONFil] Annali dell'Istituto universitario orientale di Napoli. Dipartimento di studi del mondo classico e del Mediterraneo antico. Sezione filologicoletteraria 
\item[AIONLing] Annali dell'Istituto universitario orientale di Napoli. Dipartimento di studi del mondo classico e del Mediterraneo antico. Sezione linguistica 
\item[AIPhOr] Annuaire de l'Institut de philologie et d'histoire orientales et slaves (Université Libre de Bruxelles) 
\item[Aitna] Aitna. Quaderni di topografia antica 
\item[AJA] American Journal of Archaeology 
\item[AJahrBay] Das archäologische Jahr in Bayern 
\item[AJPh] American Journal of Philology 
\item[AJug] Archaeologia Jugoslavica 
\item[AKorrBl] Archäologisches Korrespondenzblatt 
\item[AlbaRegia] Alba Regia. Annales Musei Stephani regis 
\item[AlmaMaterSt] Alma mater studiorum Almanacco calabrese Almoraima 
\item[AlmanachWien] Österreichische Akademie der Wissenschaften. Almanach 
\item[AlonJisrael] Alon mahleqat ha-'atiqot šel medinat Jisra'el 
\item[Al-Qannis] Al-Qanniš. Boletín del Taller de arqueología de Alcañiz
\item[Altamura] Altamura. Bollettino dell'Archivio-biblioteca-museo civico 
\item[AltoMed] Alto medioevo 
\item[Alt-Paphos] Ausgrabungen in Alt-Paphos auf Cypern 
\item[AltThuer] Alt-Thüringen %*Abweichung!
\item[AM] Mitteilungen des Deutschen Archäologischen Instituts, Athenische Abteilung 
\item[AMediev] Archaeologia medievale. Cultura materiale, insediamenti, territorio 
\item[AMethTh] Advances in Archaeological Method and Theory 
\item[AMI] Archäologische Mitteilungen aus Iran 
\item[AMIT] Archäologische Mitteilungen aus Iran und Turan 
\item[AmJAncHist] American Journal of Ancient History 
\item[AmJNum] American Journal of Numismatics 
\item[AMold] Arheologia Moldovei 
\item[AMosel] Archaeologia Mosellana 
\item[AMS] Asia Minor Studien 
\item[AmStP] American Studies in Papyrology 
\item[AMuGS] Antike Münzen und geschnittene Steine 
\item[ANachr] Archäologisches Nachrichtenblatt 
\item[ANachrBad] Archäologische Nachrichten aus Baden 
\item[AnadoluAra] Anadolu araştırmaları. Jahrbuch für kleinasiatische Forschung 
\item[AnadoluKonf] Anadolu Medeniyetleri Müzesi konferansları 
\item[AnadoluYil] Anadolu Medeniyetleri Müzesi yıllığı 
\item[AnAe] Analecta Aegyptiaca 
\item[Anagennesis] Anagennesis. A Papyrological Journal 
\item[AnalBolland] Analecta Bollandiana 
\item[AnalP] Analecta papyrologica 
\item[AnalRom] Analecta Romana Instituti Danici 
\item[AnArqAnd] Anuario arqueológico de Andalucía 
\item[Anas] Anas. Museo nacional de arte romano de Mérida 
\item[AnatA] Anatolian Archaeology. Reports on Research Conducted in Turkey 
\item[Anatolia] Anatolia. Revue annuelle de l’Institut d’archéologie de l’Université d’Ankara 
\item[ANaturwiss] Archäologie und Naturwissenschaften 
\item[AncCivScytSib] Ancient Civilizations from Scythia to Siberia. An International Journal of Comparative Studies in History and Archaeology 
\item[AncHistB] The Ancient History Bulletin 
\item[AncInd] The Ancient India 
\item[AncNearEastSt] Ancient Near Eastern Studies. An Annual 
\item[AnCord] Anales de arqueología cordobesa 
\item[AncSoc] Ancient Society 
\item[AncW] The Ancient World 
\item[AncWestEast] Ancient West and East 
\item[AnDubr] Annali Zavoda za povijesne znanosti Istraivakog centra Jugoslavenske akademije znanosti i umjetnosti u Dubrovniku 
\item[ANews] Archaeological News 
\item[ANilMoy] Archéologie du Nil moyen 
\item[ANL] The Archaeological News Letter 
\item[AnMunFaro] Anais do municípo de Faro 
\item[AnMurcia] Anales de prehistoria y arqueología. Universidad de Murcia 
\item[AnnAcEtr] Annuario. Accademia etrusca di Cortona 
\item[AnnAcTorino] Annuario della Accademia delle scienze di Torino 
\item[AnnAStorAnt] Annali. Sezione di archeologia e storia antica. Istituto universitario orientale di Napoli. Dipartimento di studi del mondo classico e del Mediterraneo antico 
\item[AnnBari] Annali della Facoltà di lettere e filosofia, Università degli Studi, Bari 
\item[AnnBenac] Annali Benacensi 
\item[AnnBiblAModena] Annuario bibliografico di archeologia. Modena 
\item[AnnBiblARom] Annuario bibliografico di archeologia. Nuove accessioni del \ldots. Biblioteca dell’Istituto nazionale di archeologia e storia dell’arte, Roma 
\item[AnnByzConf] Annual Byzantine Studies Conference. Abstracts of Papers 
\item[AnnCagl] Annnali della Facoltà di lettere e filosofia dell'Università di Cagliari 
\item[AnnCaglMag] Annali della Facoltà di magistero dell'Università di Cagliari 
\item[AnnEconSocCiv] Annales. Economies, sociétés, civilisations 
\item[AnnEgBibl] Annual Egyptological Bibliography 
\item[AnnEth] Annales d'Éthiopie 
\item[AnnFaina] Annali della Fondazione per il Museo Claudio Faina 
\item[AnnHistA] Annales d'histoire et d'archéologie 
\item[AnnHistScSoc] Annales. Histoire, sciences sociales 
\item[AnnIstGiapp] Annuario. Istituto giapponese di cultura in Roma 
\item[AnnIstItNum] Annali. Istituto italiano di numismatica 
\item[AnnLecce] Annali dell'Università di Lecce. Facoltà di lettere e filosofia e di magistero 
\item[AnnLeedsUnOrSoc] The Annual of Leeds University Oriental Society 
\item[AnnMacerata] Annali della Facoltà di lettere e filosofia, Università di Macerata 
\item[AnnMessMag] Nuovi annali della Facoltà di magistero dell'Università di Messina 
\item[AnnMusRov] Annali del Museo civico di Rovereto. Sezione archeologia, storia, scienze naturali 
\item[AnnNap] Annali della Facoltà di lettere e filosofia, Università di Napoli 
\item[AnnNivern] Les annales des pays Nivernais 
\item[AnnNoment] Annali. Associazione nomentana di storia e archeologia 
\item[AnnOrNap] Annali. Rivista del Dipartimento di studi asiatici e del Dipartimento di studi e ricerche su Africa e paesi arabi, Istituto universitario orientale di Napoli 
\item[AnnotNum] Annotazioni numismatiche 
\item[AnnPerugia] Annali della Facoltà di lettere e filosofia, Università degli studi di Perugia, 1. Studi classici 
\item[AnnPisa] Annali della Scuola normale superiore di Pisa 
\item[AnnPontAcRom] Annuario della Pontificia accademia romana di archeologia 
\item[AnnRepBSA] Annual Report of Council. British School of Archaeology at Athens 
\item[AnnRepCypr] Annual Report of the Department of Antiquities, Republic of Cyprus 
\item[AnnRepFoggArtMus] The Annual Report of the Fogg Art Museum 
\item[AnnSiena] Annali della Facoltà di lettere e filosofia, Università di Siena 
\item[AnnuarioAcLinc] Annuario della Accademia nazionale dei Lincei 
\item[AnnuarioLecce] Annuario. Liceo-ginnasio statale G. Palmieri, Lecce 
\item[AnnUnBud] Annales Universitatis scientiarum Budapestinensis de Rolando Eötvös nominatae 
\item[AnnWorcArtMus] Annual (Report). Worcester Art Museum 
\item[Anodos] Anodos. Studies of Ancient World 
\item[AnOr] Analecta orientalia. Commentationes scientificae de rebus orientis antiqui 
\item[ANRW] Aufstieg und Niedergang der römischen Welt 
\item[Anschnitt] Der Anschnitt. Mitteilungsblatt der Vereinigung der Freunde von Kunst und Kultur im Bergbau 
\item[ANSMusNotes] Museum Notes. American Numismatic Society 
\item[AnSt] Anatolian Studies 
\item[Antaeus] Antaeus. Communicationes ex Instituto archaeologico Academiae scientiarum hungaricae 
\item[AntAfr] Antiquités africaines 
\item[AntChr] Antike und Christentum 
\item[AntCl] L'antiquité classique 
\item[AnthrAChron] Ανθρωπολογικά και Αρχαιολογικά Χρονικά 
\item[Anthropos] Άνθρωπος. Όργανο της Ανθρωπολογικής Εταιρείας Ελλάδος 
\item[Antichthon] Antichthon. Journal of the Australian Society for Classical Studies 
\item[AntigCr] Antigüedad y cristianismo. Monografías históricas sobre la antigüedad tardía 
\item[Antipolis] Antipolis. A Journal of Mediterranean Archaeology 
\item[Antiqua] Antiqua. Rivista dell'Archeoclub d'Italia 
\item[Antiquity] Antiquity. A Quarterly Review of Archaeology 
\item[AntJ] The Antiquaries Journal 
\item[AntK] Antike Kunst 
\item[AntNat] Antiquités nationales. Saint-Germain-en-Laye 
\item[AntPisa] Antichità pisane 
\item[AntPl] Antike Plastik 
\item[AntSurv] Antiquity and Survival 
\item[AntTard] Antiquité tardive. Revue internationale d'histoire et d'archéologie 
\item[AnzAW] Anzeiger für die Altertumswissenschaft 
\item[AnzWien] Anzeiger. Österreichische Akademie der Wissenschaften, Philosophisch-Historische Klasse 
\item[AOAT] Alter Orient und Altes Testament. Veröffentlichungen zur Kultur und Geschichte des Alten Orients und des Alten Testaments 
\item[AoF] Altorientalische Forschungen 
\item[AOtkryt] Archeologičeskie otkrytija 
\item[APamKiiv] Archeologični pamjatki URSR 
\item[APh] L'année philologique 
\item[APol] Archaeologia Polona 
\item[Apollo] Apollo. Bolletino dei musei provinciali del Salernitano 
\item[ApolloLond] Apollo. The International Magazine of the Arts 
\item[APort] O arqueólogo português 
\item[AppRomFil] Appunti romani di filologia. Studi e comunicazioni di filologia, linguistica e letteratura greca e latina 
\item[APregl] Arheološki pregled. Arheološko društvo Jugoslavije 
\item[Apulum] Apulum. Acta Musei Apulensis 
\item[AquiLeg] Aquila legionis. Cuadernos des estudios sobre el ejército romano 
\item[AquilNost] Aquileia nostra. Bollettino dell'Associazione nazionale per Aquileia 
\item[Aquitania] Aquitania. Une revue inter-régionale d'archéologie 
\item[ArabAEpigr] Arabian Archaeology and Epigraphy 
\item[ARadRaspr] Arheolokiradovi i rasprave 
\item[ArbFBerSaechs] Arbeits- und Forschungsberichte zur sächsischen Bodendenkmalpflege 
\item[Archaeographie] Archäographie. Archäologie und elektronische Datenverarbeitung %*Abweichung!
\item[Archaeologia] Archaeologia or Miscellaneous Tracts Relating to Antiquity Published by the Society of Antiquaries of London 
\item[Archaeology] Archaeology. A Magazine Dealing with the Antiquity of the World 
\item[Archaeometry] Archaeometry. Bulletin of the Research Laboratory for Archaeology and History of Art, Oxford University 
\item[Archaiognosia] Αρχαιογνωσία 
\item[ArchBegriffsGesch] Archiv für Begriffsgeschichte 
\item[ArchByzMnem] Αρχείον των Βυζαντινών Μνημείων της Ελλάδος  
\item[ArchCl] Archeologia classica 
\item[Archeo] Archeo. Attualità del passato 
\item[ArcheogrTriest] Archeografo triestino 
\item[ArcheologiaParis] Archeologia, Paris. L'archéologie dans le monde et tout ce qui concerne les recherches historiques, artistiques et scientifiques sur terre et dans les mers 
\item[ArcheologiaRoma] Archeologia. Rivista bimestrale. Roma 
\item[ArcheologiaWarsz] Archeologia. Rocznik Instytutu archeologii i etnologii, Polskiej akademii nauk 
\item[ArcheologijaKiiv] Archeologija. Nacional'na akademija nauk Ukraini. Institut archeologii 
\item[ArcheologijaSof] Archeologija. Organ na Archeologičeskija institut i muzej (pri Bălgarskata akademii nauk) 
\item[ArchEubMel] Αρχείον Ευβοϊκών Μελετών 
\item[ArchHom] Archaeologia Homerica 
\item[Architectura] Architectura. Zeitschrift für Geschichte der Baukunst 
\item[Archivi] Archivi. Archivi d'Italia e rassegna internazionale degli archivi 
\item[ArchPF] Archiv für Papyrusforschung und verwandte Gebiete 
\item[ArchPrehistLev] Archivo de prehistoria levantina 
\item[ArchRel] Archiv für Religionsgeschichte 
\item[ArchStorCal] Archivio storico per la Calabria e la Lucania 
\item[ArchStorPugl] Archivio storico pugliese 
\item[ArchStorRom] Archivio della Società romana di storia patria 
\item[ArchStorSicOr] Archivio storico per la Sicilia orientale 
\item[ArchStorSir] Archivio storico siracusano 
\item[Arctos] Arctos. Acta philologica Fennica 
\item[ARepLond] Archaeological Reports 
\item[Argo] Argo. časopis slovenskih muzejev, Narodni Muzej Slovenije 
\item[ArOr] Archív orientální. Quarterly Journal of African and Asian Studies 
\item[ArOrMono] Archív orientální. Quarterly Journal of African, Asian and Latin American Studies. Monografie Archívu orientálního 
\item[ArOrSuppl] Archív orientální. Quarterly Journal of African, Asian and Latin American Studies. Supplementa 
\item[ARozhl] Archeologické rozhledy 
\item[ArqBeja] Arquivo de Beja. Boletim, estudos, arquivo 
\item[Arse] Arse. Boletín del Centro arqueológico saguntino 
\item[ArsGeorg] Ars Georgica 
\item[ArtAntMod] Arte antica e moderna 
\item[ArtARhone] Art et archéologie en Rhône-Alpes %*Abweichung!
\item[ArtAs] Artibus Asiae 
\item[ArtB] The Art Bulletin 
\item[ArtJ] Art Journal 
\item[ArtLomb] Arte lombarda 
\item[ArtMediev] Arte medievale 
\item[ArtVirg] Arts in Virginia 
\item[ARV2] J. D. Beazley, Attic Red-figure Vase-painters \textsuperscript{2}(Oxford 1963) 
\item[ASachs] Archäologie in Sachsen-Anhalt 
\item[ASAE] Annales du Service des antiquités de l’Égypte 
\item[ASammlUnZuerch] Archäologische Sammlung der Universität Zürich %*Abweichung!
\item[ASAtene] Annuario della Scuola archeologica di Atene e delle missioni italiane in Oriente 
\item[ASbor] Archeologičeskij sbornik. Gosudarstvennyj ordena Lenina Ermitaž 
\item[ASchw] Archäologie der Schweiz. Mitteilungsblatt der Schweizerischen Gesellschaft für Ur- und Frühgeschichte 
\item[ASoc] Archeologia e società 
\item[ASoloth] Archäologie des Kantons Solothurn 
\item[ASR] Die antiken Sarkophagreliefs 
\item[Assaph] Assaph. Studies in Art History 
\item[AssyrMisc] Assyriological Miscellanies 
\item[AST] AratirmaSonuçlariToplantisi 
\item[ASub] L'archeologo subacqueo. Quadrimestrale di archeologia subacquea e navale 
\item[ASubacq] Archeologia subacquea. Studi, ricerche e documenti 
\item[Athenaeum] Athenaeum. Studi di letteratura e storia dell'antichità 
\item[Atiqot] `Atiqot. Journal of the Israel Department of Antiquities 
\item[AtiqotHeb] `Atiqot. Journal of the Israel Department of Antiquities. Hebrew Series 
\item[Atlal] Atlal. The Journal of Saudi Arabian Archaeology 
\item[AttiAcPontan] Atti della Accademia pontaniana 
\item[AttiAcRov] Atti della Accademia Roveretana degli Agiati. Contributi della classe di scienze umane, di lettere ed arti 
\item[AttiAcTorino] Atti della Accademia delle scienze di Torino, 2. Classe di scienze morali, storiche e filologiche 
\item[AttiCAntCl] Atti. Centro ricerche e documentazione sull'antichità classica 
\item[AttiCItRom] Atti. Centro studi e documentazione sull'Italia romana 
\item[AttiMemBologna] Atti e memorie. Deputazione di storia patria per le province di Romagna 
\item[AttiMemDal] Atti e memorie della Società dalmata di storia patria 
\item[AttiMemFirenze] Atti e memorie dell'Academia toscana di scienze e lettere »La Columbaria« 
\item[AttiMemIstria] Atti e memorie della Società istriana di archeologia e storia patria 
\item[AttiMemMagnaGr] Atti e memorie della Società Magna Grecia 
\item[AttiMemModena] Atti e memorie. Deputazione di storia patria per le antiche provincie modenesi 
\item[AttiMemTivoli] Atti e memorie della Società tiburtina di storia e d'arte 
\item[AttiMusTrieste] Atti dei Civici musei di storia ed arte di Trieste 
\item[AttiPalermo] Atti della Accademia di scienze, lettere e arti di Palermo 
\item[AttiRovigno] Atti. Centro di ricerche storiche, Rovigno 
\item[AttiSocFriuli] Atti della Società per la preistoria e protostoria della regione Friuli -- Venezia Giulia 
\item[AttiVenezia] Atti. Istituto veneto di scienze, lettere ed arti 
\item[AuA] Antike und Abendland 
\item[AulaOr] Aula orientalis. Revista de estudios del Próximo Oriente antiguo 
\item[AusgrFu] Ausgrabungen und Funde. Nachrichtenblatt der Landesarchäologie 
\item[AusgrFuWestf] Ausgrabungen und Funde in Westfalen-Lippe 
\item[AustrRom] Pro Austria Romana 
\item[AUTerr] Archeologia, uomo, territorio. Rivista dei Gruppi archeologici Nord Italia 
\item[AUWE] Ausgrabungen in Uruk-Warka. Endberichte 
\item[AV] Archäologische Veröffentlichungen. Deutsches Archäologisches Institut, Abteilung Kairo 
\item[AVen] Archeologia veneta 
\item[AVes] Arheološki vestnik (Ljubljana) 
\item[AViva] Archeologia viva 
\item[AvP] Altertümer von Pergamon 
\item[AW] Antike Welt. Zeitschrift für Archäologie und Kulturgeschichte 
\item[AyasofyaMuezYil] Ayasofia Müzesi yıllığı. Annual of Ayasofya Museum \label{AyasofyaMuezYil-lang} %*Abweichung!
\item[AZ] Archäologische Zeitung 
\item[Azotea] Azotea. Revista de cultura del Ayuntamiento de Coria del Río 
\item[BA] Bollettino di archeologia 
\item[Baalbek] Baalbek. Ergebnisse der Ausgrabungen und Untersuchungen in den Jahren 1898 bis 1905 
\item[BAAlger] Bulletin d'archéologie algérienne 
\item[BABarcel] Butlletí informatiu de l'Institut de prehistòria i arqueologia de la Diputació provincial de Barcelona 
\item[BABesch] Bulletin antieke beschaving. Annual Papers on Classical Archaeology 
\item[BAcRHist] Boletín de la Real academia de la historia 
\item[BACopt] Bulletin de la Société d'archéologie copte 
\item[BadFuBer] Badische Fundberichte 
\item[Baetica] Baetica. Estudios de arte, geografía e historia 
\item[BaF] Baghdader Forschungen 
\item[BalacaiKoez] Balácai közlemények %*Abweichung!
\item[BalkSt] Balkan Studies 
\item[BALond] Bulletin of the Institute of Archaeology, University of London 
\item[BALux] Bulletin d'archéologie luxembourgeoise 
\item[BaM] Baghdader Mitteilungen 
\item[BAMaroc] Bulletin d'archéologie marocaine 
\item[BAmSocP] The Bulletin of the American Society of Papyrologists 
\item[BAncOrMus] Bulletin of the Ancient Orient Museum (Tokyo) 
\item[BAngers] Bulletin du Centre de recherches et d'enseignement de l'antiquité, Angers 
\item[BAngloIsrASoc] Bulletin of the Anglo-Israel Archaeological Society 
\item[BAnnMusFerr] Bollettino annuale. Musei ferraresi 
\item[BAntFr] Bulletin de la Société nationale des antiquaires de France 
\item[BAntLux] Bulletin des antiquités luxembourgeoises 
\item[BAParis] Bulletin archéologique du Comité des travaux historiques et scientifiques. Antiquités nationales 
\item[BAProv] Bulletin archéologique de Provence 
\item[BAR] British Archaeological Reports. British Series 
\item[BArchAlex] Bulletin. Société archéologique d'Alexandrie 
\item[BArchit] Bollettino del Centro di studi per la storia dell'architettura 
\item[BARIntSer] British Archaeological Reports. International Series 
\item[BASard] Nuovo bullettino archeologico sardo 
\item[BAsEspA] Boletín. Asociación española de amigos de la arqueología 
\item[BAsInst] Bulletin of the Asia Institute 
\item[BASOR] Bulletin of the American Schools of Oriental Research 
\item[BAssBude] Bulletin de l'Association Guillaume Budé %*Abweichung!
\item[BAssMosAnt] Bulletin d'information de l'Association internationale pour l'étude de la mosaïque antique 
\item[BASub] Bollettino di archeologia subacquea 
\item[BASudEstEur] Bulletin d'archéologie sud-est européenne 
\item[BATarr] Bulletí arqueològic. Reial societat arqueològica tarraconense. Boletín arqueológico. Real sociedad arqueológica tarragonese 
\item[BAur] Boletín auriense 
\item[BAVA] Beiträge zur Allgemeinen und Vergleichenden Archäologie 
\item[BayVgBl] Bayerische Vorgeschichtsblätter 
\item[BBasil] Bollettino storico della Basilicata 
\item[BBelgRom] Bulletin de l'Institut historique belge de Rome 
\item[BBolsena] Bollettino di studi e ricerche. Biblioteca comunale di Bolsena 
\item[BBrByzSt] Bulletin of British Byzantine Studies 
\item[BCamuno] Bullettino del Centro Camuno di studi preistorici 
\item[BCASic] Beni culturali e ambientali. Sicilia 
\item[BCercleNum] Bulletin du Cercle d'études numismatiques 
\item[BCH] Bulletin de correspondance hellénique 
\item[BCircNumNap] Bollettino del Circolo numismatico napoletano 
\item[BCl] Bollettino dei classici 
\item[BClevMus] The Bulletin of The Cleveland Museum of Art 
\item[BCom] Bullettino della Commissione archeologica comunale di Roma 
\item[BCord] Bulletin d'information de l'Association internationale pour l'étude de la mosaïque antique 
\item[BdA] Bollettino d'arte 
\item[BdE] Bibliothèque d'études, Institut français d'archéologie orientale, Kairo 
\item[BdEC] Bibliothèque d'études coptes, Institut français d'archéologie orientale, Kairo 
\item[BdI] Bullettino dell'Instituto di corrispondenza archeologica 
\item[BDirRom] Bullettino dell'Istituto di diritto romano »Vittorio Scialoja« 
\item[BeazleyAddenda2] T. H. Carpenter (Hrsg.), Beazley Addenda \textsuperscript{2}(Oxford 1989) %*Abweichung!
\item[BeazleyPara] J. D. Beazley, Paralipomena. Additions to Attic Black-figure Vase-painters and to Attic Red-figure Vase-painters (Oxford 1971) %*Abweichung!
\item[BEcAntNimes] Bulletin (annuel) de l'École antique de Nîmes %*Abweichung!
\item[BediKart] Bedi Kartlisa. Revue de kartvélologie 
\item[BEFAR] Bibliothèque des Écoles françaises d'Athènes et de Rome 
\item[BeitrESkAr] Beiträge zur Erschließung hellenistischer und kaiserzeitlicher Skulptur und Architektur 
\item[BeitrNamF] Beiträge zur Namenforschung 
\item[BeitrSudanF] Beiträge zur Sudanforschung 
\item[BelArt] Bellas artes (Madrid) 
\item[Belleten] Belleten. Türk Tarih Kurumu 
\item[Benacus] Benàcus. Museo archeologico della Val Tenesi 
\item[BerBayDenkmPfl] Bericht der Bayerischen Bodendenkmalpflege 
\item[BerDFG] Bericht der Deutschen Forschungsgemeinschaft 
\item[BerlBeitrArchaeom] Berliner Beiträge zur Archäometrie %*Abweichung!
\item[BerlBlVFruehGesch] Berliner Blätter für Vor- und Frühgeschichte %*Abweichung!
\item[BerlJbVFruehGesch] Berliner Jahrbuch für Vor- und Frühgeschichte %*Abweichung!
\item[BerlMus] Berliner Museen 
\item[BerlNumZ] Berliner numismatische Zeitschrift 
\item[BerOudhBod] Berichten van de Rijksdienst voor het oudheidkundig bodemonderzoek 
\item[BerRGK] Bericht der Römisch-Germanischen Kommission 
\item[BerVerhLeipz] Berichte über die Verhandlungen der Sächsischen Akademie der Wissenschaften zu Leipzig 
\item[Berytus] Berytus. Archaeological Studies 
\item[BEspA] Boletín. Asociación española de amigos de la arqueología 
\item[BEspOr] Boletín de la Asociación española de orientalistas 
\item[BEtOr] Bulletin d'études orientales 
\item[BFilGrPadova] Bollettino dell'Istituto di filologia greca, Università di Padova 
\item[BFilLingSic] Bollettino. Centro di studi filologici e linguistici siciliani 
\item[BFlegr] Bollettino flegreo. Rivista di storia, arte e scienze 
\item[BFoligno] Bollettino storico della città di Foligno 
\item[BHarvMus] Harvard University Art Museums Bulletin 
\item[BIasos] Bollettino dell'Associazione Iasos di Caria 
\item[BibAr] The Biblical Archaeologist. The American School of Oriental Research New Haven 
\item[BiblClOr] Bibliotheca classica orientalis 
\item[BiblSymb] Bibliographie zur Symbolik, Ikonographie und Mythologie. Internationales Referateorgan 
\item[BIBulg] Izvestija na Archeologičeskija institut. Bulletin de l’Institut d’archéologie 
\item[BICS] Bulletin of the Institute of Classical Studies of the University of London 
\item[BIFAO] Bulletin de l'Institut français d'archéologie orientale 
\item[BInfCentumcellae] Bollettino di informazioni. Associazione archeologica Centumcellae 
\item[BInfCESDAE] Bollettino di informazioni del Centro di studi e documentazione sull'area elima 
\item[BiogrZbor] Biogradski zbornik 
\item[BiOr] Bibliotheca orientalis 
\item[BIstOrvieto] Bollettino dell'Istituto storico artistico orvietano 
\item[BJaen] Boletín del Instituto de estudios giennenses %*Abweichung!
\item[BJb] Bonner Jahrbücher des Rheinischen Landesmuseums in Bonn 
\item[BJerus] Bulletin. The Hebrew University, Jerusalem, Rabinowitz Fund 
\item[BLaborMusLouvre] Bulletin du laboratoire du Musée du Louvre 
\item[BLazioMerid] Bollettino dell'Istituto di storia e di arte del Lazio meridionale 
\item[BLikUm] Bulletin Razreda za likovne umjetnosti Hrvatske akademije znanosti i umjetnosti 
\item[BlMueFreundeF] Blätter für Münzfreunde und Münzforschung %*Abweichung!
\item[BLugo] Boletín de la Comisión provincial de monumentos históricos y artísticos de Lugo 
\item[BMCGreekCoins] Catalogue of the Greek Coins in the British Museums %*Abweichung!
\item[BMCOR] Catalogue of Oriental Coins in the British Museum I--X 
\item[BMCRE] H. Mattingly (u. a.), Coins of the Roman Empire in the British Museum (London 1923--1950; \textsuperscript{2}1975) 
\item[BMCRRI-III] H. A. Grueber, Coins of the Roman Republic in the British Museum I--III (London 1910) %*Abweichung!
\item[BMetrMus] The Metropolitan Museum of Art Bulletin 
\item[BMon] Bulletin monumental 
\item[BMonMusPont] Bollettino. Monumenti, musei e gallerie pontificie 
\item[BMQ] The British Museum Quarterly 
\item[BMQNSuppl] The British Museum Quarterly. News Supplement 
\item[BMusBeyrouth] Bulletin du Musée de Beyrouth 
\item[BMusBrux] Bulletin des Musées royaux d'art et d'histoire, Bruxelles 
\item[BMusCadiz] Boletín del Museo de Cádiz 
\item[BMusCivRom] Bullettino del Museo della civiltà romana 
\item[BMusFA] Bulletin. Museum of Fine Arts, Boston 
\item[BMusHongr] Bulletin du Musée hongrois des beaux-arts 
\item[BMusMadr] Boletín del Museo arqueológico nacional, Madrid 
\item[BMusMich] Bulletin. Museums of Art and Archaeology, University of Michigan 
\item[BMusMonaco] Bulletin du Musée d'anthropologie préhistorique de Monaco 
\item[BMusPadova] Bollettino del Museo civico di Padova 
\item[BMusPBelArt] Boletín del Museo provincial de bellas artes 
\item[BMusRom] Bollettino dei Musei comunali di Roma 
\item[BMusVars] Bulletin du Musée national de Varsovie 
\item[BMusZaragoza] Museo de Zaragoza. Boletín 
\item[BNumParis] Bulletin de la Société française de numismatique 
\item[BNumRoma] Bollettino di numismatica 
\item[Bogazkoey-Hattusa] Boğazköy-Hattuša. Ergebnisse der Ausgrabungen %*Abweichung!
\item[Bolskan] Bolskan. Revista de arqueología del Instituto de estudios Altoaragoneses, revista de arqueología Oscense 
\item[BonnHVg] Bonner Hefte zur Vorgeschichte 
\item[BOntMus] Bulletin of the Royal Ontario Museum of Archaeology, University of Toronto 
\item[Boreas] Boreas. Münstersche Beiträge zur Archäologie 
\item[BoreasUpps] Boreas. Uppsala Studies in Ancient Mediterranean and Near Eastern Civilization 
\item[BPeintRom] Bulletin de liaison. Centre d'études des peintures murales romaines 
\item[BPI] Bullettino di paletnologia italiana 
\item[BPrehistAlp] Bulletin d'études préhistoriques alpines %*Abweichung!
\item[BProAvent] Bulletin de l'Association Pro Aventico 
\item[BProvidence] Bulletin of the Rhode Island School of Design. Museum Notes 
\item[BracAug] Bracara Augusta. Revista cultural da Câmara municipal de Braga 
\item[BracaraAugusta] Bracara Augusta. Revista cultural da Câmara municipal de Braga 
\item[BremABl] Bremer archäologische Blätter 
\item[BRest] Bollettino dell'Istituto centrale del restauro 
\item[Brigantium] Brigantium. Museo arqueolóxico e histórico 
\item[BrMusYearbook] The British Museum Yearbook 
\item[BSA] The Annual of the British School at Athens 
\item[BSAA] Boletín del Seminario de estudios de arte y arqueología, Universidad de Valladolid 
\item[BSFE] Bulletin de la Société française d'égyptologie 
\item[BSiena] Bullettino senese di storia patria 
\item[BSOAS] Bulletin of the School of Oriental and African Studies (London) 
\item[BSocAChamp] Bulletin de la Société archéologique champenoise 
\item[BSocBiblReinach] Bulletin de liaison de la Société des amis de la Bibliothèque Salomon Reinach 
\item[BSocNumRom] Buletinul Societăţii numismatice române 
\item[BSR] Papers of the British School at Rome 
\item[BStLat] Bollettino di studi latini 
\item[BStorArt] Bollettino della Unione storia ed arte 
\item[BTextilAnc] Bulletin du Centre international d'étude des textiles anciens 
\item[BTorino] Bollettino della Società piemontese di archeologia e belle arti 
\item[BTravTun] Bulletin des travaux de l'Institut national du patrimoine. Comptes rendus 
\item[BudReg] Budapest régiségei 
\item[BulletinGetty] Bulletin. J. Paul Getty Museum of Art 
\item[BulletinNorthampton] Bulletin. Smith College Museum of Art 
\item[BVallad] Boletín del Seminario de estudios de arte y arqueología, Universidad de Valladolid 
\item[BVitoria] Boletín de la Institución »Sancho el Sabio« 
\item[BWaltersArtGal] The Walters Art Gallery Bulletin 
\item[BWPr] Winckelmannsprogramm der Archäologischen Gesellschaft zu Berlin 
\item[Byzantina] Βυζαντινά. Επιστημονικόν Όργανον Κέντρου Βυζαντινών Ερευνών Φιλοσοφικής Σχολής Αριστοτελείου Πανεπιστημίου Θεσσαλονίκης
\item[ByzF] Byzantinische Forschungen. Internationale Zeitschrift für Byzantinistik 
\item[ByzJb] Byzantinisch-neugriechische Jahrbücher 
\item[ByzZ] Byzantinische Zeitschrift 
\item[BZ] Biblische Zeitschrift 
\item[CAA] Corpus antiquitatum Aegyptiacarum 
\item[CAD] The Assyrian Dictionary of the Oriental Institute of the University of Chicago 
\item[CadA] Cadernos de arqueologia 
\item[Caesaraugusta] Caesaraugusta. Publicaciones del Seminario de Arqueología y Numismática Aragonesas 
\item[Caesarodunum] Caesarodunum. Bulletin de l'Institut d'études latines et du Centre de recherches A. Piganiol 
\item[CAH] The Cambridge Ancient History 
\item[CahArmeeRom] Cahiers du Groupe de recherches sur l'armée romaine et les provinces 
\item[CahASubaqu] Cahiers d'archéologie subaquatique 
\item[CahByrsa] Cahiers de Byrsa 
\item[CahCEC] Cahier. Centre d'études chypriotes 
\item[CahCerEg] Cahiers de la céramique égyptienne 
\item[CahDelFrIran] Cahiers de la Délégation française en Iran 
\item[CahGlotz] Cahiers du Centre Gustave-Glotz. Revue reconnue par le CNRS 
\item[CahKarnak] Cahiers de Karnak 
\item[CahLig] Cahiers ligures de préhistoire et de protohistoire 
\item[CahMariemont] Les cahiers de Mariemont. Bulletin du Musée royal de Mariemont 
\item[CahMusChampollion] Cahiers du Musée Champollion. Histoire et archéologie 
\item[CahPEg] Cahiers de recherches de l'Institut de papyrologie et d'égyptologie de Lille. Sociétés urbaines en Égypte et au Soudan 
\item[CahRhod] Cahiers rhodaniens 
\item[CahTun] Cahiers de Tunisie 
\item[CalifStClAnt] California Studies in Classical Antiquity 
\item[CambrAJ] Cambridge Archaeological Journal 
\item[CArch] Cahiers archéologiques 
\item[CarinthiaI] Carinthia I. Geschichtliche und volkskundliche Beiträge zur Heimatkunde Kärntens %*Abweichung!
\item[CarnuntumJb] Carnuntum-Jahrbuch. Zeitschrift für Archäologie und Kulturgeschichte des Donauraumes 
\item[Carpica] CarpicaMuzeuljudeeandeistorieiartBacu 
\item[Carrobbio] Il Carrobbio. Rivista di studi bolognesi 
\item[CAT] Ch. W. Clairmont, Classical Attic Tombstones (Kilchberg 1993--1995) 
\item[CE] Cuadernos emeritenses 
\item[CEDAC] CEDAC. Bulletin. Centre d'études et de documentation archéologique de la conservation de Cartage 
\item[CEFR] Collection de l'École française de Rome 
\item[Celticum] Celticum. Supplément à Ogam 
\item[CercA] Cercetriarheologice 
\item[CercNum] Cercetări numismatice. Muzeul naţional de istorie 
\item[Chiron] Chiron. Mitteilungen der Kommission für Alte Geschichte und Epigraphik des Deutschen Archäologischen Instituts 
\item[ChronEg] Chronique d'Égypte 
\item[CIA] Corpus inscriptionum Atticarum 
\item[CIE] Corpus inscriptionum Etruscarum 
\item[CIG] Corpus inscriptionum Graecarum 
\item[CIH] Corpus inscriptionum Semiticarum. Pars quarta. Inscriptiones himyariticas et sabaeas continens 
\item[CIL] Corpus inscriptionum Latinarum 
\item[CincArtB] The Cincinnati Art Museum Bulletin 
\item[CIS] Corpus inscriptionum Semiticarum 
\item[CIstAMilano] Contributi dell'Istituto di archeologia. Pubblicazioni dell'Università cattolica del Sacro Cuore, Milano 
\item[CivClCr] Civiltà classica e cristiana 
\item[CivPad] Civiltà padana. Archeologia e storia del territorio 
\item[ClAnt] Classical Antiquity 
\item[ClevStHistArt] Cleveland Studies in the History of Art 
\item[Clio] Clio. Revista do Centro de história da Universidade Lisboa 
\item[ClIre] Classics Ireland 
\item[ClJ] The Classical Journal 
\item[ClMediaev] Classica et mediaevalia. Revue danoise de philologie et d'histoire 
\item[ClPhil] Classical Philology 
\item[ClQ] The Classical Quarterly 
\item[ClR] The Classical Review 
\item[ClRh] Clara Rhodos 
\item[CMatAOr] Contributi e materiali di archeologia orientale 
\item[CMGr] Convegni di studi sulla Magna Grecia 
\item[CMS] Corpus der minoischen und mykenischen Siegel 
\item[CoinHoards] Coin Hoards. The Royal Numismatic Society, London %*Abweichung!
\item[ColloquiSod] Colloqui del Sodalizio 
\item[CommunicAHung] Communicationes archaeologicae hungaricae 
\item[Complutum] Complutum. Publicaciones del Departamento de prehistoria de la Universidad complutense de Madrid 
\item[Conoscenze] Conoscenze. Rivista annuale della Soprintendenza archeologica e per i beni ambientali, architettonici, artistici e storici del Molise 
\item[Corduba] Corduba archaeologica 
\item[Corinth] Corinth. Results of Excavations Conducted by the American School of Classical Studies at Athens 
\item[CRAI] Académie des inscriptions et belles-lettres. Comptes rendus des séances de l'Académie 
\item[CretAnt] Creta antica. Rivista annuale di studi archeologici, storici ed epigrafici 
\item[CretSt] Cretan Studies 
\item[CronA] Cronache di archeologia 
\item[CronErcol] Cronache ercolanesi. Bollettino del Centro internazionale per lo studio dei papiri ercolanesi 
\item[CronPomp] Cronache pompeiane 
\item[CRPetersbourg] Compte-rendu de la Commission impériale archéologique, St. Pétersbourg %*Abweichung!
\item[CSE] Corpus speculorum Etruscorum 
\item[CSIR] Corpus signorum Imperii Romani 
\item[CSSpecPisa] Contributi della Scuola di specializzazione in archeologia dell'Università degli studi di Pisa 
\item[CuadAMed] Cuadernos de arqueología mediterránea 
\item[CuadArquitRom] Cuadernos de arquitectura romana 
\item[CuadCastellon] Cuadernos de prehistoria y arqueología castellonense 
\item[CuadCat] Cuaderni catanesi di studi classici e medievali 
\item[CuadFilCl] Cuadernos de filología clásica. Facultad de letras y filosofía, Universidad de Madrid 
\item[CuadGallegos] Cuadernos de estudios gallegos 
\item[CuadGranada] Cuadernos de prehistoria de la Universidad de Granada 
\item[CuadNavarra] Cuadernos de arqueología de la Universidad de Navarra 
\item[CuadPrehistA] Cuadernos de prehistoria y arqueología. Universidad autónoma de Madrid 
\item[CuadRom] Cuadernos de trabajos de la Escuela española de historia y arqueología en Roma 
\item[CuPaUAM] Cuadernos de prehistoria y arqueología. Universidad autónoma de Madrid 
\item[CVA] Corpus vasorum antiquorum 
\item[CZero] Cota Zero. Revista d'arqueologia i ciencia 
\item[DAA] Denkmäler antiker Architektur 
\item[Dacia] Dacia. Revue d'archéologie et d'histoire ancienne 
\item[DACL] Dictionnaire d'archéologie chrétienne et de liturgie 
\item[Dacoromania] Dacoromania. Jahrbuch für östliche Latinität 
\item[DaF] Damaszener Forschungen 
\item[Daidalos] Daidalos. Studi e ricerche del Dipartimento di scienze del mondo antico 
\item[DAIGeschDok] Das Deutsche Archäologische Institut. Geschichte und Dokumente 
\item[DaM] Damaszener Mitteilungen 
\item[Daremberg-Saglio] Dictionnaire des antiquités grecques et romaines d'après les textes et les monuments. Ouvrage rédigé sous la direction de Ch. Daremberg et E. Saglio %*Abweichung!
\item[DebrecMuzEvk] A Debreceni Déri múzeum évkönyve 
\item[Dedalo] Dédalo. Revista de arte e arqueologia %*Abweichung!
\item[Delos] Exploration archéologique de Délos faite par l'École française d'Athènes %*Abweichung!
\item[DeltChrA] Δελτίον της Χριστιανικής Αρχαιολογικής Εταιρείας 
\item[Demircihueyuek] Demircihüyük. Die Ergebnisse der Ausgrabungen 1975--1978 %*Abweichung!
\item[DeMuseus] De museus. Quaderns de museologia i museografia 
\item[DenkmPflBadWuert] Denkmalpflege in Baden-Württemberg %*Abweichung!
\item[DenkschrWien] Österreichische Akademie der Wissenschaften, Philosophisch-Historische Klasse. Denkschriften 
\item[Diadora] Diadora. Glasilo arheoloskog muzeja u Zadru 
\item[DialA] Dialoghi di archeologia 
\item[DialHistAnc] Dialogues d'histoire ancienne 
\item[Dike] Dike. Rivista di storia del diritto greco ed ellenistico. Università degli studi di Milano. Facoltà di giurisprudenza 
\item[Dioniso] Dioniso. Annale della Fondazione INDA, Istituto nazionale del dramma antico 
\item[DiskAB] Diskussionen zur archäologischen Bauforschung 
\item[DKuDenkmPfl] Deutsche Kunst und Denkmalpflege 
\item[DLZ] Deutsche Literaturzeitung für Kritik der internationalen Wissenschaft 
\item[DNP] Der Neue Pauly. Enzyklopädie der Antike 
\item[DocAlb] Documenta Albana 
\item[DocALouv] Documents d'archéologie régionale. Université catholique de Louvain 
\item[DocAMerid] Documents d'archéologie méridionale 
\item[DocEmRom] Documenti. Istituto per i beni artistici, culturali, naturali della regione Emilia-Romagna 
\item[Dodone] Δωδώνη 
\item[DOP] Dumbarton Oaks Papers 
\item[DossAlet] Les dossiers du Centre régional archéologique d'Alet 
\item[DossAParis] Les dossiers d'archéologie 
\item[Dura-Europos] The Excavations at Dura-Europos Conducted by Yale University and the French Academy of Inscriptions and Letters 
\item[EAA] Enciclopedia dell'arte antica classica e orientale 
\item[EAE] Excavaciones arqueológicas en España 
\item[EastWest] East and West 
\item[EcAntNimes] École antique de Nîmes. Bulletin annuel 
\item[EchosCl] Echos du monde classique. Classical Views 
\item[eDAI-F] e-Forschungsberichte des Deutschen Archäologischen Instituts 
\item[eDAI-J] e-Jahresberichte des Deutschen Archäologischen Instituts 
\item[EgA] Egyptian Archaeology. The Bulletin of the Egypt Exploration Society 
\item[Egnatia] Εγνατία. Επιστημονική Επετηρίδα της Φιλοσοφικής Σχολής, Αριστοτέλειο Πανεπιστήμιο Θεσσαλονίκης, Τμήμα Ιστορίας και Αρχαιολογίας 
\item[EgVicOr] Egitto e Vicino Oriente 
\item[Eikasmos] Εικασμός. Quaderni bolognesi di filologia classica 
\item[Eirene] Eirene. Studia Graeca et Latina 
\item[Elenchos] Elenchos. Rivista di studi sul pensiero antico 
\item[Ellenika] Ελληνικά. Φιλολογικόν, Ιστορικόν και Λαογραφικόν Περιοδικόν Σύγγραμμα 
\item[Emerita] Emerita. Revista de linguistica y filología clasica 
\item[EmPrerom] Emilia preromana 
\item[Empuries] Empúries. Revista de prehistòria, arqueologia i etnologia %*Abweichung!
\item[Enalia] Ενάλια 
\item[EnaliaAnn] Enalia. Annual. English Edition of the Hellenic Institute of Marine Archaeology 
\item[Enchoria] Enchoria. Zeitschrift für Demotistik und Koptologie 
\item[Eos] Eos. Commentarii Societatis philologae Polonorum 
\item[EpetBoiotMel] Επετηρίς της Εταιρείας Βοιωτικών Μελετών 
\item[EpetByzSpud] Επετηρίς της Εταιρείας Βυζαντινών Σπουδών 
\item[EpetKyklMel] Επετηρίς της Εταιρείας Κυκλαδικών Μελετών 
\item[EphemDac] Ephemeris Dacoromana. Annuario della Scuola Romena di Roma 
\item[EphemNapoc] Ephemeris Napocensis 
\item[EpigrAnat] Epigraphica Anatolica. Zeitschrift für Epigraphik und historische Geographie Anatoliens 
\item[EpistEpetAth] Επιστημονική Επετηρίς της Φιλοσοφικής Σχολής του Πανεπιστημίου Αθηνών 
\item[EpistEpetPolytThess] Επιστημονική Επετηρίδα της Πολυτεχνικής Σχολής, Αριστοτέλειο Πανεπιστήμιο Θεσσαλονίκης, Τμήμα Αρχιτεκτόνων 
\item[EpistEpetThess] Επιστημονική Επετηρίδα της Φιλοσοφικής Σχολής του Πανεπιστημίου Θεσσαλονίκης 
\item[EPRO] Études préliminaires aux religions orientales dans l'empire romain 
\item[Eranos] Eranos. Acta philologica Suecana 
\item[EranosJb] Eranos-Jahrbuch 
\item[Eretria] Eretria. Fouilles et recherches 
\item[Eretz-Israel] Eretz-Israel. Archaeological, Historical and Geographical Studies 
\item[Ergon] Το Έργον της Αρχαιολογικής Εταιρείας 
\item[ESA] Eurasia septentrionalis antiqua 
\item[EspacioHist] Espacio, tiempo y forma. Revista de la Facultad de geografia e historia. Serie 2, Historia antigua 
\item[EstMadr] Estudios de prehistoria y arqueología madrileñas 
\item[EstZaragoza] Estudios del Seminario de prehistoria, arqueología e historia antigua de la Facultad de filosofía y letras de Zaragoza 
\item[EtACl] Études d'archéologie classique 
\item[EtCl] Les études classiques. Revue trimestrielle de recherche et d'enseignement 
\item[EtClAix] Études classiques. Faculté des lettres et sciences humaines d'Aix 
\item[EtCret] Études crétoises 
\item[Ethnos] Ethnos. Revista do Instituto português de arqueología, história e etnografia 
\item[EtP] Études de papyrologie 
\item[EtPezenas] Études sur Pézenas et l'Hérault %*Abweichung!
\item[EtrSt] Etruscan Studies. Journal of the Etruscan Foundation 
\item[Etruscans] Etruscans. Bulletin of the Etruscan Foundation 
\item[EtTrav] Études et travaux. Studia i prace. Travaux du Centre d'archéologie méditerranéenne de l'Académie des sciences polonaise 
\item[Eulimene] Ευλιμένη (Μεσογειακή Αρχαιολογική Εταιρεία) 
\item[Eunomia] Eunomia. Ephemeridis Listy filologické supplementum 
\item[Euphrosyne] Euphrosyne. Revista de filologia clássica 
\item[EurAnt] Eurasia antiqua 
\item[EurRHist] European Review of History. Revue européenne d'histoire 
\item[Eutopia] Eutopia. Commentarii novi de antiquitatibus totius Europae 
\item[EVP] J. D. Beazley, Etruscan Vase Painting (Oxford 1947) 
\item[ExcIsr] Excavations and Surveys in Israel 
\item[Expedition] Expedition. The Magazine of Archaeology, Anthropology 
\item[ExtremA] Extremadura arqueológica 
\item[FA] Fasti archaeologici 
\item[FAAK] Forschungen zur Archäologie Außereuropäischer Kulturen 
\item[Faenza] Faenza. Bollettino del Museo internazionale delle ceramiche in Faenza. Rivista bimestrale di studi storici e di tecnica dell'arte ceramica 
\item[FAVA] Forschungen zur Allgemeinen und Vergleichenden Archäologie 
\item[Faventia] Faventia. Departement de clássiques, Facultat de letres, Universitat autónoma de Barcelona 
\item[FBerBadWuert] Forschungen und Berichte zur Vor- und Frühgeschichte in Baden-Württemberg %*Abweichung!
\item[FdC] Fouilles de Conimbriga. Publiées sous la direction de J. Alarcão et R. Etienne 
\item[FdD] Fouilles de Delphes 
\item[FdX] Fouilles de Xanthos 
\item[FeddersenWierde] Feddersen Wierde. Die Ergebnisse der Ausgrabung der vorgeschichtlichen Wurt Feddersen Wierde bei Bremerhaven in den Jahren 1955 bis 1963 %*Abweichung!
\item[FelRav] Felix Ravenna 
\item[FGrHist] F. Jacoby, Die Fragmente der griechischen Historiker 
\item[FHG] Fragmenta historicorum Graecorum 
\item[FiA] Forschungen in Augst 
\item[FichEpigr] Ficheiro epigráfico. Suplemento de »Conimbriga« 
\item[FiE] Forschungen in Ephesos 
\item[FIFAO] Fouilles de l'Institut français d'archéologie orientale du Caire 
\item[Figlina] Figlina. Documents du Laboratoire de céramologie de Lyon 
\item[Florentia] Florentia. Studi di archeologia 
\item[FlorIl] Florentia Iliberritana. Revista de estudios de antigüedad clásica 
\item[FMRD] Die Fundmünzen der römischen Zeit in Deutschland 
\item[FMROe] Fundmünzen der römischen Zeit in Österreich %*Abweichung!
\item[FoggArtMusAcqu] Fogg Art Museum. Acquisitions 
\item[FolA] Folia archaeologica 
\item[FolOr] Folia orientalia 
\item[Fonaments] Fonaments. Prehistòria i mon antic als Paisos Catalans 
\item[Fondamenti] Fondamenti. Rivista quadrimestrale di cultura 
\item[FontAPos] Fontes archaeologici Posnanienses 
\item[Fontes] Fontes. Rivista di filologia, iconografia e storia della tradizione classica 
\item[Forlimpopoli] Forlimpopoli. Documenti e studi 
\item[Fornvaennen] Fornvännen. Tidskrift för svensk antikvarisk forskning %*Abweichung!
\item[Forum] Forum. Revue du Groupe d'archéologie antique 
\item[FR] A. Furtwängler – K. Reichhold, Griechische Vasenmalerei (München 1900--1925) 
\item[FruehMitAltSt] Frühmittelalterliche Studien. Jahrbuch des Instituts für Frühmittelalterforschung der Universität Münster %*Abweichung!
\item[FuAusgrTrier] Funde und Ausgrabungen im Bezirk Trier 
\item[FuB] Forschungen und Berichte. Staatliche Museen zu Berlin 
\item[FuBerBadWuert] Fundberichte aus Baden-Württemberg %*Abweichung!
\item[FuBerHessen] Fundberichte aus Hessen 
\item[FuBerOe] Fundberichte aus Österreich %*Abweichung!
\item[FuBerSchwab] Fundberichte aus Schwaben 
\item[FuF] Forschungen und Fortschritte 
\item[FuWien] Fundort Wien. Berichte zur Archäologie 
\item[GacNum] Gaceta numismática 
\item[Gades] Gades. Revista del Colegio universitario de filosofía y letras 
\item[Gallaecia] Gallaecia. Publicación del Departamento de prehistoria y arqueología 
\item[Gallia] Gallia. Fouilles et monuments archéologiques en France metropolitaine 
\item[GalliaInf] Gallia informations. Préhistoire et histoire 
\item[GalliaInfAReg] Gallia informations. L'archéologie des régions 
\item[GalliaPrehist] Gallia préhistoire. Archéologie de la France préhistorique 
\item[GaR] Greece and Rome 
\item[GazBA] Gazette des beaux-arts 
\item[Genava] Genava. Revue d'histoire de l'art et d'archéologie 
\item[GeoAnt] Geographia antiqua. Rivista di geografia storica del mondo antico e di storia della geografia 
\item[Germania] Germania. Anzeiger der Römisch-Germanischen Kommission des Deutschen Archäologischen Instituts 
\item[Gesta] Gesta. International Center of Medieval Art 
\item[GettyMusJ] The J. Paul Getty Museum Journal 
\item[GFA] Göttinger Forum für Altertumswissenschaft 
\item[GGA] Göttingische Gelehrte Anzeigen 
\item[GiornFilFerr] Giornale filologico ferrarese 
\item[GiornItFil] Giornale italiano di filologia 
\item[GiornStorLun] Giornale storico della Lunigiana e del territorio lucense 
\item[GiRoccPalermo] Giglio di roccia 
\item[Gladius] Gladius. Estudios sobre armas antiguas, armamento, arte militar y vida cultural en Oriente y Occidente 
\item[GlasAJ] Glasgow Archaeological Journal 
\item[GlasBeograd] GlasnikSrpskoarheolokodrutvo 
\item[GlasSarajevo] Glasnik Zemaljskog muzeja Bosne i Hercegovie u Sarajevu. Arheologija 
\item[Glotta] Glotta. Zeitschrift für griechische und lateinische Sprache 
\item[Gnomon] Gnomon. Kritische Zeitschrift für die gesamte klassische Altertumswissenschaft 
\item[GodDepA] Godišnik na Departament archeologija 
\item[GodMuzPlov] Godišnik na Archeologičeski muzej Plovdiv. Annuaire du Musée archéologique Plovdiv 
\item[GodMuzSof] Godišnik na Nacionalnija archeologičeski muzej. Annuaire du Musée national archéologique (Sofia) 
\item[GodZborSkopje] Godišen zbornik na Filozofskiot fakultet na Universitetot vo Skopje 
\item[GorLet] Goriški letnik. Zbornik Goriškega muzeja 
\item[GoettMisz] Göttinger Miszellen. Beiträge zur ägyptologischen Diskussion %*Abweichung!
\item[GraRaspr] Građa i rasprave. Arheološki muzej Istre, Pula 
\item[GrazBeitr] Grazer Beiträge. Zeitschrift für die Klassische Altertumswissenschaft 
\item[GrLatOr] Graecolatina et orientalia. Zborník Filozofickej fakulty Univerzity Komenského 
\item[GrLatPrag] Graecolatina Pragensia. Acta Universitatis Carolinae. Philologica 
\item[GrRomByzSt] Greek, Roman and Byzantine Studies 
\item[Gymnasium] Gymnasium. Zeitschrift für Kultur der Antike und humanistische Bildung 
\item[Habis] Habis. Universidad de Sevilla. Arqueología, filología clásica 
\item[HallWPr] Hallisches Winckelmannsprogramm 
\item[Hama] Hama. Fouilles et recherches de la Fondation Carlsberg 
\item[HambBeitrA] Hamburger Beiträge zur Archäologie 
\item[HambBeitrNum] Hamburger Beiträge zur Numismatik 
\item[Handlingar] Kungliga vitterhets historie och antikvitets akademiens handlingar. Antikvariska serien 
\item[HarvStClPhil] Harvard Studies in Classical Philology 
\item[HarvTheolR] The Harvard Theological Review 
\item[HASB] Hefte des Archäologischen Seminars der Universität Bern 
\item[HAW] Handbuch der Altertumswissenschaften 
\item[HdArch] Handbuch der Archäologie 
\item[Head] B. V. Head, Historia Numorum. A Manual of Greek Numismatics (Oxford 1887; 1911) 
\item[Helbig] W. Helbig, Führer durch die öffentlichen Sammlungen klassischer Altertümer in Rom 
\item[Helike] Helike. Universidad nacional de educación a distancia, Centro regional de Elche 
\item[Helikon] Helikon. Rivista di tradizione e cultura classica 
\item[Helinium] Helinium. Revue consacrée à l'archéologie des Pays-Bas, de la Belgique et du Grand-Duché de Luxembourg 
\item[Helios] Helios. A Journal Devoted to Critical and Methodological Studies of Classical Culture, Literature and Society 
\item[HellenikaJb] Hellenika. Jahrbuch für die Freunde Griechenlands 
\item[HelvA] Helvetia archaeologica 
\item[Hephaistos] Hephaistos. Kritische Zeitschrift zur Theorie und Praxis der Archäologie und angrenzender Wissenschaften 
\item[Hermes] Hermes. Zeitschrift für klassische Philologie 
\item[Herrscherbild] Das römische Herrscherbild 
\item[Hesperia] Hesperia. Journal of the American School of Classical Studies at Athens 
\item[Hispania] Hispania. Revista española de historia 
\item[HispAnt] Hispania antiqua. Revista de historia antigua 
\item[HispAntEpigr] Hispania antiqua epigraphica 
\item[HispEpigr] Hispania epigraphica 
\item[HistAnthr] Historische Anthropologie. Kultur, Gesellschaft, Alltag 
\item[HistArt] Histoire de l'art. Bulletin d'information de l'Institut National d'Histoire de l'Art 
\item[Historia] Historia. Zeitschrift für Alte Geschichte 
\item[Historica] Historica. Academia RSR. Centrul de istorie, filologie şi etnografie din Craiova 
\item[Histria] Histria. Les résultats des fouilles 
\item[HistriaA] Histria archaeologica 
\item[HistriaAnt] Histria antiqua. Casopis Meunarodnog Istraivakog Centra za Arheologiju. Journal of the International Research Centre for Archeology 
\item[HistSprF] Historische Sprachforschung 
\item[HKL] R. Borger, Handbuch der Keilschriftliteratur (Berlin 1967--1975) 
\item[Horos] Ηόρος. Ἔνα Ἀρχαιογνωστικò Περιοδικό 
\item[HSS] Harvard Semitic Series 
\item[HuelvaA] Huelva arqueológica 
\item[HumBild] Humanistische Bildung 
\item[Hyp] Hyperboreus. Studia classica 
\item[HZ] Historische Zeitschrift 
\item[IA] Iberia archaeologica 
\item[Iberia] Iberia. Revista della antigüedad 
\item[IEJ] Israel Exploration Journal 
\item[IG] Inscriptiones Graecae 
\item[IGCH] M. Thompson – C. M. Kraay – O. Mørkholm, An Inventory of Greek Coin Hoards (New York 1973) 
\item[IGR] Inscriptiones Graecae ad res Romanas pertinentes 
\item[IK] Inschriften griechischer Städte aus Kleinasien 
\item[Ilerda] Ilerda. Instituto de estudios ilerdenses 
\item[Iliria] Iliria. Revistë arkeologjike 
\item[IllinClSt] Illinois Classical Studies 
\item[ILN] The Illustrated London News 
\item[ILS] H. Dessau, Inscriptiones Latinae selectae (Berlin 1892--1916) 
\item[IndexQuad] Index. Quaderni camerti di studi romanistici 
\item[IndogermF] Indogermanische Forschungen 
\item[IndUnArtB] Indiana University Art Museum Bulletin 
\item[InsFulc] Insula Fulcheria 
\item[InstNautAQ] The Institute of Nautical Archaeology Quarterly 
\item[IntJClTrad] International Journal of the Classical Tradition 
\item[IntJNautA] International Journal of Nautical Archaeology 
\item[IntZSchauBibelWiss] Internationale Zeitschriftenschau für Bibelwissenschaft und Grenzgebiete 
\item[InvLuc] Invigilata lucernis 
\item[Ipek] Jahrbuch für prähistorische und ethnographische Kunst 
\item[Iran] Iran. Journal of the British Institute of Persian Studies 
\item[IrAnt] Iranica antiqua 
\item[IsrMusJ] The Israel Museum Journal 
\item[IsrMusN] The Israel Museum News 
\item[IsrMusStA] Israel Museum Studies in Archaeology. An Annual Publication by the Samuel Bronfman Biblical and Archaeological Museum of the Israel Museum, Jerusalem 
\item[IsrNumJ] Israel Numismatic Journal 
\item[IstanbAMuezYil] Istanbul Arkeoloji Müzeleri yıllığı %*Abweichung!
\item[IstForsch] Istanbuler Forschungen 
\item[Isthmia] Isthmia. Excavations by the University of Chicago under the Auspices of the American School of Classical Studies at Athens 
\item[IstMitt] Istanbuler Mitteilungen 
\item[Italica] Italica. Cuadernos de trabajos de la Escuela española de historia y arqueología en Roma 
\item[ItNostr] Italia nostra 
\item[IzvBurgas] Izvestija na Narodnija muzej Burgas. Bulletin du Musée national de Bourgas 
\item[IzvMuzJuzBalg] Izvestija na muzeite ot Južna Bălgarija. Bulletin des musées de la Bulgarie du Sud %*Abweichung
\item[IzvVarna] Izvestija na Narodnija muzej Varna 
\item[Jabega] Jábega. Revista de la Disputación provincial de Málaga %*Abweichung!
\item[JadrZbor] Jadranski zbornik. Prilozi za povijest Istre, Rijeke i Hrvatskog primorja 
\item[JAOS] Journal of the American Oriental Society 
\item[JARCE] Journal of the American Research Center in Egypt 
\item[JASc] Journal of Archaeological Science 
\item[JbAC] Jahrbuch für Antike und Christentum 
\item[JbAkMainz] Jahrbuch. Akademie der Wissenschaften und der Literatur, Mainz 
\item[JbBadWuert] Jahrbuch der Staatlichen Kunstsammlungen in Baden-Württemberg %*Abweichung!
\item[JbBerlMus] Jahrbuch der Berliner Museen 
\item[JbBernHistMus] Jahrbuch des Bernischen Historischen Museums in Bern 
\item[JberAugst] Jahresberichte aus Augst und Kaiseraugst 
\item[JberBasel] Jahresbericht der Archäologischen Bodenforschung des Kantons Basel-Stadt 
\item[JberBayDenkmPfl] Jahresbericht der Bayerischen Bodendenkmalpflege 
\item[JberProVindon] Jahresbericht. Gesellschaft Pro Vindonissa 
\item[JberVgFrankf] Jahresbericht des Instituts für Vorgeschichte der Universität Frankfurt a. M. 
\item[JberZuerich] Jahresbericht. Schweizerisches Landesmuseum Zürich %*Abweichung!
\item[JbGoett] Jahrbuch der Akademie der Wissenschaften in Göttingen %*Abweichung!
\item[JbHambKuSamml] Jahrbuch der Hamburger Kunstsammlungen 
\item[JbKHMWien] Jahrbuch des Kunsthistorischen Museums Wien 
\item[JbKHSWien] Jahrbuch der Kunsthistorischen Sammlungen in Wien 
\item[JbKleinasF] Jahrbuch für kleinasiatische Forschung 
\item[JbMuench] Bayerische Akademie der Wissenschaften. Jahrbuch %*Abweichung!
\item[JbMusKGHamb] Jahrbuch des Museums für Kunst und Gewerbe, Hamburg 
\item[JbMusLinz] Jahrbuch des Oberösterreichischen Musealvereins 
\item[JbOeByz] Jahrbuch der Österreichischen Byzantinistik %*Abweichung!
\item[JbPreussKul] Jahrbuch Preussischer Kulturbesitz 
\item[JbRGZM] Jahrbuch des Römisch-Germanischen Zentralmuseums Mainz 
\item[JbSchwUrgesch] Jahrbuch der Schweizerischen Gesellschaft für Ur- und Frühgeschichte 
\item[JCS] Journal of Cuneiform Studies 
\item[JdI] Jahrbuch des Deutschen Archäologischen Instituts 
\item[JEA] The Journal of Egyptian Archaeology 
\item[JEChrSt] Journal of Early Christian Studies. Journal of the North American Patristics Society 
\item[JEOL] Jaarbericht van het Vooraziatisch-Egyptisch Genootschap Ex Oriente Lux 
\item[JewelSt] Jewellery Studies 
\item[JFieldA] Journal of Field Archaeology 
\item[JGS] Journal of Glass Studies 
\item[JHS] The Journal of Hellenic Studies 
\item[JIbA] Journal of Iberian Archaeology 
\item[JJurP] The Journal of Juristic Papyrology 
\item[JKuGesch] Journal für Kunstgeschichte 
\item[JMedA] Journal of Mediterranean Archaeology 
\item[JMedAnthrA] Journal of Mediterranean Anthropology and Archaeology 
\item[JMithrSt] Journal of Mithraic Studies 
\item[JNES] Journal of Near Eastern Studies 
\item[JNG] Jahrbuch für Numismatik und Geldgeschichte 
\item[JPrehistRel] Journal of Prehistoric Religion 
\item[JRA] Journal of Roman Archaeology 
\item[JRomMilSt] Journal of Roman Military Equipment Studies 
\item[JRomPotSt] Journal of Roman Pottery Studies 
\item[JRS] The Journal of Roman Studies 
\item[JSav] Journal des savants 
\item[JSchrVgHalle] Jahresschrift für mitteldeutsche Vorgeschichte 
\item[JSS] Journal of Semitic Studies 
\item[JTheorA] Journal of Theoretical Archaeology 
\item[Jura] Iura. Rivista internazionale di diritto romano e antico 
\item[JWaltersArtGal] The Journal of the Walters Art Gallery 
\item[JWCI] Journal of the Warburg and Courtauld Institutes 
\item[Kadmos] Kadmos. Zeitschrift für vor- und frühgriechische Epigraphik 
\item[Kairos] Kairos. Zeitschrift für Judaistik und Religionswissenschaft 
\item[Kalapodi] Kalapodi. Ergebnisse der Ausgrabungen im Heiligtum der Artemis und des Apollon von Hyampolis in der antiken Phokis 
\item[Kalathos] Kalathos. Revista del Seminario de arqueología y etnología Turolense, Universidad de Teruel 
\item[Kalos] Kalós. Arte in Sicilia %*Abweichung!
\item[Karthago] Karthago. Revue d'archéologie mediterranéenne 
\item[Kemi] Kêmi. Revue de philologie et d'archéologie égytiennes et coptes %*Abweichung!
\item[Kenchreai] Kenchreai. Eastern Port of Corinth. Results of Investigations by the University of Chicago and Indiana University for the American School of Classical Studies at Athens 
\item[KentAR] Kent Archaeological Review 
\item[Keos] Keos. Results of Excavations Conducted by the University of Cincinnati under the Auspices of the American School of Classical Studies at Athens 
\item[Kerameikos] Kerameikos. Ergebnisse der Ausgrabungen 
\item[Kernos] Kernos. Revue internationale et pluridisciplinaire de religion grecque antique 
\item[Klearchos] Klearchos. Bollettino dell'Associazione Amici del Museo nazionale di Reggio Calabria 
\item[Kleos] Kleos. Estemporaneo di studi e testi sulla fortuna dell'antico 
\item[Klio] Klio. Beiträge zur alten Geschichte 
\item[Kodai] Kodai. Journal of Ancient History 
\item[KoelnJb] Kölner Jahrbuch %*Abweichung!
\item[KoelnMusB] Kölner Museums-Bulletin. Berichte und Forschungen aus den Museen der Stadt Köln %*Abweichung!
\item[Kokalos] Kώκαλος. Studi pubblicati dall’Istituto di storia antica dell’Università di Palermo %*Abweichung!
\item[KollAVA] Kolloquien zur Allgemeinen und Vergleichenden Archäologie 
\item[Kratylos] Kratylos. Kritisches Berichts- und Rezensionsorgan für indogermanische und allgemeine Sprachwissenschaft 
\item[KretChron] Κρητικά Χρονικά 
\item[KSIA] Kratkie soobščenija o dokladach i polevych issledovanijach Instituta archeologii 
\item[KSIAKiev] Kratkie soobščenija Instituta archeologii, Kiev 
\item[KST] Kazı Sonuçları Toplantısı 
\item[Ktema] Ktema. Civilisations de l'Orient, de la Grèce et de Rome antiques 
\item[KuGeschAnz] Kunstgeschichtliche Anzeigen 
\item[Kuml] Kuml. Årbog for Jysk Arkaeologisk Selskab 
\item[Kunstchronik] Kunstchronik. Monatsschrift für Kunstwissenschaft, Museumswesen und Denkmalpflege, Mitteilungsblatt des Verbandes Deutscher Kunsthistoriker 
\item[KuOr] Kunst des Orients 
\item[Kush] Kush. Journal of the National Corporation for Antiquities and Museums (NCAM) 
\item[KuWeltBerlMus] Kunst der Welt in den Berliner Museen 
\item[KypA] Κυπριακή Αρχαιολογία. Archaeologia Cypria 
\item[KypSpud] Κυπριακαì Σπουδαί 
\item[Labeo] Labeo. Rassegna di ritritto romano 
\item[LAe] Lexikon der Ägyptologie %*Abweichung!
\item[Laietania] Laietania. Estudios d'arqueologia del Maresme 
\item[Lampas] Lampas. Tijdschrift voor nederlandse classici 
\item[Lancia] Lancia. Revista de prehistoria, arqueología e historia antigua del noreste peninsular 
\item[LandKunVierJBl] Landeskundliche Vierteljahrsblätter. Trier 
\item[LangOrAnc] Langues orientales anciennes. Philologie et linguistique 
\item[Latinitas] Latinitas. Commentarii linguae Latinae excolendae provehendae 
\item[Latomus] Latomus. Revue d'études latines 
\item[Laverna] Laverna. Beiträge zur Wirtschafts- und Sozialgeschichte der Alten Welt 
\item[LCS] A. D. Trendall, The Red-figured Vases of Lucania, Campania and Sicily (Oxford 1967--­1983) 
\item[Levant] Levant. Journal of the British School of Archaeology in Jerusalem and the British Institute at Amman for Archaeology and History 
\item[Lexis] Lexis. Poetica, retorica e comunicazione nella tradizione classica 
\item[LF] Listy filologické 
\item[LibSt] Libyan Studies 
\item[LibyaAnt] Libya antiqua 
\item[LibycaBServAnt] Libyca. Bulletin du Service des antiquités. Archéologie, épigraphie 
\item[LibycaTrav] Libyca. Travaux du Laboratoire d'anthropologie et d'archéologie préhistorique du Musée du Bardo 
\item[LSJ] G. Liddell – R. Scott – H. S. Jones, A Greek-English Lexikon \textsuperscript{9}(1996); Suppl. (1996) %*Abweichung!
\item[LIMC] Lexikon iconographicum mythologiae classicae 
\item[Limesforschungen] Limesforschungen. Studien zur Organisation der römischen Reichsgrenze an Rhein und Donau 
\item[Lindos] Lindos. Fouilles et recherches 
\item[LingIt] Linguistica, epigrafia, filologia italica 
\item[LTUR] Lexikon topographicum urbis Romae 
\item[Lucentum] Lucentum. Anales de la Universidad de Alicante. Prehistoria, arqueología e historia antigua 
\item[LundAR] Lund Archaeological Review 
\item[Lustrum] Lustrum. Internationale Forschungsberichte aus dem Bereich des klassischen Altertums 
\item[Lykia] Lykia. Anadolu-akdeniz kültürleri 
\item[MacActaA] Macedoniae acta archaeologica 
\item[Maecenas] Maecenas. Studi sul mondo classico 
\item[MAGesGraz] Mitteilungen der Archäologischen Gesellschaft Graz 
\item[MAGesStei] Mitteilungen der Archäologischen Gesellschaft Steiermark 
\item[Maia] Maia. Rivista di letterature classiche 
\item[Mainake] Mainake. Estudios de arqueología Malagueña 
\item[MAInstUngAk] Mitteilungen des Archäologischen Instituts der Ungarischen Akademie der Wissenschaften 
\item[MainzZ] Mainzer Zeitschrift 
\item[MakedNasl] Makedonsko nasledstvo. Spisanie za arheologija, istorija, istorija na umetnosta i etnologija 
\item[Makedonika] Μακεδονικά. Σύγγραμμα Περιοδικόν της Εταιρείας Μακεδονικών Σπουδών 
\item[MAMA] Monumenta Asiae Minoris antiqua. Publications of the American Society for Archaeological Research in Asia Minor 
\item[MAnthrWien] Mitteilungen der Anthropologischen Gesellschaft in Wien 
\item[MARo] Monumenta artis Romanae 
\item[MarbWPr] Marburger Winckelmann-Programm 
\item[Marche] Le Marche. Archeologia, storia, territorio 
\item[Mari] Mari. Annales de recherches interdisciplinaires 
\item[Marisia] Marisia. Studii şi materiale. Arheologie, istorie, etnografie 
\item[MarNero] Il Mar Nero. Annali di archeologia e storia 
\item[Marsyas] Marsyas. Studies in the History of Art 
\item[MascaJ] Masca Journal. Museum Applied Science Center for Archaeology, University of Pennsylvania 
\item[MascaP] Masca Research Papers in Science and Archaeology 
\item[Mastia] Mastia. Revista del Museo arqueológico municipal de Cartagena 
\item[MatABSSR] Materialy po archeologii BSSR 
\item[MatASevPri] Materialy po archeologii severnogo Pričernomor’ja 
\item[MatCercA] Materiale şi cercetări arheologice 
\item[MatIsslA] Materialy i issledovanija po archeologii SSSR 
\item[MatStar] Materiały starożytne 
\item[MatStarWczes] Materiały starożytne i wczesnośredniowieczne 
\item[MatTestiCl] Materiali e discussioni per l'analisi dei testi classici 
\item[MatWczes] Materiały wczesnośredniowieczne 
\item[MAVA] Materialien zur Allgemeinen und Vergleichenden Archäologie 
\item[MB] Madrider Beiträge 
\item[MBAH] Marburger Beiträge zur antiken Handels-, Wirtschafts- und Sozialgeschichte, vor Jahrgang 2009 Münstersche Beiträge zur antiken Handelsgeschichte 
\item[MBlVFruehGesch] Mitteilungsblatt für Vor- und Frühgeschichte %*Abweichung!
\item[MDAIK] Mitteilungen des Deutschen Archäologischen Instituts, Abteilung Kairo 
\item[MDAVerb] Mitteilungen des Deutschen Archäologen-Verbandes e.V. 
\item[MdI] Mitteilungen des Deutschen Archäologischen Instituts 
\item[MDOG] Mitteilungen der Deutschen Orient-Gesellschaft zu Berlin 
\item[Meander] Meander. Miesięcznik poświęcony kulturze świata starożytnego 
\item[MedA] Mediterranean Archaeology 
\item[MedAnt] Mediterraneo antico. Economie, società, culture 
\item[MeddelGlypt] Meddelelser fra Ny Carlsberg Glyptotek 
\item[MeddelLund] Meddelanden från Lunds universtitets historiska museum 
\item[MeddelThor] Meddelelser fra Thorvaldsens Museum 
\item[MededRom] Mededelingen van het Nederlands instituut te Rome 
\item[MedelhavsMusB] Medelhavsmuseet. Bulletin 
\item[MedHistR] Mediterranean Historical Review 
\item[MedievA] Medieval Archaeology. Journal of the Society for Medieval Archaeology 
\item[MediSec] Medicina nei secoli. Arte e scienza 
\item[MEFRA] Mélanges de l'École française de Rome. Antiquité 
\item[MelBeyrouth] Mélanges de l'Université Saint-Joseph 
\item[MelCasaVelazquez] Mélanges de la Casa de Velázquez. Antiquité et moyen âge 
\item[MemAcInscr] Mémoires de l'Académie des inscriptions et belles-lettres 
\item[MemAmAc] Memoirs of the American Academy in Rome 
\item[MemAnt] Memoria antiquitatis. Acta Musei Petrodavensis. Revista Muzeului arheologic Piatra Neamţ 
\item[MemAntFr] Mémoires de la Société nationale des antiquaires de France 
\item[MemBarcelA] Memoria. Universidad de Barcelona, Instituto de arqueología y prehistoria 
\item[MemBologna] Atti de la Accademia delle scienze dell'Istituto di Bologna. Memorie 
\item[MemHistAnt] Memorias de historia antigua (Universidad de Oviedo) 
\item[MemInstNatFr] Mémoires de l'Institut national de France 
\item[MemLinc] Atti dell'Accademia nazionale dei Lincei, Classe di scienze morali, storiche e filologiche. Memorie 
\item[MemNap] Memorie dell'Accademia di archeologia, lettere e belle arti di Napoli 
\item[MemPontAc] Atti della Pontificia accademia romana di archeologia. Memorie 
\item[MemStor] Memoria storica. Rivista del Centro studi storici terni 
\item[MemStorFriuli] Memorie storiche forogiuliesi 
\item[Merida] Mérida. Ciudad y patrimonio %*Abweichung!
\item[MeridaMem] Mérida. Excavaciones arqueológicas. Memoria %*Abweichung!
\item[Meroitica] Meroitica. Schriften zur altsudanesischen Geschichte und Archäologie 
\item[Mesopotamia] Mesopotamia. Rivista di archeologia 
\item[Messana] Messana. Rassegna di studi filologici, linguistici e storici 
\item[Metis] Métis. Revue d'anthropologie du monde grec ancien %*Abweichung!
\item[MetrMusJ] Metropolitan Museum Journal 
\item[MetrMusSt] Metropolitan Museum Studies 
\item[MF] Madrider Forschungen 
\item[MFruehChrOe] Mitteilungen zur frühchristlichen Archäologie in Österreich %*Abweichung!
\item[Milet] Milet. Ergebnisse der Ausgrabungen und Untersuchungen seit dem Jahr 1899 
\item[MilForsch] Milesische Forschungen 
\item[MinEpigrP] Minima epigraphica et papyrologica. Taccuini della cattedra e del laboratorio di epigrafia e papirologia giuridica dell'Università degli studi di Catanzaro »Magna Graecia« 
\item[Minerva] Minerva. Revista de filología clásica 
\item[Minos] Minos. Revista de filología egea 
\item[MInstWasser] Mitteilungen. Leichtweiss-Institut für Wasserbau der Technischen Universität Braunschweig 
\item[MIO] Mitteilungen des Instituts für Orientforschung 
\item[MiscCrAnt] Miscellanea di studi di letteratura christiana antica 
\item[MiscStStor] Miscellanea di studi storici 
\item[MitChrA] Mitteilungen zur christlichen Archäologie 
\item[MKT] Menschen – Kulturen – Traditionen. Studien aus den Forschungsclustern des Deutschen Archäologischen Instituts 
\item[MKuHistFlorenz] Mitteilungen des Kunsthistorischen Institutes in Florenz 
\item[MKul] Mitteilungen zur Kulturkunde 
\item[MM] Madrider Mitteilungen 
\item[Mnemosyne] Mnemosyne. A Journal of Classical Studies 
\item[MOeNumGes] Mitteilungen der Österreichischen Numismatischen Gesellschaft %*Abweichung!
\item[MonAnt] Monumenti antichi 
\item[MonInst] Monumenti inediti pubblicati dall'Istituto Archeologico 
\item[MonPiot] Monuments et mémoires. Fondation E. Piot 
\item[MonPitt] Monumenti della pittura antica scoperti in Italia 
\item[Mozia] Mozia. Rapporto preliminare della missione congiunta con la Soprintendenza alle antichità della Sicilia occidentale 
\item[MPraehistKomWien] Mitteilungen der Prähistorischen Kommission der Österreichischen Akademie der Wissenschaften %*Abweichung!
\item[MSAtene] Monografie della Scuola archeologica di Atene e delle missioni italiane in Oriente 
\item[MSchliemann] Mitteilungen aus dem Heinrich-Schliemann-Museum Ankershagen 
\item[MSchwUrFruehGesch] Mitteilungsblatt der Schweizerischen Gesellschaft für Ur- und Frühgeschichte %*Abweichung!
\item[MSpaetAByz] Mitteilungen zur spätantiken Archäologie und byzantinischen Kunstgeschichte %*Abweichung!
\item[MueJb] Münchner Jahrbuch der bildenden Kunst %*Abweichung!
\item[MuenchBeitrVFG] Münchner Beiträge zur Vor- und Frühgeschichte %*Abweichung!
\item[MuenchStSprWiss] Münchener Studien zur Sprachwissenschaft %*Abweichung!
\item[MuM] Münzen und Medaillen 
\item[MusAfr] Museum Africum. West African Journal of Classical and Related Studies 
\item[MusBenaki] Μουσείο Μπενάκη 
\item[MusCrit] Museum criticum 
\item[Muse] Muse. Annual of the Museum of Art and Archaeology, University of Missouri, Columbia 
\item[Museon] Le Muséon. Revue d'études orientales %*Abweichung!
\item[MuseumUnesco] Museum. A Quarterly Review Published by UNESCO 
\item[MusFerr] Musei Ferraresi. Bollettino annuale 
\item[MusGalIt] Musei e gallerie d'Italia 
\item[MusHaaretz] Museum Haaretz, Tel Aviv. Yearbook 
\item[MusHelv] Museum Helveticum 
\item[MusKoeln] Museen in Köln. Bulletin %*Abweichung!
\item[MusNotAmNumSoc] Museum Notes. The American Numismatic Society 
\item[MusPontevedra] El Museo de Pontevedra 
\item[MusRiv] Museo in rivista. Notiziario dei Musei civici di Pavia 
\item[MusTusc] Museum Tusculanum. København 
\item[MuzEvkSzeged] A Móra Ferenc múzeum évkönyve 
\item[MuzNat] Muzeul naţional. Bucureşti
\item[MuzPamKul] Muzei i pametnici na kulturata 
\item[NachrArbUWA] Nachrichtenblatt Arbeitskreis Unterwasserarchäologie 
\item[NapNobil] Napoli nobilissima 
\item[NassAnn] Nassauische Annalen 
\item[NAWG] Nachrichten der Akademie der Wissenschaften in Göttingen. Philologisch-Historische Klasse 
\item[NBWorcArtMus] News Bulletin and Calendar. Worcester Art Museum 
\item[NEphemSemEpigr] Neue Ephemeris für semitische Epigraphik 
\item[NewsletterAthen] Newsletter of the Netherlands Institute at Athens 
\item[NewsletterPotTech] Newsletter. Department of Pottery Technology, University of Leiden 
\item[NGWG] Nachrichten von der Gesellschaft der Wissenschaften zu Göttingen. Philologisch-Historische Klasse 
\item[NigCl] Nigeria and the Classics 
\item[Nikephoros] Nikephoros. Zeitschrift für Sport und Kultur im Altertum 
\item[Nin] Nin. Journal of Gender Studies in Antiquity 
\item[NNM] American Numismatic Society. Numismatic Notes and Monographs 
\item[NomChron] Νομισματικά Χρονικά. Περιοδική Έκδωσις της Ελληνικής Νομισματικής Εταιρείας 
\item[Norba] Norba. Revista de arte, geografía e historia 
\item[NordNumArs] Nordisk numismatisk årsskrift. Utgiven av Kungliga vitterhets historie och antikvitets akademien i samarbete med Nordisk numismatisk union 
\item[NotABerg] Notizie archeologiche bergomensi. Periodico di archeologia del Civico museo archeologico di Bergamo 
\item[NotAHisp] Noticiario arqueológico hispánico. Arqueología 
\item[NotAHispPrehistoria] Noticiario arqueológico hispánico. Prehistoria 
\item[NotAllumiere] Notiziario. Museo civico, Associazione archeologica, Allumiere 
\item[NotALomb] Notiziario. Soprintendenza archeologica della Lombardia 
\item[NotMilano] Notizie dal chiostro del monastero maggiore. Rassegna di studi del Civico museo archeologico e del Civico gabinetto numismatico di Milano 
\item[NouvClio] La nouvelle Clio 
\item[Novaensia] Novaensia. Badania Ekspedycji archeologicznej Uniwersytetu warszawskiego w Novae 
\item[NSc] Notizie degli scavi di antichità 
\item[NStFan] Nuovi studi fanesi 
\item[NubChr] Nubia christiana 
\item[NubLet] Nubian Letters 
\item[NueBlA] Nürnberger Blätter zur Archäologie. Publikationsreihe des Bildungszentrums der Stadt Nürnberg, Fachbereich Archäologie %*Abweichung!
\item[NumAntCl] Numismatica e antichità classiche. Quaderni ticinesi 
\item[Numantia] Numantia. Investigaciones arqueológicas en Castilla y León 
\item[NumChron] The Numismatic Chronicle. The Journal of the Royal Society 
\item[Numen] Numen. International Review for the History of Religions 
\item[NumEpigr] Numizmatika i epigrafika 
\item[Numisma] Numisma. Revista de la Sociedad iberoamericana de estudios numismáticos 
\item[NumismaticaRom] Numismatica. Periodico di cultura e di informazione numismatica 
\item[Numizmaticar] Numizmatičar. Casopis za anticki i stari jugoslavenski novac %*Abweichung!
\item[Nummus] Nummus. Sociedade portuguesa de numismatica 
\item[NumZ] Numismatische Zeitschrift 
\item[NuovDidask] Nuovo Didaskaleion 
\item[OccasPublClSt] Occasional Publications in Classical Studies 
\item[OccOr] Occident and Orient 
\item[OGIS] W. Dittenberger, Orientis Graeci inscriptiones selectae (Leipzig 1903--­1905) 
\item[OeJh] Jahreshefte des Österreichischen Archäologischen Institutes in Wien %*Abweichung!
\item[OF] Olympische Forschungen 
\item[Offa] Offa. Berichte und Mitteilungen zur Urgeschiche, Frühgeschichte und Mittelalterarchäologie 
\item[Ogam] Ogam. Bulletin des Amis de la tradition celtique 
\item[Oikumene] Oikumene. Studia ad historiam antiquam classicam et orientalem pertinentia 
\item[OIP] Oriental Institute Publications 
\item[Olba] Olba. Mersin Üniversitesi Kilikia Arkeolojisini Araştırma Merkezi yayınları 
\item[OlBer] Bericht über die Ausgrabungen in Olympia 
\item[Olympia] Olympia. Die Ergebnisse der von dem Deutschen Reich veranstalteten Ausgrabung (Berlin 1890--­1897) 
\item[Olynthus] Excavations at Olynthus 
\item[OLZ] Orientalistische Literaturzeitung 
\item[OpArch] Skrifter utgivna av Svenska institutet i Rom. Opuscula archaeologica 
\item[OpAth] Opuscula Atheniensia 
\item[OpFin] Opuscula Instituti Romani Finlandiae 
\item[OpPomp] Opuscula Pompeiana 
\item[OpRom] Opuscula Romana 
\item[Opus] Opus. Rivista internazionale per la storia economica e sociale dell’antichità 
\item[Or] Orientalia (Pontificio Istituto biblico) 
\item[OrA] Orient-Archäologie 
\item[OrAnt] Oriens antiquus. Rivista del Centro per le antichità e la storia dell’arte del Vicino Oriente 
\item[OrbTerr] Orbis terrarum. Internationale Zeitschrift für historische Geographie der Alten Welt 
\item[OrChr] Oriens christianus 
\item[OrChrPer] Orientalia christiana periodica 
\item[Ordona] Ordona. Rapports et études 
\item[Orient] Orient. The Reports of the Society for Near Eastern Studies in Japan 
\item[Origini] Origini. Preistoria e protostoria delle civiltà antiche 
\item[Orizzonti] Orizzonti. Rassegna di archeologia 
\item[Orpheus] Orpheus. Rivista di umanità classica e cristiana 
\item[OrpheusThracSt] Orpheus. Journal of Indo-European, Palaeo-Balkan and Thracian Studies 
\item[OrSu] Orientalia Suecana. An International Journal of Indological, Iranian, Semitic, Turkic Studies 
\item[OsjZbor] Osjekizbornik 
\item[OstbGrenzm] Ostbairische Grenzmarken. Passauer Jahrbuch für Geschichte, Kunst und Volkskunde 
\item[Ostraka] Ostraka. Rivista di antichità 
\item[OudhMeded] Oudheidkundige mededelingen uit het Rijksmuseum van oudheden te Leiden 
\item[OxfJA] Oxford Journal of Archaeology 
\item[OxfStPhilos] Oxford Studies in Ancient Philosophy 
\item[Pact] Pact. Revue du Groupe européen d’études pour les techniques physiques, chimiques et mathématiques appliquées à l’archéologie 
\item[Padusa] Padusa. Bolletino del Centro polesano di studi storici, archeologici ed etnografici, Rovigo 
\item[PagA] Pagine di archeologia. Studi e materiali 
\item[PAI] Polevye archeologičeskie issledovanija 
\item[Paideuma] Paideuma. Mitteilungen zur Kulturkunde 
\item[Palaeohistoria] Palaeohistoria. Acta et communicationes Instituti bio-archaeologici universitatis Groninganae 
\item[Paleohispanica] Paleohispánica. Revista sobre lenguas y culturas de la Hispania antigua %*Abweichung!
\item[Palladio] Palladio. Rivista di storia dell’architettura 
\item[Pallas] Pallas. Revue d’études antiques 
\item[Palmet] Palmet. Sadberk Hanım Müzesi yıllığı 
\item[PamA] Památky archeologické 
\item[Pan] Pan. Studi dell’Istituto di filologia latina 
\item[PapBilb] Papeles Bilbilitanos 
\item[Papyri] Papyri. Bollettino del Museo del papiro 
\item[Parthica] Parthica. Incontri di culture nel mondo antico 
\item[Partenope] Partenope. Rivista trimestrale di cultura napoletana 
\item[PAS] Prähistorische Archäologie in Südosteuropa 
\item[PBF] Prähistorische Bronzefunde 
\item[Pegasus] Pegasus. Berliner Beiträge zum Nachleben der Antike 
\item[PEQ] Palestine Exploration Quarterly 
\item[Peristil] Peristil. Zbornik radova za povijest umjetnosti 
\item[Persica] Persica. Jaarboek van het Genootschap Nederland-Iran, Stichting voor culturele betrekkingen 
\item[Peuce] Peuce. Studii şi comunicări de istorie veche, arheologie şi numismatică 
\item[PF] Pergamenische Forschungen 
\item[Pharos] Pharos. Journal of the Netherlands Institute at Athens 
\item[Philologus] Philologus. Zeitschrift für das klassische Altertum 
\item[Phoenix] Phoenix. The Journal of the Classical Association of Canada 
\item[PhoenixExOrLux] Phoenix. Bulletin uitgegeven door het Vooraziatisch-Egyptisch Genootschap »Ex Oriente Lux« 
\item[Phoibos] Phoibos. Bulletin du Cercle de philologie classique et orientale de l’Université libre de Bruxelles 
\item[Phronesis] Phronesis. A Journal for Ancient Philosophy 
\item[Picus] Picus. Studi e ricerche sulle Marche nell’antichità 
\item[PIR] Prosopographia Imperii Romani 
\item[PKG] Propyläen Kunstgeschichte 
\item[Platon] Πλάτον. Δελτίον της Εταιρείας Ελλήνων Φιλολόγων 
\item[PLup] Papyrologica Lupiensia 
\item[PolAMed] Polish Archaeology in the Mediterranean 
\item[Polemon] Πολέμων. Αρχαιολογικόν Περιοδικόν 
\item[Polis] Polis. Revista de ideas y formas políticas de la antigüedad clásica 
\item[PompHercStab] Pompeii, Herculaneum, Stabiae. Bolletino. Associazione internazionale Amici di Pompei 
\item[Pontica] Pontica. Studii şi materiale de istorie, arheologie şi muzeografie, Constanţa 
\item[Portugalia] Portugália. Revista do Instituto de arqueologia da Faculdade de letras da Universidade do Porto 
\item[PP] La parola del passato 
\item[PPM] Pompei: Pitture e mosaici. Enciclopedia dell’arte antica classica e orientale 
\item[PraceA] Prace archeologiczne 
\item[PraceMatLodz] Prace i materialy Muzeum archeologicznego i etnograficznego w Łodzi %*Abweichung!
\item[Prakt] Πρακτικά της εν Αθήναις Αρχαιολογικής Εταιρείας 
\item[PraktAkAth] Πρακτικά της Ακαδημίας Αθηνών 
\item[PreistAlp] Preistoria alpina. Museo tridentino di scienze naturali 
\item[PriloziZagreb] Prilozi Instituta za arheologiju u Zagrebu 
\item[PrincViana] Principe di Viana 
\item[ProblIsk] Problemi na izkustvoto. Trimesecno spisanie za estetika, teorija, istorija i kritika na izkustvoto 
\item[ProcAfrClAss] The Proceedings of the African Classical Associations 
\item[ProcCambrPhilSoc] Proceedings of the Cambridge Philological Society 
\item[ProcDanInstAth] Proceedings of the Danish Institute at Athens 
\item[ProcPrehistSoc] Proceedings of the Prehistoric Society 
\item[Prometheus] Prometheus. Rivista quadrimestrale di studi classici 
\item[ProspAQuad] Prospezioni archeologiche. Quaderni 
\item[Prospettiva] Prospettiva. Rivista di storia dell’arte antica e moderna 
\item[Prospezioni] Prospezioni. Bollettino di informazioni scientifiche 
\item[ProvHist] Provence historique 
\item[ProvLucca] La provincia di Lucca 
\item[PublInstTTMeneses] Publicaciones de la Institución »Tello Téllez de Meneses« 
\item[Pulpudeva] Pulpudeva. Semaines philippopolitaines de l’histoire et de la culture thrace 
\item[Puteoli] Puteoli. Studi di storia antica 
\item[Pyrenae] Pyrenae. Crónica arqueológica 
\item[PZ] Prähistorische Zeitschrift 
\item[QDAP] The Quarterly of the Department of Antiquities in Palestine 
\item[QuadABarcel] Quaderns d’arqueologia i història de la ciutat (Barcelona) 
\item[QuadACagl] Quaderni. Soprintendenza archeologica per la provincie di Cagliari e Oristano 
\item[QuadACal] Quaderni del Dipartimento delle arti, Università della Calabria 
\item[QuadALibya] Quaderni di archeologia della Libia 
\item[QuadAMant] Quaderni di archeologia del Mantovano 
\item[QuadAMess] Quaderni di archeologia, Università di Messina 
\item[QuadAOst] Quaderni del Gruppo archeologico ostigliese 
\item[QuadAPiem] Quaderni della Soprintendenza archeologica del Piemonte 
\item[QuadAquil] Quaderni aquileiesi 
\item[QuadAReggio] Quaderni d’archeologia reggiana 
\item[QuadAVen] Quaderni di archeologia del Veneto 
\item[QuadCast] Quaderns de prehistòria i arqueologia de Castelló 
\item[QuadCat] Quaderni catanesi di studi classici e medievali 
\item[QuadChieti] Quaderni dell’Istituto di archeologia e storia antica, Università di Chieti 
\item[QuadErb] Quaderni erbesi. Civico museo archeologico di Erba 
\item[QuadFriulA] Quaderni friulani di archeologia 
\item[QuadGerico] Quaderni di Gerico 
\item[QuadIstFilGr] Quaderni dell’Istituto di filologia greca 
\item[QuadIstLat] Quaderni dell’Istituto di lingua e letteratura latina, Università di Roma 
\item[QuadLecce] Quaderni. Istituto di lingue e letterature classiche, Facoltà di magistero, Università degli studi, Lecce 
\item[QuadMusPontecorvo] Museo civico Pontecorvo. Quaderni 
\item[QuadMusSalinas] Quaderni del Museo archeologico regionale Antonino Salinas 
\item[QuadProtost] Quaderni di protostoria. Università di Perugia 
\item[QuadStLun] Quaderni. Centro studi lunensi 
\item[QuadStor] Quaderni di storia 
\item[QuadStorici] Quaderni storici 
\item[QuadUrbin] Quaderni urbinati di cultura classica 
\item[Quaternaria] Quaternaria. Storia naturale e culturale del quaternario 
\item[RA] Revue archéologique 
\item[RAArtLouv] Revue des archéologues et historiens d’art de Louvain 
\item[RAC] Reallexikon für Antike und Christentum 
\item[RACFr] Revue archéologique du Centre de la France 
\item[RAComo] Rivista archeologica dell’antica provincia e diocesi di Como 
\item[RACr] Rivista di archeologia cristiana 
\item[RadAkZadar] Radovi Zavoda za povijesne znanosti, Hrvatska akademija znanosti i umjetnosti u Zadru 
\item[Radiocarbon] Radiocarbon. An International Journal of Cosmogenic Isotope Research 
\item[RadSplit] Radovi. Razdio povijesnih znanosti 
\item[RAE] Revue archéologique de l’Est et du Centre-Est 
\item[Raggi] Raggi. Zeitschrift für Kunstgeschichte und Archäologie 
\item[RAMadrid] Revista de arqueología 
\item[RANarb] Revue archéologique de Narbonnaise 
\item[RAPon] Revista d’arqueologia de Ponent 
\item[RapWyk] Raporty wykopaliskowe 
\item[RArchBiblMus] Revista de archivos, bibliotecas y museos 
\item[RArcheom] Revue d’archéométrie 
\item[RArtMus] La revue des arts. Musées de France 
\item[RassAPiomb] Rassegna di archeologia. Associazione archeologica piombinese 
\item[RassLazio] Rassegna del Lazio 
\item[RassStorSalern] Rassegna storica salernitana 
\item[RassVolt] Rassegna volterrana 
\item[RAssyr] Revue d’assyriologie et d’archéologie orientale 
\item[Ratiariensia] Ratiariensia. Studi e materiali mesici e danubiani 
\item[RAtlMed] Revista atlántica-mediterránea de prehistoria y arqueología social 
\item[RavStRic] Ravenna studi e ricerche 
\item[Raydan] Raydan. Journal of the Ancient Yemeni Antiquities and Epigraphy 
\item[RB] Revue biblique 
\item[RBelgNum] Revue belge de numismatique et de sigillographie 
\item[RBelgPhilHist] Revue belge de philologie et d’histoire 
\item[RBK] Reallexikon zur byzantinischen Kunst 
\item[RCulClMedioev] Rivista di cultura classica e medioevale 
\item[RdA] Rivista di archeologia 
\item[RDAC] Report of the Department of Antiquities, Cyprus 
\item[RdE] Revue d’égyptologie (Kairo) 
\item[RDroitsAnt] Revue internationale des droits de l’antiquité 
\item[RE] Paulys Realencyclopädie der classischen Altertumswissenschaft 
\item[REA] Revue des études anciennes 
\item[REByz] Revue des études byzantines 
\item[RecConstantine] Recueil des notices et memoires de la Société archéologique du département de Constantine 
\item[RechACrac] Recherches archéologiques. Institut d’archéologie de l’Université de Cracovie 
\item[RechAlb] Recherches albanologiques 
\item[RecMusAlcoi] Recerques del Museu d’Alcoi 
\item[RecTrav] Recueil de travaux relatifs à philologie et l’archéologie égyptiennes et assyriennes 
\item[REG] Revue des études grecques 
\item[ReiCretActa] Rei Cretariae Romanae Fautorum acta 
\item[ReiCretCommunic] Rei Cretariae Romanae Fautores. Communicationes 
\item[REL] Revue des études latines 
\item[Rema] Ρήμα. Mitteilungen zur indogermanischen, vornehmlich indo-iranischen Wortkunde sowie zur holothetischen Sprachtheorie 
\item[RendBologna] Atti della Accademia delle scienze dell’Istituto di Bologna. Rendiconti 
\item[RendIstLomb] Rendiconti. Classe di lettere e scienze morali e storiche, Istituto lombardo, Accademia di scienze e lettere 
\item[RendLinc] Rendiconti dell’Accademia nazionale dei Lincei, Classe di scienze morali, storiche e filologiche 
\item[RendNap] Rendiconti della Accademia di archeologia, lettere e belle arti, Napoli 
\item[RendPontAc] Rendiconti. Atti della Pontificia accademia romana di archeologia 
\item[RepMalta] Report on the Working of the Museum Department, Malta, Department of Information 
\item[Reppal] Reppal. Revue des études phénicienne-puniques et des antiquités libyques 
\item[RepSocLibSt] Annual Report. The Society for Libyan Studies 
\item[RES] Répertoire d’épigraphie sémitique (Paris 1900--1950) 
\item[REstIber] Revista de estudios ibéricos 
\item[REtArm] Revue des études arméniennes 
\item[RevEg] Revue égyptologique (Paris) 
\item[RFil] Rivista di filologia e di istruzione classica 
\item[RGeorgCauc] Revue des études géorgiennes et caucasiennes 
\item[RGF] Römisch-Germanische Forschungen 
\item[RGuimar] Revista de Guimarães 
\item[RHA] Revue hittite et asianique 
\item[RheinMusBonn] Das Rheinische Landesmuseum Bonn. Berichte aus der Arbeit des Museums 
\item[RHistArmees] Revue historique des armées %*Abweichung!
\item[RHistRel] Revue de l’histoire des religions 
\item[RhM] Rheinisches Museum für Philologie 
\item[RIA] Rivista dell’Istituto nazionale d’archeologia e storia dell’arte 
\item[RIC] H. Mattingly – E. A. Sydenham, The Roman Imperial Coinage 
\item[RicEgAntCopt] Ricerche di egittologia e di antichità copte 
\item[RicognA] Ricognizioni archeologiche 
\item[RicStBrindisi] Ricerche e studi. Museo Francesco Ribezzo, Brindisi 
\item[RIngIntem] Rivista Ingauna e Intemelia 
\item[RItNum] Rivista italiana di numismatica e scienze affini 
\item[RlA] Reallexikon der Assyriologie und vorderasiatischen Archäologie 
\item[RLouvre] Revue du Louvre. La revue des musées de France 
\item[RM] Mitteilungen des Deutschen Archäologischen Instituts, Römische Abteilung 
\item[RNum] Revue numismatique 
\item[RoczMuzWarsz] Rocznik Muzeum narodowego w Warszawie 
\item[RoemOe] Römisches Österreich. Jahresschrift der Österreichischen Gesellschaft für Archäologie %*Abweichung!
\item[RoemQSchr] Römische Quartalschrift für christliche Altertumskunde und Kirchengeschichte %*Abweichung!
\item[Romanobarbarica] Romanobarbarica. Contributi allo studio dei rapporti culturali tra il mondo latino e mondo barbarico 
\item[RomGens] Romana gens. Bollettino dell’Associazione archeologica romana 
\item[RoscherML] W. H. Roscher, Ausführliches Lexikon der griechischen und römischen Mythologie %*Abweichung!
\item[RossA] Rossijskaja archeologija 
\item[RPC] Roman Provincial Coinage 
\item[RPhil] Revue de philologie, de littérature et d’histoire anciennes 
\item[RPortA] Revista portuguesa de arqueologia 
\item[RPorto] Revista da Facultade de letras. Serie de historia. Universidade do Porto 
\item[RRC] M. Crawford, Roman Republican Coinage (London 1974) 
\item[RSaintonge] Revue de la Saintonge et de l’Aunis 
\item[RScPreist] Rivista di scienze preistoriche 
\item[RSO] Rivista degli studi orientali 
\item[RStBiz] Rivista di studi bizantini e neoellenici 
\item[RStCl] Rivista di studi classici 
\item[RStFen] Rivista di studi fenici 
\item[RStLig] Rivista di studi liguri 
\item[RStMarch] Rivista di studi marchigiani 
\item[RStorAnt] Rivista storica dell’antichità 
\item[RStorCal] Rivista storica calabrese 
\item[RStPomp] Rivista di studi pompeiani 
\item[RStPun] Rivista di studi punici 
\item[RTopAnt] Rivista di topografia antica 
\item[Rudiae] Rudiae. Ricerche sul mondo classico 
\item[SaalbJb] Saalburg-Jahrbuch. Bericht des Saalburg-Museums 
\item[SaarBeitr] Saarbrücker Beiträge zur Altertumskunde 
\item[SaarStMat] Saarbrücker Studien und Materialien zur Altertumskunde 
\item[Sacer] Sacer. Bollettino della Associazione storica saccarese 
\item[Saeculum] Saeculum. Jahrbuch für Universalgeschichte 
\item[SAGA] Studien zur Archäologie und Geschichte Altägyptens 
\item[SaggiFen] Saggi fenici 
\item[Saguntum] Saguntum. Papeles del Laboratorio de arqueología de Valencia 
\item[Saitabi] Saitabi. Noticiario de historia, arte y arqueología de Levante 
\item[SAK] Studien zur altägyptischen Kultur 
\item[Salduie] Salduie. Estudios de prehistoria y arqueología 
\item[Samothrace] Samothrace. Excavations Conducted by the Institute of Fine Arts of New York University 
\item[Sandalion] Sandalion. Quaderni di cultura classica, cristiana e medievale 
\item[Sardis] Sardis. Publications of the American Society for the Excavation of Sardis 
\item[Sargetia] Sargetia. Acta Musei regionalis Devensis 
\item[SarkSt] Sarkophag-Studien 
\item[Sautuola] Sautuola. Revista del Instituto de prehistoria y arqueologia Sautuola 
\item[Savaria] Savaria. Bulletin der Museen des Komitats Vas 
\item[SBBerlin] Sitzungsberichte der Deutschen Akademie der Wissenschaften zu Berlin. Klasse für Sprache, Literatur und Kunst 
\item[SBLeipzig] Sitzungsberichte der Sächsischen Akademie der Wissenschaften zu Leipzig 
\item[SBMuenchen] Bayerische Akademie der Wissenschaften. Philosophisch-Historische Klasse. Sitzungsberichte %*Abweichung!
\item[SborBrno] Sborník prací Filozofické fakulty Brněnské univerzity. M, Rada archeologická 
\item[SBWien] Sitzungsberichte. Österreichische Akademie der Wissenschaften 
\item[ScAnt] Scienze dell’antichità. Storia, archeologia, antropologia 
\item[SCE] The Swedish Cyprus Expedition 
\item[SchildStei] Schild von Steier. Beiträge zur Steirischen Vor- und Frühgeschichte und Münzkunde 
\item[Scholia] Scholia. Natal Studies in Classical Antiquity 
\item[SchwMueBl] Schweizer Münzblätter %*Abweichung!
\item[SchwNumRu] Schweizerische numismatische Rundschau 
\item[ScrCiv] Scrittura e civiltà 
\item[ScrClIsr] Scripta classica Israelica. Yearbook of the Israel Society for the Promotion of Classical Studies 
\item[ScrHieros] Scripta Hierosolymitana. Publications of the Hebrew University, Jerusalem 
\item[ScrMed] Scripta mediterranea. Bulletin of the Society for Mediterranean Studies, Toronto 
\item[SDAIK] Sonderschriften des Deutschen Archäologischen Instituts, Abteilung Kairo 
\item[SEG] Supplementum epigraphicum Graecum 
\item[SeminRom] Seminari romani di cultura greca 
\item[Semitica] Semitica. Cahiers publiés par l’Institut d’études sémitiques du College de France. Avec le concours du Centre national de la recherche scientifique 
\item[SetubalA] Setúbal arqueológica 
\item[Sibrium] Sibrium. Collana di studi e documentazioni 
\item[SicA] Sicilia archeologica 
\item[SicGymn] Siculorum gymnasium 
\item[SIG] W. Dittenberger, Sylloge inscriptionum Graecarum (Leipzig 1915--­1924) 
\item[Sileno] Sileno. Rivista di studi classici e cristiani 
\item[SilkRoadArtA] Silk Road Art and Archaeology. Journal of the Institute of Silk Road Studies, Kamakura 
\item[SIMA] Studies in Mediterranean Archaeology 
\item[Simblos] Simblos. Scritti di storia antica 
\item[Skyllis] Skyllis. Zeitschrift für Unterwasserarchäologie 
\item[SlovA] Slovenská archeológia 
\item[SlovNum] Slovenská numizmatika 
\item[SMEA] Studi micenei ed egeo-anatolici 
\item[SNG] Sylloge nummorum Graecorum 
\item[SocGeoAOran] (Bulletin trimestriel de la) Société de géographie et d’archéologie (de la province) d’Oran 
\item[SoobErmit] Soobščenija Gosudarstvennogo Ėrmitaža 
\item[SoobMuzMoskva] Soobščenija Gosudarstvennogo muzeja izobrazitel’nych isskustv imeni A. S. Puškina 
\item[SovA] Sovetskaja archeologija 
\item[Spal] Spal. Revista de prehistoria y arqueología de la Universidad de Sevilla 
\item[SpNov] Specimina nova dissertationum ex Instituto historico Universitatis Quinqueecclesiensis de Iano Pannonio nominatae 
\item[Spoletium] Spoletium. Rivista di arte, storia, cultura 
\item[StA] Studia archaeologica 
\item[Stadion] Stadion. Internationale Zeitschrift für Geschichte des Sports 
\item[StaedelJb] Städel-Jahrbuch %*Abweichung!
\item[StAeg] Studia Aegyptiaca. Budapest 
\item[StAlb] Studia Albanica 
\item[StAnt] Studi di antichità. Università di Lecce 
\item[Starinar] StarinarArheoloki institut Beograd 
\item[StAWarsz] Studia archeologiczne. Uniwersytet Warszawski, Instytut archeologii 
\item[StBiFranc] Studium biblicum Franciscanum. Liber annuus 
\item[StBitont] Studi bitontini 
\item[StBoT] Studien zu den Bogazköy-Texten 
\item[StCercIstorV] Studii şi cercetări de istorie veche şi arheologie 
\item[StCercNum] Studii şi cercetări de numismatică 
\item[StCl] Studii clasice. Societatea de studii clasice din Republica socialistă Romănia 
\item[StClOr] Studi classici e orientali 
\item[StDocA] Studi e documenti di archeologia 
\item[StDocHistIur] Studia et documenta historiae et iuris 
\item[StEbla] Studi eblaiti 
\item[StEgAntPun] Studi di egittologia e di antichità puniche 
\item[SteMat] Studi e materiali. Soprintendenza ai beni archeologici per la Toscana 
\item[StEpigrLing] Studi epigrafici e linguistici sul Vicino Oriente antico 
\item[StEtr] Studi etruschi 
\item[StGenu] Studi genuensi 
\item[StHist] Studia historica. Historia antigua 
\item[StiftHambKuSamml] Stiftung zur Förderung der Hamburgischen Kunstsammlungen. Erwerbungen 
\item[StItFilCl] Studi italiani di filologia classica 
\item[StLatIt] Studi latini e italiani 
\item[StMagreb] Studi magrebini 
\item[StMatStorRel] Studi e materiali di storia delle religioni 
\item[StOliv] Studia Oliveriana 
\item[StOr] Studia orientalia, Helsinki 
\item[StOrCr] Studi sull’Oriente cristiano 
\item[StP] Studia papyrologica 
\item[StrennaRom] Strenna dei romanisti 
\item[StRom] Studi romani 
\item[StRomagn] Studi romagnoli 
\item[StSalent] Studi salentini 
\item[StSard] Studi sardi 
\item[StStorRel] Studi storico-religiosi 
\item[StTardoant] Studi tardoantichi 
\item[StTrentStor] Studi trentini di scienze storiche. Sezione seconda 
\item[StTroica] Studia Troica 
\item[StUrbin] Studi urbinati. B, Scienze umane. 3, Linguistica, letteratura, arte 
\item[Sumer] Sumer. A Journal of Archaeology (and History) in Iraq 
\item[SylvaMala] Sylva Mala. Bollettino del Centro di studi archeologici di Boscoreale, Boscotrecase e Trecase %*Abweichung!
\item[SymbOslo] Symbolae Osloenses 
\item[Syria] Syria. Revue d’art oriental et d’archéologie 
\item[SyrMesopSt] Syro-Mesopotamian Studies 
\item[TAD] Türk arkeoloji dergisi 
\item[Talanta] Τάλαντα. Proceedings of the Dutch Archaeological and Historical Society 
\item[TAM] E. Kalinka (Hrsg.), Tituli Asiae Minoris (Wien 1901--­1941) 
\item[Taras] Taras. Rivista di archeologia 
\item[Tarsus] Excavations at Gözlü Kule, Tarsus 
\item[TAVO] Tübinger Atlas des Vorderen Orients 
\item[TeherF] Teheraner Forschungen 
\item[Teiresias] Teiresias. A Review and Continuing Bibliography of Boiotian Studies 
\item[TelAvivJA] Tel Aviv. Journal of the Institute of Archaeology of Tel Aviv University 
\item[TerraAntBalc] Acta Associationis internationalis »Terra antiqua balcanica« 
\item[TerraVolsci] Terra dei Volsci. Annali del Museo archeologico di Frosinone 
\item[Teruel] Teruel. Instituto de estudios turolenses 
\item[TextilAnc] Textiles anciens. Bulletin du Centre international d’étude des textiles anciens 
\item[TheolRu] Theologische Rundschau 
\item[ThesCRA] Thesaurus Cultus et Rituum Antiquorum 
\item[Thessalika] Θεσσαλικά 
\item[Thessalonike] Η Θεσσαλονίκη 
\item[Thieme-Becker] U. Thieme – F. Becker (Hrsg.), Allgemeines Lexikon der bildenden Künstler %*Abweichung!
\item[ThrakChron] Θρακικά Χρονικά 
\item[ThrakEp] Θρακική Επετηρίδα 
\item[TIB] Tabula Imperii Byzantini 
\item[Tibiscus] Tibiscus. Istorie, arheologie. Muzeul Banatului Timişoara 
\item[TiLeon] Tierras de León 
\item[Tiryns] Tiryns. Forschungen und Berichte 
\item[TMA] Tijdschrift voor Mediterrane archeologie 
\item[Topoi] Τόποι. Orient – Occident 
\item[Torretta] La torretta. Rivista quadrimestrale a cura della Biblioteca comunale di Blera 
\item[TourOrleOr] La Tour de l’Orle-d’Or, Semur-en-Auxois 
\item[TrabAntrEtn] Trabalhos de antropologia e etnologia 
\item[TrabArq] Trabalhos de arqueologia 
\item[TrabAssArqPort] Trabalhos da associação dos arqueólogos portugueses 
\item[TrabNavarra] Trabajos de arqueología de Navarra 
\item[TrabPrehist] Trabajos de prehistoria 
\item[Traditio] Traditio. Studies in Ancient and Medieval History, Thought and Religion 
\item[TransactAmPhilAss] Transactions and Proceedings of the American Philological Association 
\item[TransactAmPhilosSoc] Transactions of the American Philosophical Society 
\item[TransactLond] Transactions of the London and Middlesex Archaeological Society 
\item[TravMem] Travaux et mémoires. Centre de recherche d’histoire et civilisation byzantine, Paris 
\item[TravToulouse] Travaux de l’Institut d’art préhistorique, Université de Toulouse – Le Mirail 
\item[TreMonet] Trésors monétaires 
\item[TribArq] Tribuna d’arqueologia 
\item[TrudyErmit] Trudy Gosudarstvennogo Ėrmitaža 
\item[TrWPr] Trierer Winckelmannsprogramme 
\item[TrZ] Trierer Zeitschrift für Geschichte und Kunst des Trierer Landes und seiner Nachbargebiete 
\item[TTKY] Türk Tarih Kurumu yayınları 
\item[TueBA-Ar] Türkiye Bilimler Akademisi arkeoloji dergisi %*Abweichung!
\item[Tyche] Tyche. Beiträge zur Alten Geschichte, Papyrologie und Epigraphik 
\item[UF] Ugarit-Forschungen. Internationales Jahrbuch für die Altertumskunde Syrien-Palästinas 
\item[UPA] Universitätsforschungen zur Prähistorischen Archäologie 
\item[LUrbe] L’Urbe. Rivista romana %*Abweichung!
\item[UrSchw] Ur-Schweiz. La Suisse primitive 
\item[UVB] Vorläufiger Bericht über die von dem Deutschen Archäologischen Institut und der Deutschen Orient-Gesellschaft aus den Mitteln der Deutschen Forschungsgemeinschaft unternommenen Ausgrabungen in Uruk-Warka 
\item[VarSpom] Varstvo spomenikov 
\item[VDI] Vestnik drevnej istorii 
\item[Vekove] Vekove. Dvumesečno spisanie. Bălgarsko istoričesko družestvo 
\item[Veleia] Veleia. Revista de prehistoria, historia antigua, arqueología y filología clásicas 
\item[VenArt] Venezia arti. Bolletino del Dipartimento di storia e critica delle arti dell’Università di Venezia 
\item[VerAmstMeded] Mededelingenblad. Vereniging van Vrienden Allard Pierson Museum 
\item[Verbanus] Verbanus. Rassegna per la cultura, l’arte, la storia del lago 
\item[VeteraChr] Vetera christianorum 
\item[VGesVind] Veröffentlichungen der Gesellschaft Pro Vindonissa 
\item[Vichiana] Vichiana. Rassegna di studi filologici e storici 
\item[VicOr] Vicino Oriente 
\item[VigChr] Vigiliae christianae 
\item[Viminacium] Viminacium. Zbornik radova Narodnog muzeja 
\item[VisRel] Visible Religion 
\item[Vitudurum] Beiträge zum römischen Oberwinterthur – Vitudurum. Ausgrabungen im Unteren Bühl 
\item[VivScyl] Vivarium Scyllacense. Bollettino dell’Istituto di studi su Cassiodoro e sul medioevo in Calabria 
\item[VizVrem] Vizantijskij vremennik 
\item[VjesAMuzZagreb] Vjesnik Arheološkog muzeja u Zagrebu 
\item[VjesDal] Vjesnik za arheologiju i historiju dalmatinsku. Bulletin d’archéologie et d’histoire dalmates 
\item[Wad-al-Hayara] Wad-al-Hayara. Revista de estudios de la Institución provincial de cultura »Marqués de Santillana« de Guadalajara 
\item[WeltGesch] Die Welt als Geschichte 
\item[WiadA] Wiadomości archeologiczne. Bulletin archéologique polonais 
\item[WissMBosn] Wissenschaftliche Mitteilungen des Bosnischen Landesmuseums, A. Archäologie 
\item[WissZBerl] Wissenschaftliche Zeitschrift der Humboldt-Universität zu Berlin. Gesellschafts- und sprachwissenschaftliche Reihe 
\item[WissZHalle] Wissenschaftliche Zeitschrift. Martin-Luther-Universität Halle-Wittenberg 
\item[WissZJena] Wissenschaftliche Zeitschrift der Friedrich-Schiller-Universität Jena 
\item[WissZRostock] Wissenschaftliche Zeitschrift der Universität Rostock 
\item[WO] Die Welt des Orients. Wissenschaftliche Beiträge zur Kunde des Morgenlandes 
\item[WorldA] World Archaeology 
\item[WSt] Wiener Studien 
\item[WuerzbJb] Würzburger Jahrbücher für die Altertumswissenschaft %*Abweichung!
\item[WVDOG] Wissenschaftliche Veröffentlichungen der Deutschen Orient-Gesellschaft 
\item[WZKM] Wiener Zeitschrift für die Kunde des Morgenlandes 
\item[Xenia] Xenia. Semestrale di antichità 
\item[XeniaAnt] Xenia antiqua 
\item[XeniaKonst] Xenia. Konstanzer althistorische Vorträge und Forschungen 
\item[YaleClSt] Yale Classical Studies 
\item[YaleUnivB] Yale University Art Gallery Bulletin 
\item[ZA] Zeitschrift für Assyriologie und vorderasiatische Archäologie 
\item[ZAAK] Zeitschrift für Archäologie Außereuropäischer Kulturen 
\item[ZAeS] Zeitschrift für ägyptische Sprache und Altertumskunde %*Abweichung!
\item[ZAKSSchriften] Schriften des Zentrums für Archäologie und Kulturgeschichte des Schwarzmeerraumes 
\item[ZAntChr] Zeitschrift für antikes Christentum 
\item[ZAW] Zeitschrift für die alttestamentliche Wissenschaft 
\item[ZborMuzBeograd] Zbornik Narodnog muzeja Beograd 
\item[ZborRadBeograd] Zbornik radova Vizantološkog instituta. Recueil des travaux de l’Institut d’études byzantines, Beograd 
\item[ZborZadar] Zbornik Instituta za historijske nauke u Zadru 
\item[ZDMG] Zeitschrift der Deutschen Morgenländischen Gesellschaft 
\item[ZDPV] Zeitschrift des Deutschen Palästina-Vereins 
\item[Zephyrus] Zephyrus. Revista de prehistoria y arqueología 
\item[ZEthn] Zeitschrift für Ethnologie der Deutschen Gesellschaft für Völkerkunde und der Berliner Gesellschaft für Anthropologie, Ethnologie und Urgeschichte 
\item[ZfA] Zeitschrift für Archäologie 
\item[ZfNum] Zeitschrift für Numismatik 
\item[ZivaAnt] Živa antika. Antiquité vivante 
\item[ZKuGesch] Zeitschrift für Kunstgeschichte 
\item[ZNW] Zeitschrift für die neutestamentliche Wissenschaft und die Kunde der älteren Kirche 
\item[ZPE] Zeitschrift für Papyrologie und Epigraphik 
\item[ZSav] Zeitschrift der Savigny-Stiftung für Rechtsgeschichte. Romanistische Abteilung 
\item[ZSchwA] Zeitschrift für Schweizerische Archäologie und Kunstgeschichte 
\item[ZVerglSprF] Zeitschrift für vergleichende Sprachforschung 
\end{description}
\end{footnotesize}
%\end{multicols}


 \changes{v1.1}{2015/07/06}{Erstellung der Liste mit Abkürzungen}


%%\clearpage
%%\section{Umsetzung}
%%\label{driver}
%%|archaeologie| consists of various files:
%%There is one file that takes care of the bibliography (|bbx|) 
%%another looks after the citation-style  (|cbx|)
%%and several files which are necessary for the individual languages (|lbx|),
%%furthermore there is a |dbx|-file.
%%
%%%http://tex.stackexchange.com/questions/95036/continue-line-numbers-in-listings-package
%%\subsection{archaeologie.bbx}
%%|archaeologie| baut auf dem |standard|-Stil von |biblatex| auf, der entsprechend geladen werden muss.
%% \DescribeMacro{bbx}
%% \StartLineAt{13}
%%\begin{code}[style=code]
%%\ProvidesFile{archaeologie.bbx}%
%%               [2016/05/31 v2.0  archaeologie -- %
%%                biblatex for archaeologists, 
%%                historians and philologists, bbx-file]
%%\RequireBibliographyStyle{standard}
%%\end{code}
%%
%%It continues with all required settings
%%\ContinueLineNumber
%%\begin{code}[style=code]
%%\AtBeginDocument{%
%%    \urlstyle{sf}%
%%    \typeout{* * * archaeologie * * *  
%%        biblatex for archaeologists, 
%%               historians and philologists}
%%}
%%\ExecuteBibliographyOptions{%
%%pagetracker=true,%
%%citecounter=true,%
%%giveninits=true,%
%%sortlocale=auto,%
%%language=auto,%
%%autolang=other,%
%%bibencoding=utf8,%
%%dateabbrev=false, %
%%sorting=nyt,%
%%maxnames=2,% 
%%minnames=1,%
%%maxitems=1,%
%%maxbibnames=999,%
%%}
%%\end{code}
%% \StartLineAt{426}
%%\begin{code}[style=code]
%%\renewbibmacro*{journal}{%
%%  \ifboolexpr{test {\iffieldundef{shortjournal}} %
%%            or bool {bbx:noabbrevs}}%
%%    {\printtext[journaltitle]{%
%%       \printfield[titlecase]{journaltitle}%
%%       \setunit{\subtitlepunct}%
%%       \printfield[titlecase]{journalsubtitle}}}%
%%    {\printfield{shortjournal}}%
%%    }
%%\end{code}
%%    

%\subsection{archaeologie.cbx}

 \changes{v0.1}{2015/06/04}{Started Project}
 \changes{v1}{2015/09/15}{First public version}
%\PrintChanges
%\PrintIndex

%\fi
\end{document}