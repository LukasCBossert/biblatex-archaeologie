% archaeologie --%
%            biblatex for archaeologists, 
%				historians and philologists
% Copyright (c) 2016 Lukas C. Bossert | Johannes Friedl
%  
% This work may be distributed and/or modified under the
% conditions of the LaTeX Project Public License, either version 1.3
% of this license or (at your option) any later version.
% The latest version of this license is in
%   http://www.latex-project.org/lppl.txt
% and version 1.3 or later is part of all distributions of LaTeX
% version 2005/12/01 or later.

\documentclass[a4paper,
10pt,
greek,
french,
spanish,
italian, %ngerman,
ngerman,
english
]{ltxdoc}
\CodelineNumbered
\AtBeginDocument{\RecordChanges}
\AtEndDocument{\PrintChanges}
  \usepackage[T1,LGR]{fontenc}		% font types and character verification
%  \renewcommand*\ttdefault{lcmtt}
\usepackage{libertine}
\renewcommand*\ttdefault{lmvtt}

\usepackage[					% use  for bibliography
	backend=biber,
	style=archaeologie,
	%corpora,
	%seenote,
	%translation,
	%publisher,
	lstabbrv,
]{biblatex}
\renewcommand\bibfont{\normalfont
\footnotesize}
\usepackage{metalogo}
\usepackage{hologo}
\usepackage{babel}
\usepackage{coolthms}
\usepackage[					% advanced quotes
	strict=true,					% 	- warning are errors now
	style=ngerman,					% 	- german quotes
]{csquotes}
\usepackage{multicol}
\setlength{\columnsep}{1.5cm}
\setlength{\columnseprule}{0.2pt}
\usepackage{framed}
\usepackage{enumitem}
\setlength{\parindent}{0pt}
\setlength{\parskip}{6pt plus 2pt minus 2pt}
\setenumerate[1]{label=(\alph*),leftmargin=*,nolistsep,parsep=\parskip}
\usepackage{changepage}
\makeindex
   
\newenvironment{bsp}{\begin{framed}\begin{footnotesize}
\begin{adjustwidth}{.3cm}{.3cm}}{\end{adjustwidth}
\end{footnotesize}\end{framed}}
\defbibheading{empty}{}

\newcommand{\printbib}[2][4em]{%
\begingroup
\begin{bsp}
\footnotesize
\begin{refsection}
\setlength{\labwidthsameline}{#1} 
\nocite{#2}
\printbibliography[heading=none]
\end{refsection}
\end{bsp}
\endgroup
}

\newcommand{\printbiball}[2][4em]{%
\begingroup
\setlength{\labwidthsameline}{#1} 
\begin{bsp}
\footnotesize%
\begin{itemize}%[\bfseries]
\begin{refsection}%
\nocite{#2}%
\item[English:]{\printbibliography[heading=none]}
\item[German:]\foreignlanguage{ngerman}{\printbibliography[heading=none]}
\item[Italian:]\foreignlanguage{italian}{\printbibliography[heading=none]}
\item[French:]\foreignlanguage{french}{\printbibliography[heading=none]}
\item[Spanish:]\foreignlanguage{spanish}{\printbibliography[heading=none]}
\end{refsection}
\end{itemize}%
\end{bsp}
\endgroup
}

\listfiles
\EnableCrossrefs
\CodelineIndex
\RecordChanges
% \usepackage[disable]{todonotes} % notes not showed
\addbibresource{archaeologie.bib}
%\addbibresource{archaeologie-ancient.bib}
%\addbibresource{antike-corpora.bib} %Version 2.0
\usepackage{listings}
\usepackage{xcolor}
\definecolor{codeblue}{RGB}{0,65,137}
\definecolor{codegreen}{RGB}{147,193,26}

%\definecolor{codegreen}{rgb}{0,0.6,0}
\definecolor{codegray}{rgb}{0.5,0.5,0.5}
\definecolor{codepurple}{rgb}{0.58,0,0.82}
\definecolor{backcolour}{rgb}{0.95,0.95,0.92}
 
\lstdefinestyle{mystyle}{%
	language=[LaTeX]TeX,
    backgroundcolor=\color{backcolour},   
    commentstyle=\color{codegreen},
    keywordstyle=\color{codeblue},
    numberstyle=\tiny\color{codegray},
    stringstyle=\color{codepurple},
    escapeinside={*@}{@*},          % if you want to add LaTeX within your code
    texcsstyle=*\color{codeblue},
    morekeywords={cites, parencites, parencite, textcite, textcites, citeauthor, citetitle,
    						@Article, @Book,@Collection,@Proceedings,@Reference,@Thesis,
    						@Inproceedings,@Talk,@Review,@Inreference,@Incollection,
    						},
    basicstyle=\ttfamily\footnotesize,
    breakatwhitespace=false,         
    breaklines=true,                 
    captionpos=b,                    
    keepspaces=true,                 
    numbers=left,                    
    numbersep=5pt,            
    showspaces=false,                
    showstringspaces=false,
    showtabs=false,                  
    tabsize=2,
    literate=
            *{\{}{{{\color{codegreen}{\{}}}}{1}
            {\}}{{{\color{codegreen}{\}}}}}{1}
            {[}{{{\color{codegreen}{[}}}}{1}
            {]}{{{\color{codegreen}{]}}}}{1},
}
 \lstdefinestyle{code}{%
	language=[LaTeX]TeX,
    backgroundcolor=\color{white},   
    commentstyle=\color{codegreen},
    keywordstyle=\color{codeblue},
    numberstyle=\small\color{codegray},
    stringstyle=\color{codepurple},
	escapeinside={*@}{@*},
    texcsstyle=*\color{codeblue},
    morekeywords={},
    basicstyle=\ttfamily\footnotesize,
    breakatwhitespace=false,         
    breaklines=true,                 
    captionpos=b,                    
    keepspaces=true,                 
    numbers=left,                    
    stepnumber=1,
    numbersep=5pt,            
    showspaces=false,                
    showstringspaces=false,
    showtabs=false,                  
    tabsize=2,
    literate=
            *{\{}{{{\color{codeblue}{\{}}}}{1}
            {\}}{{{\color{codeblue}{\}}}}}{1}
            {[}{{{\color{codeblue}{[}}}}{1}
            {]}{{{\color{codeblue}{]}}}}{1},
}
%%% Always I forget this so I created some aliases
\def\ContinueLineNumber{\lstset{firstnumber=last}}
\def\StartLineAt#1{\lstset{firstnumber=#1}}
\let\numberLineAt\StartLineAt

\lstset{style=mystyle}
\lstMakeShortInline[style=mystyle]{|}

\hypersetup{  colorlinks   = true, %Colours links instead of ugly boxes
  urlcolor     =  blue, %Colour for external hyperlinks
  linkcolor    = black, %Colour of internal links
  citecolor   = blue, %Colour of citations
  }	
\begin{document}
%\MakeShortVerb{\|}
% \newenvironment{syntax}{\begin{small}\medskip\hspace*{2em}}{\par\medskip\end{small}}
 %\def\verbatimchar{;}

\title{\texttt{archaeologie} -- \\\texttt{bib\LaTeX} for archaeologists\footnote{Also very handy as well for (ancient) History or Classics.
The development of the code is done at \url{https://github.com/LukasCBossert/biblatex-archaeologie}: 
Comments and critics are welcome.
We thank  ›moewew‹ and Herbert Voß for their big help on the code.%
}}
\author{Lukas C. Bossert\thanks{\href{mailto:lukas@digitales-altertum.de}{lukas@digitales-altertum.de}} \and Johannes Friedl}
\date{Version: 2.0 (2016-05-31)}
 \maketitle
 \begin{abstract}
\noindent This citation-style covers the citation and bibliography rules of the German Archaeological Institute. 
Various options are available to change and adjust the outcome according to one's own preferences. 
The style is compatible with the English, Italian, Spanish and French languages, since all |bibstrings| used are defined in each language.

For a short introduction in German see \url{http://mirrors.ctan.org/macros/latex/contrib/biblatex-contrib/archaeologie/archaeologie-ger.pdf}
 \end{abstract}

\section{Last Changes}
Changes since Version 1.42
\begin{itemize}
\item Documentation now in English; short introduction in German
\item Adding  bibliography entry  |@talk| (\cref{talk})
\item Adding additional bibliography for ancient authors and works (\cref{ancient-list,ancient})
\end{itemize}

%\changes{v2.0}{2015/11/10}{Ergänzung: Feld |number| bei |@inreference| wird jetzt ausgelesen.}

\newpage
\begin{multicols}{2}
{\parskip=0mm \tableofcontents}
\end{multicols}

\section{Installation}
|archaeologie| is part of the distributions MiK\TeX \footnote{Website: \url{http://www.miktex.org}.} 
and \TeX Live\footnote{Website: \url{http://www.tug.org/texlive}.}~-- thus, you
can easily install it using the respective package manager. 
If you would like to
install |archaeologie| manually, do the following:
Download the folder |archaeologie| with all relevant files from the CTAN-server\footnote{\url{http://mirrors.ctan.org/macros/latex/contrib/biblatex-contrib/archaeologie.zip}} and copy it to the \texttt{\$LOCALTEXMF} directory of
 your system.\footnote{If you don't know what that is, have a look at
\url{http://www.tex.ac.uk/cgi-bin/texfaq2html?label=tds} or 
\url{http://mirror.ctan.org/tds/tds.html}.} 
Refresh your filename database. 
Here is some additional information from the UK \TeX\ FAQ:
\begin{itemize}
	\item \href{%
    http://www.tex.ac.uk/cgi-bin/texfaq2html?label=install-where}{%
    Where to install packages}
	\item \href{%
	  http://www.tex.ac.uk/cgi-bin/texfaq2html?label=inst-wlcf}{%
	  Installing files \enquote{where \LaTeX /TeX\ can find them}}
	\item \href{%
	  http://www.tex.ac.uk/cgi-bin/texfaq2html?label=privinst}{%
	  \enquote{Private} installations of files}
\end{itemize}
%%introduction from biblatex-dw copied and applied. might to be rewritten.

\section{Usage}
 \DescribeMacro{archaeologie}  The name of the biblatex-style is  |archaeologie| has to be activated in the preamble. 

\begin{lstlisting}
\usepackage[style=archaeologie,%
					*@\meta{further options}@*]{biblatex}
\bibliography*@\marg{|bib|-Datei}@*
\end{lstlisting}

Without any further options the style covers now the rules of the German Archaeological Institute. No additional settings are needed.
But you can change the outcome by using some options which are explained below.


At the end of your document you can write the command |\printbibliography| to print 
the bibliography. 
Since |archaeologie| supports different citations of various texts like from ancient authors and from modern scholars we suggest to have them listed in separted bibliographies. 
Further information are found below   (\cref{bibliographie}).

\section{Overview}
Following there is a quick overview of the possible options of the style |archaeologie|.
Furthermore one can -- at your own risk -- also use the conventional  |biblatex|-options which are related of indent, etc. 
For that please see the excellent documentation of  |biblatex|.

 \subsection{Options in the preamble}\label{preamble_options}
 \DescribeMacro{seenote}
By default the style |archaeologie| follows the author-year-system. 
With this option you can change it to a different outcome (but still according to the rules of the German Archaeological Institute) so the first citation will be a full citation and all the following citations will refer to the first full citation cf. \cref{seenote}


 \DescribeMacro{lstabbrv}
 This activates the additional bibliography file  |archaeologie-abbrv.bib|.
 With this bibliography one can cite journals and series according to the abbreviations of the German Archaeological Institute cf. \cref{abbrv}. 


 \DescribeMacro{ancient}
A separate bibliography-file is loaded, in which round about 600 ancient authors and works are listed and can be cited right away cf. \cref{ancient}.

 \DescribeMacro{corpora}
A separate bibliography-file is loaded, in which the common corpora for ancient studies are stored cf. \cref{corpora}.
This activates the other additional bibliography |archaeologie-abbrv.bib| automatically.

 \DescribeMacro{edby}
This activates a switch to have instead of  \enquote{ed.}/\enquote{Hrsg.} now  \enquote{ed. by}/\enquote{hrsg. v.} cf. \cref{edby}.

 \DescribeMacro{initials}
This option takes care of an automatically abreviation of first names keeping digraphs and trigraphs cf. \cref{initials}


 \DescribeMacro{yearseries}
This changes the order of series and year  cf. \cref{yearseries}.
 
 \DescribeMacro{yearinparens}
 The year is shown in parens cf. \cref{yearinparens}.

 \DescribeMacro{translation}
This shows in the bibliography the original title, the translator and the original language.
If the bibliography entry is set with  |option={ancient}| this behaviour is default cf. \cref{translation}. 

 \DescribeMacro{noabbrv}
By default the short titles of journals and series (|shortjournal| and |shortseries|) are shown  in the bibliography.
With this option full name is shown instead (|journaltitle| and |series|) cf.  \cref{noabbrevs}.

 \DescribeMacro{publisher}
All the locations and the publisher is shown. This changes the format of the edition and the first print cf.  \cref{publisher}.

 \DescribeMacro{bibfullname}
In the bibliography the full names of the authors and/or editors are shown cf. \cref{bibfullname}.

 \DescribeMacro{inreferences}
Each bibliography entry which is a  |@inreference| is fully referenced according to the rules of the German Archaeological Institute cf. \cref{inreferences}.

 \DescribeMacro{lastnames}
Writes only the family names of the authors or editors when theyare cited with  |citeauthor|\-\marg{bibtex-key}.
In footnotes this is default cf.  \cref{lastnames}.

 \DescribeMacro{fullnames}
Writes only the first and family names of the authors or editors when they are cited with  |citeauthor|\-\marg{bibtex-key} cf.  \cref{fullnames}.


 \DescribeMacro{scshape}
Names in referenced citations are shown with small capital letters cf. \cref{scshape}.
Bibliography entries with |option={ancient}| or |option={frgancient}| (\cref{ancient,frgancient}) are not touched by this option.


 \DescribeMacro{width}
The value of |width={value}| defines in the bibliography the width between the label and the reference cf. \cref{width}.

 \DescribeMacro{counter}
This gives the sum of the citations in the text cf. \cref{counter}.


 \subsection{Options in the entries}
 Additionally a single bibliography entry can contain a value in the |options|-field.
 This changes the behaviour of the entry when it is cited cf. \cref{optionen-literatur,beispiele}. 


 \DescribeMacro{ancient}
 The entry is an ancient source (e.\,g. Cicero, Plutarch, etc) cf. \cref{ancient}.

 \DescribeMacro{frgancient}
 The entry is a fragmented ancient source (e.\,g. Festus) cf. \cref{frgancient}.

 \DescribeMacro{corpus} 
 Only the  |shorthand|-field is printed.
 This is needed especially for corpora of inscriptions or coins (CIL, AE, RIC, etc.) cf. \cref{corpus}.


 \changes{v1.1}{2015/06/04}{Neue Optionen in Zusammenfassung ergänzt.}


 


 \subsection{cite-comands}\label{cite-commands}
 \DescribeMacro{\cite}%
As always citing is done with  |\cite|:
\begin{lstlisting}
\cite*@\oarg{prenote}\oarg{postnote}\marg{bibtex-key}%@*
\end{lstlisting}

With \meta{prenote} one can make a short preliminary note  (e.\,g. \enquote{Vgl.}) and  \meta{postnote} is usually used for page numbers.
If only one optional argument is used then it is the \oarg{postnote}.
\begin{lstlisting}
\cite*@\oarg{postnote}\marg{bibtex-key}%@*
\end{lstlisting}
The \meta{bibtex-key} is the key from the bibliography-file.

 \DescribeMacro{\cites}
If one wants to cite several authors or works a very convenient way is the following using the |\cites|-command:
\begin{lstlisting}
\cites(pre-prenote)(post-postnote)*@\oarg{prenote}\oarg{postnote}\marg{bibtex-key}@*%
 																	*@\oarg{prenote}\oarg{postnote}\marg{bibtex-key}@*%
 																	*@\oarg{prenote}\oarg{postnote}\marg{bibtex-key}\ldots@*
\end{lstlisting}
 

\begin{center} * * * \end{center}
\DescribeMacro{\parencite}
Sometimes a citation has to be put in parens. 
Therefore we implemented the command |\parencite|:
\begin{lstlisting}
\parencite*@\oarg{postnote}\marg{bibtex-key}%@*
\end{lstlisting} 
This cite-command takes care of the correct corresponding parentheses and brackets.
Especially if one cites an inreference the parentheses are changing to (square) brackets.
The example shown in \cref{inreference} makes it clear.

\DescribeMacro{\parencites}
Of course there is also the possibility to cite several authors/works in parentheses.
This is done with |\parencites|:
 \begin{lstlisting}
\parencites(pre-prenote)(post-postnote)*@\oarg{prenote}\oarg{postnote}\marg{bibtex-key}@*%
 																			*@\oarg{prenote}\oarg{postnote}\marg{bibtex-key}@*%
 																			*@\oarg{prenote}\oarg{postnote}\marg{bibtex-key}\ldots@*
\end{lstlisting}

 \begin{center} * * * \end{center}
 
 \DescribeMacro{\textcite}
Additionally to the listed |\cite|-commands above there is a third way of citing:
This is needed if one wants to |\textcite|  the author in the text, then the year and pages will be put in parentheses. 
\begin{lstlisting}
\textcite*@\oarg{postnote}\marg{bibtex-key}%@*
\end{lstlisting} 

\DescribeMacro{\textcites}
And again there is also a  |\textcites| in case of several authos: 
  \begin{lstlisting}
\textcites(pre-prenote)(post-postnote)*@\oarg{prenote}\oarg{postnote}\marg{bibtex-key}@*%
 																			*@\oarg{prenote}\oarg{postnote}\marg{bibtex-key}@*%
 																			*@\oarg{prenote}\oarg{postnote}\marg{bibtex-key}\ldots@*
\end{lstlisting}

\begin{center} * * * \end{center}
 \DescribeMacro{\citeauthor}   \DescribeMacro{\citetitle}\label{citeauthor}
Furthermore and additionally to the ›normal‹ |\cite|-commands one can also cite the authors and the work in the text and in the footnotes. 
This shows how it is done:
\begin{lstlisting}
\citeauthor*@\oarg{prenote}\oarg{postnote}\marg{bibtex-key}%@*
\end{lstlisting} 
  and for the works 
\begin{lstlisting}
\citetitle*@\oarg{prenote}\oarg{postnote}\marg{bibtex-key}%@*
\end{lstlisting} 
For further information cf. \cref{fullnames}.
   

   

\subsection{Entries with @string}
The citation rule of the German Archaeological Institute says that the journals and series have to be abbreviated according to a given list.\footnote{\url{www.dainst.org/documents/10180/70593/02_Liste-Abkürzungen_quer.pdf}  (\today)}

 \DescribeMacro{@string} The style  |archaeologie| respects the guidelines of the German Archaeological Institute and is compatible with the given 
 abbreviations of journals and series.
 To minimize 	the susceptibility to errors and to omit unnecessary typing of sometimes very long journaltitles |archaeologie| works with socalled |@strings|.
 The advantage is that with  |@string| several bibliography entries can be defined by one central given value. 
 The  |@string| is loaded at the begin of the  |bib|-file, therefore all the |@strings| have to be before all other bibliography entries.
 
 To use this offer of simplification the fields for the title of the journal (|journaltitle|) and its shortform (|shortjournal|), as well as the name of the series (|series|) and its shortform  (|shortseries|) have to be written with a |@string|.
In \cref{listen} there is a list with all the abbreviations given by the DAI, in which the |@string| (with the endings |-short|, resp. |-long|) are listed in the left column.  
This  |@string| has to be written \textsc{without} any curly brackets in the fiels (e.g. |journaltitle| or |shortseries|).\footnote{If you use the program \emph{JabRef} in its non-coding window, then you have to write  \#|AyasofyaMuezYil|\#. JabRef converts this internally to a |@string| and omits the \# in the coding window.}

An example shows how to use it:
\begin{lstlisting}
@Article{Koyunlu1990,
  author       = {Koyunlu, A.},
  title        = {Die Bodenbelage und der Errichtungsort der Hagia Sophia},
  journaltitle = AyasofyaMuezYil-long,
  shortjournal = AyasofyaMuezYil-short,
  volume       = {11},
  pages        = {147--156},
  year         = {1990},
}
\end{lstlisting}
That article appeared within a rather unusual journal, 
which should be abbreviated with ›AyasofyaMüzYıl‹.
To save the time to look for the special character and insert  ›ı‹ manually 
it is written in the  |@string| with an  ›i‹  (for further information see \cref{listen}) 
but will be replaced after compiling with the correct character:

\printbib[6em]{Koyunlu1990}

The default behaviour of abbreviations can be switched off, of course.
In case you use the option |noabbrevs| (see \cref{noabbrevs}), then the output changes as well:
\begin{bsp}
A. Koyunlu, Die Bodenbelage und der Errichtungsort der Hagia Sophia, {\color{red}Ayasofia Müzesi yıllığı. Annual of Ayasofya Museum} 11, 1990, 147–156
%\fullcite{Koyunlu_1990}
\end{bsp}
The advantage lies in the possibility to create a separate bibliography with all the abbreviations of used journaltitles and series (see  \cref{bibliographie}).

If a journal or a series is  {\color{red}not} not included in the list (\cref{listen}) 
then this journal/series will {\color{red}not}  be abbreviated and with its full name written in  |{}| in the field e.g. |journaltitle=|\marg{Zeitschriftentitel}.
For the following examples we use |@string| whenever it is clever and possible.

\section{Optional preferences}
Following we give a more detailed insight into the various options of |archaeologie| 
and show their results on the bases of concrete examples.
Changes made by the options are {\color{red}coloured}.
%In {\color{blue}blau} sind bibliographische Einträge, die mit dem nächsten Literaturverzeichnis verknüpft sind.

 \subsection{Options set in the preamble}\label{options-preamble}
The optional preferences   In the preamble are loaeded within the package  |biblatex|:
 \begin{lstlisting}
\usepackage[%				
		backend=biber,	% activates biber which is done by default anyway 
										%	(but will give an error if not done here)
		style=archaeologie, 	% loads the style *@|archaeologie|@*
		inreferences=true,		%option *@|inreferences|@* is loaded
		lstabbrv,							%option *@|lstabbrv|@* is loaded as well
		]{biblatex}
\end{lstlisting}
In this example the style |archaeologie| is loaded with the option |inreferences|.
It doesn't matter if you write |inreferences,| or |inreferences=true|.


\subsubsection{lstabbrv}\label{abbrv}
 \DescribeMacro{lstabbrv}
 If you want to use the above mentioned method with |@string| you have to activate the option called  |lstabbrv| (list of abbreviations) in the preamble.
Once activated the additional bibliography  |archaeologie-abbrv.bib| is loaded. 
In this bibliography all abbreviations listed in \cref{listen} are stored.


\subsubsection{ancient}\label{ancient}
 \DescribeMacro{ancient}
In the case you cite ancient authors and their works you can do it with the common  |\cite|-command.
This is recommended and will  guarantee a high level of consistency and  minimize  error-proneness.
For this case especially we included a modification that respects the different citation of  ancient authors and works.
With the option |ancient| you load an additional bibliography called |bibliography-ancient.bib| in which we inserted almost 600 ancient authors and works with their abbreviation according to The New Pauly/Thesaurus Linguae Latinae.
For the complete list of those see  \cref{ancient-list}.

You can cite the authors or works with their |bibtex-key| which is on the left. 
Authors and works are separated by a colon.
The bold entry on the right is the |shorthand| which will be printed in your paper.
Let me make it clear with an example:
\begin{refsection}
With the loaded option |ancient| you write  
|\cite[3,2,5--7]{Apul:met}|  and will get in return after compiling: \cite[3,2,5--7]{Apul:met}.
\end{refsection}
The bibliography-entry looks like this
 \begin{lstlisting}
@Book{Apul:met,
  author      = {Apuleius Madaurensis, Lucius},
  title       = {metamorphoses},
  shorthand   = {Apul. met.},
  shortauthor = {Apuleius},
  keywords    = {ancient},
  options     = {ancient},
}
\end{lstlisting}
All entries in the mentioned additional bibliography contain the line |keywords = {ancient}|.
With that you can print all ancient authors in a separated bibliography.
The needed command is:

|\printbibliography[keyword = ancient]|

\printbib{Apul:met}

\begin{refsection}
With the  |bibtexkey| (e.g. |Apul:met|) of the entry you can also cite the authors and the titles name 
like this: 
|\citeauthor{Apul:met} in \citetitle{Apul:met}| will give 
\citeauthor{Apul:met} in \citetitle{Apul:met}
\end{refsection}
 \changes{v1.5}{2016/05/31}{Extra Bibliographie}

\subsubsection{seenote}\label{seenote}
 \DescribeMacro{seenote}
In case you don’t like the  author-year-citation style you can also switch to the other allowed citation rule by the German Archaeological Institute
which works like this:
If you cite a work for the first time it will be cited fully in the footnote.
All following citations will be as a short cite and refer to the footnote with the full citation.
Bibliography-entries with |options={ancient}| are excluded form this speciality and be cited as always.

You can use the cite-commands |\cite(s)|, |\parencite(s}| but |\textcite(s)| will be like |\cite(s)| ausgegeben.

% \footnote{\cite{Ball_2013}}
% \footnote{\cite{Ball_2013}}
First citation of |\footnote{\cite{Ball_2013}}|:
 \begin{bsp}
L. F. Ball – J. J. Dobbins, Pompeii Forum Project. Current  thinking on the Pompeii Forum, 117/3, 2013, 461–492
 \end{bsp}

Second citation of |\footnote{\cite[470]{Ball_2013}}|:

 \begin{bsp}
Ball – Dobbins loc. cit. (see n. [number of the footnote with full citation]) 470
 \end{bsp}
 \changes{v1.5}{2016/05/31}{Rückverweis}

\subsubsection{translation}\label{translation}
\DescribeMacro{translation}
Once this option is activated the original title |origtitle|, the original language and the translator of the work is printed.
Works with |options={ancient}| or |options={frgancient}| 
(e.g. ancient author and ancient fragments) will always be printed out with their original title, language and translator.
By default this option is turned off (|=false|). 

An example will  clarify the possibility.
This bibliographical entry |Lefebvre_2011| contains following fields:
 \begin{lstlisting}
@Book{Lefebvre_2011,
  author       = {Lefebvre,Henri},
  title        = {The Production of Space},
  publisher    = {Blackwell Publishing Ltd},
  location     = {Maien, MA and Oxford and Victoria},
  year         = {2011},
  edition      = {30},
  origlocation = {Oxford},
  origyear     = {1991},
  origtitle    = {La production de l’espace},
  origlanguage = {french},
  translator   = {Donald Nicholson-Smith},
}
\end{lstlisting}
In the bibliography it will be shown as:
\printbib[6em]{Lefebvre_2011}

With the activated option |translation| it will change to:
\begin{bsp}%
%\fullcite{Lefebvre_2011}
 Lefebvre 2011\hspace{2em} 
H. Lefebvre,  The Production of Space, {\color{red} La production de l’espace, trans. from French by D. Nicholson-Smith} \textsuperscript{30}(Oxford 1991; repr. Maien, MA 2011)
\end{bsp}
 
It will work not only with entries like |@Book| but also with e.g. |@Article|:
\begin{lstlisting}
@Article{Lefebvre_1977,
  author       = {Lefebvre, Henri},
  title        = {Die Produktion des städtischen Raums},
  journaltitle = {ARCH+},
  volume       = {34},
  pages        = {52--57},
  year         = {1977},
  translator   = {Franz Hiss and Hans-Ulrich Wegener},
  origlanguage = {french},
  number       = {9},
  origtitle    = {Introduction à l'espace urbain},
}
\end{lstlisting}

Once again the entry in the bibliography with switched on option:

\begin{bsp}%
%\fullcite{Lefebvre_1977}
 Lefebvre 1977\hspace{2em} H. Lefebvre, Die Produktion des städtischen Raums, \emph{Introduction à l’espace urbain}, {\color{red} trans. from French by F. Hiss – H.-U. Wegener}, ARCH+ 34/9, 1977, 52–57
\end{bsp}


\subsubsection{inreferences}\label{inreferences}
\DescribeMacro{inreferences}  
There is the possibility to cite inreferences in the footnote as a full citation.
It is only required that the bibliography-entry is an |@Inreference|  (cf. \cref{inreference}).
 
Another example makes it clear: 
\begin{lstlisting}
@Inreference{Nieddu_1995,
  author    = {Nieddu, Giuseppe},
  title     = {Dei Consentes},
  booktitle = LTUR-short,
  pages     = {9--10},
  year      = {1995},
  volume    = {2},
}
\end{lstlisting}
  \begin{refsection}
  Die Ausgabe von |\cite{Nieddu_1995}| ist nun auf zwei Arten möglich:
   \begin{bsp}

 \begin{enumerate}
 \item by default it will give:  %Nieddu 1995
 \cite{Nieddu_1995}
 \item with the option |inreferences| it will change to:
 LTUR 2 (1995) 9-10 s. v. Dei Consentes (G. Nieddu)
 % \cite{Nieddu_1995}
  \end{enumerate}
   \end{bsp}
If the \oarg{postnote} is defined with the columns/page number, (e.g. |\cite[9]{Nieddu_1995}|), 
then it will change the position for the \oarg{postnote}:
 \begin{bsp}
\begin{enumerate} 
 \item by default it will give:  %Nieddu 1995
 \cite[9]{Nieddu_1995}
 \item with the option |inreferences| it will change to:
  LTUR 2 (1995) 9 s. v. Dei Consentes (G. Nieddu)
  \end{enumerate}
\end{bsp}
With the activated option  (|inreferences|) the cited entries which are defined as a|@inreference| will be automatically omitted in the (final) bibliography,
because the entries are fully cited before in the footnotes.

If the option is not used (|=flase|) the entry will look like this in the bibliography:
  \end{refsection}

\printbib[6em]{Nieddu_1995} 

 
\subsubsection{yearseries}\label{yearseries}
 \DescribeMacro{yearseries}
The option |yearseries| will lead to a switched order of  the fields |series| and |number|.
The |series| of a |@Book| or a |@Collection| will be printed \emph{after} the year.
An example with an |@Incollection| will demonstrate the effect of the option:
 
 \begin{lstlisting}
@Incollection{Mundt_2015,
  author       = {Mundt, Felix},
  title        = {Der Mensch, das Licht und die Stadt},
  subtitle     = {Rhetorische Theorie und Praxis antiker und humanistischer Städtebeschreibung},
  pages        = {179--206},
  editor       = {Therese Fuhrer and Felix Mundt and Jan Stenger},
  booktitle    = {Cityscaping},
  booksubtitle = {Constructing and Modelling Images of the City},
  publisher    = {de Gruyter},
  location     = {Berlin and Boston},
  year         = {2015},
  series       = Philologus-long #{ Supplement},
  number       = {3},
  shortseries  = Philologus-short #{ Suppl.},
}
\end{lstlisting}

Without any option activated it will look like this:
\printbib[5em]{Mundt_2015}

 
With the activated option  |yearseries| it will change to:
 \begin{bsp}
Mundt 2015 \hspace{3em} F. Mundt, Der Mensch, das Licht und die Stadt. Rhetorische Theorie und Praxis antiker und humanistischer Städtebeschreibung, in: T. Fuhrer – F. Mundt – J. Stenger (ed.), Cityscaping. Constructing and Modelling Images of the City (Berlin 2015) {\color{red}Philologus Suppl. 3,} 179–206
\end{bsp}

\subsubsection{fullnames / lastnames}\label{fullnames}\label{lastnames}
\DescribeMacro{fullnames}
\DescribeMacro{lastnames}
In your text you can cite authors directly with their names (|\citeauthor|\marg{bibtex-key}) or their works  (|\citetitle|\marg{bibtex-key}) and they will be linked to your bibliography (cf. \cref{citeauthor})

By default the author’s names will be the abbreviated first name\footnote{Usually only the first letter, but if you have the option |initials| set to true, it might change (cf. \cref{initials}).} and the last name.
If you prefer to have the full name printed (not in the bibliography!) switch on the option |fullnames|.
But in case you want the authors shorten only to their last names use |lastnames|.

Let us make it clear with an example of the following bibliographical entry:

\begin{lstlisting}
@Article{Boehmer_1985,
  author       = {Boehmer, Rainer Michael and Wrede, Nadja},
  title        = {Astragalspiele in und um Warka},
  journaltitle = BaM-long,
  shortjournal = BaM-short,
  volume       = {16},
  pages        = {399--404},
  year         = {1985},
}
\end{lstlisting}

Let’s assume that is your text and you write:
\begin{refsection}
\begin{lstlisting}
*@\ldots , this is also shown by @*\citeauthor{Boehmer_1985} 
*@ in their latest article @*\citetitle{Boehmer_1985}*@.@* \end{lstlisting}

After compiling and without any activated option it will look like this:\footnote{If you cite an author in a footnote only the last names of the authors or editors will always printed,
it doesn’t matter if you activated |fullnames| (\cref{name:a}) or |lastnames| (\cref{name:b}): 
\begin{bsp}
\ldots , this is also shown by  \citeauthor{Boehmer_1985} in their latest article \citetitle{Boehmer_1985}.
\end{bsp}
}

\begin{bsp} 
\ldots , this is also shown by  \citeauthor{Boehmer_1985} in their latest article \citetitle{Boehmer_1985}.
\end{bsp}

Oder mit den Optionen |fullnames| (\cref{name:a}) und |lastnames| (\cref{name:b}):
\begin{bsp}
 \begin{enumerate}
\item\label{name:a} 
 \ldots , this is also shown by  {\color{red}Rainer Michael Boehmer  und Nadja Wrede} in their latest article \emph{Astragalspiele in und um Warka} (1985).
\item\label{name:b}  
\ldots ,  this is also shown by  {\color{red}Boehmer und  Wrede} in their latest article \emph{Astragalspiele in und um Warka} (1985).

 \end{enumerate}
\end{bsp}
\end{refsection}
\printbib[9em]{Boehmer_1985}

There is a slightly different behaviour if you use  |\citeauthor|, or |\citetitle|  with ancient authors and ancient works |options=antik|;
Instead of the full name of the field |author| the field |shortauthor| will be considered in which you can write the more common name of the ancient author.
Ancient works will be printed without the year in parentheses.

\begin{lstlisting}
@Book{Quint_inst,
  author       = {Fabius Quintilianus, Marcus},
  title        = {Ausbildung des Redners},
  subtitle     = {Institutio oratoria},
  location     = {Darmstadt},
  year         = {2015},
  edition      = {6},
  origlanguage = {latin},
  translator   = {Rahn, Helmut},
  shorthand    = {Quint. inst.},
  shortauthor  = {Quintilian},
  keywords     = {ancient},
  options      = {ancient},
}
\end{lstlisting}

And the concrete example:
\begin{refsection}
\begin{lstlisting}
*@ \ldots\ and @* \citeauthor{Quint_inst} *@ names in @* \citetitle{Quint_inst} 
*@ the  necessary  physical qualities of an orator, too.@* \end{lstlisting}
 
\begin{bsp}
\ldots\ and \citeauthor{Quint_inst} names in \citetitle{Quint_inst} the  necessary  physical qualities of an orator, too.
\end{bsp}
\end{refsection}

\printbib[5em]{Quint_inst}


\subsubsection{yearinparens}\label{yearinparens}
\DescribeMacro{yearinparens}%
The year of the publication (|year| or year from |date|) of the cited entries will be put in parentheses,
in the footnote and in the bibliography, too. 
The ›Klammerregel‹ (correct order of different brackets) will be respected.

Instead of e.g. |\cite[475]{Ball_2013}|
\begin{bsp} Ball – Dobbins 2013, 475 \end{bsp}
 it will turn with |yearinparens| into
\begin{bsp}
Ball – Dobbins {\color{red}(}2013{\color{red})}, 475
\end{bsp}




\subsubsection{scshape}\label{scshape}
\DescribeMacro{scshape}
You can also change the look of your citations. With |scshape| the authors names will be set to small capitals -- in the footnote and in the bibliography.

Entries with no author or editor but with a |label| (\cref{unbekannt}) are excluded from this option
because it is not an author name but a  self-defined expression.
Further excluded are ancient authors (|options={ancient}|, or. |options={frgancient}|).

By default we have again with |\cite[475]{Ball_2013}|
\begin{bsp} \cite[475]{Ball_2013} \end{bsp}
 But with |schape| it will turn into:
\begin{bsp}
{\scshape {\color{red}Ball – Dobbins}} 2013, 475
\end{bsp}
And since the entry looks like this
\begin{lstlisting}
@Article{Ball_2013,
  author       = {Larry F. Ball and John J. Dobbins},
  title        = {Pompeii Forum Project},
  subtitle     = {Current Thinking on the Pompeii Forum},
  journaltitle = AJA-long,
  shortjournal = AJA-short,
  volume       = {117},
  pages        = {461--492},
  year         = {2013},
  month        = {July},
  number       = {3},
}
\end{lstlisting}

The output in the bibliography will be:
\begin{bsp}
{\scshape {\color{red}Ball – Dobbins}} 2013\hspace{3em} L. F. Ball – J. J. Dobbins, Pompeii Forum Project. Current Thinking on the Pompeii Forum, AJA 117/3, 2013, 461–492
\end{bsp}



\subsubsection{bibfullname}\label{bibfullname}
\DescribeMacro{bibfullname}
This will show the full name of an author and/or editor in the bibliography.

Without any options this entry
\begin{lstlisting}
@Article{Osland_2016,
  author       = {Osland, Daniel},
  title        = {Abuse or Reuse?},
  subtitle     = {Public Space in Late Antique Emerita},
  journaltitle = AJA-long,
  shortjournal = AJA-short,
  volume       = {120},
  pages        = {67--97},
  year         = {2016},
  eprint       = {10.3764/aja.120.1.0067},
  eprinttype   = {jstor},
  number       = {1},
}
\end{lstlisting}
looks like this
\printbib[5em]{Osland_2016}

and with |bibfullname| will change to:

\begin{bsp}
Osland 2016\hspace{3em} {\color{red}Daniel} Osland, Abuse or Reuse? Public Space in Late Antique Emerita, AJA 120/ 1, 2016, 67–97,\\
JSTOR: 10.3764/aja.120.1.0067
\end{bsp}


\subsubsection{noabbrv}\label{noabbrevs}
\DescribeMacro{noabbrv}
According to the guidelines of the German Archaeological Institute the journaltitles and series have to be abbreviated.
Therefore the fields |shortjournal| or |shortseries| will be considered. 
If there is no abbreviation (in the list cf. \cref{listen}) the fields |shortjournal| or |shortseries| remain empty -- unless you want to insert a shortname.  
If you like to have the full names of journals and series you can switch on the option |noabbrv|.

\begin{lstlisting}
@Article{Koyunlu1990,
  author       = {Koyunlu, A.},
  title        = {Die Bodenbelage und der Errichtungsort der Hagia Sophia},
  journaltitle = AyasofyaMuezYil-long,
  shortjournal = AyasofyaMuezYil-short,
  volume       = {11},
  pages        = {147--156},
  year         = {1990},
}
\end{lstlisting}
\printbib[6em]{Koyunlu1990}

\begin{bsp}
Koyunlu 1990 \hspace{3em}A. Koyunlu, Die Bodenbelage und der Errichtungsort der Hagia Sophia, {\color{red}Ayasofia Müzesi yıllığı. Annual of Ayasofya Museum} 11, 1990, 147–156
%\fullcite{Koyunlu_1990}
\end{bsp}


\subsubsection{publisher}\label{publisher}
\DescribeMacro{publisher} 
Once activated all locations and the publisher will be printed. This will lead to a different output of the edition which will be right in front of the year.
In case of a reprint or a second edition the first edition |origyear| will be put in square brackets after the year.

\begin{lstlisting}
@Book{Emme_2013,
  author    = {Burkhard Emme},
  title     = {Peristyl und Polis},
  subtitle  = {Entwicklung und Funktionen öffentlicher griechischer Hofanlagen},
  publisher = {Walter de Gruyter},
  location  = {Berlin and New York},
  year      = {2013},
  series    = {Urban Spaces},
  number    = {1},
}
\end{lstlisting}

\begin{refsection}\end{refsection}%sicherheitshalber
Without any option:
\printbib[5em]{Emme_2013}

 
With the option |publisher| the order has changed:
 \begin{bsp}
Emme 2013\hspace{3em} B. Emme, Peristyl und Polis. Entwicklung und Funktionen öffentlicher griechischer Hofanlagen, Urban Spaces 1 (Berlin {\color{red} – New York: Walter de Gruyter} 2013)
 \end{bsp}
 
 And here a more detailed example with |origlocation|, |origyear| and |origpublisher|:
 \begin{lstlisting}
 @Book{Neufert_2002,
  author       = {Neufert, Ernst},
  editor       = {Neufert, Peter and Neufert, Cornelius and Neff, Ludwig and Franken, Corinna},
  title        = {Bauentwurfslehre},
  subtitle     = {Grundlagen, Normen, Vorschriften ...},
  publisher    = {Vieweg},
  origpublisher = {Mann},
  location     = {Wiesbaden},
  year         = {2002},
  edition      = {37},
  origlocation = {Berlin},
  origyear     = {1936},
}
 \end{lstlisting}
 \printbib[5em]{Neufert_2002}
  \begin{bsp}
Neufert  2002\hspace{3em} E. Neufert, Bauentwurfslehre. Grundlagen, Normen, Vorschriften ... Ed. by P.~Neufert – C. Neufert – L. Neff  – C. Franken {\color{red}  (Wiesbaden: Vieweg \textsuperscript{37}2002 [Berlin: Mann 1936])}
 \end{bsp}
 
\subsubsection{edby}\label{edby}
\DescribeMacro{edby}
Herausgeber werden nicht mehr zu Beginn des Sammelbandes aufgelistet und mit einem |(Hrsg.)| gekennzeichnet, sondern nach dem Titel des Sammelbandes mit dem Zusatz |hrsg. v.|

\begin{lstlisting}
@Inproceedings{Wulf-Rheidt_2013,
  author       = {Wulf-Rheidt, Ulrike},
  title        = {Der Palast auf dem Palatin -- Zentrum im Zentrum},
  subtitle     = {Geplanter Herrschersitz oder Produkt eines langen Entwicklungsprozesses?},
  pages        = {277--289},
  editor       = {Dally, Ortwin and Fless, Friederike and Haensch, Rudolf and Pirson, Felix and Sievers, Susanne},
  booktitle    = {Politische Räume in vormodernen Gesellschaften},
  booksubtitle = {Gestaltung – Wahrnehmung – Funktion},
  location     = {Rahden/Westf\adddot},
  publisher    = {Verlag Marie Leidorf},
  year         = {2013},
  venue        = {Berlin},
  eventdate    = {2009-11-18/2009-11-22},
  eventtitle   = {Internationale Tagung des DAI und des DFG-Exzellenzclusters TOPOI},
  number       = {6},
  series       = MKT-long,
  shortseries  = MKT-short,
}
\end{lstlisting}

 Ohne eine Option wird in der Bibliographie daraus:
 
 \printbib[7em]{Wulf-Rheidt_2013}

 
 mit der aktivierten Option |edby| verändert sich die Reihenfolge:
 \begin{bsp}
Wulf-Rheidt 2013\hspace{3em} U. Wulf-Rheidt, Der Palast auf dem Palatin – Zentrum im Zentrum. Geplanter Herrschersitz oder Produkt eines langen Entwicklungsprozesses?, in:  {\color{red}Politische Räume in vormodernen Gesellschaften. Gestaltung – Wahrnehmung – Funktion, ed. by O. Dally – F. Fless – R. Haensch – F. Pirson – S. Sievers}. Internationale Tagung des DAI und des DFG-Exzellenzclusters TOPOI Berlin November 18–22, 2009, MKT 6 (Rahden/Westf. 2013) 277–289
 \end{bsp}
 
 
\subsubsection{width}\label{width}
\DescribeMacro{width}
In der Bibliographie ist der Abstand zwischen dem Label (Autor und Jahr)  und der vollständigen bibliographischen Angabe mit dem Wert |4em| angegeben.
Wünscht man einen anderen Abstand, dann kann dieser selbst gewählt werden:

|width = XY|

Dafür kann XY für jede Länge stehen (bspw. |3em|, |7pt| oder |4cm|), auch |-1em| ist möglich, womit der Einzug aufgehoben wird.

\subsubsection{counter}\label{counter}
 \DescribeMacro{counter} 
 Möchte man die Anzahl an Zitationen eines Werks im Text erfahren, 
 dann reicht es wenn man in der Präambel die Option |counter| aktiviert (|counter=true|).\footnote{Idee basiert auf \href{http://tex.stackexchange.com/a/14159/98739}{http://tex.stackexchange.com/a/14159/98739} und wurde entsprechend angepasst.}
 Dadurch wird am Ende eines jeden Literatureintrages die Anzahl der zitierten Einträge geschrieben. 
 Die Ausgabe erfolgt bei eingestellter Sprache |ngerman|(im Paket |babel| oder in der |documentclass|)  entsprechend auf Deutsch,
 ansonsten auf Englisch.
\begin{bsp}
Böhm – Eickstedt 2001\\
\indent S. Böhm – K.-V. v. Eickstedt (Hrsg.), Ithake. Festschrift Jörg Schäfer (Würzburg 2001)  $\vert$  {\scshape  wurde 1-mal zitiert.}
\end{bsp} 

Ist ein Eintrag ohne Zitation in die Bibliographie geraten, wird dies entsprechend angezeigt:
\begin{bsp}
Böhm – Eickstedt 2001\\
\indent S. Böhm – K.-V. v. Eickstedt (Hrsg.), Ithake. Festschrift Jörg Schäfer (Würzburg 2001)  $\vert$  {\scshape  wurde {\color{red}{keinmal}} zitiert.}
\end{bsp} 



Für alle Sprachen außer Deutsch:
 \begin{bsp}
Böhm – Eickstedt 2001\\
 \indent S. Böhm – K.-V. v. Eickstedt (ed.), Ithake. Festschrift Jörg Schäfer (Würzburg 2001) $\vert$  {\scshape cited {{\color{red}{not once}}}.}
 \end{bsp}
  
 \begin{bsp}
Böhm – Eickstedt 2001\\
 \indent S. Böhm – K.-V. v. Eickstedt (ed.), Ithake. Festschrift Jörg Schäfer (Würzburg 2001) $\vert$  {\scshape cited 1 time.}
 \end{bsp}

Bei mehrmaligeb Zitationen ändert sich die Ausgabe entsprechend:
  \begin{bsp}
 Böhm – Eickstedt 2001\\
 \indent S. Böhm – K.-V. v. Eickstedt (ed.), Ithake. Festschrift Jörg Schäfer (Würzburg 2001) $\vert$  {\scshape cited 3 times.}
\end{bsp}
 
 \subsubsection{initials}\label{initials}
\footnote{\url{http://tex.stackexchange.com/a/295486/98739}}


\subsection{Optionen der Literatureinträge}\label{optionen-literatur}
\subsubsection{ancient}\label{ancient}
 \DescribeMacro{ancient} \emph{Die Formatierungen von |ancient| und |frgancient| sind vom Bibliographiestil |geschichtsfrkl| v.1.1 (Jonathan Zachhuber) inspiriert und modifiziert worden.}
 
Bei dem Zitieren antiker Autoren empfiehlt es sich diese Werke mit der Option |antik| zu versehen. Wir betrachten wieder ein Beispiel:
\begin{lstlisting}
@Book{Cic_Att,
  author       = {Tullius Cicero, Marcus},
  editor       = {Kasten, Helmut},
  title        = {Atticus-Briefe},
  publisher    = {Artemis {\&} Winkler},
  location     = {Düsseldorf and Zürich},
  year         = {1980},
  series       = {Tusculum Bücherei},
  edition      = {3},
  origyear     = {1959},
  origtitle    = {epistulae ad Atticum},
  origlanguage = {latin},
  translator   = {Kasten, Helmut},
  shorthand    = {Cic. Att.},
  shortauthor  = {Cicero},
  keywords     = {ancient},
  options      = {ancient},
}
\end{lstlisting}
 Beim Zitieren wird  nur das Feld |shorthand| berücksichtigt: 
 \begin{lstlisting}
 \cite[1, 3,3]{Cic_Att}\end{lstlisting} 
 liefert
 \begin{refsection}
\begin{bsp}
\cite[1, 3,3]{Cic_Att}
%Cic. Att. 1, 3,3
\end{bsp}\end{refsection}
Dieses Feld |shorthand| wird für das Literaturverzeichnis verwendet, wo es als ›Schlüssel‹ auftaucht.: 
\printbib{Cic_Att}




Es gibt auch antike Texte, die in einem Sammelband (|@incollection|) herausgegeben sind. 
Dieser Fall stellt jedoch kein Problem dar und wird analog zu |@book| geplottet.
Ein Beispiel verschafft Klarheit:
\begin{lstlisting}
@Inbook{Cic_Sest,
  author       = {Tullius Cicero, Marcus},
  title        = {Rede für P.\ Sestius},
  booktitle    = {Die politischen Reden},
  year         = {1993},
  editor       = {Fuhrmann, Manfred},
  volume       = {II},
  publisher    = {Artemis \& Winkler},
  pages        = {110--185},
  origlanguage = {latin},
  series       = {Sammlung Tusculum},
  location     = Munich,%bibstring: @String{Munich = {\iflanguage{english}{Munich}{\iflanguage{italian}{Monaco}{\iflanguage{french}{Munich}{München}}}}}
  intranslator = {Fuhrmann, Manfred},
  keywords     = {ancient},
  options      = {ancient},
  origtitle    = {pro P. Sestio},
  shortauthor  = {Cicero},
  shorthand    = {Cic. Sest.},
}
\end{lstlisting}


\printbiball{Cic_Sest}


\subsubsection{frgantik}\label{frgantik}
 \DescribeMacro{frgantik}
 Mit der Option |frgantik| werden die Bibliographieeinträge versehen, die antike Fragmente beinhalten, da die Herausgeber dieser Fragmente relevant sind. Dies wird in der Zitierweise berücksichtigt. 
\begin{lstlisting}
@Book{Fest,
  author      = {Pompeius Festus, {Sex}tus},
  editor      = {Lindsay, Wallace Martin},
  title       = {De verborum significatu quae supersunt cum Pauli epitome},
  publisher   = {Teubner},
  location    = {Leipzig},
  year        = {1965},
  series      = {Bibliotheca scriptorum et Grecorum et Romanorum Teubnerina},
  origyear    = {1913},
  shorthand   = {Fest.},
  shortauthor = {Festus},
  keywords    = {ancient},
  options     = {frgancient},
  shorteditor = {L},
}
\end{lstlisting}
 Zitiert man diesen Eintrag durch |\cite[3]{Fest}|, so wird der oder die Herausgeber genannt; ist das Feld |Shorteditor| ausgefüllt, dann wird dieses angegeben, ansonsten die Nachnamen aus |Editor|.
 \begin{bsp}
 Fest. 3 L
 \end{bsp}
 In der Bibliographie  unterscheidet sich der Eintrag gegenüber |antik|  geringfügig:
\printbib{Fest}



\subsubsection{corpus}\label{corpus}
\DescribeMacro{corpus}
Für bestimmte Corpora (Inschriften, Münzen, etc.) wird für gewöhnlich mit einer gängigen Abkürzung zitiert.
Diese Abkürzung des Corpus wird im Bibliographieeintrag unter |shorthand| eingetragen.
Nun kann man sehr einfach das gewünschte Corpus in der Fußnote zitieren, mit  |prenote| und |postnote|-Feldern. Bei anderen Autoren-Einträge, die mittels |shorthand| zitiert werden, wird ein Komma zwischen Nachname und |postnote| gesetzt. Dank |options=corpus| fällt dieses Komma weg.

Das Beispiel zeigt die Option für die lateinischen Inschriften:
\begin{lstlisting}
@Book{CIL,
  title     = CIL-lang,
  location  = {Berlin},
  year      = {1863--},
  shorthand = CIL-short,
  keywords  = {corpus}, %!
  options   = {corpus},
}
\end{lstlisting}


Zitiert wird, wie gewöhnlich, mit 
\begin{lstlisting}
\cite[06, 01234]{CIL}.\end{lstlisting}
 Daraus wird:


\begin{bsp}
CIL 06, 01234
%Cic. Att. 1, 3,3
\end{bsp}
Dieses Feld |shorthand| wird für das Literaturverzeichnis verwendet, wo es als ›Schlüssel‹ auftaucht.: 
\printbib{CIL}

Aufgrund der Setzung von |keywords=Sigel| können diese Art von Corpora in einer separaten Bibliographie aufgeführt werden. Siehe dazu \cref{bibliographie}

\changes{v1.1}{2015/06/15}{Modifikation der Option |corpus|.}


 \section{Beschreibung der Eintragtypen (Beispiele)} \label{beispiele}


 Der |archaeologie|-Zitierstil definiert unterschiedliche bibliography driver, die es erlauben verschiedene Arten Werke zu zitieren. 
 Diese werden im Folgenden zusammen mit den für sie relevanten Optionen beschrieben.



 \subsection{Typ \texttt{@book}}\label{book}
 \DescribeMacro{@book}
 \DescribeMacro{@collection}\footnote{Der Typ |@collection| entspricht hier dem Typ |@book|.}
 Fangen wir ganz einfach an: Zu einem einfachen Buch sieht der Eintrag in der |bib|-Datei ungefähr folgendermaßen aus:
\begin{lstlisting}
@Book{Mann_2011,
  author    = {Mann, Christian},
  title     = {\enquote{Um keinen Kranz, um das Leben kämpfen wir!}},
  subtitle  = {Gladiatoren im Osten des Römischen Reiches und die Frage der Romanisierung},
  publisher = {Verlag Antike},
  location  = {Berlin},
  year      = {2011},
  series    = {Studien zur Alten Geschichte},
  number    = {14},
}
\end{lstlisting}
Zitiert man daraus mit 
\begin{lstlisting}
\footnote{\cite[Vgl.][142--144]{Mann_2011}.}
\end{lstlisting} dann erscheint in der Fußnote\footnote{Vgl. Mann 2011, 142--144.}.
In der Bibliographie wird der Eintrag wiedergegeben mit:
\printbib[5em]{Mann_2011}

\subsubsection{Festschriften, Gedenkschriften u.\,ä.}
Um Festschriften/Gedenkschriften/Ausstellungskataloge/Auktionskataloge entsprechend zu zitieren, gehört der Zusatz ins Feld |titleaddon|, bzw. wenn es sich um ein |@inbook| oder |@inproceedings| handelt, entsprechend ins Feld |Booktitleaddon| (\cref{inbook}).
\begin{lstlisting}
@Book{Boehm_2001,
  editor     = {Böhm, Stephanie and Eickstedt, Klaus-Valtin von},
  title      = {Ithake},
  publisher  = {Ergon-Verlag},
  location   = {Würzburg},
  year       = {2001},
  titleaddon = {Festschrift Jörg Schäfer},
}
\end{lstlisting}

\printbib[9em]{Boehm_2001}
 
 \subsubsection{Übersetzungen u.ä.}
Liegt ein Werk in Übersetzung vor, gibt es die Möglichkeit den Übersetzer/in, Ausgangssprache und ursprünglichen Titel automatisch anzeigen zu lassen.
Dies funktioniert mit den Literatureinträgen |related| und |relatedtype|  (\cref{review}).

Im Beispiel wird die Verwendung klar:
Das Ausgangswerk ist |Zanker_2014|.
\begin{lstlisting}
@Book{Zanker_2014,
  author    = {Zanker, Paul},
  title     = {Die römische Stadt},
  subtitle  = {Eine kurze Geschichte},
  publisher = {CHB},
  location  = Munich,
  year      = {2014},
  language  = {german},
}
\end{lstlisting}
Die Übersetzung davon ist |Zanker_2013|:
\begin{lstlisting}
@Book{Zanker_2013,
  author      = {Zanker, Paul},
  title       = {La città romana},
  publisher   = {GLF},
  location    = {Roma-Bari},
  year        = {2013},
  series      = {Storia della città},
  translator  = {Senatore, Anna Maria},
  language    = {italian},
  related     = {Zanker_2014},
  relatedtype = {translationof},
}
\end{lstlisting}


\printbib[5em]{Zanker_2013}


 \subsubsection{Mehrbändige Monographien mit Untertitel}
Es kommt vor, dass man einen Band einer mehrbändigen Monographie zitiert, der einen eigenen Untertitel hat.
Damit die jeweilige Zahl des Bandes an korrekter Stelle ausgegen wird, 
muss ein Bibliographieeintrag wie folgt aussehen:
\begin{lstlisting}
@Book{MacDonald_1986,
  author    = {MacDonald, William L.},
  title     = {An urban Appraisal},
  publisher = {YUP},
  location  = {New Haven and London},
  year      = {1986},
  maintitle = {The Architecture of the Roman Empire},
  volume    = {II},
  series    = {Yale Publictions in the History of Art},
  number    = {35},
}
\end{lstlisting}


Damit werden Haupttitel der Monographie (|Maintitle|) und Titel des Bandes (|Titel|) getrennt voneinander eingegeben, sodass die Bandzahl (|volume|) vor den Titel gesetzt werden kann.


\printbib[7em]{MacDonald_1986}


 \subsection{Typ \texttt{@inbook / @incollection}}\label{inbook}

 \DescribeMacro{@incollection}
 Kapitel aus Sammelbändern macht man am Besten mit dem Typ |@incollection|. 
 Am besten sieht man das wieder an Hand eines Beispiels:
 
 \begin{lstlisting}
@Incollection{Carter_2014,
  author    = {Carter, Michael J. and Edmondson, Jonathan},
  title     = {Spectacle in Rome, Italy, and the Provinces},
  pages     = {537--558},
  editor    = {Bruun, Christer and Edmondson, Jonathan},
  booktitle = {The Oxford Handbook of Roman Epigraphy},
  publisher = {Oxford University Press},
  location  = {Oxford},
  year      = {2014},
}
\end{lstlisting}

\printbib[12em]{Carter_2014}

 
Zitiert man einen Beitrag aus einer Festschrift, o.ä., dann kann das folgendermaßen aussehen:
\begin{lstlisting}
@Incollection{Hoelscher_2001,
  author         = {Hölscher, Tonio},
  title          = {Schatzhäuser -- Banketthäuser?},
  pages          = {143--152},
  editor         = {Böhm, Stephanie and Eickstedt, Klaus-Valtin von},
  booktitle      = {Ithake},
  publisher      = {Ergon-Verlag},
  location       = {Würzburg},
  year           = {2001},
  booktitleaddon = {Festschrift Jörg Schäfer},
}
\end{lstlisting}

In der Bibliographie ist die Darstellung:


\printbib[6em]{Hoelscher_2001}
 \subsubsection{Kleinere Kurzreihen}
Es kommt auch vor, dass man einen Band einer kleinen Reihe (nicht Serie) zitieren muss.
Zum Beispiel dieses Buch:
\begin{lstlisting}
@Incollection{Fentress_2003,
  author       = {Fentress, Elizabeth and John Bodel and Adam Rabinowitz and Rabun Taylor},
  title        = {Cosa in the Republic and Early Empire},
  pages        = {13--62},
  editor       = {Fentress, Elizabeth},
  booktitle    = {An Intermittent Town},
  booksubtitle = {Excavations 1991--1997},
  publisher    = UMP,
  location     = {Ann Arbor, Mich.},
  year         = {2003},
  volume       = {V},
  series       = MemAmAc-long,
  number       = {2},
  maintitle    = {Cosa},
  shortseries  = MemAmAc-short,
}
\end{lstlisting}

Es handelt sich also um den fünften Band (|volume|) mit dem Titel (|title|) \emph{Cosa in the Republic and Early Empire} der Reihe \emph{Cosa} (|maintitle|) darstellt, was wiederum den zweiten Band (Number) der Serie \emph{MemAmAc} ist.


\printbiball[7em]{Fentress_2003}

 
 
 \subsubsection{Bestandskatalog}
 Die Angabe von Bestandskatalogen weicht in einer Kleinigkeit von Sammelbänden o.ä. ab:
 Es wird kein Titel genannt. Damit fällt auch das Komma nach dem Autor weg.
Zwei Beispiele verdeutlichen dies.

\begin{lstlisting}
@Inbook{Kohlmeyer_1983,
  author       = {K. Kohlmeyer},
  booktitle    = {Tierbilder aus vier Jahrtausenden},
  year         = {1983},
  editor       = {U. Gehrig},
  booksubtitle = {Antiken der Sammlung Mildenberg},
  pages        = {20 Nr. 9},
  location     = {Mainz},
}
\end{lstlisting}
und

\begin{lstlisting}
@Inbook{Parlasca_1969,
  author    = {K. Parlasca},
  booktitle = {Helbig},
  year      = {1969},
  volume    = {III},
  edition   = {4},
  pages     = {98\psq\ Nr. 2176},%! \psq = following page
  location  = {Tübingen},
}
\end{lstlisting}


\printbib[6em]{Kohlmeyer_1983}
\begin{refsection}\end{refsection}
\printbiball[6em]{Parlasca_1969}

 Der Typ |@inbook| entspricht hier dem Typ |@incollection|.


 \subsection{Typ \texttt{@article}}\label{article}
\DescribeMacro{@article}
\begin{lstlisting}
@Article{Evangelidis_2014,
  author       = {Evangelidis, Vasilis},
  title        = {Agoras {and} Fora},
  subtitle     = {Developments in the Central Public Space of the Cities of Greece during the {Roman} Period},
  journaltitle = BSA-long,
  shortjournal = BSA-short,
  volume       = {109},
  pages        = {335--356},
  year         = {2014},
  doi          = {10.1017/s006824541400015x},
}
\end{lstlisting}
\printbib[7em]{Evangelidis_2014}
\begin{refsection}\end{refsection}

Es gibt die Möglichkeit allen Einträgenl den Publikationszustand anzugeben.
Die beliebige Eingabe erfolgt im Feld |pubstate| und wird ans Ende des bibliographischen Eintrages gesetzt.

Es empfiehlt sich allerdings sich auf ein paar Standarts zu beschränken, da diese sprachabhängig ausgegeben werden:
\begin{description}
\item[inpreparation] Typoskript wird für die Publikation vorbereitet. \\(|pubstate = {inpreparation}|)
\item[submitted] Typoskript wurde bei der Zeitschrift eingereicht.\\(|pubstate = {submitted}|)
\item[forthcoming] Typoskript wurde von der Zeitschrift akzeptiert.\\(|pubstate = {forthcoming}|)
\item[inpress] Typoskript liegt bereits als fertig gesetzter Artikel vor.\\(|pubstate = {inpress}|)
\item[prepublished] Artikel liegt in einer (online) Vorabversion vor. \\(|pubstate = {prepublished}|)
\end{description}

Ein Beispiel zeigt die Anwendung für |pubstate = {forthcoming}|
\begin{lstlisting}
@Article{Bossert,
  author       = {Lukas C. Bossert},
  title        = {\ldots\ \textsc{in formam anitqvam restitvto}?},
  subtitle     = {Überlegungen zur Inschrift der ›Porticus Deorum Consentium‹ (CIL\,VI 102) und ihren Ergänzungen im 19.{\,}Jh.},
  journaltitle = {BeStAR. Berliner Studien zum Antiken Rom},
  shortjournal = {BeStAR},
  volume       = {2},
  pubstate     = {forthcoming},
}
\end{lstlisting}

\printbib[8em]{Bossert}



\subsection{Typ \texttt{@proceedings}}\label{proceedings}

 Für Beiträge innerhalb eines Konferenzbandes müssen die Felder |venue|, |eventdate| und |eventtitle| ausgefüllt werden. Ansonsten alle anderen Felder entsprechend wie bei |@book|:
 \begin{lstlisting}
@Proceedings{Kurapkat_2014,
  title        = {Die Architektur des Weges},
  year         = {2014},
  editor       = {Kurapkat, Dietmar and Schneider, Peter I. and Wulf-Rheidt, Ulrike},
  number       = {11},
  series       = DiskAB-long,
  publisher    = {Schnell + Steiner},
  organization = {Architekturreferat des DAI},
  eventdate    = {2012-02-08/2012-02-11},
  eventtitle   = {Kolloquium Architekturreferat des DAI},
  location     = {Regensburg},
  shortseries  = DiskAB-short,
  subtitle     = {Gestaltete Bewegung im gebauten Raum},
  venue        = {Berlin},
}
\end{lstlisting}

So wird daraus: 
 
\printbiball[7em]{Kurapkat_2014}

\subsection{Typ \texttt{@inproceedings}}\label{inproceedings}

Wie bei |@proceedings| so auch hier:
 \begin{lstlisting}
@Inproceedings{Torelli_1991,
  author       = {Torelli, Mario},
  title        = {Il \enquote{diribitorium} di Alba Fucens e il \enquote{campus} eroico di Herdonia},
  pages        = {39--63},
  editor       = {Mertens, Josef},
  booktitle    = {Comunitá indigene e problemi della romanizzazione nell’Italia centro\--meri\-dionale (IV--III sec. a.C.)},
  booksubtitle = {Actes du Colloque International Organisé à l'Occasion du 50. Anniversaire de l'Academia Belgica et du 40. Anniversaire des Fouilles Belges en Italie},
  location     = {Bruxelles},
  publisher    = {Institut Historique Belge de Rome},
  year         = {1991},
  venue        = {Roma, Academia Belgica},
  eventdate    = {1990-02-01/1990-02-03},
  hyphenate    = {italian},
  language     = {italian},
  number       = {29},
  series       = {Études de philologie, d'archéologie et d'histoire anciennes},
  shorttitle   = {Il \enquote{diribitorium}},
}
\end{lstlisting}

So wird daraus: 
 
\printbiball[4em]{Torelli_1991}

 \subsection{Typ \texttt{@inreference}}\label{inreference}
 \DescribeMacro{@inreference}
 Mit dem Typ |@inreference| können  Lexikonartikel zitiert werden.
 Dabei gibt es generall zwei Möglichkeiten wie der Eintrag dargestellt wird:
 Entweder gleich wie alle anderen Einträge mit ›Autor Jahr‹-Angabe, oder speziell nach den Vorgaben des Deutschen Archäologischen Instituts, sodass der Eintrag in der Fußnote vollzitiert und -referenziert ist:
›Lexikon Band (Jahr) Seitenzahlen s.\,v. Titel des Artikels (Autor)‹.
Für die erste Variante bedarf es keine Änderung, präferiert man die zweite Variante, so muss in der Präambel die Option |inreferences=true| gesetzt werden. \DescribeMacro{inreferences=true}

An einem Beispiel wird die Anwendung deutlich.
  \begin{lstlisting}
@Inreference{Neils_1994,
  author    = {Neils, Jenifer},
  title     = {Theseus},
  booktitle = LIMC-short,
  pages     = {922--951},
  year      = {1994},
  volume    = {7.1},
  keywords  = {lexikon},
}
\end{lstlisting}

Zusätzlich lohnt es sich den Eintrag über |related| mit dem Hauptwerk (in diesem Falle ›LIMC‹) zu verküpfen:
  \begin{lstlisting}
@Reference{LIMC,
  Title                    = LIMC-long,
  Keywords                 = {corpus},
  Options                  = {corpus},
  Shorthand                = LIMC-short
}
 \end{lstlisting}
 Dies hat den Vorteil, dass man die Abkürzung ›LIMC‹ in der Teilbibliographie ›Lexika und Corpora‹ (\cref{bib:corpus}) danke des |keywords = {corpus}| automatisiert auflösen kann. Dafür muss der Bibliographieeintrag |@reference{LIMC}| nicht manuell zitiert werden, da dies über den Eintrag |related| bei |@inreference{Neils_1994}| dynamischt erfolgt.
 
 

 Verwendet man den normalen |\cite|\oarg{prenote}\oarg{postnote}\marg{Schlüssel}-Befehl, dann wird bei eingetragener Option (|lexikon|) die \oarg{postnote} nicht nachgestellt, sondern nach der Jahresangabe des Lexikons gesetzt.
 Ist die \oarg{postnote} leer, dann werden die Seiten aus dem Bibliographieeintrag ausgelesen und an diese Stelle gesetzt:
  
 \begin{bsp}
	\ldots\ (|\cite[vgl.][930 Nr. 283]{Neils_1994}|).
	
	...\ (vgl. LIMC 7.1 (1994) 930 Nr. 283 s. v. Theseus (J. Neils)). \textcolor{red}{WRONG!}
	% \ldots (\fullcite[vgl.][930 Nr. 283]{Neils_1994}).
 \end{bsp}
 
\DescribeMacro{\parencite} 
Dieses Beispiel zeigt, dass keine Klammerregelung angewendet wurde, was nicht gewünscht ist. 
Es ist zu beachten, dass bei gewählter Option  |inreferences=true| (in der Präambel) und bei einer gewünschten  Zitation in Klammern (), der Befehl |\parencite|\marg{Schlüssel} verwendet wird (\ref{cite-befehle}), damit die Klammerregelung automatisch angewendet wird.
Dies funktioniert so: 
 
 \begin{bsp}
	\ldots\ |\parencite[vgl.][930 Nr. 283]{Neils_1994}.|
 
 	...\ (vgl. LIMC 7.1 [1994] 930 Nr. 283 s. v. Theseus [J. Neils]). \textcolor{green}{CORRECT!}

	%\ldots \parencite{\fullcite[vgl.][930 Nr. 283]{Neils_1994}}.
 \end{bsp}

\DescribeMacro{\parencites}
Gleiches gilt auch für das Zitieren von mehreren Einträgen in Klammern: Dafür wird analog der Befehl |\parencites|\marg{Schlüssel} verwendet:
 \begin{bsp}
	\ldots\ |\parencites[vgl.][930 Nr. 283]{Neils_1994}[9]{Nieddu_1995}.|
 
 	...\ (vgl. Neils 1994, 930 Nr. 283; Nieddu 1995, 9).
 	 \end{bsp}

 	bzw. mit der Präambel-Option |inreferences=true|:
 	 \begin{bsp}

 	\ldots (vgl. LIMC 7.1 [1994] 930 Nr. 283 s. v. Theseus [J. Neils]; LTUR 2 [1995] 9 s. v. Dei Consentes [G. Nieddu])

	%\ldots\ \parencites[vgl.][930 Nr. 283]{Neils_1994}[9]{Nieddu_1995}
	%\ldots \parencite{\fullcite[vgl.][930 Nr. 283]{Neils_1994}}.
 \end{bsp}


Da der Bibliographieeintrag ein Lexikonartikel ist (|@inreference|), werden bei gewählter Präambel-Option |inreferences=true| alle Lexikoneinträge von der Bibliographie ausgeschlossen, da sie ja in der Fußnote vollzitiert werden.

Ansonsten  (Präambel-Option |inreferences=false| = default)  sieht der Eintrag in der Bibliographie so aus:


\printbib{Neils_1994}
\printbib{Nieddu_1995}

 



 Diese Handhabe gilt nicht nur für ›kanonische Lexika‹ der Altertumswissenschaften (RE, LIMC, DNP, LTUR, LÄ, etc.) sondern kann auf alle Lexika angewendet werden.
 
 Da es nicht für alle Lexika eine Abkürzungskonvention  des  Titels gibt, können und sollten diese selbst abkürzt werden.\footnote{Dies ist sehr emphehlenswert, da in der Bibliographie der Abstand zwischen der Autor-Jahr-Abkürzung und dem Vollzitat sich nach der längsten ›Shorthand‹ richtet.}
 
 Ein weiteres Beispiel:
\begin{lstlisting}
@Inreference{Weinbrenner_1914,
  author    = {Weinbrenner},
  title     = {Rennbahn},
  booktitle = {Lexikon d. T.},
  pages     = {636--637},
  year      = {1914},
  related   = {Lexikon-der-Technik},
  volume    = {9},
  number    = {2},
}
\end{lstlisting}
Der Lexikoneintrag ist über |related| verknüpft mit:
\begin{lstlisting}
@Reference{Lexikon-der-Technik,
  title     = {Lexikon der gesamten Technik und ihrer Hilfswissenschaften},
  year      = {1904--1920},
  edition   = {2},
  editor    = {Otto Lueger},
  keywords  = {corpus},
  location  = {Stuttgart},
  shorthand = {Lexikon d. T.},
}
\end{lstlisting}
Wichtig ist, dass |booktitle| des |@inreference| und |shorthand| des |@reference| gleich sind, damit der Titel des Lexikons entsprechend aufgelöst werden kann.


\begin{refsection}
\begin{bsp}
\nocite{Lexikon-der-Technik,Weinbrenner_1914}
 \printbibliography[keyword=Sigel,title={Abkürzungen und Sigel}]
\printbibliography[title={Forschungsliteratur},notkeyword=Sigel]
\end{bsp}
 \end{refsection}


 
 \subsection{Typ \texttt{@review}}\label{review}
\DescribeMacro{@review}
Rezensionen, die in Zeitschriften erschienen sind, werden als |@review| verarbeitet.
Oft zitiert man die Rezension und das rezensierte Werk in seiner Arbeit.
Das folgende Beispiel geht auf diesen Fall ein und erläutert die relevanten Punkte und Besonderheiten.

Dafür sind zwei Bibliographieeinträge notwendig.
Zuerst das rezensierte Werk:
\begin{lstlisting}
@Book{Welch_2007,
  author    = {Welch, Katherine E.},
  title     = {The {Roman} Amphitheatre},
  subtitle  = {From its Origins to the Colosseum},
  publisher = {CUP},
  location  = {Cambridge and New York},
  year      = {2007},
}
\end{lstlisting}
und schließlich die dazugehörige Rezension:
\begin{lstlisting}
@Review{Bell_2011,
  author       = {Bell, Sinclair},
  number       = {1},
  pages        = {1--4},
  volume       = {115},
  journaltitle = AJA-long,
  shortjournal = AJA-short,
  related      = {Welch_2007},
  relatedtype  = {reviewof},
  year         = {2011},
  publisher    = {Archaeological Institute of America},
}
\end{lstlisting}

Im Bibliographieeintrag der Rezension (|Bell_2011|) wird im Feld |Related| mittels Eintrag |{Welch_2007}| Bezug auf das rezensierte Werk (|Welch_2007|) genommen.
Das Feld |Relatedtype| enthält die Information in welcher Beziehung das Werk (|Bell_2011|) zu |Welch_2007| steht. Die Angabe |reviewof| ist der |bibstring| für Rezensionen und enthält sprachabhängig (Deutsch und Englisch) die Bezeichnung |Rez. zu| bzw. |Review of|.
Dies ist für die Bibliographie relevant: Es ist nun nicht mehr notwendig das rezensierte Werk mit allen dazugehörigen Angaben einzutippen, da  dies automatisch über die |Related|-Funktion erfolgt.


\printbiball{Bell_2011}

 
Der Vorteil mit |Related| und |Relatedtype| zu arbeiten liegt in der Dynamik: Ändert sich eine Angabe im rezensierten Werk, wird dies bei der Rezension automatisch geändert.
Zudem wird das rezensierte Werk bei der Erwähung in der Rezension nicht automatisch in die Bibliographie als eigener Eintrag aufgenommen, sondern erst dann, wenn es selbst explizit zitiert wird.

\subsubsection{Rezensionen mit eigenem Titel}
Manche Rezensionen sind ausführlicher und haben daher einen eigenen Titel, der angegeben werden soll.
In diesem Fall wird das Feld |Title| ausgefüllt, ansonsten bleibt alles gleich: 
\begin{lstlisting}
@Review{Hufschmid_2010,
  author       = {Hufschmid, Thomas},
  title        = {Von Caesars \emph{theatron kynegetikon} zum \emph{amphitheatrum novum} Vespasians},
  pages        = {487--504},
  volume       = {23},
  journaltitle = JRA-long,
  shortjournal = JRA-short,
  related      = {Welch_2007},
  relatedtype  = {reviewof},
  year         = {2010},
}
\end{lstlisting}
In der Bibliographie wird dann zuerst der eigene Titel der Rezension ausgegeben, dann die Angaben zum rezensierten Werk.


\printbib[7em]{Hufschmid_2010}


\subsubsection{Sammelrezensionen}
Es kann vorkommen, dass Rezensionen nicht nur ein Werk unter die Lupe nehmen, sondern zwei oder noch mehr.
Dies ist jedoch kein Problem, da das Vorgehen fast analog zu den oben genannten Beispielen ist.
\begin{lstlisting}
\citeauthor{Taylor_2008} hat nicht nur das bereits erwähnte Buch
 \citetitle{Welch_2007} von \citeauthor{Welch_2007} rezensiert, 
 sondern es zugleich mit \citeauthor{Sear_2006}s \citetitle{Sear_2006} verglichen.
\end{lstlisting}

\begin{bsp}\citeauthor{Taylor_2008} hat nicht nur das bereits erwähnte Buch \citetitle{Welch_2007} 
von \citeauthor{Welch_2007} rezensiert, sondern es zugleich mit 
\citeauthor{Sear_2006}s \citetitle{Sear_2006} verglichen.\end{bsp}
\begin{lstlisting}
@re@Review{Taylor_2008,
  author       = {Taylor, Rabun},
  number       = {3},
  pages        = {443--445},
  volume       = {67},
  journaltitle = {Journal of the Society of Architectural Historians},
  related      = {Sear_2006,Welch_2007},
  relatedtype  = {reviewof},
  year         = {2008},
}
\end{lstlisting}
\begin{lstlisting}
@Book{Sear_2006,
  author    = {Sear, Frank},
  title     = {Roman Theatres},
  subtitle  = {An Architectural Study},
  publisher = OUP,
  location  = {Oxford},
  year      = {2006},
  series    = {Oxford Monographs on Classical Archeology},
}
\end{lstlisting}


\printbib[5em]{Taylor_2008}


 \subsection{Typ \texttt{@thesis}}\label{thesis}
Master- und (unpublizierte) Doktorarbeiten sind als |@thesis| aufzunehmen.
Wichtige Felder sind |type=|\marg{|phdthesis|} bzw. \marg{|mathesis|}  und |institution=|\marg{Universität}.

Beispiel:
\begin{lstlisting}
@Thesis{Arnolds_2005,
  author      = {Markus Arnolds},
  title       = {Funktionen republikanischer und frühkaiserzeitlicher Forumsbasiliken in Italien},
  type        = {phdthesis},
  institution = {Ruprecht-Karls-Universität zu Heidelberg},
  eprint      = {urn:nbn:de:bsz:16-heidok-74406},
  eprinttype  = {urn},
  date        = {2005-05-31},
}
\end{lstlisting}
In der Bibiographie wird das zu:

\printbib[5em]{Arnolds_2005}

 
 \changes{v1.1}{2015/06/04}{Umsetzung von |@thesis| im Stil.}

 \subsection{Typ \texttt{@talk}}\label{talk}
 Mit dem neu definierten Eintragstyp |@talk| können Vorträge von Einzelkolloquien oder von Konferenzen zitiert werden.
 Folgende Felder können benutzt werden: 
 | author|,
| title|,
| subtitle|,
| titleaddon|,
| date|,
| venue|,
| institution|,
| eventtitle|,
| eventdate|,
| url|,
| urldate|,
| note|.
 
 \begin{lstlisting}
@Talk{Bergmann_2015,
  author      = {Bergmann, Birgit},
  title       = {\enquote{An exciting find}},
  date        = {2015-04-27},
  eventtitle  = {Kolloquium der Klassischen Archäologie},
  institution = {Freie Universität Berlin},
  subtitle    = {Neues zum Forums-Fries der Praedia Iuliae Felicis},
  titleaddon  = {(Pompeii II, 4)},
  url         = {https://www.antikezentrum.hu-berlin.de/de/veranstaltungskalender/bibergmann},
  urldate     = {2016-05-14},
  venue       = {Berlin},
}
\end{lstlisting}
 Für die Bibliographie wird dies zu: 

		\printbib[6em]{Bergmann_2015}
	
 \changes{v1.5}{2016/05/31}{Rückverweis}

\section{Verschiedenes}
\subsection{Anonymes Werk}\label{unbekannt}

 Für manche Artikel oder Bücher lässt sich kein Autor oder Herausgeber ermitteln. 
 Diese Werke werden dann als anonym gekennzeichnet und nicht nach dem (anonymen) Autor/Herausgeber zitiert, sondern nach einem gewählten |label|. 
\begin{lstlisting}
@Article{Cosa_1949,
  title        = {Cosa},
  subtitle     = {Republican Colony in Etruria},
  journaltitle = ClJ-long,
  shortjournal = ClJ-short,
  volume       = {45},
  pages        = {141--149},
  year         = {1949},
  label        = {Cosa},
  number       = {1},
}
\end{lstlisting}
 
 Das |Label| wurde in diesem Fall analog zum Titel gewählt (Cosa).
 Zitiert man dieses Werk in einer Fußnote, dann wird
\begin{lstlisting}
 \cite[vgl.][145--146]{Cosa_1949}
\end{lstlisting}
 zu

 \begin{bsp} 
 vgl. Cosa 1949, 145--146
 \end{bsp}
 
Und für die Bibliographie:
\printbib{Cosa_1949}


\newpage
 \section{Bibliographie}\label{bibliographie}
 \DescribeMacro{\printbibliography}
Wie bei jedem Dokument mit im Text zitierten Werken bedarf es einer Stelle, an der diese auch aufgeschlüsselt werden: die Bibliographie. 
Für Altertumswissenschaftler (und auch andere) ist es manchmal hilfreich verschiedene Bibliographien im Dokument zu haben, die unterschiedliche Arten von Werke beinhalten, bspw. ein Quellenverzeichnis, Abkürzungen und Forschungsliteratur. 
Nachfolgend wird gezeigt, wie dies berwerkstelligt werden kann. 
Zunächst sollten alle Quellen in der |bib|-Datei mit dem Feld
 |keyword={ancient},|
 versehen werden. 
Es bietet sich  an, mit (nummerierten) Unterbibliographien zu arbeiten, die über die Option  |heading=bibnumbered|, bzw. |heading=subbibnumbered| geladen werden.

\begin{lstlisting}
\printbibheading[%
							%heading=bibnumbered,%
									% commented for this documentation
									% uncomment if you want it numbered		
							title={Bibliography}] %heading for bibliography

\printbibliography[%
							keyword=ancient,%
							%heading=subbibnumbered,%
									% commented for this documentation
									% uncomment if you want it numbered
							title={Ancient authors and works}]

\printbibliography[%
							notkeyword=ancient,%
							notkeyword=corpus,%
							%heading=subbibnumbered,%
									% commented for this documentation
									% uncomment if you want it numbered
							title={Secondary literature}]
\end{lstlisting}



\begin{refsection}
\nocite{*}

Damit wird zuerst die Quellen und danach das \enquote{gewöhnliche} Literaturverzeichnis getrennt voneinander ausgegeben. 

\begin{bsp}
\renewcommand\bibfont{\normalfont\footnotesize}
\printbibheading[%
							%heading=bibnumbered,%
							title={Bibliography}] %heading for bibliography

\printbibliography[%
							keyword=ancient,%
							%heading=subbibnumbered,%
							title={Ancient authors and works}]

\printbibliography[%
							notkeyword=ancient,%
							notkeyword=corpus,%
							%heading=subbibnumbered,%
							title={Secondary literature}]
\end{bsp}


Es können mehrere Bibliographien über |\printbibliography| erstellt werden, die jeweils unterschiedliche Einträge haben können.
Beispielsweise kann man eine Unterbibliographie erstellen, in der nur die Sigeln (Lexika, Handbücher, Inschriftencorpora, etc) aufgeführt werden, sodass diese dann aus der |Forschungsliteratur| herausfliegen (dort  |notkeyword=Sigel| ergänzen). Dafür wird das Feld |keyword| auf den Inhalt |Sigel| ausgelesen:
\begin{lstlisting}
\printbibliography[%
					keyword=corpus,%
            		%heading=subbibnumbered,%
							% commented for this documentation
							% uncomment if you want it numbered
            		title={Abbreviation and corpora}]
\end{lstlisting}
Die Teilbibliographie umfasst dann nur Einträge, die unter |keywords = {corpus}| stehen haben:
\begin{bsp}
\printbibliography[%
						keyword={corpus},
          				%heading=subbibnumbered,%
							% commented for this documentation
							% uncomment if you want it numbered
            			title={Abbreviation of corpora}]\label{bib:corpus}
\end{bsp}

Ebenso hilfreich kann es sein, dass die verwendeten Abkürzungen der Zeitschriften und Reihen aufgelöst werden.
Dies geschieht  für die Zeitschriften mit:
\begin{lstlisting}
\printbiblist[%
					%heading=subbibnumbered,%
						% commented for this documentation
						% uncomment if you want it numbered
					title={Zeitschriftenabkürzungenl}]{shortjournal}
\end{lstlisting}

\begin{bsp}
\printbiblist[%
					%heading=subbibnumbered,%
						% commented for this documentation
						% uncomment if you want it numbered
					title={Zeitschriftenabkürzungenl}]{shortjournal}
\end{bsp}

bzw. für die Reihen:
\begin{lstlisting}
\printbiblist[%
							heading=subbibnumbered,%
							title={Reihenabkürzungen}]{shortseries}
\end{lstlisting}


\begin{bsp}
\printbiblist[heading=subbibnumbered,title={Reihenabkürzungen}]{shortseries}\end{bsp}

\end{refsection}
\clearpage
\section{Antike Autoren und Werke}\label{ancient-list}
Bibliography |archaeologie-ancient.bib| with a list of ancient authors and texts and their abbrevviation according to  Der Neue Pauly.
The bold entry in the second line is the |bibtexkey|.
All the entries have |keywords={ancient}|, |options={ancient}|.
\begin{multicols}{2}
	\begin{footnotesize}
\begin{description}[%
			%	style=multiline,
				style=nextline,
				leftmargin=2cm,
				font=\normalfont]

\item[Acc.] Accius \\ \textbf{Acc}
\item[Ach. Tat.] Achilleus Tatios \\ \textbf{Ach\_Tat}
\item[Aet.] Aetios \\ \textbf{Aet}
\item[Aischyl. Ag.] Aischylos Agamemnon\\ \textbf{Aischyl\_Ag}
\item[Aischyl. Choeph.] Aischylos Choephoroi\\ \textbf{Aischyl\_Choeph}
\item[Aischyl. Eum.] Aischylos Eumenides\\ \textbf{Aischyl\_Eum}
\item[Aischyl. Pers.] Aischylos Persae\\ \textbf{Aischyl\_Pers}
\item[Aischyl. Prom.] Aischylos Prometheus\\ \textbf{Aischyl\_Prom}
\item[Aischyl. Sept.] Aischylos Septem adversus Thebas\\ \textbf{Aischyl\_Sept}
\item[Aischyl. Suppl.] Aischylos Supplices (Hiketides)\\ \textbf{Aischyl\_Suppl}
\item[Aisop.] Aisopos \\ \textbf{Aisop}
\item[Alex. Aphr.] Alexandros von Aphrodisias \\ \textbf{Alex\_Aphr}
\item[Alk.] Alkaios \\ \textbf{Alk}
\item[Ambr. epist.] Ambrosius epistulae\\ \textbf{Ambr\_epist}
\item[Ambr. exc. Sat.] Ambrosius de excessu fratris (Satyri)\\ \textbf{Ambr\_exc\_Sat}
\item[Ambr. obit. Theod.] Ambrosius de obitu Theodosii\\ \textbf{Ambr\_obit\_Theod}
\item[Ambr. obit. Valent.] Ambrosius de obitu Valentiniani (iunioris)\\ \textbf{Ambr\_obit\_Valent}
\item[Ambr. off.] Ambrosius de officiis magistrorum\\ \textbf{Ambr\_off}
\item[Ambr. paenit.] Ambrosius de paenitentia\\ \textbf{Ambr\_paenit}
\item[Amm.] Ammianus Marcellinus \\ \textbf{Amm}
\item[Anakr.] Anakreon \\ \textbf{Anakr}
\item[Anaxag.] Anaxagoras \\ \textbf{Anaxag}
\item[Anaximand.] Anaximandros \\ \textbf{Anaximand}
\item[Anaximen.] Anaximenes \\ \textbf{Anaximen}
\item[And.] Andokides \\ \textbf{And}
\item[Anth. Gr.] Antologia Graeca \\ \textbf{Anth\_Gr}
\item[Anth. Lat.] Antologia Latina \\ \textbf{Anth\_Lat}
\item[Anth. Pal.] Antologia Palatina \\ \textbf{Anth\_Pal}
\item[Anth. Plan.] Anthologia Planudea \\ \textbf{Anth\_Plan}
\item[Antiph.] Antiphon \\ \textbf{Antiph}
\item[Antisth.] Antisthenes \\ \textbf{Antisth}
\item[Ap.] Apostelgeschichte \\ \textbf{Ap}
\item[Ap.] Apokalypse \\ \textbf{Apk}
\item[Apoll. Rhod.] Apollonios Rhodios Argonautica\\ \textbf{Apoll\_Rhod}
\item[Apollod. bibl.] Apollodoros bibliotheke\\ \textbf{Apollod\_bibl}
\item[Apollod. epit.] Apollodoros epitome\\ \textbf{Apollod\_epit}
\item[App. Celt.] Appian Celtica\\ \textbf{App\_Celt}
\item[App. civ.] Appian bella civilia\\ \textbf{App\_civ}
\item[App. Hann.] Appian Hannibalica\\ \textbf{App\_Hann}
\item[App. Ib.] Appian Iberica\\ \textbf{App\_Ib}
\item[App. Ill.] Appian Illyrica\\ \textbf{App\_Ill}
\item[App. It.] Appian Italica\\ \textbf{App\_It}
\item[App. Lib.] Appian Libyca\\ \textbf{App\_Lib}
\item[App. Mac.] Appian Macedonica\\ \textbf{App\_Mac}
\item[App. Mithr.] Appian Mithridatius\\ \textbf{App\_Mithr}
\item[App. pr.] Appian proemium\\ \textbf{App\_pr}
\item[App. reg.] Appian regia\\ \textbf{App\_reg}
\item[App. Rom.] Appian historia romana\\ \textbf{App\_Rom}
\item[App. Samn.] Appian Samnitica\\ \textbf{App\_Samn}
\item[App. Sic.] Appian Sicula\\ \textbf{App\_Sic}
\item[App. Syr.] Appian Syriaca\\ \textbf{App\_Syr}
\item[Apul. Flor.] Apuleius florida\\ \textbf{Apul\_Flor}
\item[Apul. met.] Apuleius metamorphoses\\ \textbf{Apul\_met}
\item[Arat.] Aratos \\ \textbf{Arat}
\item[Archil.] Archilochos \\ \textbf{Archil}
\item[Archim.] Archimedes \\ \textbf{Archim}
\item[Archyt.] Archytas \\ \textbf{Archyt}
\item[Aristeid.] Ailios Aristeides \\ \textbf{Aristeid}
\item[Aristoph. Ach.] Aristophanes Acharnenses\\ \textbf{Aristoph\_Ach}
\item[Aristoph. Av.] Aristophanes Aves (Ornithes)\\ \textbf{Aristoph\_Av}
\item[Aristoph. Eccl.] Aristophanes  Ecclesiazusae\\ \textbf{Aristoph\_Eccl}
\item[Aristoph. Equ.] Aristophanes Equites (Hippeis)\\ \textbf{Aristoph\_Equ}
\item[Aristoph. Lys.] Aristophanes Lysistrata\\ \textbf{Aristoph\_Lys}
\item[Aristoph. Nub.] Aristophanes Nubes (Nephelai)\\ \textbf{Aristoph\_Nub}
\item[Aristoph. Pax] Aristophanes Pax (Eirene)\\ \textbf{Aristoph\_Pax}
\item[Aristoph. Plut.] Aristophanes Plutus\\ \textbf{Aristoph\_Plut}
\item[Aristoph. Ran.] Aristophanes Ranae (Batrachoi)\\ \textbf{Aristoph\_Ran}
\item[Aristoph. Thesm.] Aristophanes Thesmophoriazusae\\ \textbf{Aristoph\_Thesm}
\item[Aristoph. Vesp.] Aristophanes Vespae (Sphekes)\\ \textbf{Aristoph\_Vesp}
\item[Aristot. an.] Aristoteles de anima (peri psyches)\\ \textbf{Aristot\_an}
\item[Aristot. an post.] Aristoteles analytica posteriora\\ \textbf{Aristot\_an\_post}
\item[Aristot. an pr.] Aristoteles analytica priora\\ \textbf{Aristot\_an\_pr}
\item[Aristot. Ath. pol.] Aristoteles Athenaion politeia\\ \textbf{Aristot\_Ath\_pol}
\item[Aristot. cael.] Aristoteles de caelo\\ \textbf{Aristot\_cael}
\item[Aristot. cat.] Aristoteles categoriae\\ \textbf{Aristot\_cat}
\item[Aristot. col.] Aristoteles de coloribus (περὶ χρωμάτων)\\ \textbf{Aristot\_col}
\item[Aristot. div.] Aristoteles de divinatione (peri mantikes)\\ \textbf{Aristot\_div}
\item[Aristot. Eth. Eud.] Aristoteles  ethica Eudemia\\ \textbf{Aristot\_Eth\_Eud}
\item[Aristot. eth. Nic.] Aristoteles ethica Nicomachea\\ \textbf{Aristot\_eth\_Nic}
\item[Aristot. gen. an.] Aristoteles de generatione animalium\\ \textbf{Aristot\_gen\_an}
\item[Aristot. gen. corr.] Aristoteles de generatione et corruptione\\ \textbf{Aristot\_gen\_corr}
\item[Aristot. hist.] Aristoteles historia animalium\\ \textbf{Aristot\_hist}
\item[Aristot. m. mor.] Aristoteles magna moralia\\ \textbf{Aristot\_m\_mor}
\item[Aristot. metaph.] Aristoteles metaphysica\\ \textbf{Aristot\_metaph}
\item[Aristot. meteor.] Aristoteles meteorologica\\ \textbf{Aristot\_meteor}
\item[Aristot. mir.] Aristoteles mirabilia\\ \textbf{Aristot\_mir}
\item[Aristot. mot. an.] Aristoteles de motu animalium\\ \textbf{Aristot\_mot\_an}
\item[Aristot. mund.] Aristoteles de mundo (peri kosmu)\\ \textbf{Aristot\_mund}
\item[Aristot. oec.] Aristoteles oeconomica\\ \textbf{Aristot\_oec}
\item[Aristot. part. an.] Aristoteles de partibus animalium\\ \textbf{Aristot\_part\_an}
\item[Aristot. phgn.] Aristoteles physiognomica\\ \textbf{Aristot\_phgn}
\item[Aristot. phys.] Aristoteles physica\\ \textbf{Aristot\_phys}
\item[Aristot. poet.] Aristoteles poetica\\ \textbf{Aristot\_poet}
\item[Aristot. pol.] Aristoteles politica\\ \textbf{Aristot\_pol}
\item[Aristot. probl.] Aristoteles problemata\\ \textbf{Aristot\_probl}
\item[Aristot. rhet.] Aristoteles rhetorica\\ \textbf{Aristot\_rhet}
\item[Aristot. rhet. Alex.] Aristoteles rhetorica ad Alexandrum\\ \textbf{Aristot\_rhet\_Alex}
\item[Aristot. sens.] Aristoteles de sensu (περὶ αἰσθήσεως)\\ \textbf{Aristot\_sens}
\item[Aristot. somn.] Aristoteles de somno et vigilia (περὶ ὔπνου καὶ ἐγρηγόρσεως)\\ \textbf{Aristot\_somn}
\item[Aristot. soph. el.] Aristoteles sophistici elenchi\\ \textbf{Aristot\_soph\_el}
\item[Aristot. spir.] Aristoteles de spiritu\\ \textbf{Aristot\_spir}
\item[Aristot. top.] Aristoteles topica\\ \textbf{Aristot\_top}
\item[Arnob.] Arnobius adversus nationes\\ \textbf{Arnob}
\item[Arr. an.] Arrian Anabasis\\ \textbf{Arr\_an}
\item[Arr. cyn.] Arrian Cynegeticus\\ \textbf{Arr\_cyn}
\item[Arr. Ind.] Arrian Indica\\ \textbf{Arr\_Ind}
\item[Arr. per p. E.] Arrian Periplus ponti Euxeni\\ \textbf{Arr\_per\_p\_E}
\item[Arr. succ.] Arrian Istoria successorum Alexandri\\ \textbf{Arr\_succ}
\item[Arr. tact.] Arrian tactica\\ \textbf{Arr\_tact}
\item[Artem.] Artemidor von Daldis Oneirokritika\\ \textbf{Artem}
\item[Athan. ad Const.] Athanasios Apologia ad Constantium\\ \textbf{Athan\_ad\_Const}
\item[Athan. c. Ar.] Athanasios Apologia contra Arianos\\ \textbf{Athan\_c\_Ar}
\item[Athan. fuga] Athanasios Apologia de fuga sua\\ \textbf{Athan\_fuga}
\item[Athan. hist. Ar.] Athanasios Istoria Arianorum ad monachos\\ \textbf{Athan\_hist\_Ar}
\item[Athen.] Athenaios \\ \textbf{Athen}
\item[Aug. civ.] Augustinus de civitate dei\\ \textbf{Aug\_civ}
\item[Aug. conf.] Augustinus confessiones\\ \textbf{Aug\_conf}
\item[Aug. doct.christ.] Augustinus de doctrina christiana\\ \textbf{Aug\_doctchrist}
\item[Aug. epist.] Augustinus epistulae\\ \textbf{Aug\_epist}
\item[Aug. retract.] Augustinus retractationes\\ \textbf{Aug\_retract}
\item[Aug. serm.] Augustinus sermones\\ \textbf{Aug\_serm}
\item[Aug. soliloq.] Augustinus soliloquia\\ \textbf{Aug\_soliloq}
\item[Aug. trin.] Augustinus de trinitate\\ \textbf{Aug\_trin}
\item[Aur. Vict.] Aurelius Victor \\ \textbf{Aur\_Vict}
\item[Auson. Mos.] Ausonius Mosella\\ \textbf{Auson\_Mos}
\item[Auson. urb.] Ausonius ordo nobilium urbium\\ \textbf{Auson\_urb}
\item[Avien.] Avienus \\ \textbf{Avien}
\item[Babr.] Babrios \\ \textbf{Babr}
\item[Bakchyl.] Bakchylides \\ \textbf{Bakchyl}
\item[Boeth.] Boethius \\ \textbf{Boeth}
\item[Caes. Bell. Afr.] Caesar Bellum Africum (Ps.-Caes.)\\ \textbf{Caes\_Bell\_Afr}
\item[Caes. Bell. Alex.] Caesar Bellum Alexandrinum (Ps.-Caes.)\\ \textbf{Caes\_Bell\_Alex}
\item[Caes. Bell. Hisp.] Caesar Bellum Hispaniense (Ps.-Caes.)\\ \textbf{Caes\_Bell\_Hisp}
\item[Caes. civ.] Caesar de bello civili\\ \textbf{Caes\_civ}
\item[Caes. Gall.] Caesar de bello Gallico\\ \textbf{Caes\_Gall}
\item[Calp. ecl.] Calpurnius Bucolica vel Eclogae\\ \textbf{Calp\_ecl}
\item[Cass. Dio] Cassius Dio \\ \textbf{Cass\_Dio}
\item[Cassiod. inst.] Cassiodor institutiones variae\\ \textbf{Cassiod\_inst}
\item[Cato agr.] Cato maior de agri cultura\\ \textbf{Cato\_agr}
\item[Cato Orig.] Cato maior origines\\ \textbf{Cato\_Orig}
\item[Catull.] Catull carmina\\ \textbf{Catull}
\item[Cels.] Cornelius Celsus De medicina\\ \textbf{Cels}
\item[Cels. artes] Celsus \\ \textbf{Cels\_artes}
\item[Cels. Dig.] Celsus Digesta\\ \textbf{Cels\_Dig}
\item[Cens.] Censorinus de die natali\\ \textbf{Cens}
\item[Cic. ac. 1] Cicero Academicorum posteriorum liber 1\\ \textbf{Cic\_ac\_1}
\item[Cic. ac. 2] Cicero Lucullus sive Academicorum priorum liber 2\\ \textbf{Cic\_ac\_2}
\item[Cic. ad. Brut.] Cicero epistulae ad Brutum\\ \textbf{Cic\_ad\_Brut}
\item[Cic. ad Q. fr.] Cicero ad Quintum fratrem\\ \textbf{Cic\_ad\_Q\_fr}
\item[Cic. Arat.] Cicero Aratea\\ \textbf{Cic\_Arat}
\item[Cic. Arch.] Cicero pro Archia poeta\\ \textbf{Cic\_Arch}
\item[Cic. Att.] Cicero epistulae ad Atticum\\ \textbf{Cic\_Att}
\item[Cic. Balb.] Cicero pro L. Balbo\\ \textbf{Cic\_Balb}
\item[Cic. Brut.] Cicero Brutus\\ \textbf{Cic\_Brut}
\item[Cic. Caecin.] Cicero pro A. Caecina\\ \textbf{Cic\_Caecin}
\item[Cic. Cael.] Cicero pro M. Caelio\\ \textbf{Cic\_Cael}
\item[Cic. Catil.] Cicero in Catilinam\\ \textbf{Cic\_Catil}
\item[Cic. Cato] Cicero Cato maior de senectute\\ \textbf{Cic\_Cato}
\item[Cic. Cluent.] Cicero pro A. Cluentio\\ \textbf{Cic\_Cluent}
\item[Cic. de orat.] Cicero de oratore\\ \textbf{Cic\_de\_orat}
\item[Cic. Deiot.] Cicero pro rege Deiotaro\\ \textbf{Cic\_Deiot}
\item[Cic. div.] Cicero de divinatione\\ \textbf{Cic\_div}
\item[Cic. div.in Caec.] Cicero divinatio in Q. Caecilium\\ \textbf{Cic\_divin\_Caec}
\item[Cic. fam.] Cicero epistulae ad familiares\\ \textbf{Cic\_fam}
\item[Cic. fat.] Cicero de fato\\ \textbf{Cic\_fat}
\item[Cic. fin.] Cicero de finibus bonorum et malorum\\ \textbf{Cic\_fin}
\item[Cic. Flacc.] Cicero pro L. Valerio Flacco\\ \textbf{Cic\_Flacc}
\item[Cic. Font.] Cicero pro M. Fonteio\\ \textbf{Cic\_Font}
\item[Cic. har. resp.] Cicero de haruspicum responso\\ \textbf{Cic\_har\_resp}
\item[Cic. inv.] Cicero de inventione\\ \textbf{Cic\_inv}
\item[Cic. Lael.] Cicero Laelius de amicitia\\ \textbf{Cic\_Lael}
\item[Cic. leg.] Cicero de legibus\\ \textbf{Cic\_leg}
\item[Cic. leg. agr.] Cicero de lege agraria\\ \textbf{Cic\_leg\_agr}
\item[Cic. Lig.] Cicero pro Q. Ligario\\ \textbf{Cic\_Lig}
\item[Cic. Manil.] Cicero pro lege Manilia (de imperio Cn. Pompei)\\ \textbf{Cic\_Manil}
\item[Cic. Marcell.] Cicero pro M. Marcello\\ \textbf{Cic\_Marcell}
\item[Cic. Mil.] Cicero pro T. Annio Milone\\ \textbf{Cic\_Mil}
\item[Cic. Mur.] Cicero pro L. Murena\\ \textbf{Cic\_Mur}
\item[Cic. nat. deor.] Cicero de natura deorum\\ \textbf{Cic\_nat\_deor}
\item[Cic. off.] Cicero de officiis\\ \textbf{Cic\_off}
\item[Cic. opt. gen.] Cicero de optimo genere oratorum\\ \textbf{Cic\_opt\_gen}
\item[Cic. orat.] Cicero orator\\ \textbf{Cic\_orat}
\item[Cic. parad.] Cicero paradoxa\\ \textbf{Cic\_parad}
\item[Cic. part.] Cicero partitiones oratoriae\\ \textbf{Cic\_part}
\item[Cic. Phil.] Cicero in M. Antonium orationes Philippicae\\ \textbf{Cic\_Phil}
\item[Cic. Pis.] Cicero in L. Pisonem\\ \textbf{Cic\_Pis}
\item[Cic. Planc.] Cicero pro Cn. Plancio\\ \textbf{Cic\_Planc}
\item[Cic. p.red.ad Quir.] Cicero oratio post reditum ad Quirites\\ \textbf{Cic\_predad\_Quir}
\item[Cic. p.red.in sen.] Cicero oratio post reditum in senatu\\ \textbf{Cic\_predin\_sen}
\item[Cic. Q. Rosc.] Cicero pro Q. Roscio comoedo\\ \textbf{Cic\_Q\_Rosc}
\item[Cic. Quinct.] Cicero pro P. Quinctio\\ \textbf{Cic\_Quinct}
\item[Cic. Rab. perd.] Cicero pro C. Rabirioperduellionis reo\\ \textbf{Cic\_Rab\_perd}
\item[Cic. Rab post.] Cicero pro C. Rabirio Postumo\\ \textbf{Cic\_Rab\_post}
\item[Cic. rep.] Cicero de re publica\\ \textbf{Cic\_rep}
\item[Cic. S. Rosc.] Cicero pro Sex. Roscio Amerino\\ \textbf{Cic\_S\_Rosc}
\item[Cic. Scaur.] Cicero pro M. Aemilio Scauro\\ \textbf{Cic\_Scaur}
\item[Cic. Sest.] Cicero pro P. Sestio\\ \textbf{Cic\_Sest}
\item[Cic. Sull.] Cicero pro P. Sulla\\ \textbf{Cic\_Sull}
\item[Cic. Tim.] Cicero Timaeus\\ \textbf{Cic\_Tim}
\item[Cic. top.] Cicero topica\\ \textbf{Cic\_top}
\item[Cic. Tull.] Cicero pro M.Tullio\\ \textbf{Cic\_Tull}
\item[Cic. Tusc.] Cicero Tusculanae disputationes\\ \textbf{Cic\_Tusc}
\item[Cic. Vatin.] Cicero in P. Vatinium testem interrogatio\\ \textbf{Cic\_Vatin}
\item[Cic. Verr. 1,2] Cicero in Verrem actio prima, secunda\\ \textbf{Cic\_Verr\_1\_2}
\item[Claud. carm.] Claudius Claudianus carmina\\ \textbf{Claud\_carm}
\item[Claud. rapt. Pros.] Claudius Claudianus de raptu Proserpinae\\ \textbf{Claud\_rapt\_Pros}
\item[Colum.] Columella \\ \textbf{Colum}
\item[Curt.] Curtius Rufus historiae Alexandri Magni\\ \textbf{Curt}
\item[Cypr.] Cyprianus \\ \textbf{Cypr}
\item[Demokr.] Demokrit \\ \textbf{Demokr}
\item[Demosth. or.] Demosthenes orationes\\ \textbf{Demosth\_or}
\item[Diod.] Diodorus Siculus \\ \textbf{Diod}
\item[Diog. Laert.] Diogenes Laertios \\ \textbf{Diog\_Laert}
\item[Dion Chrys.] Dion Chrysostomos Orationes\\ \textbf{Dion\_Chrys}
\item[Dion. Hal. ant.] Dionysios von Halikarnassos antiquitates Romanae\\ \textbf{Dion\_Hal\_ant}
\item[Dion. Hal. comp.] Dionysios von Halikarnassos de compositione verborum\\ \textbf{Dion\_Hal\_comp}
\item[Dion. Hal. rhet.] Dionysios von Halikarnassos ars rhetorica       .\\ \textbf{Dion\_Hal\_rhet}
\item[Don.] Donatus grammaticus \\ \textbf{Don}
\item[Drac.] Dracontius \\ \textbf{Drac}
\item[Dt.] Deuteronomium (= 5. Mose) \\ \textbf{Dt}
\item[Emp.] Empedokles \\ \textbf{Emp}
\item[Enn. ann.] Ennius Annales (Skutsch, 1985)\\ \textbf{Enn\_ann}
\item[Enn. sat.] Ennius Saturae, (Vahlen, 21928)\\ \textbf{Enn\_sat}
\item[Enn. scaen.] Ennius fragmenta scenica (Vahlen, 21928)\\ \textbf{Enn\_scaen}
\item[Ennod.] Ennodius \\ \textbf{Ennod}
\item[Ep.] Epheserbrief \\ \textbf{Ep}
\item[Epik.] Epikuros \\ \textbf{Epik}
\item[Epikt.] Epiktetos \\ \textbf{Epikt}
\item[Etym. m.] Etymologicum magnum (Gaisford) \\ \textbf{Etym\_m}
\item[Eukl. elem.] Eukleides elementa\\ \textbf{Eukl\_elem.}
\item[Eur. Alc.] Euripides Alcestis\\ \textbf{Eur\_Alc}
\item[Eur. Andr.] Euripides Andromache\\ \textbf{Eur\_Andr}
\item[Eur. Bacch.] Euripides Bacchae\\ \textbf{Eur\_Bacch}
\item[Eur. Cycl.] Euripides Cyclops\\ \textbf{Eur\_Cycl}
\item[Eur. El.] Euripides Electra\\ \textbf{Eur\_El}
\item[Eur. Hec.] Euripides Hecuba (Hekabe)\\ \textbf{Eur\_Hec}
\item[Eur. Hel.] Euripides Helena\\ \textbf{Eur\_Hel}
\item[Eur. Heracl.] Euripides Heraclidae\\ \textbf{Eur\_Heracl}
\item[Eur. Herc.] Euripides Hercules (furens)\\ \textbf{Eur\_Herc}
\item[Eur. Hipp.] Euripides Hippolytos\\ \textbf{Eur\_Hipp}
\item[Eur. Ion] Euripides Ion\\ \textbf{Eur\_Ion}
\item[Eur. Iph. A.] Euripides Iphigenia Aulidensis\\ \textbf{Eur\_Iph\_A}
\item[Eur. Iph. T.] Euripides Iphigenia Taurica\\ \textbf{Eur\_Iph\_T}
\item[Eur. Med.] Euripides Medea\\ \textbf{Eur\_Med}
\item[Eur. Or.] Euripides Orestes\\ \textbf{Eur\_Or}
\item[Eur. Phoen.] Euripides Phoenissae\\ \textbf{Eur\_Phoen}
\item[Eur. Rhes.] Euripides Rhesus\\ \textbf{Eur\_Rhes}
\item[Eur. Suppl.] Euripides Supplices (ἱκέτιδες)\\ \textbf{Eur\_Suppl}
\item[Eur. Tro.] Euripides Troades\\ \textbf{Eur\_Tro}
\item[Eus. De. Ev.] Eusebios Demonstratio Evangelica\\ \textbf{Eus\_De\_Ev}
\item[Eus. HE] Eusebios Historia Ecclesiastica\\ \textbf{Eus\_HE}
\item[Eus. On.] Eusebios Onomastikon\\ \textbf{Eus\_On}
\item[Eus. Pr. Ev.] Eusebios Praeparatio Evangelica\\ \textbf{Eus\_Pr\_Ev}
\item[Eust.] Eustathios \\ \textbf{Eust}
\item[Eutr.] Eutropius \\ \textbf{Eutr}
\item[Fest.] Festus \\ \textbf{Fest}
\item[Flor.  epit.] Florus epitoma de Tito Livio\\ \textbf{Flor\_epit}
\item[Frontin. aqu. strat.] Frontin de aquae ductu urbis Romae strategemata\\ \textbf{Frontin\_aqu\_strat}
\item[Fulg.] Fulgentius Afer \\ \textbf{Fulg}
\item[Ga.] Galaterbrief \\ \textbf{Ga}
\item[Gal.] Galenos \\ \textbf{Gal}
\item[Gell.] Gellius noctes Atticae\\ \textbf{Gell}
\item[Gorg.] Gorgias \\ \textbf{Gorg}
\item[Greg. M. dial.] Gregorius Magnus dialogi (de miraculis patrum Italicorum)\\ \textbf{Greg\_M\_dial}
\item[Greg. M. epist.] Gregorius Magnus epistulae\\ \textbf{Greg\_M\_epist}
\item[Greg. M. past.] Gregorius Magnus regula pastoralis\\ \textbf{Greg\_M\_past}
\item[Greg. Naz. epist.] Gregorius Nazianzienus epistulae\\ \textbf{Greg\_Naz\_epist}
\item[Greg. Naz. or.] Gregorius Nazianzienus orationes\\ \textbf{Greg\_Naz\_or}
\item[Greg. Nyss.] Gregorius Nyssenus \\ \textbf{Greg\_Nyss}
\item[Greg. Tur. Franc.] Gregorius von Tours historia Francorum\\ \textbf{Greg\_Tur\_Franc}
\item[Greg. Tur. Mart.] Gregorius von Tours de virtutibus Martini\\ \textbf{Greg\_Tur\_Mart}
\item[Greg. Tur. vit. patr.] Gregorius von Tours de vita patrum\\ \textbf{Greg\_Tur\_vit\_patr}
\item[Hdt.] Herodot \\ \textbf{Hdt}
\item[Hekat.] Hekataios \\ \textbf{Hekat}
\item[Herakl.] Herakleitos \\ \textbf{Herakl}
\item[Herakl. Pont.] Herakleides Pontikos \\ \textbf{Herakl\_Pont}
\item[Herm. Trism.] Hermes Trismegistos \\ \textbf{Herm\_Trism}
\item[Herodian.] Herodianos \\ \textbf{Herodian}
\item[Hes. cat.] Hesiodos Catalogues feminarum (ἠοίαι)\\ \textbf{Hes\_cat}
\item[Hes. erg.] Hesiodos opera et dies (ἔργα καὶ ἡμέραι)\\ \textbf{Hes\_erg}
\item[Hes. scut.] Hesiodos scutum (ἀσπίς)\\ \textbf{Hes\_scut}
\item[Hes. theog.] Hesiodos Theogonia\\ \textbf{Hes\_theog}
\item[Hesych.] Hesychios \\ \textbf{Hesych}
\item[Hier. chron.] Hieronymos chronicon\\ \textbf{Hier\_chron}
\item[Hier. comm.in Ez] Hieronymos commentaria in Ezechielem (PL 25)\\ \textbf{Hier\_commin\_Ez}
\item[Hier. epist.] Hieronymos epistulae\\ \textbf{Hier\_epist}
\item[Hier. On.] Hieronymos Onomastikon\\ \textbf{Hier\_On}
\item[Hier. vir. ill.] Hieronymos de viris illustribus\\ \textbf{Hier\_vir\_ill}
\item[Hippokr.] Hippokrates \\ \textbf{Hippokr}
\item[HL] Hohelied \\ \textbf{HL}
\item[Hom. h.] Homer Hymni Homeri\\ \textbf{Hom\_h}
\item[Hom. Il.] Homer Ilias\\ \textbf{Hom\_Il}
\item[Hom. Od.] Homer Odyssee\\ \textbf{Hom\_Od}
\item[Hor. ars] Horaz ars poetica\\ \textbf{Hor\_ars}
\item[Hor. carm.] Horaz carmina\\ \textbf{Hor\_carm}
\item[Hor. carm. saec.] Horaz carmen saeculare\\ \textbf{Hor\_carm\_saec}
\item[Hor. epist.] Horaz epistulae\\ \textbf{Hor\_epist}
\item[Hor. epod.] Horaz epodi\\ \textbf{Hor\_epod}
\item[Hor. sat.] Horaz saturae (sermones)\\ \textbf{Hor\_sat}
\item[Hyg. astr.] Hygin astronomica\\ \textbf{Hyg\_astr}
\item[Hyg. fab.] Hygin fabulae\\ \textbf{Hyg\_fab}
\item[Iambl. de myst.] Iamblichos de mysteriis\\ \textbf{Iambl\_de\_myst}
\item[Iambl. protr.] Iamblichos protrepticus in philosophiam\\ \textbf{Iambl\_protr}
\item[Iambl. v. P.] Iamblichos de vita Pythagorica\\ \textbf{Iambl\_v\_P}
\item[Ios. Ant. Iud.] Flavius Josephus antiquitates Iudaicae\\ \textbf{Ios\_Ant\_Iud}
\item[Ios. Bell. Iud.] Iosephos (Flavius Iosephus) bellum Iudaicum\\ \textbf{Ios\_Bell\_Iud}
\item[Ios. c. Ap.] Iosephos (Flavius Iosephus) contra Apionem\\ \textbf{Ios\_c\_Ap}
\item[Ios. vita] Iosephos (Flavius Iosephus) de sua vita\\ \textbf{Ios\_vita}
\item[Isid. nat.] Isidorus von Sevilla de natura rerum\\ \textbf{Isid\_nat}
\item[Isid. orig.] Isidorus von Sevilla origines (etymologiae)\\ \textbf{Isid\_orig}
\item[Isokr. or.] Isokrates orationes\\ \textbf{Isokr\_or}
\item[Iul. epist.] Iulian epistulae\\ \textbf{Iul\_epist}
\item[Iul. in Gal.] Iulian in Galilaeos\\ \textbf{Iul\_in\_Gal}
\item[Iul. mis.] Iulian Misopogon\\ \textbf{Iul\_mis}
\item[Iul. or.] Iulian orationes\\ \textbf{Iul\_or}
\item[Iul. symp.] Iulian symposion\\ \textbf{Iul\_symp}
\item[Iul. Vict. Rhet.] C. Iulius Victor ars rhetorica\\ \textbf{Iul\_Vict\_Rhet}
\item[Iust.] Iustinus epitoma historiarum Philippicarum\\ \textbf{Iust}
\item[Iuv.] Iuvenal saturae\\ \textbf{Iuv\_sat}
\item[Iuvenc.] Iuvencus evangelia\\ \textbf{Iuvenc}
\item[Jo] Johannes \\ \textbf{Jo}
\item[Kall. epigr.] Kallimachos Epigrammata\\ \textbf{Kall\_epigr}
\item[Kall. frg.] Kallimachos fragmentum\\ \textbf{Kall\_frg}
\item[Kall. h.] Kallimachos hymni\\ \textbf{Kall\_h}
\item[Lact. inst.] Lactantius divinae institutiones\\ \textbf{Lact\_inst}
\item[Lact. ira] Lactantius de ira dei\\ \textbf{Lact\_ira}
\item[Lact. mort. pers.] Lactantius de mortibus persecutorum\\ \textbf{Lact\_mort\_pers}
\item[Lact. opif.] Lactantius de opificio dei\\ \textbf{Lact\_opif}
\item[Liv.] Livius ab urbe condita\\ \textbf{Liv}
\item[Lk] Lukas \\ \textbf{Lk}
\item[Lucan.] Lucanus bellum civile\\ \textbf{Lucan}
\item[Lucil.] Lucilius saturae\\ \textbf{Lucil}
\item[Lucr.] Lukrez de rerum natura\\ \textbf{Lucr}
\item[Lukian.] Lukianos \\ \textbf{Lukian}
\item[LXX] Septuaginta \\ \textbf{LXX}
\item[Lyd. mag.] Lydos de magistratibus\\ \textbf{Lyd\_mag}
\item[Lyd. mens.] Lydos de mensibus\\ \textbf{Lyd\_mens}
\item[Lykurg.] Lykurgos \\ \textbf{Lykurg}
\item[Lys.] Lysias \\ \textbf{Lys}
\item[M. Aur.] Marcus Aurelius Antonius Augustus \\ \textbf{M\_Aur}
\item[Macr. Sat.] Macrobius Saturnalia\\ \textbf{Macr\_Sat}
\item[Manil.] Manilius astronomica\\ \textbf{Manil}
\item[Mar. Victorin.] Marius Victorinus \\ \textbf{Mar\_Victorin}
\item[Mart.] Martial Epigrammata\\ \textbf{Mart}
\item[Mart. spect.] Martial spectacula\\ \textbf{Mart\_spect}
\item[Mart. Cap.] Martianus Capella \\ \textbf{Mart\_Cap}
\item[Mel.] Pomponius Mela \\ \textbf{Mel}
\item[Men. Dysk.] Menandros Dyskolo.\\ \textbf{Men\_Dysk}
\item[Men. Epitr.] Menandros Epitrepontes\\ \textbf{Men\_Epitr}
\item[Men. Sam.] Menandros Samia\\ \textbf{Men\_Sam}
\item[Mimn.] Mimnermos \\ \textbf{Mimn}
\item[Min. Fel.] Oct. Minucius Felix \\ \textbf{Min\_Fel}
\item[Mk] Markus \\ \textbf{Mk}
\item[Mosch.] Moschos \\ \textbf{Mosch}
\item[Mt] Matthäus \\ \textbf{Mt}
\item[Naev.] Naevius \\ \textbf{Naev}
\item[Nep. Att.] Cornelius Nepos Atticus\\ \textbf{Nep\_Att}
\item[Nep. Hann.] Cornelius Nepos Hannibal\\ \textbf{Nep\_Hann}
\item[Nik. Alex.] Nikander Alexipharmaka\\ \textbf{Nik\_Alex}
\item[Nik. Ther.] Nikander Theriaka\\ \textbf{Nik\_Ther}
\item[Nm] Numeri (= 4. Mose) \\ \textbf{Nm}
\item[Non.] Nonius \\ \textbf{Non}
\item[Nonn.  Dion.] Dion. \\ \textbf{Nonn\_Dion}
\item[Orig.] Origines \\ \textbf{Orig}
\item[Oros..] Orosius \\ \textbf{Oros}
\item[Ov. am.] Ovid amores\\ \textbf{Ov\_am}
\item[Ov. ars] Ovid ars amatoria\\ \textbf{Ov\_ars}
\item[Ov. epist.] Ovid epistulae (heroides)\\ \textbf{Ov\_epist}
\item[Ov. fast.] Ovid fasti\\ \textbf{Ov\_fast}
\item[Ov. Ib.] Ovid Ibis\\ \textbf{Ov\_Ib}
\item[Ov. medic.] Ovid medicamina faciei femineae\\ \textbf{Ov\_medic}
\item[Ov. met.] Ovid metamorphoses\\ \textbf{Ov\_met}
\item[Ov. Pont.] Ovid epistulae ex Ponto\\ \textbf{Ov\_Pont}
\item[Ov. rem.] Ovid remedia amoris\\ \textbf{Ov\_rem}
\item[Ov. trist.] Ovid tristia\\ \textbf{Ov\_trist}
\item[Papin.] Aemilius Papinianus \\ \textbf{Papin}
\item[Paus.] Pausanias \\ \textbf{Paus}
\item[Pers.] Persius saturae\\ \textbf{Pers}
\item[Petron.] Petronius satyrica\\ \textbf{Petron}
\item[Phaedr.] Phaedrus fabulae\\ \textbf{Phaedr}
\item[Pind. frg.] Pindar fragmentum\\ \textbf{Pind\_frg}
\item[Pind. I.] Pindar Isthmien\\ \textbf{Pind\_I}
\item[Pind. N.] Pindar Nemeen\\ \textbf{Pind\_N}
\item[Pind. O.] Pindar Olympien\\ \textbf{Pind\_O}
\item[Pind. P.] Pindar Pythien\\ \textbf{Pind\_P}
\item[Plat. Alk. 1] Platon Alkibiades 1\\ \textbf{Plat\_Alk\_1}
\item[Plat. Alk. 2] Platon Alkibiades 2\\ \textbf{Plat\_Alk\_2}
\item[Plat. apol.] Platon apologia\\ \textbf{Plat\_apol}
\item[Plat. Ax.] Platon Axiochos\\ \textbf{Plat\_Ax}
\item[Plat. Charm.] Platon Charmides\\ \textbf{Plat\_Charm}
\item[Plat. def.] Platon Definitiones (ὅροι)\\ \textbf{Plat\_def}
\item[Plat. Dem.] Platon Demodokos\\ \textbf{Plat\_Dem}
\item[Plat. epin.] Platon epinomis\\ \textbf{Plat\_epin}
\item[Plat. epist.] Platon epistulae\\ \textbf{Plat\_epist}
\item[Plat. eras.] Platon erastae\\ \textbf{Plat\_eras}
\item[Plat. Eryx.] Platon Eryxias\\ \textbf{Plat\_Eryx}
\item[Plat. Euthyd.] Platon Euthydemos\\ \textbf{Plat\_Euthyd}
\item[Plat. Euthyphr.] Platon Euthyphron\\ \textbf{Plat\_Euthyphr}
\item[Plat. Gorg.] Platon Gorgias\\ \textbf{Plat\_Gorg}
\item[Plat. Hipp. mai.] Platon Hippias maior\\ \textbf{Plat\_Hipp\_mai}
\item[Plat. Hipp. min.] Platon Hippias minor\\ \textbf{Plat\_Hipp\_min}
\item[Plat. Hipparch.] Platon Hipparchos\\ \textbf{Plat\_Hipparch}
\item[Plat. Ion] Platon Ion\\ \textbf{Plat\_Ion}
\item[Plat. Kleit.] Platon Kleitophon\\ \textbf{Plat\_Kleit}
\item[Plat. Krat.] Platon Kratylos\\ \textbf{Plat\_Krat}
\item[Plat. Krit.] Platon Kriton\\ \textbf{Plat\_Krit}
\item[Plat. Kritias] Platon Kritias\\ \textbf{Plat\_Kritias}
\item[Plat. Lach.] Platon Laches\\ \textbf{Plat\_Lach}
\item[Plat. leg.] Platon leges (νόμοι)\\ \textbf{Plat\_leg}
\item[Plat. Lys.] Platon Lysis\\ \textbf{Plat\_Lys}
\item[Plat. Men.] Platon Menon\\ \textbf{Plat\_Men}
\item[Plat. Min.] Platon Minos\\ \textbf{Plat\_Min}
\item[Plat. Mx.] Platon Menexenos\\ \textbf{Plat\_Mx}
\item[Plat. Parm.] Platon Parmenides\\ \textbf{Plat\_Parm}
\item[Plat. Phaid.] Platon Phaidon\\ \textbf{Plat\_Phaid}
\item[Plat. Phaidr.] Platon Phaidros\\ \textbf{Plat\_Phaidr}
\item[Plat. Phil.] Platon Philebos\\ \textbf{Plat\_Phil}
\item[Plat. polit.] Platon politicus\\ \textbf{Plat\_polit}
\item[Plat. Prot.] Platon Protagoras\\ \textbf{Plat\_Prot}
\item[Plat. rep.] Platon de re publica (πολιτεία)\\ \textbf{Plat\_rep}
\item[Plat. Sis.] Platon Sisyphos\\ \textbf{Plat\_Sis}
\item[Plat. soph.] Platon sophista\\ \textbf{Plat\_soph}
\item[Plat. symp.] Platon symposium\\ \textbf{Plat\_symp}
\item[Plat. Thg.] Platon Theages\\ \textbf{Plat\_Thg}
\item[Plat. Tht.] Platon Theaitetos\\ \textbf{Plat\_Tht}
\item[Plat. Tim.] Platon Timaios\\ \textbf{Plat\_Tim}
\item[Plaut. Amph.] Plautus Amphitruo\\ \textbf{Plaut\_Amph}
\item[Plaut. Asin.] Plautus Asinaria\\ \textbf{Plaut\_Asin}
\item[Plaut. Aul.] Plautus Aulularia\\ \textbf{Plaut\_Aul}
\item[Plaut. Bacch.] Plautus Bacchides\\ \textbf{Plaut\_Bacch}
\item[Plaut. Cist.] Plautus Cistellaria\\ \textbf{Plaut\_Cist}
\item[Plaut. Curc.] Plautus Curculio\\ \textbf{Plaut\_Curc}
\item[Plaut. Epid.] Plautus Epidicus\\ \textbf{Plaut\_Epid}
\item[Plaut. Men.] Plautus Menaechmi\\ \textbf{Plaut\_Men}
\item[Plaut. Merc.] Plautus Mercator\\ \textbf{Plaut\_Merc}
\item[Plaut. Mil.] Plautus Miles gloriosus\\ \textbf{Plaut\_Mil}
\item[Plaut. Most.] Plautus Mostellaria\\ \textbf{Plaut\_Most}
\item[Plaut. Poen.] Plautus Poenulus\\ \textbf{Plaut\_Poen}
\item[Plaut. Pseud.] Plautus Pseudolus\\ \textbf{Plaut\_Pseud}
\item[Plaut. Rud.] Plautus Rudens\\ \textbf{Plaut\_Rud}
\item[Plaut. Stich.] Plautus Stichus\\ \textbf{Plaut\_Stich}
\item[Plaut. Trin.] Plautus Trinummus\\ \textbf{Plaut\_Trin}
\item[Plaut. Truc.] Plautus Truculentus\\ \textbf{Plaut\_Truc}
\item[Plaut. Vid.] Plautus Vidularia\\ \textbf{Plaut\_Vid}
\item[Plin. epist.] Plinius minor Epistulae\\ \textbf{Plin\_epist}
\item[Plin. nat.] Plinius maior naturalis historia\\ \textbf{Plin\_nat}
\item[Plin. paneg.] Plinius minor panegyricus\\ \textbf{Plin\_paneg}
\item[Plot.] Plotin \\ \textbf{Plot}
\item[Plut.] Plutarch vitae parallelae\\ \textbf{Plut}
\item[Plut. am.] Plutarch Amatorius (erotikos)\\ \textbf{Plut\_am}
\item[Plut. de Pyth.Or.] Plutarch de Pythiae oraculis\\ \textbf{Plut\_de\_PythOr}
\item[Plut. Is.] Plutarch de Iside et Osiride\\ \textbf{Plut\_Is}
\item[Plut. mor.] Plutarch moralia\\ \textbf{Plut\_mor}
\item[Plut. qu.Gr.] Plutarch quaestiones Graecae\\ \textbf{Plut\_quGr}
\item[Plut. qu.R.] Plutarch quaestiones Romanae\\ \textbf{Plut\_quR}
\item[Plut. symp.] Plutarch quaestiones convivales\\ \textbf{Plut\_symp}
\item[Pol.] Polybios \\ \textbf{Pol}
\item[Poll.] Pollux \\ \textbf{Poll}
\item[Porph.] Porphyrios \\ \textbf{Porph}
\item[Poseid.] Poseidonios \\ \textbf{Poseid}
\item[Priap.] Priapea \\ \textbf{Priap}
\item[Prisc.] Priscianus \\ \textbf{Prisc}
\item[Prok. aed.] Prokop de aedificiis (περὶ κτισμάτων)\\ \textbf{Prok\_aed}
\item[Prok. BG] Prokop bellum Gothicum\\ \textbf{Prok\_BG}
\item[Prok. BP] Prokop bellum Persicum\\ \textbf{Prok\_BP}
\item[Prok. BV] Prokop bellum Vandalicum\\ \textbf{Prok\_BV}
\item[Prok. HA] Prokop historia arcana\\ \textbf{Prok\_HA}
\item[Prokl.] Proklos \\ \textbf{Prokl}
\item[Prop.] Properz elegiae\\ \textbf{Prop}
\item[Prud.] Prudentius \\ \textbf{Prud}
\item[Quint. decl.] Quintilian declamationes minores\\ \textbf{Quint\_decl}
\item[Quint. inst.] Quintilian institutio oratoria\\ \textbf{Quint\_inst}
\item[Rhet. Her.] Rhetorica ad Herennium \\ \textbf{Rhet\_Her}
\item[Röm] Römerbrief \\ \textbf{Roem}
\item[Rut. Nam.] Rutilius Claudius Namatianus de reditu suo\\ \textbf{Rut\_Nam}
\item[S. Emp.] Sextus Empiricus \\ \textbf{S\_Emp}
\item[Sall. Catil.] Sallust de coniuratione Catilinae\\ \textbf{Sall\_Catil}
\item[Sall. epist.] Sallust epistulae ad Caesarem\\ \textbf{Sall\_epist}
\item[Sall. hist.] Sallust historiae\\ \textbf{Sall\_hist}
\item[Sall. Iug.] Sallust de bello Iugurthino\\ \textbf{Sall\_Iug}
\item[Sen. apocol.] Seneca minor divi Claudii apocolocynthosis\\ \textbf{Sen\_apocol}
\item[Sen. benef.] Seneca minor de beneficiis\\ \textbf{Sen\_benef}
\item[Sen. clem.] Seneca minor de clementia\\ \textbf{Sen\_clem}
\item[Sen. contr.] Seneca maior controversiae\\ \textbf{Sen\_contr}
\item[Sen. dial.] Seneca minor dialogi\\ \textbf{Sen\_dial}
\item[Sen. epist.] Seneca minor epistulae morales ad Lucilium\\ \textbf{Sen\_epist}
\item[Sen. Herc. f.] Seneca minor Hercules furens\\ \textbf{Sen\_Herc\_f}
\item[Sen. Med.] Seneca minor Medea\\ \textbf{Sen\_Med}
\item[Sen. nat.] Seneca minor naturales quaestiones\\ \textbf{Sen\_nat}
\item[Sen. Oed.] Seneca minor Oedipus\\ \textbf{Sen\_Oed}
\item[Sen. Phaedr.] Seneca minor Phaedra\\ \textbf{Sen\_Phaedr}
\item[Sen. Phoen.] Seneca minor Phoenissae\\ \textbf{Sen\_Phoen}
\item[Sen. suas.] Seneca maior suasoriae\\ \textbf{Sen\_suas}
\item[Sen. Thy.] Seneca minor Thyestes\\ \textbf{Sen\_Thy}
\item[Sen. Tro.] Seneca minor Troades\\ \textbf{Sen\_Tro}
\item[Serv. Aen.] Servius commentarius in Vergilii Aeneida\\ \textbf{Serv\_Aen}
\item[Serv. ecl.] Servius commentarius in Vergilii eclogas\\ \textbf{Serv\_ecl}
\item[Serv. georg.] Servius commentarius in Vergilii georgica\\ \textbf{Serv\_georg}
\item[Serv. auct.] Servius auctus Danielis \\ \textbf{Serv\_auct}
\item[Sil.] Silius Italicus Punica\\ \textbf{Sil}
\item[Sim.] Simonides \\ \textbf{Sim}
\item[Simpl.] Simplikios \\ \textbf{Simpl}
\item[Sol.] Solon \\ \textbf{Sol}
\item[Soph. Ai.] Sophokles Aias\\ \textbf{Soph\_Ai}
\item[Soph. Ant.] Sophokles Antigone\\ \textbf{Soph\_Ant}
\item[Soph. El.] Sophokles Electra\\ \textbf{Soph\_El}
\item[Soph. Oid. K.] Sophokles Oedipus Coloneus\\ \textbf{Soph\_Oid\_K}
\item[Soph. Oid. T.] Sophokles Oedipus Rex (Oedipus Tyrannus)\\ \textbf{Soph\_Oid\_T}
\item[Soph. Phil.] Sophokles Philoctetes\\ \textbf{Soph\_Phil}
\item[Stat. Ach.] Statius Achilleis\\ \textbf{Stat\_Ach}
\item[Stat. silv.] Statius silvae\\ \textbf{Stat\_silv}
\item[Stat. Theb.] Statius Thebais\\ \textbf{Stat\_Theb}
\item[Stesich.] Stesichoros \\ \textbf{Stesich}
\item[Stob.] Stobaios \\ \textbf{Stob}
\item[Strab.] Strabon \\ \textbf{Strab}
\item[Suet. Aug.] Sueton divus Augustus\\ \textbf{Suet\_Aug}
\item[Suet. Cal.] Sueton Caligula\\ \textbf{Suet\_Cal}
\item[Suet. Claud.] Sueton divus Claudius\\ \textbf{Suet\_Claud}
\item[Suet. Dom.] Sueton Domitianus\\ \textbf{Suet\_Dom}
\item[Suet. gramm.] Sueton de grammaticis\\ \textbf{Suet\_gramm}
\item[Suet. Iul.] Sueton divus Iulius\\ \textbf{Suet\_Iul}
\item[Suet. Tib.] Sueton divus Tiberius\\ \textbf{Suet\_Tib}
\item[Suet. Tit.] Sueton divus Titus\\ \textbf{Suet\_Tit}
\item[Suet. Vesp.] Sueton divus Vespasianus\\ \textbf{Suet\_Vesp}
\item[Suet. Vit.] Sueton Vitellius\\ \textbf{Suet\_Vit}
\item[Symm. epist.] Symmachus epistulae\\ \textbf{Symm\_epist}
\item[Symm. or.] Symmachus orationes\\ \textbf{Symm\_or}
\item[Symm. rel.] Symmachus relationes\\ \textbf{Symm\_rel}
\item[Synes. Epist.] epist. epistulae\\ \textbf{Synes\_epist}
\item[Tac. Agr.] Tacitus Agricola\\ \textbf{Tac\_Agr}
\item[Tac. ann.] Tacitus annales\\ \textbf{Tac\_ann}
\item[Tac. dial.] Tacitus dialogus de oratoribus\\ \textbf{Tac\_dial}
\item[Tac. Germ.] Tacitus Germania\\ \textbf{Tac\_Germ}
\item[Tac. hist.] Tacitus historiae\\ \textbf{Tac\_hist}
\item[Ter. Ad.] Terenz Adelphoe\\ \textbf{Ter\_Ad}
\item[Ter. Andr.] Terenz Andria\\ \textbf{Ter\_Andr}
\item[Ter. Eun.] Terenz Eunuchus\\ \textbf{Ter\_Eun}
\item[Ter. Haut.] Terenz Heautontimorumenos\\ \textbf{Ter\_Haut}
\item[Ter. Hec.] Terenz Hecyra\\ \textbf{Ter\_Hec}
\item[Ter. Phorm.] Terenz Phormio\\ \textbf{Ter\_Phorm}
\item[Ter. Maur.] Terentianus Maurus \\ \textbf{Ter\_Maur}
\item[Tert. apol.] Tertullianus apologeticum\\ \textbf{Tert\_apol}
\item[Tert. nat.] Tertullianus ad nationes\\ \textbf{Tert\_nat}
\item[Theokr.] Theokritos \\ \textbf{Theokr}
\item[Thgn.] Theognis \\ \textbf{Thgn}
\item[Thuk.] Thukydides \\ \textbf{Thuk}
\item[Tib.] Tibullus elegiae\\ \textbf{Tib}
\item[Ulp.  Reg.] Ulpianus (Ulpiani regulae)\\ \textbf{Ulp\_reg}
\item[Val. Fl.] Valerius Flaccus Argonautica\\ \textbf{Val\_Fl}
\item[Val. Max.] Valerius Maximus facta et dicta memorabilia\\ \textbf{Val\_Max}
\item[Varro ling.] Varro de lingua Latina\\ \textbf{Varro\_ling}
\item[Varro Men.] Varro saturae Menippeae\\ \textbf{Varro\_Men}
\item[Varro rust.] Varro res rusticae\\ \textbf{Varro\_rust}
\item[Vell. ] Velleius Paterculus historiae Romanae\\ \textbf{Vell}
\item[Ven. Fort.] Venantius Fortunatus \\ \textbf{Ven\_Fort}
\item[Verg. Aen.] Vergilius Aeneis\\ \textbf{Verg\_Aen}
\item[Verg. catal.] Vergilius catalepton\\ \textbf{Verg\_catal}
\item[Verg. ecl.] Vergilius eclogae\\ \textbf{Verg\_ecl}
\item[Verg. georg.] Vergilius georgica\\ \textbf{Verg\_georg}
\item[Vitr.] Vitruv de architectura\\ \textbf{Vitr}
\item[Xen. an.] Xenophon anabasis\\ \textbf{Xen\_an}
\item[Xen. apol.] Xenophon apologia\\ \textbf{Xen\_apol}
\item[Xen. Ath. pol.] Xenophon Athenaion politeia\\ \textbf{Xen\_Ath\_pol}
\item[Xen. hell.] Xenophon hellenica\\ \textbf{Xen\_hell}
\item[Xen. kyn.] Xenophon cynegeticus\\ \textbf{Xen\_kyn}
\item[Xen. Kyr.] Xenophon Cyrupaideia\\ \textbf{Xen\_Kyr}
\item[Xen. mem.] Xenophon memorabilia (Apomnemoneumata)\\ \textbf{Xen\_mem}
\item[Xen. oec.] Xenophon oeconomicus\\ \textbf{Xen\_oec}
\item[Xen. symp.] Xenophon symposium\\ \textbf{Xen\_symp}
\item[Xenophan.] Xenophanes \\ \textbf{Xenophan}
\item[Zen.] Zenon \\ \textbf{Zen}
\item[Zenob.] Zenobios \\ \textbf{Zenob}
\item[Zenod.] Zenodotos \\ \textbf{Zenod}
\item[Zos.] Zosimos \\ \textbf{Zos}
\item[ Cod. Iust.]  Corpus Iuris Civilis, Codex Iustinianus\\ \textbf{Cod\_Iust}
\item[ Dig.]  Corpus Iuris Civilis, Digesta (Pandekten.\\ \textbf{Dig}
\item[ Herc. O.]  Hercules Oetaeus\\ \textbf{Herc\_O}
\item[ Inst. Iust.]  Corpus Iuris Civilis, Institutiones\\ \textbf{Inst\_Iust}
\item[ Itin. Anton.]  Itinerarium Antonini\\ \textbf{Itin\_Anton}
\item[ Nov.]  Corpus Iuris Civilis, Leges Novellae\\ \textbf{Nov}
\item[ Pass. mart.]  Passiones martyrum\\ \textbf{Pass\_mart}
\item[ R. Gest. div. Aug.]  Res gestae divi Augusti\\ \textbf{R\_Gest\_div\_Aug}
\item[ Vulg.]  Vulgata\\ \textbf{Vulg}
\end{description}
\end{footnotesize}

\end{multicols}
 \changes{v1.5}{2016/05/31}{Antike Bibliographie}


\section{Corpora}\label{corpora}
List of corpora in ancient studies |archaeologie-corpora.bib|.
This activates the other additional bibliography |archaeologie-abbrv.bib| automatically.
You can cite them directly using the left entry as \marg{bibtex-key}.
All the entries have |keywords={corpus}|, |options={corpus}|.
\begin{multicols}{2}
	\begin{footnotesize}
\begin{description}[%
			%	style=multiline,
				style=nextline,
				leftmargin=1.5cm,
				%font=\normalfont\bfseries
				]
\item[ABV] Attic Black-figure Vase-painters
\item[AE] L'année épigraphique 
\item[AHw] Akkadisches Handwörterbuch
\item[ARV2] Attic Red-figure Vase-painters
\item[CAD] The Assyrian Dictionary of the Oriental Institute of the University of Chicago 
\item[CIL] Corpus inscriptionum Latinarum 
\item[DACL] Dictionnaire d'archéologie chrétienne et de liturgie 
\item[Daremberg-Saglio] Dictionnaire des antiquités grecques et romaines d'après les textes et les monuments. Ouvrage rédigé sous la direction de Ch. Daremberg et E. Saglio 
\item[DNP] Der Neue Pauly. Enzyklopädie der Antike 
\item[EAA] Enciclopedia dell'arte antica classica e orientale 
\item[FGrHist] Die Fragmente der griechischen Historiker
\item[FHG] Fragmenta historicorum Graecorum 
\item[FR] A. Furtwängler – K. Reichhold, Griechische Vasenmalerei (München 1900--1925) 
\item[HAW] Handbuch der Altertumswissenschaften 
\item[HdArch] Handbuch der Archäologie 
\item[Head] B. V. Head, Historia Numorum. A Manual of Greek Numismatics (Oxford 1887; 1911)
\item[Helbig] W. Helbig, Führer durch die öffentlichen Sammlungen klassischer Altertümer in Rom 
\item[IG] Inscriptiones Graecae 
\item[IGR] Inscriptiones Graecae ad res Romanas pertinentes 
\item[IK] Inschriften griechischer Städte aus Kleinasien
\item[ILS] Inscriptiones Latinae selectae
\item[LAe] Lexikon der Ägyptologie
\item[LIMC] Lexikon iconographicum mythologiae classicae
\item[LSJ] G. Liddell – R. Scott – H. S. Jones, A Greek-English Lexikon \textsuperscript{9}(1996); Suppl. (1996)
\item[LTUR] Lexikon topographicum urbis Romae 
\item[PIR] Prosopographia Imperii Romani 
\item[PPM] Pompei: Pitture e mosaici. Enciclopedia dell’arte antica classica e orientale 
\item[PPP] Pitture e Pavimenti di Pompei
\item[RAC] Reallexikon für Antike und Christentum 
\item[RBK] Reallexikon zur byzantinischen Kunst
\item[RE] Paulys Realencyclopädie der classischen Altertumswissenschaft 
\item[RES] Répertoire d’épigraphie sémitique (Paris 1900--1950) 
\item[RIA] Rivista dell’Istituto nazionale d’archeologia e storia dell’arte
\item[RIC] H. Mattingly – E. A. Sydenham, The Roman Imperial Coinage 
\item[RoscherML] W. H. Roscher, Ausführliches Lexikon der griechischen und römischen Mythologie
\item[RPC] Roman Provincial Coinage
\item[RRC] M. Crawford, Roman Republican Coinage (London 1974) 
\item[SEG] Supplementum epigraphicum Graecum 
\item[SIG] Sylloge inscriptionum Graecarum
\item[SNG] Sylloge nummorum Graecorum 
\item[TAM] Tituli Asiae Minoris
\item[ThesCRA] Thesaurus Cultus et Rituum Antiquorum
\item[Thieme-Becker] U. Thieme – F. Becker (Hrsg.), Allgemeines Lexikon der bildenden Künstler
\item[TIB] Tabula Imperii Byzantini
\end{description}
\end{footnotesize}
\end{multicols}
 \changes{v1.5}{2016/05/31}{Antike Bibliographie}




\section{Abkürzungen nach DAI-Richtlinie}\label{listen}
Da die Schreibweise von manchen Kürzeln der Zeitschriften oder Reihen nicht als |@string| übernommen werden konnten, musste die Liste an die \TeX-verarbeitende Lesart angepasst werden.
Generell gilt, dass Akzente weggelassen wurden, Umlaute ausgeschrieben (ä--> ae), ähnliche Buchstaben mit Äquivalenten ersetzt wurden (ı --> i).\footnote{Für problemloses Kompilieren und Verarbeiten der nichtlateinischen Buchstaben wird \hologo{XeLaTeX} oder  \hologo{LuaTeX} empfohlen.}
Nachfolgend werden zwei Listen angeführt, die die Auflistung nach @string (links) und Ausgabe (rechts) zeigen.
Die erste Liste enthält die Abkürzungen (\cref{liste-kurz}), die zweite Liste die ausgeschriebenen Namen (\cref{liste-lang}).
Es empfiehlt sich, in diesen Listen die |@string|-Angaben nachzuschauen und dann für |journaltitle| und |shortjournal|, bzw. |series| und |shortseries| einzutragen.

Die Unterscheidung, ob in der Bibliographie die Abkürzung (Standard) oder der voller Titel genannt werden soll, erfolgt automatisch über die Option |noabbrevs| (Siehe \cref{noabbrevs}). Ist diese Option nicht aktiviert, dann werden standardmäßig die Abkürzungen angegeben.

%%Beim Kompilieren wird eine Bibliographiedatei (|dai-abkuerzungen.bib|) erstellt, in der die Angaben der |@abbrv| gespeichert sind.
%%Diese Bibliographiedatei wird zusätzlich zur eigenen Bibliographie geladen.
%%Ist |dai-abkuerzungen.bib| bereits vorhanden, dann wird sie nicht neu generiert.


\subsection{Kurzformen\label{liste-kurz}}
\begin{multicols}{2}
\input{archaeologie-abbrv-short.tex}
\end{multicols}


\subsection{Langformen\label{liste-lang}}
Die Ausgabe erfolgt über die Option |noabbrevs|.
%\begin{multicols}{1}
\input{archaeologie-abbrv-long.tex}
%\end{multicols}


 \changes{v1.1}{2015/07/06}{Erstellung der Liste mit Abkürzungen}


\clearpage
\section{Umsetzung}
\label{driver}
|archaeologie| consists of various files:
There is one file that takes care of the bibliography (|bbx|) 
another looks after the citation-style  (|cbx|)
and several files which are necessary for the individual languages (|lbx|),
furthermore there is a |dbx|-file.

%http://tex.stackexchange.com/questions/95036/continue-line-numbers-in-listings-package
\subsection{archaeologie.bbx}
|archaeologie| baut auf dem |standard|-Stil von |biblatex| auf, der entsprechend geladen werden muss.
 \DescribeMacro{bbx}
 \StartLineAt{13}
\begin{lstlisting}[style=code]
\ProvidesFile{archaeologie.bbx}%
               [2016/05/31 v2.0  archaeologie -- %
                biblatex for archaeologists, 
                historians and philologists, bbx-file]
\RequireBibliographyStyle{standard}
\end{lstlisting}

It continues with all required settings
\ContinueLineNumber
\begin{lstlisting}[style=code]
\AtBeginDocument{%
		\urlstyle{sf}%
		\typeout{* * * archaeologie * * *  
				biblatex for archaeologists, 
               historians and philologists}
}
\ExecuteBibliographyOptions{%
pagetracker=true,%
citecounter=true,%
giveninits=true,%
sortlocale=auto,%
language=auto,%
autolang=other,%
bibencoding=utf8,%
dateabbrev=false, %
sorting=nyt,%
maxnames=2,% 
minnames=1,%
maxitems=1,%
maxbibnames=999,%
}
\end{lstlisting}
 \StartLineAt{426}
\begin{lstlisting}[style=code]
\renewbibmacro*{journal}{%
  \ifboolexpr{test {\iffieldundef{shortjournal}} %
  					or bool {bbx:noabbrevs}}%
    {\printtext[journaltitle]{%
       \printfield[titlecase]{journaltitle}%
       \setunit{\subtitlepunct}%
       \printfield[titlecase]{journalsubtitle}}}%
    {\printfield{shortjournal}}%
    }
\end{lstlisting}
    

%\subsection{archaeologie.cbx}

 \changes{v0.1}{2015/06/04}{Started Project}
 \changes{v1}{2015/09/15}{First public version}
\PrintChanges
\PrintIndex

\end{document}
