% archaeologie --%
%               biblatex fuer Archaeologen, Historiker und Philologen
% Copyright (c) 2016 Lukas C. Bossert | Johannes Friedl
%  
% This work may be distributed and/or modified under the
% conditions of the LaTeX Project Public License, either version 1.3
% of this license or (at your option) any later version.
% The latest version of this license is in
%   http://www.latex-project.org/lppl.txt
% and version 1.3 or later is part of all distributions of LaTeX
% version 2005/12/01 or later.

\documentclass[a4paper,
10pt,
ngerman
]{ltxdoc}
\CodelineNumbered
\AtBeginDocument{\RecordChanges}
\AtEndDocument{\PrintChanges}
  \usepackage[T1,LGR]{fontenc}		% font types and character verification
%  \renewcommand*\ttdefault{lcmtt}
\usepackage{libertine}
%\renewcommand*\ttdefault{cmvtt}
%\renewcommand*\ttdefault{txtt}
%\usepackage{inconsolata}
\renewcommand*\ttdefault{lmvtt}

\usepackage[					% use  for bibliography
	backend=biber,
	style=archaeologie,
%	eprint=false,
%	doi=false,
%	url=false,
%inreferences,
%seenote,
	%translation,
	%bibfullname,
	%noabbrevs,
	%publisher,
	%edby,
	%yearseries,
%yearinparens,
	%lastnames,
	%fullnames,
%	scshape,
%	width=6em,
%counter,
%ancient,
strings,
]{biblatex}
\renewcommand\bibfont{\normalfont
\footnotesize}


\usepackage{marginnote}
	\usepackage{metalogo}
	\usepackage{hologo}
 \usepackage{babel}
\usepackage{coolthms}
\usepackage[					% advanced quotes
	strict=true,					% 	- warning are errors now
	style=ngerman,					% 	- german quotes
]{csquotes}
\usepackage{multicol}
\setlength{\columnsep}{1.5cm}
\setlength{\columnseprule}{0.2pt}
\usepackage{framed}
\usepackage{enumitem}
\setlength{\parindent}{0pt}
\setlength{\parskip}{6pt plus 2pt minus 2pt}
\setenumerate[1]{label=(\alph*),leftmargin=*,nolistsep,parsep=\parskip}
\usepackage{changepage}
\makeindex
   
\newenvironment{bsp}{\begin{framed}\begin{small}
\begin{adjustwidth}{1cm}{0cm}}{\end{adjustwidth}
\end{small}\end{framed}}
\defbibheading{empty}{}

\newcommand{\printbib}[2][4em]{
\begingroup
\begin{bsp}
\footnotesize
\begin{refsection}
\setlength{\labwidthsameline}{#1} 
\nocite{#2}
\printbibliography[heading=none]
\end{refsection}
\end{bsp}
\endgroup
}

\newcommand{\printbiball}[2][4em]{
\begingroup
\setlength{\labwidthsameline}{#1} 

\begin{bsp}
\footnotesize%
\begin{itemize}%[\bfseries]
\begin{refsection}%
\nocite{#2}%
\item[English:]{\printbibliography[heading=none]}
\item[German:]\foreignlanguage{ngerman}{\printbibliography[heading=none]}
\item[Italian:]\foreignlanguage{italian}{\printbibliography[heading=none]}
\item[French:]\foreignlanguage{french}{\printbibliography[heading=none]}
\end{refsection}
\end{itemize}%
\end{bsp}

\endgroup
}


\listfiles
\EnableCrossrefs
\CodelineIndex
\RecordChanges


 \usepackage[disable]{todonotes} % notes not showed
 
 



\addbibresource{archaeologie.bib}
%\addbibresource{archaeologie-ancient.bib}
%\addbibresource{antike-corpora.bib} %Version 2.0



\usepackage{listings}
\lstMakeShortInline[basicstyle=\footnotesize\ttfamily\bfseries]{|}

\hypersetup{  colorlinks   = true, %Colours links instead of ugly boxes
  urlcolor     =  blue, %Colour for external hyperlinks
  linkcolor    = black, %Colour of internal links
  citecolor   = blue, %Colour of citations
  }	
\begin{document}
%  \DocInput{archaeologie.dtx}

%\MakeShortVerb{\|}
 \newenvironment{syntax}{\begin{small}\medskip\hspace*{2em}}{\par\medskip\end{small}}
 %\def\verbatimchar{;}



 \title{\texttt{archaeologie} -- \\\texttt{bib\LaTeX} für Archäologen\footnote{Ebenso geeignet für (Alt-)Philologen und (Alt-)Historiker.
 Die Entwicklung des Codes findet statt auf \url{https://github.com/LukasCBossert/biblatex-archaeologie}: 
 Anmerkungen und Kritik sind willkommen.
 Dank an   ›moewew‹ und Herbert Voß für ihre Hilfe.
} }
\author{Lukas C. Bossert\thanks{\url{lukas@digitales-altertum.de}} \and Johannes Friedl}
\date{Version: 1.5 (2016-XX-XX)}
 
 
 \maketitle
 \begin{abstract}
Der Stil setzt die Zitations- und Bibliographievorgaben  des Deutschen Archäologischen Instituts (DAI) (Stand 2014) um. Verschiedene zusätzliche Optionen erlauben das Erscheinungsbild jedoch auch zu verändern, um eigenen Vorlieben anzupassen.
Der Stil ist nicht nur für Dokumente in deutscher Sprache geeignet, sondern wurde ebenso für die Sprachen Englisch, Italienisch und Französisch angepasst.
 \end{abstract}


\section{Verwendung}
 \DescribeMacro{archaeologie}  Der Name des Stils ist |archaeologie| und wird an entsprechender Stelle geladen.

\begin{syntax} |\usepackage[style=archaeologie,%|\\
 |            |\meta{weitere Optionen}|]{biblatex}|\\
 |\bibliography|\marg{|bib|-Datei}\end{syntax}
 
Dabei kann man weitere der \enquote{konventionellen} |biblatex|-Optionen oder der -- weiter unten beschriebenen -- von |archaeologie| zur Verfügung gestellten Optionen laden.

 |archaeologie| lädt standardmäßig den DAI-Stil im Autor-Jahr-System. 
 Um schnell und einfach im DAI-Stil zu zitieren, benötigt es keine weiteren Einstellungen und Optionen.

Am Ende des Dokuments oder an gewünschter Stelle muss der  |\printbibliography|-Befehl stehen, sodass  eine Bibliographie angelegt und ausgegeben wird. 
 Da |archaeologie| unterschiedliche Zitierweisen von Textsorten wie antiker Primärliteratur oder wissenschaftlicher Sekundärliteratur unterstützt, empfiehlt es sich, die Bibliographieaufteilung dementsprechend anzupassen. Verschiedene Möglichkeiten zur Gestaltung der Bibliographie  (\cref{bibliographie}).

\section{Übersicht}
Im Folgenden sind   kurz die möglichen Optionen, mit denen der Stil |archaeologie| aufgerufen werden kann, aufgeführt. 
 Dazu kann man -- quasi auf eigene Gefahr -- noch die konventionellen |biblatex|-Optionen (insbesondere zur Formatierung der Abstände etc. des Literaturverzeichnisses) verwenden. Näheres zu diesen findet man in der Dokumentation von |biblatex|.

 \subsection{Paketoptionen (Präambel)}\label{preamble_options}
 \begin{description}
 \item[strings] Aktiviert die extra Bibliographiedatei |archaeologie-abbrv.bib|, mit der man Zeitschrifen oder Reihen mit entsprechenden Abkürzungen zitieren kann. s. \cref{strings}. 
 \item[edby] (zuvor |hrsgv|) Bei Sammelbänden steht anstatt \enquote{Hrsg.} nun \enquote{hrsg. v.}. Siehe \cref{edby}.

\item[seenote] Aktiviert die Möglichkeit in Fußnoten zunächst als Vollzitat, dann als Rückverweis auf die Erstnennung zu zitieren. \cref{seenote}

\item[ancient] Damit wird eine separate Bibliographie geladen,
in der knapp 600 antike Autoren und Werke gespeichert sind, die direkt im Text zitiert werden können. \cref{ancient}


\item[yearseries] (zuvor |jahrreihe|) Die Reihe wird erst nach der Jahreszahl ausgegeben.  \cref{yearseries}.
 
 \item[yearinparens] (zuvor |jahrinklammer|)  Die Jahreszahl wird in Klammern gesetzt.  \cref{yearinparens}.

\item[translation] (zuvor |uebersetzung|)  Anzeige der Angaben des Originaltitels, Übersetzers und der Sprache, aus der übersetzt wurde. Bei Einträgen, die |options=antik| haben, ist dies Standard.  \cref{translation}. 

\item[noabbrevs] (zuvor |keineabkuerzung|)  Die Abkürzungen von Zeitschriften und Serien (|shortjournal|\linebreak |shortseries|) werden ausgeschrieben, wofür die Felder |journaltitle| und |series| ausgelesen werden.  \cref{noabbrevs}.

\item[publisher] (zuvor |verlag|)  Angabe aller Verlagsorte und Verlag selbst. Ändert die Formatierung der Edition und Erstausgabe.  \cref{publisher}.

\item[bibfullname] (zuvor |bibvollername|)  Schreibt die Autoren/-Herausgeber mit vollem Namen in der Bibliographie.  \cref{bibfullname}.

\item[inreferences] (zuvor |lexika|)  Bibliographieeinträge mit |@inreference| werden bei aktivierter Funktion vollreferenziert geschrieben. Siehe \cref{inreferences}.

\item[lastnames] (zuvor |nurnachname|)  Schreibt die Autoren/-Herausgeber, die über  |citeauthor|\-\marg{key}  im Fließtext aufgerufen werden, nur mit Nachnamen (= Erscheinen in Fußnoten).  \cref{lastnames}.

\item[fullnames] (zuvor |vollername|)  Schreibt die Autoren/-Herausgeber, die über |citeauthor|\-\marg{key} im Fließtext aufgerufen werden,  mit vollem Vor- und Nachnamen -- sofern diese im Bibliographieeintrag vorhanden sind.  \cref{fullnames}.


\item[scshape] (zuvor |kapitaelchen|)  Die Namen in den Fußnoten werden in Kapitälchen gesetzt \cref{scshape}, ausgenommen sind Werke unbekannter Herkunft (\cref{unbekannt}) und Werke antiker Autoren (\cref{antik,frgantik}).


\item[width] (zuvor |abstand|)  Mit dieser Option wird in der Bibliographie der Abstand zwischen dem Label und dem Vollzitat nach Belieben gesteuert.
\cref{width}.

\item[counter] Angabe über die Anzahl an Zitationen des Werks im Text. \cref{counter}.
\end{description}


 \subsection{Literaturoptionen (Bibliographieeintrag)}
 Zusätzlich kann ein einzelner Eintrag durch folgende Werte in seinem |options|-Feld manipuliert werden. Siehe dazu auch \cref{optionen-literatur} und \cref{beispiele}. 

 \begin{description}
 \item[antik] Zeichnet den Eintrag als antike Quelle aus. Siehe \cref{antik}.
% \item[frg] Zeichnet den Eintrag als Fragment aus. Siehe \cref{frg}.
 \item[frgantik] Zeichnet den Eintrag als antikes Fragment aus. Siehe \cref{frgantik}.
 \item[corpus] Nur das |shorthand|-Feld wird beim Folgezitat ausgegeben. Wichtig für beispielsweise Inschriften- oder Münzcorpora (CIL, AE, RIC, etc.). Siehe \cref{corpus}.
% \item[lexikon] Zeichnet den Eintrag als ein zitierfähiges Lexikon aus, das über den abgekürzten Haupttitel zitiert wird (RE, DNP, LTUR, LIMC, etc.). Siehe \cref{lexikon}.
 %\item[unbekannt] Zeichnet den Eintrag als anonymes Werk aus, sodass nach dem Feld |shorthand| zitiert wird. Siehe \cref{unbekannt}.
 \end{description}

 \changes{v1.1}{2015/06/04}{Neue Optionen in Zusammenfassung ergänzt.}


 


 \subsection{cite-Befehle}\label{cite-befehle}
 \DescribeMacro{\cite}
Die einfachste Weise zum Zitieren wird mit |\cite| bewerkstelligt:\par
 \begin{syntax}|\cite|\oarg{prenote}\oarg{postnote}\marg{Schlüssel}\end{syntax}
 wobei \meta{prenote} eine einleitende Bemerkung (z.B. \enquote{Vgl.}) ist und \meta{postnote} für gewöhnlich die Seitenzahl. Wenn nur ein optionales Argument gegeben wird, so ist das die Seitenzahl:\par
 \begin{syntax}|\cite|\oarg{postnote}\marg{Schlüssel}\end{syntax}
 \meta{Schlüssel} ist dabei in jedem Fall der Schlüssel des Eintrags aus der |bib|-Datei.

 \DescribeMacro{\cites}
 Möchte man mehrere Autoren/Werke zugleich zitieren, eignet sich am besten der |\cites|-Befehl:\par
 \begin{syntax}\noindent|\cites|(pre-prenote)(post-postnote)\oarg{prenote}\oarg{postnote}\marg{Schlüssel}\%\\
 				\indent\oarg{prenote}\oarg{postnote}\marg{Schlüssel}\%\\%
 				\indent\oarg{prenote}\oarg{postnote}\marg{Schlüssel}\ldots
 				\end{syntax}
\begin{center} * * * \end{center}
 
 \DescribeMacro{\parencite}
 Möchte man Literaturangaben (bspw. in den Fußnoten) in Klammern setzen, dann empfiehlt sich dies mittels\par
   \begin{syntax}|\parencite|\oarg{prenote}\oarg{postnote}\marg{Schlüssel}\end{syntax} zu tun.
 Dieser Befehl berücksichtigt die Klammerregelung, die besagt, dass runde Klammern innerhalb einer Klammerumgebung als eckigen Klammern geschrieben werden müssen.
 Dies ist vor allem bei Lexikaeinträgen der Fall, wie das Beispiel unter \cref{inreference} zeigt.

 \DescribeMacro{\parencites}
 Möchte man mehrere Literaturangaben (bspw. in den Fußnoten) in Klammern setzen, dann empfiehlt sich dies mittels\par
 \begin{syntax}\noindent|\parencites|(pre-prenote)(post-postnote)\oarg{prenote}\oarg{postnote}\marg{Schlüssel}\%\\
 				\indent\oarg{prenote}\oarg{postnote}\marg{Schlüssel}\%\\%
 				\indent\oarg{prenote}\oarg{postnote}\marg{Schlüssel}\ldots
 				\end{syntax} zu tun.
 Dieser Befehl berücksichtigt die Klammerregelung, die besagt, dass runde Klammern innerhalb einer Klammerumgebung als eckigen Klammern geschrieben werden müssen.
 Dies ist vor allem bei Lexikaeinträgen der Fall, wie das Beispiel unter \cref{inreference} zeigt.
 \begin{center} * * * \end{center}
 
 \DescribeMacro{\textcite}
Zu den bisher aufgeführten |\cite|-Befehlen gibt es zusätzlich die Möglichkeit einen Eintrag bspw. im Fließtext mit |\textcite|  zu zitieren: \par
 \begin{syntax}\noindent|\textcite|\oarg{prenote}\oarg{postnote}\marg{Schlüssel}
 				\end{syntax} 

\DescribeMacro{\textcites}
Wiederum gibt es die Möglichkeit mehrere Werke mit |\textcites| anzugeben: \par
 \begin{syntax}\noindent|\textcites|(pre-prenote)(post-postnote)\oarg{prenote}\oarg{postnote}\marg{Schlüssel}\\%
 \indent\oarg{prenote}\oarg{postnote}\marg{Schlüssel}\%\\%
 				\indent\oarg{prenote}\oarg{postnote}\marg{Schlüssel}\ldots
 				\end{syntax} 




\begin{center} * * * \end{center}
 \DescribeMacro{\citeauthor}   \DescribeMacro{\citetitle}
Zum ›normalen‹ |\cite|-Befehl kann man im Fließtext und in den Anmerkungen auf den Autor/Herausgeber und das Werk verweisen.
Dies wird über den Befehl\par
  \begin{syntax}|\citeauthor|\oarg{prenote}\oarg{postnote}\marg{Schlüssel}\end{syntax} \par 
  und \par 
   \begin{syntax}|\citetitle|\oarg{prenote}\oarg{postnote}\marg{Schlüssel}\end{syntax}ausgeführt. Weitere Informationen unter \cref{fullnames}.
   


\end{document}
