% archaeologie --%
%            biblatex for archaeologists, 
%				historians and philologists
% Copyright (c) 2016 Lukas C. Bossert | Johannes Friedl
%  
% This work may be distributed and/or modified under the
% conditions of the LaTeX Project Public License, either version 1.3
% of this license or (at your option) any later version.
% The latest version of this license is in
%   http://www.latex-project.org/lppl.txt
% and version 1.3 or later is part of all distributions of LaTeX
% version 2005/12/01 or later.

\documentclass[a4paper,
10pt,ngerman
]{ltxdoc}
\CodelineNumbered
\AtBeginDocument{\RecordChanges}
\AtEndDocument{\PrintChanges}
\usepackage[T1,LGR]{fontenc}		% font types and character verification
%  \renewcommand*\ttdefault{lcmtt}
\usepackage{libertine}
%\renewcommand*\ttdefault{cmvtt}
%\renewcommand*\ttdefault{txtt}
%\usepackage{inconsolata}
\renewcommand*\ttdefault{lmvtt}

\usepackage[					% use  for bibliography
	backend=biber,
	style=archaeologie,
%	eprint=false,
%	doi=false,
%	url=false,
%inreferences,
%seenote,
	%translation,
	%bibfullname,
	%noabbrevs,
	%publisher,
	%edby,
	%yearseries,
%yearinparens,
	%lastnames,
	%fullnames,
%	scshape,
%	width=6em,
%counter,
%ancient,
lstabbrv,
]{biblatex}
\renewcommand\bibfont{\normalfont
\footnotesize}


	\usepackage{metalogo}
	\usepackage{hologo}
 \usepackage{babel}
\usepackage{coolthms}
\usepackage[					% advanced quotes
	strict=true,					% 	- warning are errors now
	style=ngerman,					% 	- german quotes
]{csquotes}
\usepackage{multicol}
\setlength{\columnsep}{1.5cm}
\setlength{\columnseprule}{0.2pt}
\usepackage{framed}
\usepackage{enumitem}
\setlength{\parindent}{0pt}
\setlength{\parskip}{6pt plus 2pt minus 2pt}
\setenumerate[1]{label=(\alph*),leftmargin=*,nolistsep,parsep=\parskip}
\usepackage{changepage}
\makeindex
    
\newenvironment{bsp}{\begin{framed}\begin{footnotesize}
\begin{adjustwidth}{0cm}{0cm}}{\end{adjustwidth}
\end{footnotesize}\end{framed}}
\defbibheading{empty}{}

\newcommand{\printbib}[2][4em]{
\begingroup
\begin{bsp}
\footnotesize
\begin{refsection}
\setlength{\labwidthsameline}{#1} 
\nocite{#2}
\printbibliography[heading=none]
\end{refsection}
\end{bsp}
\endgroup
}

\newcommand{\printbiball}[2][4em]{
\begingroup
\setlength{\labwidthsameline}{#1} 
\begin{bsp}
\footnotesize%
\begin{itemize}%[\bfseries]
\begin{refsection}%
\nocite{#2}%
\item[English:]{\printbibliography[heading=none]}
\item[German:]\foreignlanguage{ngerman}{\printbibliography[heading=none]}
\item[Italian:]\foreignlanguage{italian}{\printbibliography[heading=none]}
\item[French:]\foreignlanguage{french}{\printbibliography[heading=none]}
\end{refsection}
\end{itemize}%
\end{bsp}
\endgroup
}


\listfiles
\EnableCrossrefs
\CodelineIndex
\RecordChanges


 \usepackage[disable]{todonotes} % notes not showed
 
 



\addbibresource{archaeologie.bib}
%\addbibresource{archaeologie-ancient.bib}
%\addbibresource{antike-corpora.bib} %Version 2.0



\usepackage{listings}
 \definecolor{myblue}{RGB}{0,65,137}
\definecolor{codegreen}{RGB}{147,193,26}

%\definecolor{codegreen}{rgb}{0,0.6,0}
\definecolor{codegray}{rgb}{0.5,0.5,0.5}
\definecolor{codepurple}{rgb}{0.58,0,0.82}
\definecolor{backcolour}{rgb}{0.95,0.95,0.92}
 
\lstdefinestyle{mystyle}{
	language=[LaTeX]TeX,
    backgroundcolor=\color{backcolour},   
    commentstyle=\color{codegreen},
    keywordstyle=\color{myblue},
    numberstyle=\tiny\color{codegray},
    stringstyle=\color{codepurple},
    escapeinside={<@}{@>},
    texcsstyle=*\color{myblue},
    morekeywords={cites, parencites, parencite, textcite, textcites, citeauthor, citetitle,
    						@article, @book,@inproceedings@,@proceedings,@talk,@reference,@inreference,@incollection,
    						},
    basicstyle=\ttfamily\footnotesize,
    breakatwhitespace=false,         
    breaklines=true,   
    inputencoding=utf8/latin1,              
    captionpos=b,                    
    keepspaces=true,                 
    numbers=left,                    
    numbersep=5pt,            
    showspaces=false,                
    showstringspaces=false,
    showtabs=false,                  
    tabsize=2,
    literate=
            *{\{}{{{\color{codegreen}{\{}}}}{1}
            {\}}{{{\color{codegreen}{\}}}}}{1}
            {[}{{{\color{codegreen}{[}}}}{1}
            {]}{{{\color{codegreen}{]}}}}{1},
}
 
\lstset{style=mystyle}
\lstMakeShortInline[style=mystyle]{|}

\hypersetup{  colorlinks   = true, %Colours links instead of ugly boxes
  urlcolor     =  blue, %Colour for external hyperlinks
  linkcolor    = black, %Colour of internal links
  citecolor   = blue, %Colour of citations
  }	


%\MakeShortVerb{\|}
 \newenvironment{syntax}{\begin{small}\medskip\hspace*{2em}}{\par\medskip\end{small}}
 %\def\verbatimchar{;}

\begin{document}
%  \DocInput{archaeologie.dtx}

 \title{\texttt{archaeologie} -- \\\texttt{bib\LaTeX} für Archäologen\footnote{Ebenso geeignet für (Alt-)Philologen und (Alt-)Historiker.
 Die Entwicklung des Codes findet statt auf \url{https://github.com/LukasCBossert/biblatex-archaeologie}: 
 Anmerkungen und Kritik sind willkommen.
 Dank an   ›moewew‹ und Herbert Voß für ihre Hilfe.
}\\ --- {\scshape kurzanleitung} --\footnote{Für eine vollständige Anleitung siehe die englische Stilbeschreibung unter \url{http://mirrors.ctan.org/macros/latex/contrib/biblatex-contrib/archaeologie/archaeologie.pdf}} }
\author{Lukas C. Bossert\thanks{\url{lukas@digitales-altertum.de}} \and Johannes Friedl}
\date{Version: 2.0 (2016-05-31)}
 
 
 \maketitle
 \begin{abstract}
\noindent Der Stil setzt die Zitations- und Bibliographievorgaben des Deutschen Archäologischen Instituts (DAI) (Stand 2014) um. Verschiedene zusätzliche Optionen erlauben jedoch, das Erscheinungsbild den eigenen Vorlieben anzupassen. Der Stil ist nicht nur für Dokumente in deutscher Sprache geeignet, sondern unterstützt ebenso die Sprachen Englisch, Italienisch und Französisch. Dieses Dokument ist die deutsche Kurzfassung der englischen Hauptdokumentation, Details findet man dort.

 \end{abstract}


\section{Verwendung}
 \DescribeMacro{archaeologie}  Der Name des Stils ist |archaeologie| und wird an entsprechender Stelle geladen.

\begin{lstlisting}
\usepackage[style=archaeologie,%
					<@\meta{weitere Optionen}@>]{biblatex}
\bibliography<@\marg{|bib|-Datei}@>
\end{lstlisting}

|archaeologie| funktioniert out-of-the-box und lädt hierfür standardmäßig den DAI-Stil im Autor-Jahr-System. 
Um schnell und einfach im DAI-Stil zu zitieren, bedarf es also keiner weiteren Einstellungen und Optionen. 

Um den Stil den eigenen Bedürfnissen anzupassen können darüber hinaus verschiedene -- weiter unten beschriebene -- Optionen von |archaeologie| oder \enquote{konventionelle} |biblatex|-Optionen geladen werden.

Wie bei |bib|\LaTeX{} gewohnt werden Zitate über einen der |\cite|-Befehle eingebunden, der |\printbibliography|-Befehl am Ende des Dokuments oder an gewünschter Stelle gibt eine Liste der verwendeten Bibliographie aus. Da |archaeologie| jedoch unterschiedliche Zitierweisen je Textsorte wie antiker Primärliteratur oder wissenschaftlicher Sekundärliteratur unterstützt, empfiehlt es sich, die Bibliographie dementsprechend aufzuteilen (z.\,B. Antike Quellen, Abkürzungen etc.). 
Einige Möglichkeiten zur Gestaltung der Bibliographie geben wir in \cref{bibliographie}.

Um die externen Bibliographien, die |archaeologie| optional bereitstellt, korrekt einzubinden, ist ein Schrift-Encoding von Nöten, das auch griechische Schriftzeichen enthält. Für \LaTeX{} ist ein Unicode-Encoding z.\,B. über |\inputenc{utf8}| eine mögliche Lösung. \XeLaTeX{} unterstützt Unicode von Hause aus.

\section{Übersicht}
Im Folgenden stellen wir etwas knapp die Optionen und Einstellungen vor, mit denen der Stil |archaeologie| aufgerufen werden kann. Wer mehr wissen will, sei auf die englische Hauptdokumentation verwiesen. Die Optionen sind nicht Teil der oben angeführten Standardaktivierung von |archaeologie|, sondern können im Bedarfsfall ausgewählt werden.

Neben den |archaeologie|-Optionen kann man -- quasi auf eigene Gefahr -- noch die konventionellen |biblatex|-Optionen (insbesondere zur Formatierung der Abstände etc. des Literaturverzeichnisses) verwenden. Näheres zu diesen findet man in der Dokumentation von |biblatex|.

\subsection{Paketoptionen (Präambel)}\label{preamble_options}

\DescribeMacro{seenote} 
Gibt bei erster Verwendung zunächst das Vollzitat, im Anschluss dann den Rückverweis auf die Erstnennung aus.

\DescribeMacro{lstabbrv} Aktiviert die separate Bibliographiedatei |archaeologie-abbrv.bib|, mit der man über ein Makro in der eigenen Bibliographie automatisch Zeitschriften oder Reihen nach den DAI-Abkürzungsrichtlinien zitieren kann. 

\DescribeMacro{ancient} Damit wird eine separate Bibliographie geladen,
in der knapp 600 antike Autoren und Werke gespeichert sind, die direkt im Text zitiert werden können. 
Alle Einträge enthalten |keywords={ancient}|, |options={ancient}|. 
Diese Einträge steuern die Zitationsform und ermöglichen nach Textsorten getrennte Bibliographien zu erstellen (s. \cref{bibliographie}).
 
\DescribeMacro{corpora} Damit wird eine separate Bibliographie geladen,
in der über 40 Corpora fertig zitierfähig gespeichert sind.
Die Option lädt zusätzlich die Bibliographie |archaeologie-abbrv.bib|, deren separate Einbindung durch die Option |lstabbrv| (s. oben) erfolgt.
Alle Corpora-Einträge enthalten |keywords={corpus}|, |options={corpus}|. Diese Einträge steuern die Zitationsform und ermöglichen nach Textsorten getrennte Bibliographien zu erstellen (s. \cref{bibliographie}).

\DescribeMacro{edby}  Bei Sammelbänden steht anstatt \enquote{Hrsg.} nun \enquote{hrsg. v.}. 

\DescribeMacro{yearseries}  Die Reihe wird erst nach der Jahreszahl ausgegeben.  
 
\DescribeMacro{yearinparens}  Die Jahreszahl wird in Klammern gesetzt.  

\DescribeMacro{translation}   Anzeige der Angaben des Originaltitels, Übersetzers und der Sprache, aus der übersetzt wurde. 
Keinen Einfluss hat die Wahl auf Bibliographie-Einträge, die |options=ancient| haben. 
Dort wird dies standardmäßig angegeben.

\DescribeMacro{noabbrv}   Die Abkürzungen von Zeitschriften und Serien (|shortjournal| |shortseries|) werden ausgeschrieben, wofür die Felder |journaltitle| und |series| ausgelesen werden. 

\DescribeMacro{publisher}   Angabe aller Verlagsorte und Verlag selbst. Ändert die Formatierung von Auflage (|edition|) und Erstausgabe (|origyear|).  

\DescribeMacro{bibfullname}   Schreibt die Autoren/-Herausgeber mit vollem Namen in die Bibliographie. Standard ist sonst die Abkürzung der Vornamen. 

\DescribeMacro{inreferences}   Bibliographieeinträge mit |@inreference| werden bei aktivierter Funktion vollreferenziert geschrieben. 
Dies verwirft die vom DAI angegebene, spezielle Lexika-Zitation (LIMC, DNP etc.), die  nicht in der Bibliographie aufgeführt werden. 

\DescribeMacro{lastnames}   Schreibt die Autoren/-Herausgeber, die über  |\citeauthor|\-\marg{bibtex-key}  im Fließtext aufgerufen werden, nur mit Nachnamen (dies entspricht der Autoren-Darstellung in Fußnoten).  

\DescribeMacro{fullnames}   Schreibt die Autoren/-Herausgeber, die über |\citeauthor|\-\marg{bibtex-key} im Fließtext aufgerufen werden,  mit vollem Vor- und Nachnamen -- sofern diese im Bibliographieeintrag vorhanden sind.  

\DescribeMacro{scshape}   Die Namen im Fließtext und in den Fußnoten werden in Kapitälchen gesetzt, ausgenommen sind Werke unbekannter Herkunft und Werke antiker Autoren.

\DescribeMacro{width}  Mit dieser Option wird in der Bibliographie der Abstand zwischen dem Label (Kurzzitat) und dem Vollzitat nach Belieben gesteuert.

\DescribeMacro{counter} Angabe über die Anzahl an Zitationen des Werks im Fließtext. 


 \subsection{Literaturoptionen (Bibliographieeintrag)}
 Zusätzlich kann ein einzelner Eintrag durch folgende Werte in seinem |options|-Feld manipuliert werden. 

\DescribeMacro{ancient} Zeichnet den Eintrag als antike Quelle aus. 

\DescribeMacro{frgancient} Zeichnet den Eintrag als antikes Fragment aus. 

\DescribeMacro{corpus} Nur das |shorthand|-Feld wird beim Folgezitat ausgegeben. 
 Wichtig für beispielsweise Inschriften- oder Münzcorpora (CIL, AE, RIC, etc.). 


 \changes{v1.1}{2015/06/04}{Neue Optionen in Zusammenfassung ergänzt.}


 

 \subsection{cite-Befehle}\label{cite-befehle}
 \DescribeMacro{\cite} Die einfachste Weise zum Zitieren wird mit |\cite| bewerkstelligt:
\begin{lstlisting}
\cite<@\oarg{prenote}\oarg{postnote}\marg{bibtex-key}%@>
\end{lstlisting}

Dabei \meta{prenote} eine einleitende Bemerkung (z.B. \enquote{Vgl.}) ist und \meta{postnote} für gewöhnlich die Seitenzahl. 
Wenn nur ein optionales Argument gegeben wird, so ist das die Seitenzahl:
\begin{lstlisting}
\cite<@\oarg{postnote}\marg{bibtex-key}%@>
\end{lstlisting}

 \meta{bibtex-key} ist dabei in jedem Fall der Schlüssel des Eintrags aus der |bib|-Datei.


 \DescribeMacro{\cites} Möchte man mehrere Autoren/Werke zugleich zitieren, eignet sich am besten der |\cites|-Befehl:
\begin{lstlisting}
\cites(pre-prenote)(post-postnote)<@\oarg{prenote}\oarg{postnote}\marg{bibtex-key}@>%
 																	<@\oarg{prenote}\oarg{postnote}\marg{bibtex-key}@>%
 																	<@\oarg{prenote}\oarg{postnote}\marg{bibtex-key}\ldots@>
\end{lstlisting}
 

\begin{center} * * * \end{center} 
 \DescribeMacro{\parencite} Möchte man Literaturangaben (bspw. in den Fußnoten) in Klammern setzen, dann empfiehlt sich dies  zu tun mittels:

\begin{lstlisting}
\parencite<@\oarg{prenote}\oarg{postnote}\marg{bibtex-key}%@>
\end{lstlisting} 
 Dieser Befehl berücksichtigt die Klammerregelung, die besagt, dass runde Klammern innerhalb einer Klammerumgebung als eckigen Klammern geschrieben werden müssen.
 Dies ist vor allem bei Lexikaeinträgen der Fall.


 \DescribeMacro{\parencites}  Möchte man mehrere Literaturangaben (bspw. in den Fußnoten) in Klammern setzen, dann empfiehlt sich dies zu tun mittels
\begin{lstlisting}
\parencites(pre-prenote)(post-postnote)<@\oarg{prenote}\oarg{postnote}\marg{bibtex-key}@>%
 																			<@\oarg{prenote}\oarg{postnote}\marg{bibtex-key}@>%
 																			<@\oarg{prenote}\oarg{postnote}\marg{bibtex-key}\ldots@>
\end{lstlisting}
 Dieser Befehl berücksichtigt die Klammerregelung, die besagt, dass runde Klammern innerhalb einer Klammerumgebung als eckigen Klammern geschrieben werden müssen.
 Dies ist vor allem bei Lexikaeinträgen der Fall.
 \begin{center} * * * \end{center}
 \DescribeMacro{\textcite}
Zu den bisher aufgeführten |\cite|-Befehlen gibt es zusätzlich die Möglichkeit einen Eintrag bspw. im Fließtext mit |\textcite|  zu zitieren: 
\begin{lstlisting}
\textcite<@\oarg{prenote}\oarg{postnote}\marg{bibtex-key}%@>
\end{lstlisting} 

\DescribeMacro{\textcites}
Wiederum gibt es die Möglichkeit mehrere Werke mit |\textcites| anzugeben: 
  \begin{lstlisting}
\textcites(pre-prenote)(post-postnote)<@\oarg{prenote}\oarg{postnote}\marg{bibtex-key}@>%
 																			<@\oarg{prenote}\oarg{postnote}\marg{bibtex-key}@>%
 																			<@\oarg{prenote}\oarg{postnote}\marg{bibtex-key}\ldots@>
\end{lstlisting}

\begin{center} * * * \end{center}
 \DescribeMacro{\citeauthor}   \DescribeMacro{\citetitle}
Zum ›normalen‹ |\cite|-Befehl kann man im Fließtext und in den Anmerkungen auf den Autor/Herausgeber und das Werk verweisen.
Dies wird über folgenden Befehl ausgeführt:
\begin{lstlisting}
\citeauthor<@\oarg{prenote}\oarg{postnote}\marg{bibtex-key}%@>
\end{lstlisting} 
  und 
\begin{lstlisting}
\citetitle<@\oarg{prenote}\oarg{postnote}\marg{bibtex-key}%@>
\end{lstlisting} 
   

 \section{Kategorien}
 Es werden die auslesbaren Felder jeder zur Verfügung stehenden Kategorie aufgelistet.
 \subsection{@article}
 Für Artikel aus periodischen Zeitschriften.
 \begin{itemize}
\item[Notwendig:] |author|, |title|, |subtitle|,  |titleaddon|, |pages|,
 |journaltitle|, |shortjournal|, |volume|, |number|, |year|,
\item[Optional:] 
 |doi|, |url|, |urldate|, |eprint|, |eprinttype|, |note|, |addendum|, |pubstate|, 
 \end{itemize}

 \printbib[8em]{Ball_2013}

 \subsection{@review}
 Für Rezensionen.
 \begin{itemize}
\item[Notwendig:] |author|, |title|, |subtitle|, |titleaddon|, |pages|,
 |journaltitle|, |shortjournal|, |volume|, |number|, |year|,
 |related|, |relatedtype|
\item[Optional:] 
 |doi|, |url|, |urldate|, |eprint|, |eprinttype|, |note|, |addendum|, |pubstate|, 
 \end{itemize}
 
  Beispiel:
 \printbib[5em]{Taylor_2008}
 
  \subsection{@book/@collection}
  Für Monographien und Sammelbände.
 \begin{description}
\item[Notwendig:] |author|/|editor|, |title|, |subtitle|, |titleaddon|,
 |location|, |year|,
\item[Optional:] |maintitle|, |mainsubtitle|, |maintitleaddon|,
|related|, |relatedtype|,
|publisher|, |series|, |number|, |edition|, |origyear|, |origlanguage|, |origlocation|, |translator|,
|volume|,
 |doi|, |url|, |urldate|, |eprint|, |eprinttype|, |note|, |addendum|, |pubstate|, 
 \end{description}
 
  Beispiel:
 \printbib[7em]{Mann_2011}
 
 \subsection{@inbook/@incollection}
 Für Beiträge innerhalb eines Sammelbandes.
  \begin{description}
\item[Notwendig:] |author|, |editor|, |title|, |subtitle|, |titleaddon|,
 |location|, |year|, |pages|, 
 |booktitle|, |booksubtitle|, |booktitleaddon|,
\item[Optional:] |maintitle|, |mainsubtitle|, |maintitleaddon|,
|related|, |relatedtype|,
|publisher|, |series|, |number|, |edition|, |origyear|, |origlanguage|, |origlocation|, |translator|,
|volume|,
 |doi|, |url|, |urldate|, |eprint|, |eprinttype|, |note|, |addendum|, |pubstate|, 
 \end{description}
\printbib[5em]{Mundt_2015}
 
 \subsection{@proceedings}
Für Konferenz-/Kolloquiumssammelbände, bei denen ein Datum angegeben ist. 
  \begin{description}
\item[Notwendig:] |author|/|editor|, |title|, |subtitle|, |titleaddon|,
 |location|, |year|,
 |eventdate|, |eventtitle|, |venue|
\item[Optional:] |maintitle|, |mainsubtitle|, |maintitleaddon|,
|related|, |relatedtype|,
|publisher|, |series|, |number|, |edition|, |origyear|, |origlanguage|, |origlocation|, |translator|,
|volume|,
 |doi|, |url|, |urldate|, |eprint|, |eprinttype|, |note|, |addendum|, |pubstate|, 
 \end{description}
 
 
\printbib[7em]{Kurapkat_2014}

 
  \subsection{@inproceedings}
Für Beiträge innerhalb eines   Konferenz-/Kolloquiumssammelbandes, bei denen ein Datum angegeben ist. 
\begin{description}
\item[Notwendig:] |author|, |editor|, |title|, |subtitle|, |titleaddon|,
 |location|, |year|, |pages|, 
 |booktitle|, |booksubtitle|, |booktitleaddon|,
 |eventdate|, |eventtitle|, |venue|
\item[Optional:] |maintitle|, |mainsubtitle|, |maintitleaddon|,
|related|, |relatedtype|,
|publisher|, |series|, |number|, |edition|, |origyear|, |origlanguage|, |origlocation|, |translator|,
|volume|,
 |doi|, |url|, |urldate|, |eprint|, |eprinttype|, |note|, |addendum|, |pubstate|, 
\end{description}
 
 \printbib[7em]{Wulf-Rheidt_2013}
 
\subsection{@talk}
  Für (mündliche) Vorträge (Kolloquium, Konferenz) 
\begin{description}
\item[Notwendig:]  |author|, |title|, |subtitle|, |titleaddon|, |date|, |venue|,
|institution|, |eventtitle|, |eventdate|
\item[Optional:] 
 |doi|, |url|, |urldate|, |eprint|, |eprinttype|, |note|, |addendum| 
\end{description}

\printbib[6em]{Bergmann_2015}


  \subsection{@reference}
Für Lexika als Gesamtwerk.
\begin{description}
\item[Notwendig:] |editor|, |title|, |subtitle|, |titleaddon|,
 |location|, |year|
\item[Optional:] |maintitle|, |mainsubtitle|, |maintitleaddon|,
|related|, |relatedtype|,
|publisher|, |series|, |number|, |edition|, |volume|,
 |doi|, |url|, |urldate|, |eprint|, |eprinttype|, |note|, |addendum|, |pubstate|, 
\end{description}


\printbib{ThesCRA}

   \subsection{@inreference}
Für  Beiträge innerhalb eines Lexikon.
\begin{description}
\item[Notwendig:] |author|, |title|, |subtitle|, |titleaddon|,
 |location|, |year|, |pages|, |volume|, 
 |booktitle|, |booksubtitle|, |booktitleaddon|
\item[Optional:] |maintitle|, |mainsubtitle|, |maintitleaddon|,
|related|, |relatedtype|,
|publisher|, |series|, |number|, |edition|, |origyear|, 
 |doi|, |url|, |urldate|, |eprint|, |eprinttype|, |note|, |addendum|, |pubstate|, 
\end{description}

\printbib[5em]{Nieddu_1995}


   \subsection{@thesis}
Für (unpublizierte) eingereichte Dissertationen oder Master/Magisterarbeiten.
\begin{description}
\item[Notwendig:] |author|, |title|, |subtitle|, |titleaddon|,
 |location|, |year|, |institution|,  |type|, 
\item[Optional:] 
|publisher|, |series|, |number|, |edition|, |origyear|, 
 |doi|, |url|, |urldate|, |eprint|, |eprinttype|, |note|, |addendum|, |pubstate|, 
\end{description}

Für eine Dissertation:  |type = {phdthesis}|, \\ für eine Magister/Masterarbeit: |type={mathesis}|.
\printbib[5em]{Arnolds_2005}

 \section{Bibliographie}\label{bibliographie}
 \DescribeMacro{\printbibliography}
Wie bei jedem Dokument mit im Text zitierten Werken bedarf es einer Stelle, an der diese auch aufgeschlüsselt werden: die Bibliographie. 
Für Altertumswissenschaftler (und auch andere) ist es manchmal hilfreich verschiedene Bibliographien im Dokument zu haben, die unterschiedliche Arten von Werke beinhalten, bspw. ein Quellenverzeichnis, Abkürzungen und Forschungsliteratur. 

Nachfolgend wird gezeigt, wie dies berwerkstelligt werden kann. 
Zunächst sollten alle Quellen in der |bib|-Datei mit dem Feld
 |keyword={ancient},|
 versehen werden. 
Es bietet sich  an, mit (nummerierten) Unterbibliographien zu arbeiten, die über die Option  |heading=bibnumbered|, bzw. |heading=subbibnumbered| geladen werden.

\begin{lstlisting}
\printbibheading[%
							heading=bibnumbered,%
							title={Bibliographie}]% Überschrift für Bibliographie

\printbibliography[%
							keyword=ancient,%
							heading=subbibnumbered,%
							title={Antike Quellen}]

\printbibliography[%
							notkeyword=ancient,%
							notkeyword=corpus,%
							heading=subbibnumbered,%
							title={Forschungsliteratur}]
\end{lstlisting}



\begin{refsection}
\nocite{Cic:Att,Fest,%
Ball_2013,Taylor_2008,Mann_2011,Mundt_2015,Kurapkat_2014,%
Wulf-Rheidt_2013,Bergmann_2015,ThesCRA,Nieddu_1995,Arnolds_2005}

Damit wird zuerst die Quellen und danach das \enquote{gewöhnliche} Literaturverzeichnis getrennt voneinander ausgegeben. 
\setcounter{section}{0}
\begin{bsp}
\renewcommand\bibfont{\normalfont\footnotesize}
\printbibheading[%
							heading=bibnumbered,%
							title={Bibliographie}] %Überschrift für Bibliographieumgebung}

\printbibliography[%
							keyword=ancient,%
							heading=subbibnumbered,%
							title={Antike Quellen}]

\printbibliography[%
							notkeyword=corpus,%
							notkeyword=ancient,%
							heading=subbibnumbered,%
							title={Forschungsliteratur}]
\end{bsp}


Es können mehrere Bibliographien über |\printbibliography| erstellt werden, die jeweils unterschiedliche Einträge haben können.
Beispielsweise kann man eine Unterbibliographie erstellen, in der nur die Sigeln (Lexika, Handbücher, Inschriftencorpora, etc) aufgeführt werden, sodass diese dann aus der |Forschungsliteratur| herausfliegen (dort  |notkeyword=Sigel| ergänzen). Dafür wird das Feld |keyword| auf den Inhalt |Sigel| ausgelesen:

\begin{lstlisting}
\printbibliography[%
						keyword=corpus,%
						heading=subbibnumbered,%
						title={Abkürzungen und Sigel}]
\end{lstlisting}

Die Teilbibliographie umfasst dann nur Einträge, die unter |keywords = {Sigel}| stehen haben:
\begin{bsp}
\printbibliography[%
				keyword={corpus},
           		heading=subbibnumbered,
            	title={Abkürzungen und Sigel}]\label{bib:sigel}
\end{bsp}

Ebenso hilfreich kann es sein, dass die verwendeten Abkürzungen der Zeitschriften und Reihen aufgelöst werden.
Dies geschieht  für die Zeitschriften mit:
\begin{lstlisting}
\printbiblist[%
					heading=subbibnumbered,%
					title={Zeitschriftenabkürzungenl}]{shortjournal}
\end{lstlisting}

\begin{bsp}
\printbiblist[heading=subbibnumbered,
title={Zeitschriftenabkürzungen}]{shortjournal}
\end{bsp}

bzw. für die Reihen:
\begin{lstlisting}
\printbiblist[%
					heading=subbibnumbered,%
					title={Reihenabkürzungen}]{shortseries}
\end{lstlisting}


\begin{bsp}
\printbiblist[heading=subbibnumbered,title={Reihenabkürzungen}]{shortseries}\end{bsp}

\end{refsection}
\end{document}
