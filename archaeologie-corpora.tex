\begin{footnotesize}
\begin{description}[%
			%	style=multiline,
				style=nextline,
				leftmargin=1.5cm,
				%font=\normalfont\bfseries
				]
\item[ABV] Attic Black-figure Vase-painters
\item[AE] L'année épigraphique 
\item[AHw] Akkadisches Handwörterbuch
\item[ARV2] Attic Red-figure Vase-painters
\item[CAD] The Assyrian Dictionary of the Oriental Institute of the University of Chicago 
\item[CIL] Corpus inscriptionum Latinarum 
\item[DACL] Dictionnaire d'archéologie chrétienne et de liturgie 
\item[Daremberg-Saglio] Dictionnaire des antiquités grecques et romaines d'après les textes et les monuments. Ouvrage rédigé sous la direction de Ch. Daremberg et E. Saglio 
\item[DNP] Der Neue Pauly. Enzyklopädie der Antike 
\item[EAA] Enciclopedia dell'arte antica classica e orientale 
\item[FGrHist] Die Fragmente der griechischen Historiker
\item[FHG] Fragmenta historicorum Graecorum 
\item[FR] A. Furtwängler – K. Reichhold, Griechische Vasenmalerei (München 1900--1925) 
\item[HAW] Handbuch der Altertumswissenschaften 
\item[HdArch] Handbuch der Archäologie 
\item[Head] B. V. Head, Historia Numorum. A Manual of Greek Numismatics (Oxford 1887; 1911)
\item[Helbig] W. Helbig, Führer durch die öffentlichen Sammlungen klassischer Altertümer in Rom 
\item[IG] Inscriptiones Graecae 
\item[IGR] Inscriptiones Graecae ad res Romanas pertinentes 
\item[IK] Inschriften griechischer Städte aus Kleinasien
\item[ILS] Inscriptiones Latinae selectae
\item[LAe] Lexikon der Ägyptologie
\item[LIMC] Lexikon iconographicum mythologiae classicae
\item[LSJ] G. Liddell – R. Scott – H. S. Jones, A Greek-English Lexikon \textsuperscript{9}(1996); Suppl. (1996)
\item[LTUR] Lexikon topographicum urbis Romae 
\item[PIR] Prosopographia Imperii Romani 
\item[PPM] Pompei: Pitture e mosaici. Enciclopedia dell’arte antica classica e orientale 
\item[PPP] Pitture e Pavimenti di Pompei
\item[RAC] Reallexikon für Antike und Christentum 
\item[RBK] Reallexikon zur byzantinischen Kunst
\item[RE] Paulys Realencyclopädie der classischen Altertumswissenschaft 
\item[RES] Répertoire d’épigraphie sémitique (Paris 1900--1950) 
\item[RIA] Rivista dell’Istituto nazionale d’archeologia e storia dell’arte
\item[RIC] H. Mattingly – E. A. Sydenham, The Roman Imperial Coinage 
\item[RoscherML] W. H. Roscher, Ausführliches Lexikon der griechischen und römischen Mythologie
\item[RPC] Roman Provincial Coinage
\item[RRC] M. Crawford, Roman Republican Coinage (London 1974) 
\item[SEG] Supplementum epigraphicum Graecum 
\item[SIG] Sylloge inscriptionum Graecarum
\item[SNG] Sylloge nummorum Graecorum 
\item[TAM] Tituli Asiae Minoris
\item[ThesCRA] Thesaurus Cultus et Rituum Antiquorum
\item[Thieme-Becker] U. Thieme – F. Becker (Hrsg.), Allgemeines Lexikon der bildenden Künstler
\item[TIB] Tabula Imperii Byzantini
\end{description}
\end{footnotesize}