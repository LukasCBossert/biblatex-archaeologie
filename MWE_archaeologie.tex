\documentclass[%
	english,ngerman,%	Sprachen immer gleich global und hier definieren
	11pt,
	paper=A4,					% paper size --> A4 is default in Germany
draft=false,						% value for draft version
demo,
	parskip=half+,				% spacing value / method for paragraphs --> Eine halbe Zeile Abstand nach Absatz
]{scrreprt}%
\usepackage{babel}
\usepackage{libertine}
\usepackage[					% use  for bibliography
	backend=biber,
	bibencoding=utf8,				% 	- use auto file encode
%style=authortitle-dw, 
	style=archaeologie,
 autorjahr=true,		%% true = Zitation im Autor-Jahr-System; Anpassung der Bibliographie; false = Zitation im Autor-Titel-System
shortjournal=false,	%% true = nur Kurzname der Zeitschrift (Feld: "shortjournal"), wenn "shortjournal" leer, dann Feld "journal"; false = Angabe von Feld "journal"
notranslator=false,	%% true = keine Angabe von Originaltitle, Originalsprache, Übersetzer; false = "Originaltitel: XY, übers. v. Z"
lexika=true, %true = in Fußnote Zitation nach DAI (LTUR 2 (19994} 123 s.v. ); false = Zitation nach Autor-Jahr oder Autor-Zitation
hrsg=true,		%% false = "hrsg. v. XY"; true = "XY (Hrsg.)“
xref=true,
citeinit=false, 	%% false = keine Initiale in Fußnoten, true =  in Fußnoten Initiale nachgestellt
mitvn=false,		%% false = keine Vornamen in Fußnote; true = Vornamen in Fußnote nach Nachname
jahrkeineklammern=false,	%% false = (2015); true = 2015
mitjahr=true,%		%% WENN "kurztitel = true", DANN:  false = keine Jahresangabe in der Fußnote; true = Jahresangabe nach Kurztitel in Klammer
jahrreihe=false,%	%% false = Serie und Nummer VOR Erscheinungsort und -jahr; true = Serie und Nummer NACH Erscheinungsort und -jahr
	%%%%%%%%%%%%%%%%%%%%%%%
	sorting=nyt, % nyt lassen, damit Shorthands sortiert werden
	%indexing=cite,	%%Aktiviert die Indexerstellung nur für Verweise
	uniquename=full,
	maxalphanames=2,
	maxnames=3,%
	minnames=1,%
%	maxbibnames=4,%
%	minbibnames=1,%
%	maxcitenames=2,%
%	mincitenames=1%
]{biblatex}

                  

\usepackage{url}
\urlstyle{same}


\addbibresource{MWE_archaeologie.bib}



\usepackage[colorlinks   = true, %Colours links instead of ugly boxes
]{hyperref}

\begin{document}
ok, das ist mein erster Versuch etwas zu schreiben. 


%hello world\footnote{%
%\cite{Baker_2011,Lackner_2008}}

\footnote{\cite{MacDonald_1986,MacDonald_1982}}

\renewcommand\bibfont{\normalfont\small}
%\setlength{\biblabelsep}{3em} %geht nur bei autorjahr
%\setlength{\bibhang}{{3em}\space}
%\setlength{\bibitemsep}{0.5\baselineskip plus 0.5\baselineskip}
%\nocite{*} % alle Bibliographieeinträge in die Auflistung

%%%%%%%%%%%%%%%  Bibliographie (-Überschriften  %%%%%%%%%%%%%%%


\printbibheading[heading=bibintoc,title={Bibliographie}] 
%\printshorthands[keyword=Sigel,heading=subbibintoc,title={Abkürzungen}] 
\printbibliography[keyword = Sigel,heading=subbibintoc,title={Abkürzungen}] 
\printbibliography[keyword = unbekannt,heading=subbibintoc,title={Unbekannte Autoren}] 
\printbibliography[keyword=Quelle,heading=subbibintoc,title={Primärliteratur}] 
\printbibliography[%
		notkeyword=Quelle,%
		notkeyword=Sigel,%
		notkeyword=unbekannt,%
		%notkeyword=lexikon,%
		heading=subbibintoc,title={Forschungsliteratur}] 


%\printbibliography[nottype=online,nottype=shorthand] %alle, die nicht online sind
%\printbibliography[heading=subbibliography,title={Webseiten},type=online,prefixnumbers={@}]
\end{document}
