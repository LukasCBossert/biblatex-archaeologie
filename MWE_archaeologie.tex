\documentclass[%
	english,ngerman,%	Sprachen immer gleich global und hier definieren
	11pt,
	paper=A4,					% paper size --> A4 is default in Germany
draft=false,						% value for draft version
demo,
	parskip=half+,				% spacing value / method for paragraphs --> Eine halbe Zeile Abstand nach Absatz
]{scrreprt}%
\usepackage{babel}
\usepackage{libertine}
\usepackage{imakeidx}
			\makeindex
\usepackage[					% use  for bibliography
	backend=biber,
	bibencoding=utf8,				% 	- use auto file encode
%style=authortitle-dw, 
	%style=geschichtsfrkl, fnverweise=true,
	style=archaeologie,
	dai=true,			%% true=Bibliographie und Zitation nach DAI-Richtlinie
	%dai-verweis=true,			%% true=Bibliographie und Zitation nach DAI-Richtlinie
	dateabbrev=false, %false default: true --> muss notwendigerweise für DAI auf "false" stehen, -> damit aus Feb. -> Februar wird ((in)proceedings)
	%vollername=true,		%\citeauthor --> Angabe des vollen Vor- und Zunamens im Fließtext (Fußnote nur Nachname)
	%nachname=true,		%\citeauthor --> Angabe nur des Nachnamens im Fließtext (Fußnote nur Nachname)
	initialnachname=true,		%\citeauthor --> Angabe der Initialien und des Nachnamens im Fließtext (Fußnote nur Nachname)
	lexika=true, %true = in Fußnote Zitation nach DAI (LTUR 2 (19994} 123 s.v. ); false = Zitation nach Autor-Jahr oder Autor-Zitation
%%%%%%%%%% OPTIONALE OPTIONEN - NICHT für dai %%%%%%%%%%%% 	
 	%autorjahr=true,		%% true = Zitation im Autor-Jahr-System; Anpassung der Bibliographie; false = Zitation im Autor-Titel-System
	%shortjournal=false,	%% true = nur Kurzname der Zeitschrift (Feld: "shortjournal"), wenn "shortjournal" leer, dann Feld "journal"; false = Angabe von Feld "journal"
%notranslator=false,	%% true = keine Angabe von Originaltitle, Originalsprache, Übersetzer; false = "Originaltitel: XY, übers. v. Z"
%hrsg=true,		%% false = "hrsg. v. XY"; true = "XY (Hrsg.)“		
%nurinit=true,		%% true = Vornamen des Autors abgekürzt; false = ausgeschrieben
%citeinit=false, 	%% false = keine Initiale in Fußnoten, true =  in Fußnoten Initiale nachgestellt
%mitvn=false,		%% false = keine Vornamen in Fußnote; true = Vornamen in Fußnote nach Nachname
%jahrkeineklammern=false,	%% false = (2015); true = 2015
%mitjahr=true,%		%% WENN "kurztitel = true", DANN:  false = keine Jahresangabe in der Fußnote; true = Jahresangabe nach Kurztitel in Klammer
%jahrreihe=false,%	%% false = Serie und Nummer VOR Erscheinungsort und -jahr; true = Serie und Nummer NACH Erscheinungsort und -jahr
	%%%%%%%%%%%%%%%%%%%%%%%
	sorting=nyt, % nyt lassen, damit Shorthands sortiert werden
	indexing=true,	%%Aktiviert die Indexerstellung nur für Verweise
	uniquename=full,
	maxalphanames=2,
	maxnames=3,%
	minnames=1,%
%	maxitems=1,
%	maxbibnames=4,%
%	minbibnames=1,%
%	maxcitenames=2,%
%	mincitenames=1%
]{biblatex}

                  

\usepackage{url}
\urlstyle{same}
\usepackage{nth}

\addbibresource{MWE_archaeologie.bib}



\usepackage[colorlinks   = false, %Colours links instead of ugly boxes
]{hyperref}

\begin{document}
ohne Autor/Hrsg\footnote{\cite{Italie_1976,Cosa_1949,Paestum_1854,Ercolano_1762}}


@Article\footnote{
\cite{Alfoeldy_2003,Allison_2001,Andreae_1957,Anichini_2012,Babcock_1962,Ball_2013,Bartosiewicz_2003,Baumgart_1935}
}
\begin{footnotesize}
\begin{verbatim}{\citeauthor{Alfoeldy_2003}} \end{verbatim} --> \citeauthor{Alfoeldy_2003}\footnote{In einer Fußnote nur Nachnamen: \citeauthor{Alfoeldy_2003}}
\end{footnotesize}

@Book\footnote{\cite{Beard_2008,Haug_2003,Hilger_2011,Kienast_2004,Kleinwaechter_2001,Kreikenbom_2011,Pedley_1990,Rich_2002}}\\
Übersetzung\footnote{\cite{Lefebvre_2000,Lefebvre_2011}}\\
Austellungskatalog\footnote{\cite{Horn_1976}}

@Incollection\footnote{\cite{Brogiolo_2006,Burgio_2012,Calapa_2009,Christie_2009,Colin_2000,Davies_2014,Demandt_1982}}

@Proceedings\footnote{\cite{Dickenson_2013,Giannikouri_2011,Hekster_2009,Kurapkat_2014,Maggi_2011}}
\\
\begin{footnotesize}
\begin{verbatim}{\citeauthor{Dickenson_2013}} \end{verbatim} --> \citeauthor{Dickenson_2013}\footnote{Bei fehlender Author-Angabe, wird der Herausgeber ausgegeben. In einer Fußnote nur Nachnamen:  \citeauthor{Dickenson_2013}}
\begin{verbatim}{\citetitle{Dickenson_2013}} \end{verbatim}  --> \citetitle{Dickenson_2013}
\end{footnotesize}

@Inproceedings\footnote{\cite{Bacchetta_2011,Coqueugniot_2011,Danner_2014,Poupaki_2011,Santoriello_1999,Torelli_1991,Torelli_1988,Tosi_1995}}

@Inreference\footnote{\cite{Rosenberger_2012,Booms_2014,CiancioRossetto_1993,Eder_2001a,Graffunder_1914,Jongman_2001,Kornemann_1933}}

@review\footnote{\cite{Baker_2011,Bell_2011,Bernard_2013,Chamberland_1999,Dyson_2013,Earl_2009,Frost_2001,Giuliano_1978,Hufschmid_2010,Kaiser_2014a}}

@thesis\footnote{\cite{Arnolds_2005,Evangelidis_2007,Hoevelborn_1983,Johanson_2008,Kienlin_2004a}}

@online\footnote{\cite{Selle_2008,Selle_2008a,Selle_2008b}}



ANTIKE

@incollection\footnote{\cite{Cic_Arch,Cic_predadQuir,Cic_prov,Cic_fam}}

@book\footnote{\cite{Caes_Gall,Ios_bell_Iud,Plin_nat,Mart_Epigr}}

@book - frgantik\footnote{\cite[532]{Fest}, \cite[23]{Paul_Fest}}

\begin{footnotesize}
\begin{verbatim}{\citeauthor{Caes_Gall}} \end{verbatim}  --> \citeauthor{Caes_Gall}\\
\begin{verbatim}{\citetitle{Caes_Gall}} \end{verbatim} --> \citetitle{Caes_Gall}
\begin{verbatim}{\citeauthor{Mart_Epigr}} \end{verbatim}  --> \citeauthor{Mart_Epigr}\\
\begin{verbatim}{\citetitle{Mart_Epigr}} \end{verbatim} --> \citetitle{Mart_Epigr}

\end{footnotesize}

CORPORA

CIL\footnote{\cite[06, 12345]{CIL}}


Rückverweise\footnote{%
\cite{Bacchetta_2011,Selle_2008b,Paestum_1854,Hekster_2009,Graffunder_1914,Alfoeldy_2003,Kaiser_2014a}\\
Antike: \cite{Caes_Gall,Paul_Fest} <-- Darf sich NICHT ändern
}

\renewcommand\bibfont{\normalfont\small}
%\setlength{\biblabelsep}{3em} %geht nur bei autorjahr
%\setlength{\bibhang}{{3em}\space}
%\setlength{\bibitemsep}{0.5\baselineskip plus 0.5\baselineskip}
%\nocite{*} % alle Bibliographieeinträge in die Auflistung

%%%%%%%%%%%%%%%  Bibliographie (-Überschriften  %%%%%%%%%%%%%%%


\printbibheading[heading=bibintoc,title={Bibliographie}] 
%\printshorthands[keyword=Sigel,heading=subbibintoc,title={Abkürzungen}] 
\printbibliography[keyword = Sigel,heading=subbibintoc,title={Abkürzungen}] 
\printbibliography[keyword = unbekannt,heading=subbibintoc,title={Unbekannte Autoren}] 
\printbibliography[keyword=Quelle,heading=subbibintoc,title={Primärliteratur}] 
\printbibliography[%
		notkeyword=Quelle,%
		notkeyword=Sigel,%
		notkeyword=unbekannt,%
		%notkeyword=lexikon,%
		heading=subbibintoc,title={Forschungsliteratur}] 
%\printbibliography
\printindex
%\printbibliography[nottype=online,nottype=shorthand] %alle, die nicht online sind
%\printbibliography[heading=subbibliography,title={Webseiten},type=online,prefixnumbers={@}]
\end{document}
